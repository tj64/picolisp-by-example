%%%%%%%%%%%%%%%%%%%%%%acknow.tex%%%%%%%%%%%%%%%%%%%%%%%%%%%%%%%%%%%%%%%%%
% sample acknowledgement chapter
%
% Use this file as a template for your own input.
%
%%%%%%%%%%%%%%%%%%%%%%%% Springer %%%%%%%%%%%%%%%%%%%%%%%%%%

\extrachap{Acknowledgements}

This book was produced using the following software tools:

\begin{itemize}
\item Archlinux
\item GNU Emacs (AucTex)
\item LaTeX
\item Git
\item Pandoc
\item Gimp
\item PicoLisp
\end{itemize}

The book-layout is based on the freely available \emph{Springer}
\texttt{LaTeX} template for monographs
(\href{http://www.springer.com/authors/book+authors?SGWID=0-154102-12-970131-0}{\texttt{svmono}}).

The core part of the book (Rosetta Code Tasks) is based on the
programming tasks published on
\href{http://rosettacode.org/wiki/Rosetta_Code}{\texttt{rosettacode.org}}.
The first part (99 Lisp Problems) is based on a Prolog problem list by
\texttt{werner.hett@hti.bfh.ch}. The original is at

\href{https://prof.ti.bfh.ch/hew1/informatik3/prolog/p-99}{\texttt{https://prof.ti.bfh.ch/hew1/informatik3/prolog/p-99}}.
     
All the PicoLisp solutions presented in this book are written by
\emph{Alexander Burger}, the creator of PicoLisp. 

The source code for this book can be found on Github (\href{https://github.com/tj64/picolisp-by-example}{\url{https://github.com/tj64/picolisp-by-example}}

% \vfill

% \begin{wideverbatim}
%   Copyright (c)  2012  Alexander Burger, Thorsten Jolitz
%   Permission is granted to copy, distribute and/or modify this document
%   under the terms of the GNU Free Documentation License, Version 1.2
%   or any later version published by the Free Software Foundation;
%   with no Invariant Sections, no Front-Cover Texts, and no Back-Cover
%   Texts.  A copy of the license is included in the chapter entitled "GNU
%   Free Documentation License".
% \end{wideverbatim}



