%%%%%%%%%%%%%%%%%%%%%%acknow.tex%%%%%%%%%%%%%%%%%%%%%%%%%%%%%%%%%%%%%%%%%
% sample acknowledgement chapter
%
% Use this file as a template for your own input.
%
%%%%%%%%%%%%%%%%%%%%%%%% Springer %%%%%%%%%%%%%%%%%%%%%%%%%%

\extrachap{Acknowledgements}

This book was produced using the following software tools:

\begin{itemize}
\item Archlinux
\item GNU Emacs (AucTex)
\item LaTeX
\item Git
\item Pandoc
\item Gimp
\item PicoLisp
\end{itemize}

The book-layout is based on the freely available \emph{Springer}
\texttt{LaTeX} template for monographs
(\href{http://www.springer.com/authors/book+authors?SGWID=0-154102-12-970131-0}{\texttt{svmono}}).

The first part of the book (\textbf{99 Lisp Problems}) is based on a Prolog
problem list by \texttt{werner.hett@hti.bfh.ch}. The original is at

\href{https://prof.ti.bfh.ch/hew1/informatik3/prolog/p-99}{\texttt{https://prof.ti.bfh.ch/hew1/informatik3/prolog/p-99}}.

The core part of the book (\textbf{Rosetta Code Tasks}) is based on
the programming tasks published on
\href{http://rosettacode.org/wiki/Rosetta_Code}{\texttt{http://rosettacode.org/wiki/Rosetta\_Code}}.
These sometimes quite elaborated task descriptions have been
contributed by members of the \emph{Rosetta Code} community. A task
description might be the work of one or several community members.
Often one person\footnote{as an outstanding example for the great work
  of the \emph{Rosetta Code} community, \emph{Mr. Donald McCarthy} (aka
  \href{http://rosettacode.org/wiki/User:Paddy3118}{Paddy3118}) alone
  has contributed 117 initial task descriptions (\emph{accessed online:
  05 Sept. 2012})} delivers the initial task, that is then discussed and
refined by the community.

At the time of this writing, it is technically challenging to
correctly credit the task authors for their work. Therefore we
added \emph{links} to the original webpages for all of the more
than 600 Rosetta Code tasks included in this book (see
\emph{Appendix}). When visiting a task page on
\href{http://rosettacode.org/wiki/Rosetta_Code}{\texttt{http://rosettacode.org/wiki/Rosetta\_Code}},
the reader can chose the \emph{View History} tab and scroll through
the sometimes long list of contributions to the page. Most likely, the
oldest entry in the history list shows the original contributor of the
task description, but other contributions to the task description
might be buried in the many pages of 'diffs' (that include all
changes to the hundreds of task solutions too).

The \emph{Rosetta Code} community is currently engaged in improving
the visibility of the task description authors on the site:

\href{http://rosettacode.org/wiki/Task\_Description\_Authors}{\texttt{http://rosettacode.org/wiki/Task\_Description\_Authors}}

Future versions of this book will include or link-to the results of
this community process in order to better credit the members of the
\emph{Rosetta Code} community for their impressive voluntary
contributions. 
     
While the \emph{Rosetta Code} task descriptions are the work of many
people, all the \textbf{PicoLisp solutions} presented in this book are
written by one single person: \emph{Alexander Burger}, the creator of
PicoLisp.

The source code for this book can be found on Github (\href{https://github.com/tj64/picolisp-by-example}{\url{https://github.com/tj64/picolisp-by-example}}

% \vfill

% \begin{wideverbatim}
%   Copyright (c)  2012  Alexander Burger, Thorsten Jolitz
%   Permission is granted to copy, distribute and/or modify this document
%   under the terms of the GNU Free Documentation License, Version 1.2
%   or any later version published by the Free Software Foundation;
%   with no Invariant Sections, no Front-Cover Texts, and no Back-Cover
%   Texts.  A copy of the license is included in the chapter entitled "GNU
%   Free Documentation License".
% \end{wideverbatim}



