% \title*{PicoLisp Function Reference}
% \input{mainmatter/authors/alex}
% \maketitle


\chapter{Functions starting with A}
\label{sec-8-1-1}

 
\section{*Adr}
\label{sec-8-1-1-1}


A global variable holding the IP address of last recently accepted
client. See also \texttt{listen} and \texttt{accept}.


\begin{verbatim}
: *Adr
-> "127.0.0.1"
\end{verbatim}

 
\section{(adr 'var) -> num}
\label{sec-8-1-1-2}


\texttt{(adr 'num) -> var}

Converts, in the first form, a variable \texttt{var} (a symbol or a cell) into
\texttt{num} (actually an encoded pointer). A symbol will result in a negative
number, and a cell in a positive number. The second form converts a
pointer back into the original \texttt{var}.


\begin{verbatim}
: (setq X (box 7))
-> $53063416137450
: (adr X)
-> -2961853431592
: (adr @)
-> $53063416137450
: (val @)
-> 7
\end{verbatim}

 
\section{*Allow}
\label{sec-8-1-1-3}


A global variable holding allowed access patterns. If its value is
non-=NIL=, it should contain a list where the CAR is an \texttt{idx} tree of
allowed items, and the CDR a list of prefix strings. See also \texttt{allow},
\texttt{allowed} and \texttt{pre?}.


\begin{verbatim}
: (allowed ("app/")  # Initialize
   "!start" "!stop" "lib.css" "!psh" )
-> NIL
: (allow "!myFoo")  # additional item
-> "!myFoo"
: (allow "myDir/" T)  # additional prefix
-> "myDir/"

: *Allow
-> (("!start" ("!psh" ("!myFoo")) "!stop" NIL "lib.css") "app/" "myDir/")

: (idx *Allow)  # items
-> ("!myFoo" "!psh" "!start" "!stop" "lib.css")
: (cdr *Allow)  # prefixes
-> ("app/" "myDir/")
\end{verbatim}

 
\section{+Alt}
\label{sec-8-1-1-4}


Prefix class specifying an alternative class for a \texttt{+relation}. This
allows indexes or other side effects to be maintained in a class
different from the current one. See also \texttt{Database}.


\begin{verbatim}
(class +EuOrd +Ord)                    # EU-specific order subclass
(rel nr (+Alt +Key +Number) +XyOrd)    # Maintain the key in the +XyOrd index
\end{verbatim}

 
\section{+Any}
\label{sec-8-1-1-5}


Class for unspecified relations, a subclass of \texttt{+relation}. Objects of
that class accept and maintain any type of Lisp data. Used often when
there is no other suitable relation class available. See also
\texttt{Database}.

In the following example \texttt{+Any} is used simply for the reason that there
is no direct way to specify dotted pairs:


\begin{verbatim}
(rel loc (+Any))  # Locale, e.g. ("DE" . "de")
\end{verbatim}

 
\section{+Aux}
\label{sec-8-1-1-6}


Prefix class maintaining auxiliary keys for \texttt{+relation=s, in addition to =+Ref} or \texttt{+Idx} indexes. Expects a list of auxiliary attributes of the
same object, and combines all keys in that order into a single index
key. See also \texttt{+UB}, \texttt{aux} and \texttt{Database}.


\begin{verbatim}
(rel nr (+Ref +Number))                # Normal, non-unique index
(rel nm (+Aux +Ref +String) (nr txt))  # Combined name/number/text index
(rel txt (+Aux +Sn +Idx +String) (nr)) # Text/number plus tolerant text index
\end{verbatim}

 
\section{(abort 'cnt . prg) -> any}
\label{sec-8-1-1-7}


Aborts the execution of \texttt{prg} if it takes longer than \texttt{cnt} seconds, and
returns \texttt{NIL}. Otherwise, the result of \texttt{prg} is returned. \texttt{alarm} is
used internally, so care must be taken not to interfer with other calls
to \texttt{alarm}.


\begin{verbatim}
: (abort 20 (in Sock (rd)))  # Wait maximally 20 seconds for socket data
\end{verbatim}

 
\section{(abs 'num) -> num}
\label{sec-8-1-1-8}


Returns the absolute value of the \texttt{num} argument.


\begin{verbatim}
: (abs -7)
-> 7
: (abs 7)
-> 7
\end{verbatim}

 
\section{(accept 'cnt) -> cnt | NIL}
\label{sec-8-1-1-9}


Accepts a connection on descriptor \texttt{cnt} (as received by \texttt{port}), and
returns the new socket descriptor \texttt{cnt}. The global variable \texttt{*Adr} is
set to the IP address of the client. See also \texttt{listen}, \texttt{connect} and
\texttt{*Adr}.


\begin{verbatim}
: (setq *Socket
   (accept (port 6789)) )  # Accept connection at port 6789
-> 4
\end{verbatim}

 
\section{(accu 'var 'any 'num)}
\label{sec-8-1-1-10}


Accumulates \texttt{num} into a sum, using the key \texttt{any} in an association list
stored in \texttt{var}. See also \texttt{assoc}.


\begin{verbatim}
: (off Sum)
-> NIL
: (accu 'Sum 'a 1)
-> (a . 1)
: (accu 'Sum 'a 5)
-> 6
: (accu 'Sum 22 100)
-> (22 . 100)
: Sum
-> ((22 . 100) (a . 6))
\end{verbatim}

 
\section{(acquire 'sym) -> flg}
\label{sec-8-1-1-11}


Tries to acquire the mutex represented by the file \texttt{sym}, by obtaining
an exclusive lock on that file with \texttt{ctl}, and then trying to write the
PID of the current process into that file. It fails if the file already
holds the PID of some other existing process. See also \texttt{release}, \texttt{*Pid}
and \texttt{rc}.


\begin{verbatim}
: (acquire "sema1")
-> 28255
\end{verbatim}

 
\section{(alarm 'cnt . prg) -> cnt}
\label{sec-8-1-1-12}


Sets an alarm timer scheduling \texttt{prg} to be executed after \texttt{cnt} seconds,
and returns the number of seconds remaining until any previously
scheduled alarm was due to be delivered. Calling \texttt{(alarm 0)} will cancel
an alarm. See also \texttt{abort}, \texttt{sigio}, \texttt{*Hup} and \texttt{*Sig[12]}.


\begin{verbatim}
: (prinl (tim$ (time) T)) (alarm 10 (prinl (tim$ (time) T)))
16:36:14
-> 0
: 16:36:24

: (alarm 10 (bye 0))
-> 0
$
\end{verbatim}

 
\section{(align 'cnt 'any) -> sym}
\label{sec-8-1-1-13}


\texttt{(align 'lst 'any ..) -> sym}

Returns a transient symbol with all \texttt{any} arguments \texttt{pack=ed in an aligned format. In the first form, =any} will be left-aligned if \texttt{cnt}
ist negative, otherwise right-aligned. In the second form, all \texttt{any}
arguments are packed according to the numbers in \texttt{lst}. See also \texttt{tab},
\texttt{center} and \texttt{wrap}.


\begin{verbatim}
: (align 4 "a")
-> "   a"
: (align -4 12)
-> "12  "
: (align (4 4 4) "a" 12 "b")
-> "   a  12   b"
\end{verbatim}

 
\section{(all ['T | '0]) -> lst}
\label{sec-8-1-1-14}


Returns a new list of all \hyperref[ref.html-internal]{internal} symbols in the
system (if called without arguments, or with \texttt{NIL}). Otherwise (if the
argument is \texttt{T}), all current \hyperref[ref.html-transient]{transient} symbols
are returned. Else all current \hyperref[ref.html-external]{external} symbols
are returned.


\begin{verbatim}
: (all)  # All internal symbols
-> (inc> leaf nil inc! accept ...

# Find all symbols starting with an underscore character
: (filter '((X) (= "_" (car (chop X)))) (all))
-> (_put _nacs _oct _lintq _lst _map _iter _dbg2 _getLine _led ...
\end{verbatim}

 
\section{(allow 'sym ['flg]) -> sym}
\label{sec-8-1-1-15}


Maintains an index structure of allowed access patterns in the global
variable \texttt{*Allow}. If the value of \texttt{*Allow} is non-=NIL=, \texttt{sym} is added
to the \texttt{idx} tree in the CAR of \texttt{*Allow} (if \texttt{flg} is \texttt{NIL}), or to the
list of prefix strings (if \texttt{flg} is non-=NIL=). See also \texttt{allowed}.


\begin{verbatim}
: *Allow
-> (("!start" ("!psh") "!stop" NIL "lib.css") "app/")
: (allow "!myFoo")  # additionally allowed item
-> "!myFoo"
: (allow "myDir/" T)  # additionally allowed prefix
-> "myDir/"
\end{verbatim}

 
\section{(allowed lst [sym ..])}
\label{sec-8-1-1-16}


Creates an index structure of allowed access patterns in the global
variable \texttt{*Allow}. \texttt{lst} should consist of prefix strings (to be checked
at runtime with \texttt{pre?}), and the \texttt{sym} arguments should specify the
initially allowed items. See also \texttt{allow}.


\begin{verbatim}
: (allowed ("app/")  # allowed prefixes
   "!start" "!stop" "lib.css" "!psh" )  # allowed items
-> NIL
\end{verbatim}

 
\section{(and 'any ..) -> any}
\label{sec-8-1-1-17}


Logical AND. The expressions \texttt{any} are evaluated from left to right. If
\texttt{NIL} is encountered, \texttt{NIL} is returned immediately. Else the result of
the last expression is returned.


\begin{verbatim}
: (and (= 3 3) (read))
abc  # User input
-> abc
: (and (= 3 4) (read))
-> NIL
\end{verbatim}

 
\section{(any 'sym) -> any}
\label{sec-8-1-1-18}


Parses \texttt{any} from the name of \texttt{sym}. This is the reverse operation of
\texttt{sym}. See also \texttt{str}.


\begin{verbatim}
: (any "(a b # Comment^Jc d)")
-> (a b c d)
: (any "\"A String\"")
-> "A String"
\end{verbatim}

 
\section{(append 'lst ..) -> lst}
\label{sec-8-1-1-19}


Appends all argument lists. See also \texttt{conc}, \texttt{insert}, \texttt{delete} and
\texttt{remove}.


\begin{verbatim}
: (append '(a b c) (1 2 3))
-> (a b c 1 2 3)
: (append (1) (2) (3) 4)
-> (1 2 3 . 4)
\end{verbatim}

 
\section{append/3}
\label{sec-8-1-1-20}


\hyperref[ref.html-pilog]{Pilog} predicate that succeeds if appending the first
two list arguments is equal to the third argument. See also \texttt{append} and
\texttt{member/2}.


\begin{verbatim}
: (? (append @X @Y (a b c)))
 @X=NIL @Y=(a b c)
 @X=(a) @Y=(b c)
 @X=(a b) @Y=(c)
 @X=(a b c) @Y=NIL
-> NIL
\end{verbatim}

 
\section{(apply 'fun 'lst ['any ..]) -> any}
\label{sec-8-1-1-21}


Applies \texttt{fun} to \texttt{lst}. If additional \texttt{any} arguments are given, they
are applied as leading elements of \texttt{lst}.
\texttt{(apply 'fun 'lst 'any1 'any2)} is equivalent to
\texttt{(apply 'fun (cons 'any1 'any2 'lst))}.


\begin{verbatim}
: (apply + (1 2 3))
-> 6
: (apply * (5 6) 3 4)
-> 360
: (apply '((X Y Z) (* X (+ Y Z))) (3 4 5))
-> 27
: (apply println (3 4) 1 2)
1 2 3 4
-> 4
\end{verbatim}

 
\section{(arg ['cnt]) -> any}
\label{sec-8-1-1-22}


Can only be used inside functions with a variable number of arguments
(with \texttt{@}). If \texttt{cnt} is not given, the value that was returned from the
last call to \texttt{next}) is returned. Otherwise, the \texttt{cnt}'th remaining
argument is returned. See also \texttt{args}, \texttt{next}, \texttt{rest} and \texttt{pass}.


\begin{verbatim}
: (de foo @ (println (next) (arg)))    # Print argument twice
-> foo
: (foo 123)
123 123
-> 123
: (de foo @
   (println (arg 1) (arg 2))
   (println (next))
   (println (arg 1) (arg 2)) )
-> foo
: (foo 'a 'b 'c)
a b
a
b c
-> c
\end{verbatim}

 
\section{(args) -> flg}
\label{sec-8-1-1-23}


Can only be used inside functions with a variable number of arguments
(with \texttt{@}). Returns \texttt{T} when there are more arguments to be fetched from
the internal list. See also \texttt{next}, \texttt{arg}, \texttt{rest} and \texttt{pass}.


\begin{verbatim}
: (de foo @ (println (args)))       # Test for arguments
-> foo
: (foo)                             # No arguments
NIL
-> NIL
: (foo NIL)                         # One argument
T
-> T
: (foo 123)                         # One argument
T
-> T
\end{verbatim}

 
\section{(argv [var ..] [. sym]) -> lst|sym}
\label{sec-8-1-1-24}


If called without arguments, \texttt{argv} returns a list of strings containing
all remaining command line arguments. Otherwise, the \texttt{var/sym} arguments
are subsequently bound to the command line arguments. A hyphen ``=-='' can
be used to inhibit the automatic \texttt{load=ing further arguments. See also =cmd}, \hyperref[ref.html-invoc]{Invocation} and \texttt{opt}.


\begin{verbatim}
$ pil -"println 'OK" - abc 123 +
OK
: (argv)
-> ("abc" "123")
: (argv A B)
-> "123"
: A
-> "abc"
: B
-> "123"
: (argv . Lst)
-> ("abc" "123")
: Lst
-> ("abc" "123")
\end{verbatim}

 
\section{(as 'any1 . any2) -> any2 | NIL}
\label{sec-8-1-1-25}


Returns \texttt{any2} unevaluated when \texttt{any1} evaluates to non-=NIL=. Otherwise
\texttt{NIL} is returned. \texttt{(as Flg A B C)} is equivalent to
\texttt{(and Flg '(A B C))}. See also \texttt{quote}.


\begin{verbatim}
: (as (= 3 3) A B C)
-> (A B C)
\end{verbatim}

 
\section{(asoq 'any 'lst) -> lst}
\label{sec-8-1-1-26}


Searches an association list. Returns the first element from \texttt{lst} with
\texttt{any} as its CAR, or \texttt{NIL} if no match is found. ==== is used for
comparison (pointer equality). See also \texttt{assoc}, \texttt{delq}, \texttt{memq}, \texttt{mmeq}
and \hyperref[ref.html-cmp]{Comparing}.


\begin{verbatim}
: (asoq 999 '((999 1 2 3) (b . 7) ("ok" "Hello")))
-> NIL
: (asoq 'b '((999 1 2 3) (b . 7) ("ok" "Hello")))
-> (b . 7)
\end{verbatim}

 
\section{(assert exe ..) -> prg | NIL}
\label{sec-8-1-1-27}


When in debug mode (\texttt{*Dbg} is non-=NIL=), \texttt{assert} returns a \texttt{prg} list
which tests all \texttt{exe} conditions, and issues an error via \texttt{quit} if one
of the results evaluates to \texttt{NIL}. Otherwise, \texttt{NIL} is returned. Used
typically in combination with the \texttt{\textasciitilde{} tilde \texttt{read-macro} to insert the
test code only when in debug mode. See also \texttt{test}.


\begin{verbatim}
# Start in debug mode
$ pil +
: (de foo (N)
   ~(assert (>= 90 N 10))
   (bar N) )
-> foo
: (pp 'foo)                      # Pretty-print 'foo'
(de foo (N)
   (unless (>= 90 N 10)          # Assertion code exists
      (quit "'assert' failed" '(>= 90 N 10)) )
   (bar N) )
-> foo
: (foo 7)                        # Try it
(>= 90 N 10) -- Assertion failed
?

# Start in non-debug mode
$ pil
: (de foo (N)
   ~(assert (>= 90 N 10))
   (bar N) )
-> foo
: (pp 'foo)                      # Pretty-print 'foo'
(de foo (N)
   (bar N) )                     # Assertion code does not exist
-> foo
\end{verbatim}

 
\section{(asserta 'lst) -> lst}
\label{sec-8-1-1-28}


Inserts a new \hyperref[ref.html-pilog]{Pilog} fact or rule before all other
rules. See also \texttt{be}, \texttt{clause}, \texttt{assertz} and \texttt{retract}.


\begin{verbatim}
: (be a (2))            # Define two facts
-> a
: (be a (3))
-> a

: (asserta '(a (1)))    # Insert new fact in front
-> (((1)) ((2)) ((3)))

: (? (a @N))            # Query
 @N=1
 @N=2
 @N=3
-> NIL
\end{verbatim}

 
\section{asserta/1}
\label{sec-8-1-1-29}


\hyperref[ref.html-pilog]{Pilog} predicate that inserts a new fact or rule
before all other rules. See also \texttt{asserta}, \texttt{assertz/1} and \texttt{retract/1}.


\begin{verbatim}
: (? (asserta (a (2))))
-> T
: (? (asserta (a (1))))
-> T
: (rules 'a)
1 (be a (1))
2 (be a (2))
-> a
\end{verbatim}

 
\section{(assertz 'lst) -> lst}
\label{sec-8-1-1-30}


Appends a new \hyperref[ref.html-pilog]{Pilog} fact or rule behind all other
rules. See also \texttt{be}, \texttt{clause}, \texttt{asserta} and \texttt{retract}.


\begin{verbatim}
: (be a (1))            # Define two facts
-> a
: (be a (2))
-> a

: (assertz '(a (3)))    # Append new fact at the end
-> (((1)) ((2)) ((3)))

: (? (a @N))            # Query
 @N=1
 @N=2
 @N=3
-> NIL
\end{verbatim}

 
\section{assertz/1}
\label{sec-8-1-1-31}


\hyperref[ref.html-pilog]{Pilog} predicate that appends a new fact or rule
behind all other rules. See also \texttt{assertz}, \texttt{asserta/1} and \texttt{retract/1}.


\begin{verbatim}
: (? (assertz (a (1))))
-> T
: (? (assertz (a (2))))
-> T
: (rules 'a)
1 (be a (1))
2 (be a (2))
-> a
\end{verbatim}

 
\section{(assoc 'any 'lst) -> lst}
\label{sec-8-1-1-32}


Searches an association list. Returns the first element from \texttt{lst} with
its CAR equal to \texttt{any}, or \texttt{NIL} if no match is found. See also \texttt{asoq}.


\begin{verbatim}
: (assoc "b" '((999 1 2 3) ("b" . 7) ("ok" "Hello")))
-> ("b" . 7)
: (assoc 999 '((999 1 2 3) ("b" . 7) ("ok" "Hello")))
-> (999 1 2 3)
: (assoc 'u '((999 1 2 3) ("b" . 7) ("ok" "Hello")))
-> NIL
\end{verbatim}

 
\section{(at '(cnt1 . cnt2|NIL) . prg) -> any}
\label{sec-8-1-1-33}


Increments \texttt{cnt1} (destructively), and returns \texttt{NIL} when it is less
than \texttt{cnt2}. Otherwise, \texttt{cnt1} is reset to zero and \texttt{prg} is executed.
Returns the result of \texttt{prg}. If \texttt{cnt2} is \texttt{NIL}, nothing is done, and
\texttt{NIL} is returned immediately.


\begin{verbatim}
: (do 11 (prin ".") (at (0 . 3) (prin "!")))
...!...!...!..-> NIL
\end{verbatim}

 
\section{(atom 'any) -> flg}
\label{sec-8-1-1-34}


Returns \texttt{T} when the argument \texttt{any} is an atom (a number or a symbol).
See also \texttt{pair}.


\begin{verbatim}
: (atom 123)
-> T
: (atom 'a)
-> T
: (atom NIL)
-> T
: (atom (123))
-> NIL
\end{verbatim}

 
\section{(aux 'var 'cls ['hook] 'any ..) -> sym}
\label{sec-8-1-1-35}


Returns a database object of class \texttt{cls}, where the value for \texttt{var}
corresponds to \texttt{any} and the following arguments. \texttt{var}, \texttt{cls} and
\texttt{hook} should specify a \texttt{tree} for \texttt{cls} or one of its superclasses, for
a relation with auxiliary keys. For multi-key accesses, \texttt{aux} is simlar
to - but faster than - \texttt{db}, because it can use a single tree access.
See also \texttt{db}, \texttt{collect}, \texttt{fetch}, \texttt{init}, \texttt{step} and \texttt{+Aux}.


\begin{verbatim}
(class +PS +Entity)
(rel par (+Dep +Joint) (sup) ps (+Part))        # Part
(rel sup (+Aux +Ref +Link) (par) NIL (+Supp))   # Supplier
...
   (aux 'sup '+PS                               # Access PS object
      (db 'nr '+Supp 1234)
      (db 'nr '+Part 5678) )
\end{verbatim}


\chapter{Functions starting with B}
\label{sec-8-1-2}


 
\section{*Blob}
\label{sec-8-1-2-1}


A global variable holding the pathname of the database blob directory.
See also \texttt{blob}.


\begin{verbatim}
: *Blob
-> "blob/app/"
\end{verbatim}

 
\section{*Bye}
\label{sec-8-1-2-2}


A global variable holding a (possibly empty) \texttt{prg} body, to be executed
just before the termination of the PicoLisp interpreter. See also \texttt{bye}
and \texttt{tmp}.


\begin{verbatim}
: (push1 '*Bye '(call 'rm "myfile.tmp"))  # Remove a temporary file
-> (call 'rm "myfile.tmp")
\end{verbatim}

 
\section{+Bag}
\label{sec-8-1-2-3}


Class for a list of arbitrary relations, a subclass of \texttt{+relation}.
Objects of that class maintain a list of heterogeneous relations.
Typically used in combination with the \texttt{+List} prefix class, to maintain
small two-dimensional tables within oubjects. See also \texttt{Database}.


\begin{verbatim}
(rel pos (+List +Bag)         # Positions
   ((+Ref +Link) NIL (+Item))    # Item
   ((+Number) 2)                 # Price
   ((+Number))                   # Quantity
   ((+String))                   # Memo text
   ((+Number) 2) )               # Total amount
\end{verbatim}

 
\section{+Blob}
\label{sec-8-1-2-4}


Class for blob relations, a subclass of \texttt{+relation}. Objects of that
class maintain blobs, as stubs in database objects pointing to actual
files for arbitrary (often binary) data. The files themselves reside
below the path specified by the \texttt{*Blob} variable. See also \texttt{Database}.


\begin{verbatim}
(rel jpg (+Blob))  # Picture
\end{verbatim}

+Bool==

Class for boolean relations, a subclass of \texttt{+relation}. Objects of that
class expect either \texttt{T} or \texttt{NIL} as value (though, as always, only
non-=NIL= will be physically stored in objects). See also \texttt{Database}.


\begin{verbatim}
(rel ok (+Ref +Bool))  # Indexed flag
\end{verbatim}

 
\section{(balance 'var 'lst ['flg])}
\label{sec-8-1-2-5}


Builds a balanced binary \texttt{idx} tree in \texttt{var}, from the sorted list in
\texttt{lst}. Normally (if random or, in the worst case, ordered data) are
inserted with \texttt{idx}, the tree will not be balanced. But if \texttt{lst} is
properly sorted, its contents will be inserted in an optimally balanced
way. If \texttt{flg} is non-=NIL=, the index tree will be augmented instead of
being overwritten. See also \texttt{Comparing} and \texttt{sort}.


\begin{verbatim}
# Normal idx insert
: (off I)
-> NIL
: (for X (1 4 2 5 3 6 7 9 8) (idx 'I X T))
-> NIL
: (depth I)
-> 7

# Balanced insert
: (balance 'I (sort (1 4 2 5 3 6 7 9 8)))
-> NIL
: (depth I)
-> 4

# Augment
: (balance 'I (sort (10 40 20 50 30 60 70 90 80)) T)
-> NIL
: (idx 'I)
-> (1 2 3 4 5 6 7 8 9 10 20 30 40 50 60 70 80 90)
\end{verbatim}

 
\section{(basename 'any) -> sym}
\label{sec-8-1-2-6}


Returns the filename part of a path name \texttt{any}. See also \texttt{dirname} and
\texttt{path}.


\begin{verbatim}
: (basename "a/b/c/d")
-> "d"
\end{verbatim}

 
\section{(be sym . any) -> sym}
\label{sec-8-1-2-7}


Declares a \hyperref[ref.html-pilog]{Pilog} fact or rule for the \texttt{sym}
argument, by concatenating the \texttt{any} argument to the \texttt{T} property of
\texttt{sym}. See also \texttt{clause}, \texttt{asserta}, \texttt{assertz}, \texttt{retract}, \texttt{goal} and
\texttt{prove}.


\begin{verbatim}
: (be likes (John Mary))
-> likes
: (be likes (John @X) (likes @X wine) (likes @X food))
-> likes
: (get 'likes T)
-> (((John Mary)) ((John @X) (likes @X wine) (likes @X food)))
: (? (likes John @X))
 @X=Mary
-> NIL
\end{verbatim}

 
\section{(beep) -> any}
\label{sec-8-1-2-8}


Send the bell character to the console. See also \texttt{prin} and \texttt{char}.


\begin{verbatim}
: (beep)
-> "^G"
\end{verbatim}

 
\section{(bench . prg) -> any}
\label{sec-8-1-2-9}


Benchmarks \texttt{prg}, by printing the time it took to execute, and returns
the result. See also \texttt{usec}.


\begin{verbatim}
: (bench (wait 2000))
1.996 sec
-> NIL
\end{verbatim}

 
\section{(bin 'num ['num]) -> sym}
\label{sec-8-1-2-10}


\texttt{(bin 'sym) -> num}

Converts a number \texttt{num} to a binary string, or a binary string \texttt{sym} to
a number. In the first case, if the second argument is given, the result
is separated by spaces into groups of such many digits. See also \texttt{oct},
\texttt{hex}, \texttt{fmt64}, \texttt{hax} and \texttt{format}.


\begin{verbatim}
: (bin 73)
-> "1001001"
: (bin "1001001")
-> 73
: (bin 1234567 4)
-> "100 1011 0101 1010 0001 11"
\end{verbatim}

 
\section{(bind 'sym|lst . prg) -> any}
\label{sec-8-1-2-11}


Binds value(s) to symbol(s). The first argument must evaluate to a
symbol, or a list of symbols or symbol-value pairs. The values of these
symbols are saved (and the symbols bound to the values in the case of
pairs), \texttt{prg} is executed, then the symbols are restored to their
original values. During execution of \texttt{prg}, the values of the symbols
can be temporarily modified. The return value is the result of \texttt{prg}.
See also \texttt{let}, \texttt{job} and \texttt{use}.


\begin{verbatim}
: (setq X 123)                               # X is 123
-> 123
: (bind 'X (setq X "Hello") (println X))  # Set X to "Hello", print it
"Hello"
-> "Hello"
: (bind '((X . 3) (Y . 4)) (println X Y) (* X Y))
3 4
-> 12
: X
-> 123                                       # X is restored to 123
\end{verbatim}

 
\section{(bit? 'num ..) -> num | NIL}
\label{sec-8-1-2-12}


Returns the first \texttt{num} argument when all bits which are 1 in the first
argument are also 1 in all following arguments, otherwise \texttt{NIL}. When
one of those arguments evaluates to \texttt{NIL}, it is returned immediately.
See also \texttt{\&}, \texttt{|} and \texttt{x|}.


\begin{verbatim}
: (bit? 7 15 255)
-> 7
: (bit? 1 3)
-> 1
: (bit? 1 2)
-> NIL
\end{verbatim}

 
\section{(blob 'obj 'sym) -> sym}
\label{sec-8-1-2-13}


Returns the blob file name for \texttt{var} in \texttt{obj}. See also \texttt{*Blob}, \texttt{blob!}
and \texttt{pack}.


\begin{verbatim}
: (show (db 'nr '+Item 1))
{3-1} (+Item)
   jpg
   pr 29900
   inv 100
   sup {2-1}
   nm "Main Part"
   nr 1
-> {3-1}
: (blob '{3-1} 'jpg)
-> "blob/app/3/-/1.jpg"
\end{verbatim}

 
\section{(blob! 'obj 'sym 'file)}
\label{sec-8-1-2-14}


Stores the contents of \texttt{file} in a \texttt{blob}. See also \texttt{put!>}.


\begin{verbatim}
(blob! *ID 'jpg "picture.jpg")
\end{verbatim}

 
\section{(bool 'any) -> flg}
\label{sec-8-1-2-15}


Returns \texttt{T} when the argument \texttt{any} is non-=NIL=. This function is only
needed when \texttt{T} is strictly required for a ``true'' condition (Usually,
any non-=NIL= value is considered to be ``true''). See also \texttt{flg?}.


\begin{verbatim}
: (and 3 4)
-> 4
: (bool (and 3 4))
-> T
\end{verbatim}

 
\section{bool/3}
\label{sec-8-1-2-16}


\hyperref[ref.html-pilog]{Pilog} predicate that succeeds if the first argument
has the same truth value as the result of applying the \texttt{get} algorithm
to the following arguments. Typically used as filter predicate in
\texttt{select/3} database queries. See also \texttt{bool}, \texttt{isa/2}, \texttt{same/3},
\texttt{range/3}, \texttt{head/3}, \texttt{fold/3}, \texttt{part/3} and \texttt{tolr/3}.


\begin{verbatim}
: (? @OK NIL         # Find orders where the 'ok' flag is not set
   (db nr +Ord @Ord)
   (bool @OK @Ord ok) )
 @OK=NIL @Ord={3-7}
-> NIL
\end{verbatim}

 
\section{(box 'any) -> sym}
\label{sec-8-1-2-17}


Creates and returns a new anonymous symbol. The initial value is set to
the \texttt{any} argument. See also \texttt{new} and \texttt{box?}.


\begin{verbatim}
: (show (box '(A B C)))
$134425627 (A B C)
-> $134425627
\end{verbatim}

 
\section{(box? 'any) -> sym | NIL}
\label{sec-8-1-2-18}


Returns the argument \texttt{any} when it is an anonymous symbol, otherwise
\texttt{NIL}. See also \texttt{box}, \texttt{str?} and \texttt{ext?}.


\begin{verbatim}
: (box? (new))
-> $134563468
: (box? 123)
-> NIL
: (box? 'a)
-> NIL
: (box? NIL)
-> NIL
\end{verbatim}

 
\section{(by 'fun1 'fun2 'lst ..) -> lst}
\label{sec-8-1-2-19}


Applies \texttt{fun1} to each element of \texttt{lst}. When additional \texttt{lst} arguments
are given, their elements are also passed to \texttt{fun1}. Each result of
\texttt{fun1} is CONSed with its corresponding argument form the original
\texttt{lst}, and collected into a list which is passed to \texttt{fun2}. For the list
returned from \texttt{fun2}, the CAR elements returned by \texttt{fun1} are
(destructively) removed from each element.


\begin{verbatim}
: (let (A 1 B 2 C 3) (by val sort '(C A B)))
-> (A B C)
: (by '((N) (bit? 1 N)) group (3 11 6 2 9 5 4 10 12 7 8 1))
-> ((3 11 9 5 7 1) (6 2 4 10 12 8))
\end{verbatim}

 
\section{(bye 'cnt|NIL)}
\label{sec-8-1-2-20}


Executes all pending \texttt{finally} expressions, closes all open files,
executes the \texttt{VAL} of the global variable \texttt{*Bye} (should be a \texttt{prg}),
flushes standard output, and then exits the PicoLisp interpreter. The
process return value is \texttt{cnt}, or 0 if the argument is missing or \texttt{NIL}.


\begin{verbatim}
: (setq *Bye '((println 'OK) (println 'bye)))
-> ((println 'OK) (println 'bye))
: (bye)
OK
bye
$
\end{verbatim}


\chapter{Functions starting with C}
\label{sec-8-1-3}


 
\section{*Class}
\label{sec-8-1-3-1}


A global variable holding the current class. See also \texttt{OO Concepts},
\texttt{class}, \texttt{extend}, \texttt{dm} and \texttt{var} and \texttt{rel}.


\begin{verbatim}
: (class +Test)
-> +Test
: *Class
-> +Test
\end{verbatim}

 
\section{(cache 'var 'sym . prg) -> any}
\label{sec-8-1-3-2}


Speeds up some calculations, by holding previously calculated results in
an \texttt{idx} tree structure. Such an optimization is sometimes called
``memoization''. \texttt{sym} must be a transient symbol representing a unique
key for the argument(s) to the calculation. See also \texttt{hash}.


\begin{verbatim}
: (de fibonacci (N)
   (cache '(NIL) (pack (char (hash N)) N)
      (if (> 2 N)
         1
         (+
            (fibonacci (dec N))
            (fibonacci (- N 2)) ) ) ) )
-> fibonacci
: (fibonacci 22)
-> 28657
: (fibonacci 10000)
-> 5443837311356528133873426099375038013538 ...  # (2090 digits)
\end{verbatim}

 
\section{(call 'any ..) -> flg}
\label{sec-8-1-3-3}


Calls an external system command. The \texttt{any} arguments specify the
command and its arguments. Returns \texttt{T} if the command was executed
successfully.


\begin{verbatim}
: (when (call 'test "-r" "file.l")  # Test if file exists and is readable
   (load "file.l")  # Load it
   (call 'rm "file.l") )  # Remove it
\end{verbatim}

 
\section{call/1}
\label{sec-8-1-3-4}


\hyperref[ref.html-pilog]{Pilog} predicate that succeeds if the argument term
can be proven.


\begin{verbatim}
: (be mapcar (@ NIL NIL))
-> mapcar
: (be mapcar (@P (@X . @L) (@Y . @M))
   (call @P @X @Y)                        # Call the given predicate
   (mapcar @P @L @M) )
-> mapcar
:  (? (mapcar change (you are a computer) @Z))
-> NIL
:  (? (mapcar change (you are a computer) @Z) T)
-> NIL
:  (? (mapcar permute ((a b c) (d e f)) @X))
 @X=((a b c) (d e f))
 @X=((a b c) (d f e))
 @X=((a b c) (e d f))
 ...
 @X=((a c b) (d e f))
 @X=((a c b) (d f e))
 @X=((a c b) (e d f))
 ...
\end{verbatim}

 
\section{(can 'msg) -> lst}
\label{sec-8-1-3-5}


Returns a list of all classes that accept the message \texttt{msg}. See also
\texttt{OO Concepts}, \texttt{class}, \texttt{dep}, \texttt{what} and \texttt{who}.


\begin{verbatim}
: (can 'zap>)
-> ((zap> . +relation) (zap> . +Blob) (zap> . +Entity))
: (more @ pp)
(dm (zap> . +relation) (Obj Val))

(dm (zap> . +Blob) (Obj Val)
   (and
      Val
      (call 'rm "-f" (blob Obj (: var))) ) )

(dm (zap> . +Entity) NIL
   (for X (getl This)
      (let V (or (atom X) (pop 'X))
         (and (meta This X) (zap> @ This V)) ) ) )

-> NIL
\end{verbatim}

 
\section{(car 'var) -> any}
\label{sec-8-1-3-6}


List access: Returns the value of \texttt{var} if it is a symbol, or the first
element if it is a list. See also \texttt{cdr} and \texttt{c..r}.


\begin{verbatim}
: (car (1 2 3 4 5 6))
-> 1
\end{verbatim}

 
\section{(c[ad]*ar 'var) -> any}
\label{sec-8-1-3-7}


\texttt{(c[ad]*dr 'lst) -> any}

List access shortcuts. Combinations of the \texttt{car} and \texttt{cdr} functions,
with up to four letters `a' and `d'.


\begin{verbatim}
: (cdar '((1 . 2) . 3))
-> 2
\end{verbatim}

 
\section{(case 'any (any1 . prg1) (any2 . prg2) ..) -> any}
\label{sec-8-1-3-8}


Multi-way branch: \texttt{any} is evaluated and compared to the CAR elements
\texttt{anyN} of each clause. If one of them is a list, \texttt{any} is in turn
compared to all elements of that list. \texttt{T} is a catch-all for any value.
If a comparison succeeds, \texttt{prgN} is executed, and the result returned.
Otherwise \texttt{NIL} is returned. See also \texttt{state}.


\begin{verbatim}
: (case (char 66) ("A" (+ 1 2 3)) (("B" "C") "Bambi") ("D" (* 1 2 3)))
-> "Bambi"
\end{verbatim}

 
\section{(catch 'any . prg) -> any}
\label{sec-8-1-3-9}


Sets up the environment for a non-local jump which may be caused by
\texttt{throw} or by a runtime error. If \texttt{any} is an atom, it is used by
\texttt{throw} as a jump label (with \texttt{T} being a catch-all for any label), and
a \texttt{throw} called during the execution of \texttt{prg} will immediately return
the thrown value. Otherwise, \texttt{any} should be a list of strings, to catch
any error whose message contains one of these strings, and this will
immediately return the matching string. If neither \texttt{throw} nor an error
occurs, the result of \texttt{prg} is returned. See also \texttt{finally}, \texttt{quit} and
\texttt{Error Handling}.


\begin{verbatim}
: (catch 'OK (println 1) (throw 'OK 999) (println 2))
1
-> 999
: (catch '("No such file") (in "doesntExist" (foo)))
-> "No such file"
\end{verbatim}

 
\section{(cd 'any) -> sym}
\label{sec-8-1-3-10}


Changes the current directory to \texttt{any}. The old directory is returned on
success, otherwise \texttt{NIL}. See also \texttt{dir} and \texttt{pwd}.


\begin{verbatim}
: (when (cd "lib")
   (println (sum lines (dir)))
   (cd @) )
10955
\end{verbatim}

 
\section{(cdr 'lst) -> any}
\label{sec-8-1-3-11}


List access: Returns all but the first element of \texttt{lst}. See also \texttt{car}
and \texttt{c..r}.


\begin{verbatim}
: (cdr (1 2 3 4 5 6))
-> (2 3 4 5 6)
\end{verbatim}

 
\section{(center 'cnt|lst 'any ..) -> sym}
\label{sec-8-1-3-12}


Returns a transient symbol with all \texttt{any} arguments \texttt{pack=ed in a centered format. Trailing blanks are omitted. See also =align}, \texttt{tab}
and \texttt{wrap}.


\begin{verbatim}
: (center 4 12)
-> " 12"
: (center 4 "a")
-> " a"
: (center 7 "a")
-> "   a"
: (center (3 3 3) "a" "b" "c")
-> " a  b  c"
\end{verbatim}

 
\section{(chain 'lst ..) -> lst}
\label{sec-8-1-3-13}


Concatenates (destructively) one or several new list elements \texttt{lst} to
the end of the list in the current \texttt{make} environment. This operation is
efficient also for long lists, because a pointer to the last element of
the result list is maintained. \texttt{chain} returns the last linked argument.
See also \texttt{link}, \texttt{yoke} and \texttt{made}.


\begin{verbatim}
: (make (chain (list 1 2 3) NIL (cons 4)) (chain (list 5 6)))
-> (1 2 3 4 5 6)
\end{verbatim}

 
\section{(char) -> sym}
\label{sec-8-1-3-14}


\texttt{(char 'cnt) -> sym}

\texttt{(char T) -> sym}

\texttt{(char 'sym) -> cnt}

When called without arguments, the next character from the current input
stream is returned as a single-character transient symbol, or \texttt{NIL} upon
end of file. When called with a number \texttt{cnt}, a character with the
corresponding unicode value is returned. As a special case, \texttt{T} is
accepted to produce a byte value greater than any first byte in a UTF--8
character (used as a top value in comparisons). Otherwise, when called
with a symbol \texttt{sym}, the numeric unicode value of the first character of
the name of that symbol is returned. See also \texttt{peek}, \texttt{skip}, \texttt{key},
\texttt{line}, \texttt{till} and \texttt{eof}.


\begin{verbatim}
: (char)                   # Read character from console
A                          # (typed 'A' and a space/return)
-> "A"
: (char 100)               # Convert unicode to symbol
-> "d"
: (char "d")               # Convert symbol to unicode
-> 100

: (char T)                 # Special case
-> # (not printable)

: (char 0)
-> NIL
: (char NIL)
-> 0
\end{verbatim}

 
\section{(chdir 'any . prg) -> any}
\label{sec-8-1-3-15}


Changes the current directory to \texttt{any} with \texttt{cd} during the execution of
\texttt{prg}. Then the previous directory will be restored and the result of
\texttt{prg} returned. See also \texttt{dir} and \texttt{pwd}.


\begin{verbatim}
: (pwd)
-> "/usr/abu/pico"
: (chdir "src" (pwd))
-> "/usr/abu/pico/src"
: (pwd)
-> "/usr/abu/pico"
\end{verbatim}

 
\section{(chkTree 'sym ['fun]) -> num}
\label{sec-8-1-3-16}


Checks a database tree node (and recursively all sub-nodes) for
consistency. Returns the total number of nodes checked. Optionally,
\texttt{fun} is called with the key and value of each node, and should return
\texttt{NIL} for failure. See also \texttt{tree} and \texttt{root}.


\begin{verbatim}
: (show *DB '+Item)
{C} NIL
   sup (7 . {7-3})
   nr (7 . {7-1})    # 7 nodes in the 'nr' tree, base node is {7-1}
   pr (7 . {7-4})
   nm (77 . {7-6})
-> {C}
: (chkTree '{7-1})   # Check that node
-> 7
\end{verbatim}

 
\section{(chop 'any) -> lst}
\label{sec-8-1-3-17}


Returns \texttt{any} as a list of single-character strings. If \texttt{any} is \texttt{NIL}
or a symbol with no name, \texttt{NIL} is returned. A list argument is returned
unchanged.


\begin{verbatim}
: (chop 'car)
-> ("c" "a" "r")
: (chop "Hello")
-> ("H" "e" "l" "l" "o")
\end{verbatim}

 
\section{(circ 'any ..) -> lst}
\label{sec-8-1-3-18}


Produces a circular list of all \texttt{any} arguments by =cons=ing them to a
list and then connecting the CDR of the last cell to the first cell. See
also \texttt{circ?} and \texttt{list}.


\begin{verbatim}
: (circ 'a 'b 'c)
-> (a b c .)
\end{verbatim}

 
\section{(circ? 'any) -> any}
\label{sec-8-1-3-19}


Returs the circular (sub)list if \texttt{any} is a circular list, else \texttt{NIL}.
See also \texttt{circ}.


\begin{verbatim}
: (circ? 'a)
-> NIL
: (circ? (1 2 3))
-> NIL
: (circ? (1 . (2 3 .)))
-> (2 3 .)
\end{verbatim}

 
\section{(class sym . typ) -> obj}
\label{sec-8-1-3-20}


Defines \texttt{sym} as a class with the superclass(es) \texttt{typ}. As a side
effect, the global variable \texttt{*Class} is set to \texttt{obj}. See also \texttt{extend},
\texttt{dm}, \texttt{var}, \texttt{rel}, \texttt{type}, \texttt{isa} and \texttt{object}.


\begin{verbatim}
: (class +A +B +C +D)
-> +A
: +A
-> (+B +C +D)
: (dm foo> (X) (bar X))
-> foo>
: +A
-> ((foo> (X) (bar X)) +B +C +D)
\end{verbatim}

 
\section{(clause '(sym . any)) -> sym}
\label{sec-8-1-3-21}


Declares a \hyperref[ref.html-pilog]{Pilog} fact or rule for the \texttt{sym}
argument, by concatenating the \texttt{any} argument to the \texttt{T} property of
\texttt{sym}. See also \texttt{be}.


\begin{verbatim}
: (clause '(likes (John Mary)))
-> likes
: (clause '(likes (John @X) (likes @X wine) (likes @X food)))
-> likes
: (? (likes @X @Y))
 @X=John @Y=Mary
-> NIL
\end{verbatim}

 
\section{clause/2}
\label{sec-8-1-3-22}


\hyperref[ref.html-pilog]{Pilog} predicate that succeeds if the first argument
is a predicate which has the second argument defined as a clause.


\begin{verbatim}
: (? (clause append ((NIL @X @X))))
-> T

: (? (clause append @C))
 @C=((NIL @X @X))
 @C=(((@A . @X) @Y (@A . @Z)) (append @X @Y @Z))
-> NIL
\end{verbatim}

 
\section{(clip 'lst) -> lst}
\label{sec-8-1-3-23}


Returns a copy of \texttt{lst} with all whitespace characters or \texttt{NIL} elements
removed from both sides. See also \texttt{trim}.


\begin{verbatim}
: (clip '(NIL 1 NIL 2 NIL))
-> (1 NIL 2)
: (clip '(" " a " " b " "))
-> (a " " b)
\end{verbatim}

 
\section{(close 'cnt) -> cnt | NIL}
\label{sec-8-1-3-24}


Closes a file descriptor \texttt{cnt}, and returns it when successful. Should
not be called inside an \texttt{out} body for that descriptor. See also \texttt{open},
\texttt{poll}, \texttt{listen} and \texttt{connect}.


\begin{verbatim}
: (close 2)                            # Close standard error
-> 2
\end{verbatim}

 
\section{(cmd ['any]) -> sym}
\label{sec-8-1-3-25}


When called without an argument, the name of the command that invoked
the picolisp interpreter is returned. Otherwise, the command name is set
to \texttt{any}. Setting the name may not work on some operating systems. Note
that the new name must not be longer than the original one. See also
\texttt{argv}, \texttt{file} and \hyperref[ref.html-invoc]{Invocation}.


\begin{verbatim}
$ pil +
: (cmd)
-> "/usr/bin/picolisp"
: (cmd "!/bin/picolust")
-> "!/bin/picolust"
: (cmd)
-> "!/bin/picolust"
\end{verbatim}

 
\section{(cnt 'fun 'lst ..) -> cnt}
\label{sec-8-1-3-26}


Applies \texttt{fun} to each element of \texttt{lst}. When additional \texttt{lst} arguments
are given, their elements are also passed to \texttt{fun}. Returns the count of
non-=NIL= values returned from \texttt{fun}.


\begin{verbatim}
: (cnt cdr '((1 . T) (2) (3 4) (5)))
-> 2
\end{verbatim}

 
\section{(collect 'var 'cls ['hook] ['any|beg ['end [var ..]]])}
\label{sec-8-1-3-27}


Returns a list of all database objects of class \texttt{cls}, where the values
for the \texttt{var} arguments correspond to the \texttt{any} arguments, or where the
values for the \texttt{var} arguments are in the range \texttt{beg} .. \texttt{end}. \texttt{var},
\texttt{cls} and \texttt{hook} should specify a \texttt{tree} for \texttt{cls} or one of its
superclasses. If additional \texttt{var} arguments are given, the final values
for the result list are obtained by applying the \texttt{get} algorithm. See
also \texttt{db}, \texttt{aux}, \texttt{fetch}, \texttt{init} and \texttt{step}.


\begin{verbatim}
: (collect 'nr '+Item)
-> ({3-1} {3-2} {3-3} {3-4} {3-5} {3-6} {3-8})
: (collect 'nr '+Item 3 6 'nr)
-> (3 4 5 6)
: (collect 'nr '+Item 3 6 'nm)
-> ("Auxiliary Construction" "Enhancement Additive" "Metal Fittings" "Gadget Appliance")
: (collect 'nm '+Item "Main Part")
-> ({3-1})
\end{verbatim}

 
\section{(commit ['any] [exe1] [exe2]) -> T}
\label{sec-8-1-3-28}


Closes a transaction, by writing all new or modified external symbols
to, and removing all deleted external symbols from the database. When
\texttt{any} is given, it is implicitly sent (with all modified objects) via
the \texttt{tell} mechanism to all family members. If \texttt{exe1} or \texttt{exe2} are
given, they are executed as pre- or post-expressions while the database
is \texttt{lock=ed and =protect=ed. See also =rollback}.


\begin{verbatim}
: (pool "db")
-> T
: (put '{1} 'str "Hello")
-> "Hello"
: (commit)
-> T
\end{verbatim}

 
\section{(con 'lst 'any) -> any}
\label{sec-8-1-3-29}


Connects \texttt{any} to the first cell of \texttt{lst}, by (destructively) storing
\texttt{any} in the CDR of \texttt{lst}. See also \texttt{conc}.


\begin{verbatim}
: (setq C (1 . a))
-> (1 . a)
: (con C '(b c d))
-> (b c d)
: C
-> (1 b c d)
\end{verbatim}

 
\section{(conc 'lst ..) -> lst}
\label{sec-8-1-3-30}


Concatenates all argument lists (destructively). See also \texttt{append} and
\texttt{con}.


\begin{verbatim}
: (setq  A (1 2 3)  B '(a b c))
-> (a b c)
: (conc A B)                        # Concatenate lists in 'A' and 'B'
-> (1 2 3 a b c)
: A
-> (1 2 3 a b c)                    # Side effect: List in 'A' is modified!
\end{verbatim}

 
\section{(cond ('any1 . prg1) ('any2 . prg2) ..) -> any}
\label{sec-8-1-3-31}


Multi-way conditional: If any of the \texttt{anyN} conditions evaluates to
non-=NIL=, \texttt{prgN} is executed and the result returned. Otherwise (all
conditions evaluate to \texttt{NIL}), \texttt{NIL} is returned. See also \texttt{nond}, \texttt{if},
\texttt{if2} and \texttt{when}.


\begin{verbatim}
: (cond
   ((= 3 4) (println 1))
   ((= 3 3) (println 2))
   (T (println 3)) )
2
-> 2
\end{verbatim}

 
\section{(connect 'any1 'any2) -> cnt | NIL}
\label{sec-8-1-3-32}


Tries to establish a TCP/IP connection to a server listening at host
\texttt{any1}, port \texttt{any2}. \texttt{any1} may be either a hostname or a standard
internet address in numbers-and-dots/colons notation (IPv4/IPv6). \texttt{any2}
may be either a port number or a service name. Returns a socket
descriptor \texttt{cnt}, or \texttt{NIL} if the connection cannot be established. See
also \texttt{listen} and \texttt{udp}.


\begin{verbatim}
: (connect "localhost" 4444)
-> 3
: (connect "some.host.org" "http")
-> 4
\end{verbatim}

 
\section{(cons 'any ['any ..]) -> lst}
\label{sec-8-1-3-33}


Constructs a new list cell with the first argument in the CAR and the
second argument in the CDR. If more than two arguments are given, a
corresponding chain of cells is built. \texttt{(cons 'a 'b 'c 'd)} is
equivalent to \texttt{(cons 'a (cons 'b (cons 'c 'd)))}. See also \texttt{list}.


\begin{verbatim}
: (cons 1 2)
-> (1 . 2)
: (cons 'a '(b c d))
-> (a b c d)
: (cons '(a b) '(c d))
-> ((a b) c d)
: (cons 'a 'b 'c 'd)
-> (a b c . d)
\end{verbatim}

 
\section{(copy 'any) -> any}
\label{sec-8-1-3-34}


Copies the argument \texttt{any}. For lists, the top level cells are copied,
while atoms are returned unchanged.


\begin{verbatim}
: (=T (copy T))               # Atoms are not copied
-> T
: (setq L (1 2 3))
-> (1 2 3)
: (== L L)
-> T
: (== L (copy L))             # The copy is not identical to the original
-> NIL
: (= L (copy L))              # But the copy is equal to the original
-> T
\end{verbatim}

 
\section{(co 'sym [. prg]) -> any}
\label{sec-8-1-3-35}


(64-bit version only) Starts, resumes or stops a
\hyperref[ref.html-coroutines]{coroutine} with the tag given by \texttt{sym}. If \texttt{prg}
is not given, a coroutine with that tag will be stopped. Otherwise, if a
coroutine running with that tag is found (pointer equality is used for
comparison), its execution is resumed. Else a new coroutine with that
tag is initialized and started. \texttt{prg} will be executed until it either
terminates normally, or until \texttt{yield} is called. In the latter case \texttt{co}
returns, or transfers control to some other, already running, coroutine.
Trying to start more than 64 coroutines will result in a stack overflow
error. Also, a coroutine cannot resume itself directly or indirectly.
See also \texttt{stack}, \texttt{catch} and \texttt{throw}.


\begin{verbatim}
: (de pythag (N)   # A generator function
   (if (=T N)
      (co 'rt)  # Stop
      (co 'rt
         (for X N
            (for Y (range X N)
               (for Z (range Y N)
                  (when (= (+ (* X X) (* Y Y)) (* Z Z))
                     (yield (list X Y Z)) ) ) ) ) ) ) )

: (pythag 20)
-> (3 4 5)
: (pythag 20)
-> (5 12 13)
: (pythag 20)
-> (6 8 10)
\end{verbatim}

 
\section{(count 'tree) -> num}
\label{sec-8-1-3-36}


Returns the number of nodes in a database tree. See also \texttt{tree} and
\texttt{root}.


\begin{verbatim}
: (count (tree 'nr '+Item))
-> 7
\end{verbatim}

 
\section{(ctl 'sym . prg) -> any}
\label{sec-8-1-3-37}


Waits until a write (exclusive) lock (or a read (shared) lock if the
first character of \texttt{sym} is ``=+='') can be set on the file \texttt{sym}, then
executes \texttt{prg} and releases the lock. If the files does not exist, it
will be created. When \texttt{sym} is \texttt{NIL}, a shared lock is tried on the
current innermost I/O channel, and when it is \texttt{T}, an exclusive lock is
tried instead. See also \texttt{in}, \texttt{out}, \texttt{err} and \texttt{pipe}.


\begin{verbatim}
$ echo 9 >count                           # Write '9' to file "count"
$ pil +
: (ctl ".ctl"                             # Exclusive control, using ".ctl"
   (in "count"
      (let Cnt (read)                     # Read '9'
         (out "count"
            (println (dec Cnt)) ) ) ) )   # Write '8'
-> 8
:
$ cat count                               # Check "count"
8
\end{verbatim}

 
\section{(ctty 'sym|pid) -> flg}
\label{sec-8-1-3-38}


When called with a symbolic argument, \texttt{ctty} changes the current TTY
device to \texttt{sym}. Otherwise, the local console is prepared for serving
the PicoLisp process with the process ID \texttt{pid}. See also \texttt{raw}.


\begin{verbatim}
: (ctty "/dev/tty")
-> T
\end{verbatim}

 
\section{(curry lst . fun) -> fun}
\label{sec-8-1-3-39}


Builds a new function from the list of symbols or symbol-value pairs
\texttt{lst} and the functional expression \texttt{fun}. Each member in \texttt{lst} that is
a \texttt{pat?} symbol is substituted inside \texttt{fun} by its value. All other
symbols in \texttt{lst} are collected into a \texttt{job} environment.


\begin{verbatim}
: (de multiplier (@X)
   (curry (@X) (N) (* @X N)) )
-> multiplier
: (multiplier 7)
-> ((N) (* 7 N))
: ((multiplier 7) 3))
-> 21

: (def 'fiboCounter
   (curry ((N1 . 0) (N2 . 1)) (Cnt)
      (do Cnt
         (println
            (prog1
               (+ N1 N2)
               (setq N1 N2  N2 @) ) ) ) ) )
-> fiboCounter
: (pp 'fiboCounter)
(de fiboCounter (Cnt)
   (job '((N2 . 1) (N1 . 0))
      (do Cnt
         (println
            (prog1 (+ N1 N2) (setq N1 N2 N2 @)) ) ) ) )
-> fiboCounter
: (fiboCounter 5)
1
2
3
5
8
-> 8
: (fiboCounter 5)
13
21
34
55
89
-> 89
\end{verbatim}

 
\section{(cut 'cnt 'var) -> lst}
\label{sec-8-1-3-40}


Pops the first \texttt{cnt} elements (CAR) from the stack in \texttt{var}. See also
\texttt{pop} and \texttt{del}.


\begin{verbatim}
: (setq S '(1 2 3 4 5 6 7 8))
-> (1 2 3 4 5 6 7 8)
: (cut 3 'S)
-> (1 2 3)
: S
-> (4 5 6 7 8)
\end{verbatim}


\chapter{Functions starting with D}
\label{sec-8-1-4}


 
\section{*DB}
\label{sec-8-1-4-1}


A global constant holding the external symbol \texttt{\{1\}, the \texttt{database}
root. All transient symbols in a database can be reached from that root.
Except during debugging, any explicit literal access to symbols in the
database should be avoided, because otherwise a memory leak might occur
(The garbage collector temporarily sets \texttt{*DB} to \texttt{NIL} and restores its
value after collection, thus disposing of all external symbols not
currently used in the program).


\begin{verbatim}
: (show *DB)
{1} NIL
   +City {P}
   +Person {3}
-> {1}
: (show '{P})
{P} NIL
   nm (566 . {AhDx})
-> {P}
: (show '{3})
{3} NIL
   tel (681376 . {Agyl})
   nm (1461322 . {2gu7})
-> {3}
\end{verbatim}

 
\section{*Dbg}
\label{sec-8-1-4-2}


A boolean variable indicating ``debug mode''. It can be conveniently
switched on with a trailing \texttt{+} command line argument (see
\hyperref[ref.html-invoc]{Invocation}). When non-=NIL=, the \texttt{\$} (tracing) and
\texttt{!} (breakpoint) functions are enabled, and the current line number and
file name will be stored in symbol properties by \texttt{de}, \texttt{def} and \texttt{dm}.
See also \texttt{debug}, \texttt{trace} and \texttt{lint}.


\begin{verbatim}
: (de foo (A B) (* A B))
-> foo
: (trace 'foo)
-> foo
: (foo 3 4)
 foo : 3 4
 foo = 12
-> 12
: (let *Dbg NIL (foo 3 4))
-> 12
\end{verbatim}

 
\section{*Dbs}
\label{sec-8-1-4-3}


A global variable holding a list of numbers (block size scale factors,
as needed by \texttt{pool}). It is typically set by \texttt{dbs} and \texttt{dbs+}.


\begin{verbatim}
: *Dbs
-> (1 2 1 0 2 3 3 3)
\end{verbatim}

 
\section{+Date}
\label{sec-8-1-4-4}


Class for calender dates (as calculated by \texttt{date}), a subclass of
\texttt{+Number}. See also \texttt{Database}.


\begin{verbatim}
(rel dat (+Ref +Date))  # Indexed date
\end{verbatim}

 
\section{+Dep}
\label{sec-8-1-4-5}


Prefix class for maintaining depenencies between =+relation=s. Expects a
list of (symbolic) attributes that depend on this relation. Whenever
this relations is cleared (receives a value of \texttt{NIL}), the dependent
relations will also be cleared, triggering all required side-effects.
See also \texttt{Database}.

In the following example, the index entry for the item pointing to the
position (and, therefore, to the order) is cleared in case the order is
deleted, or this position is deleted from the order:


\begin{verbatim}
(class +Pos +Entity)                # Position class
(rel ord (+Dep +Joint)              # Order of that position
   (itm)                               # 'itm' specifies the dependency
   pos (+Ord) )                        # Arguments to '+Joint'
(rel itm (+Ref +Link) NIL (+Item))  # Item depends on the order
\end{verbatim}

 
\section{(d) -> T}
\label{sec-8-1-4-6}


Inserts \texttt{!} breakpoints into all subexpressions of the current
breakpoint. Typically used when single-stepping a function or method
with \texttt{debug}. See also \texttt{u} and \texttt{unbug}.


\begin{verbatim}
! (d)                            # Debug subexpression(s) at breakpoint
-> T
\end{verbatim}

 
\section{(daemon 'sym . prg) -> fun}
\label{sec-8-1-4-7}


\texttt{(daemon '(sym . cls) . prg) -> fun}

\texttt{(daemon '(sym sym2 [. cls]) . prg) -> fun}

Inserts \texttt{prg} in the beginning of the function (first form), the method
body of \texttt{sym} in \texttt{cls} (second form) or in the class obtained by
\texttt{get=ing =sym2} from \texttt{*Class} (or \texttt{cls} if given) (third form). Built-in
functions (C-function pointer) are automatically converted to Lisp
expressions. See also \texttt{expr}, \texttt{patch} and \texttt{redef}.


\begin{verbatim}
: (de hello () (prinl "Hello world!"))
-> hello

: (daemon 'hello (prinl "# This is the hello world program"))
-> (NIL (prinl "# This is the hello world program") (prinl "Hello world!"))
: (hello)
# This is the hello world program
Hello world!
-> "Hello world!"

: (daemon '* (msg 'Multiplying))
-> (@ (msg 'Multiplying) (pass $134532148))
: *
-> (@ (msg 'Multiplying) (pass $134532148))
: (* 1 2 3)
Multiplying
-> 6
\end{verbatim}

 
\section{(dat\$ 'dat ['sym]) -> sym}
\label{sec-8-1-4-8}


Formats a \texttt{date} \texttt{dat} in ISO format, with an optional delimiter
character \texttt{sym}. See also \texttt{\$dat}, \texttt{tim\$}, \texttt{datStr} and \texttt{datSym}.


\begin{verbatim}
: (dat$ (date))
-> "20070601"
: (dat$ (date) "-")
-> "2007-06-01"
\end{verbatim}

 
\section{(datStr 'dat ['flg]) -> sym}
\label{sec-8-1-4-9}


Formats a \texttt{date} according to the current \texttt{locale}. If \texttt{flg} is
non-=NIL=, the year will be formatted modulo 100. See also \texttt{dat\$},
\texttt{datSym}, \texttt{strDat}, \texttt{expDat}, \texttt{expTel} and \texttt{day}.


\begin{verbatim}
: (datStr (date))
-> "2007-06-01"
: (locale "DE" "de")
-> NIL
: (datStr (date))
-> "01.06.2007"
: (datStr (date) T)
-> "01.06.07"
\end{verbatim}

 
\section{(datSym 'dat) -> sym}
\label{sec-8-1-4-10}


Formats a \texttt{date} \texttt{dat} in in symbolic format (DDmmmYY). See also \texttt{dat\$}
and \texttt{datStr}.


\begin{verbatim}
: (datSym (date))
-> "01jun07"
\end{verbatim}

 
\section{(date ['T]) -> dat}
\label{sec-8-1-4-11}


\texttt{(date 'dat) -> (y m d)}

\texttt{(date 'y 'm 'd) -> dat | NIL}

\texttt{(date '(y m d)) -> dat | NIL}

Calculates a (gregorian) calendar date. It is represented as a day
number, starting first of March of the year 0 AD. When called without
arguments, the current date is returned. When called with a \texttt{T}
argument, the current Coordinated Universal Time (UTC) is returned. When
called with a single number \texttt{dat}, it is taken as a date and a list with
the corresponding year, month and day is returned. When called with
three numbers (or a list of three numbers) for the year, month and day,
the corresponding date is returned (or \texttt{NIL} if they do not represent a
legal date). See also \texttt{time}, \texttt{stamp}, \texttt{\$dat}, \texttt{dat\$}, \texttt{datSym},
\texttt{datStr}, \texttt{strDat}, \texttt{expDat}, \texttt{day}, \texttt{week} and \texttt{ultimo}.


\begin{verbatim}
: (date)                         # Today
-> 730589
: (date 2000 6 12)               # 12-06-2000
-> 730589
: (date 2000 22 5)               # Illegal date
-> NIL
: (date (date))                  # Today's year, month and day
-> (2000 6 12)
: (- (date) (date 2000 1 1))     # Number of days since first of January
-> 163
\end{verbatim}

 
\section{(day 'dat ['lst]) -> sym}
\label{sec-8-1-4-12}


Returns the name of the day for a given \texttt{date} \texttt{dat}, in the language of
the current \texttt{locale}. If \texttt{lst} is given, it should be a list of
alternative weekday names. See also \texttt{week}, \texttt{datStr} and \texttt{strDat}.


\begin{verbatim}
: (day (date))
-> "Friday"
: (locale "DE" "de")
-> NIL
: (day (date))
-> "Freitag"
: (day (date) '("Mo" "Tu" "We" "Th" "Fr" "Sa" "Su"))
-> "Fr"
\end{verbatim}

 
\section{(db 'var 'cls ['hook] 'any ['var 'any ..]) -> sym | NIL}
\label{sec-8-1-4-13}


Returns a database object of class \texttt{cls}, where the values for the \texttt{var}
arguments correspond to the \texttt{any} arguments. If a matching object cannot
be found, \texttt{NIL} is returned. \texttt{var}, \texttt{cls} and \texttt{hook} should specify a
\texttt{tree} for \texttt{cls} or one of its superclasses. See also \texttt{aux}, \texttt{collect},
\texttt{request}, \texttt{fetch}, \texttt{init} and \texttt{step}.


\begin{verbatim}
: (db 'nr '+Item 1)
-> {3-1}
: (db 'nm '+Item "Main Part")
-> {3-1}
\end{verbatim}

 
\section{db/3}
\label{sec-8-1-4-14}


\texttt{db/4}

\texttt{db/5}

\hyperref[ref.html-pilog]{Pilog} database predicate that returns objects
matching the given key/value (and optional hook) relation. The relation
should be of type \texttt{+index}. For the key pattern applies:

\begin{itemize}
\item a symbol (string) returns all entries which start with that string
\item other atoms (numbers, external symbols) match as they are
\item cons pairs constitute a range, returning objects
\begin{itemize}
\item in increasing order if the CDR is greater than the CAR
\item in decreasing order otherwise
\end{itemize}
\item other lists are matched for \texttt{+Aux} key combinations
\end{itemize}

The optional hook can be supplied as the third argument. See also
\texttt{select/3} and \texttt{remote/2}.


\begin{verbatim}
: (? (db nr +Item @Item))              # No value given
 @Item={3-1}
 @Item={3-2}
 @Item={3-3}
 @Item={3-4}
 @Item={3-5}
 @Item={3-6}
-> NIL

: (? (db nr +Item 2 @Item))            # Get item no. 2
 @Item={3-2}
-> NIL

: (? (db nm +Item Spare @Item) (show @Item))  # Search for "Spare.."
{3-2} (+Item)
   pr 1250
   inv 100
   sup {2-2}
   nm "Spare Part"
   nr 2
 @Item={3-2}
-> NIL
\end{verbatim}

 
\section{(db: cls ..) -> num}
\label{sec-8-1-4-15}


Returns the database file number for objects of the type given by the
\texttt{cls} argument(s). Needed, for example, for the creation of \texttt{new}
objects. See also \texttt{dbs}.


\begin{verbatim}
: (db: +Item)
-> 3
\end{verbatim}

 
\section{(dbSync) -> flg}
\label{sec-8-1-4-16}


Starts a database transaction, by trying to obtain a \texttt{lock} on the
database root object \texttt{*DB}, and then calling \texttt{sync} to synchronize with
possible changes from other processes. When all desired modifications to
external symbols are done, \texttt{(commit 'upd)} should be called. See also
\texttt{Database}.


\begin{verbatim}
(let? Obj (rd)             # Get object?
   (dbSync)                # Yes: Start transaction
   (put> Obj 'nm (rd))     # Update
   (put> Obj 'nr (rd))
   (put> Obj 'val (rd))
   (commit 'upd) )         # Close transaction
\end{verbatim}

 
\section{(dbck ['cnt] 'flg) -> any}
\label{sec-8-1-4-17}


Performs a low-level integrity check of the current (or \texttt{cnt}'th)
database file, and returns \texttt{NIL} (or the number of blocks and symbols if
\texttt{flg} is non-=NIL=) if everything seems correct. Otherwise, a string
indicating an error is returned. As a side effect, possibly unused
blocks (as there might be when a \texttt{rollback} is done before \texttt{commit=ing newly allocated (=new}) external symbols) are appended to the free list.


\begin{verbatim}
: (pool "db")
-> T
: (dbck)
-> NIL
\end{verbatim}

 
\section{(dbs . lst)}
\label{sec-8-1-4-18}


Initializes the global variable \texttt{*Dbs}. Each element in \texttt{lst} has a
number in its CAR (the block size scale factor of a database file, to be
stored in \texttt{*Dbs}). The CDR elements are either classes (so that objects
of that class are later stored in the corresponding file), or lists with
a class in the CARs and a list of relations in the CDRs (so that index
trees for these relations go into that file). See also \texttt{dbs+} and
\texttt{pool}.


\begin{verbatim}
(dbs
   (1 +Role +User +Sal)                         # (1 . 128)
   (2 +CuSu)                                    # (2 . 256)
   (1 +Item +Ord)                               # (3 . 128)
   (0 +Pos)                                     # (4 . 64)
   (2 (+Role nm) (+User nm) (+Sal nm))          # (5 . 256)
   (4 (+CuSu nr plz tel mob))                   # (6 . 1024)
   (4 (+CuSu nm))                               # (7 . 1024)
   (4 (+CuSu ort))                              # (8 . 1024)
   (4 (+Item nr sup pr))                        # (9 . 1024)
   (4 (+Item nm))                               # (10 . 1024)
   (4 (+Ord nr dat cus))                        # (11 . 1024)
   (4 (+Pos itm)) )                             # (12 . 1024)

: *Dbs
-> (1 2 1 0 2 4 4 4 4 4 4 4)
: (get '+Item 'Dbf)
-> (3 . 128)
: (get '+Item 'nr 'dbf)
-> (9 . 1024)
\end{verbatim}

 
\section{(dbs+ 'num . lst)}
\label{sec-8-1-4-19}


Extends the list of database sizes stored in \texttt{*Dbs}. \texttt{num} is the
initial offset into the list. See also \texttt{dbs}.


\begin{verbatim}
(dbs+ 9
   (1 +NewCls)                                  # (9 . 128)
   (3 (+NewCls nr nm)) )                        # (10 . 512)
\end{verbatim}

 
\section{(de sym . any) -> sym}
\label{sec-8-1-4-20}


Assigns a definition to the \texttt{sym} argument, by setting its \texttt{VAL} to the
\texttt{any} argument. If the symbol has already another value, a ``redefined''
message is issued. When the value of the global variable
\hyperref[refD.html-Dbg]{*Dbg} is non-=NIL=, the current line number and file
name (if any) are stored in the \texttt{*Dbg} property of \texttt{sym}. \texttt{de} is the
standard way to define a function. See also \texttt{def}, \texttt{dm} and \texttt{undef}.


\begin{verbatim}
: (de foo (X Y) (* X (+ X Y)))  # Define a function
-> foo
: (foo 3 4)
-> 21

: (de *Var . 123)  # Define a variable value
: *Var
-> 123
\end{verbatim}

 
\section{(debug 'sym) -> T}
\label{sec-8-1-4-21}


\texttt{(debug 'sym 'cls) -> T}

\texttt{(debug '(sym . cls)) -> T}

Inserts a \texttt{!} breakpoint function call at the beginning and all
top-level expressions of the function or method body of \texttt{sym}, to allow
a stepwise execution. Typing \texttt{(d)} at a breakpoint will also debug the
current subexpression, and \texttt{(e)} will evaluate the current
subexpression. The current subexpression is stored in the global
variable \texttt{\textasciicircum{}. See also \texttt{unbug}, \texttt{*Dbg}, \texttt{trace} and \texttt{lint}.


\begin{verbatim}
: (de tst (N)                    # Define tst
   (println (+ 3 N)) )
-> tst
: (debug 'tst)                   # Set breakpoints
-> T
: (pp 'tst)
(de tst (N)
   (! println (+ 3 N)) )         # Breakpoint '!'
-> tst
: (tst 7)                        # Execute
(println (+ 3 N))                # Stopped at beginning of 'tst'
! (d)                            # Debug subexpression
-> T
!                                # Continue
(+ 3 N)                          # Stopped in subexpression
! N                              # Inspect variable 'N'
-> 7
!                                # Continue
10                               # Output of print statement
-> 10                            # Done
: (unbug 'tst)
-> T
: (pp 'tst)                      # Restore to original
(de tst (N)
   (println (+ 3 N)) )
-> tst
\end{verbatim}

 
\section{(dec 'num) -> num}
\label{sec-8-1-4-22}


\texttt{(dec 'var ['num]) -> num}

The first form returns the value of \texttt{num} decremented by 1. The second
form decrements the \texttt{VAL} of \texttt{var} by 1, or by \texttt{num}. If the first
argument is \texttt{NIL}, it is returned immediately. \texttt{(dec 'num)} is
equivalent to \texttt{(- 'num 1)} and \texttt{(dec 'var)} is equivalent to
\texttt{(set 'var (- var 1))}. See also \texttt{inc} and \texttt{-}.


\begin{verbatim}
: (dec -1)
-> -2
: (dec 7)
-> 6
: (setq N 7)
-> 7
: (dec 'N)
-> 6
: (dec 'N 3)
-> 3
\end{verbatim}

 
\section{(def 'sym 'any) -> sym}
\label{sec-8-1-4-23}


\texttt{(def 'sym1 'sym2 'any) -> sym1}

The first form assigns a definition to the first \texttt{sym} argument, by
setting its \texttt{VAL}'s to \texttt{any}. The second form defines a property value
\texttt{any} for the first argument's \texttt{sym2} key. If any of these values
existed and was changed in the process, a ``redefined'' message is issued.
When the value of the global variable \hyperref[refD.html-Dbg]{*Dbg} is
non-=NIL=, the current line number and file name (if any) are stored in
the \texttt{*Dbg} property of \texttt{sym}. See also \texttt{de} and \texttt{dm}.


\begin{verbatim}
: (def 'b '((X Y) (* X (+ X Y))))
-> b
: (def 'b 999)
# b redefined
-> b
\end{verbatim}

 
\section{(default var 'any ..) -> any}
\label{sec-8-1-4-24}


Stores new values \texttt{any} in the \texttt{var} arguments only if their current
values are \texttt{NIL}. Otherwise, their values are left unchanged. In any
case, the last \texttt{var}'s value is returned. \texttt{default} is used typically in
functions to initialize optional arguments.


\begin{verbatim}
: (de foo (A B)               # Function with two optional arguments
   (default  A 1  B 2)        # The default values are 1 and 2
   (list A B) )
-> foo
: (foo 333 444)               # Called with two arguments
-> (333 444)
: (foo 333)                   # Called with one arguments
-> (333 2)
: (foo)                       # Called without arguments
-> (1 2)
\end{verbatim}

 
\section{(del 'any 'var) -> lst}
\label{sec-8-1-4-25}


Deletes \texttt{any} from the list in the value of \texttt{var}, and returns the
remaining list. \texttt{(del 'any 'var)} is equivalent to
\texttt{(set 'var (delete 'any var))}. See also \texttt{delete}, \texttt{cut} and \texttt{pop}.


\begin{verbatim}
: (setq S '((a b c) (d e f)))
-> ((a b c) (d e f))
: (del '(d e f) 'S)
-> ((a b c))
: (del 'b S)
-> (a c)
\end{verbatim}

 
\section{(delete 'any 'lst) -> lst}
\label{sec-8-1-4-26}


Deletes \texttt{any} from \texttt{lst}. If \texttt{any} is contained more than once in \texttt{lst},
only the first occurrence is deleted. See also \texttt{delq}, \texttt{remove} and
\texttt{insert}.


\begin{verbatim}
: (delete 2 (1 2 3))
-> (1 3)
: (delete (3 4) '((1 2) (3 4) (5 6) (3 4)))
-> ((1 2) (5 6) (3 4))
\end{verbatim}

 
\section{delete/3}
\label{sec-8-1-4-27}


\hyperref[ref.html-pilog]{Pilog} predicate that succeeds if deleting the first
argument from the list in the second argument is equal to the third
argument. See also \texttt{delete} and \texttt{member/2}.


\begin{verbatim}
: (? (delete b (a b c) @X))
 @X=(a c)
-> NIL
\end{verbatim}

 
\section{(delq 'any 'lst) -> lst}
\label{sec-8-1-4-28}


Deletes \texttt{any} from \texttt{lst}. If \texttt{any} is contained more than once in \texttt{lst},
only the first occurrence is deleted. ==== is used for comparison
(pointer equality). See also \texttt{delete}, \texttt{asoq}, \texttt{memq}, \texttt{mmeq} and
\hyperref[ref.html-cmp]{Comparing}.


\begin{verbatim}
: (delq 'b '(a b c))
-> (a c)
: (delq (2) (1 (2) 3))
-> (1 (2) 3)
\end{verbatim}

 
\section{(dep 'cls) -> cls}
\label{sec-8-1-4-29}


Displays the ``dependencies'' of \texttt{cls}, i.e. the tree of superclasses and
the tree of subclasses. See also \texttt{OO Concepts}, \texttt{class} and \texttt{can}.


\begin{verbatim}
: (dep '+Number)           # Dependencies of '+Number'
   +relation               # Single superclass is '+relation'
+Number
   +Date                   # Subclasses are '+Date' and '+Time'
   +Time
-> +Number
\end{verbatim}

 
\section{(depth 'lst) -> (cnt1 . cnt2)}
\label{sec-8-1-4-30}


Returns the maximal (\texttt{cnt1}) and the average (\texttt{cnt2}) ``depth'' of a tree
structure as maintained by \texttt{idx}. See also \texttt{length} and \texttt{size}.


\begin{verbatim}
: (off X)                                    # Clear variable
-> NIL
: (for N (1 2 3 4 5 6 7) (idx 'X N T))       # Build a degenerated tree
-> NIL
: X
-> (1 NIL 2 NIL 3 NIL 4 NIL 5 NIL 6 NIL 7)   # Only right branches
: (depth X)
-> (7 . 4)                                   # Depth is 7, average 4
\end{verbatim}

 
\section{(diff 'lst 'lst) -> lst}
\label{sec-8-1-4-31}


Returns the difference of the \texttt{lst} arguments. See also \texttt{sect}.


\begin{verbatim}
: (diff (1 2 3 4 5) (2 4))
-> (1 3 5)
: (diff (1 2 3) (1 2 3))
-> NIL
\end{verbatim}

 
\section{different/2}
\label{sec-8-1-4-32}


\hyperref[ref.html-pilog]{Pilog} predicate that succeeds if the two arguments
are different. See also \texttt{equal/2}.


\begin{verbatim}
: (? (different 3 4))
-> T
\end{verbatim}

 
\section{(dir ['any] ['flg]) -> lst}
\label{sec-8-1-4-33}


Returns a list of all filenames in the directory \texttt{any}. Names starting
with a dot `=.=' are ignored, unless \texttt{flg} is non-=NIL=. See also \texttt{cd}
and \texttt{info}.


\begin{verbatim}
: (filter '((F) (tail '(. c) (chop F))) (dir "src/"))
-> ("main.c" "subr.c" "gc.c" "io.c" "big.c" "sym.c" "tab.c" "flow.c" ..
\end{verbatim}

 
\section{(dirname 'any) -> sym}
\label{sec-8-1-4-34}


Returns the directory part of a path name \texttt{any}. See also \texttt{basename} and
\texttt{path}.


\begin{verbatim}
: (dirname "a/b/c/d")
-> "a/b/c/"
\end{verbatim}

 
\section{(dm sym . fun|cls2) -> sym}
\label{sec-8-1-4-35}


\texttt{(dm (sym . cls) . fun|cls2) -> sym}

\texttt{(dm (sym sym2 [. cls]) . fun|cls2) -> sym}

Defines a method for the message \texttt{sym} in the current class, implicitly
given by the value of the global variable \texttt{*Class}, or - in the second
form - for the explicitly given class \texttt{cls}. In the third form, the
class object is obtained by \texttt{get=ing =sym2} from \texttt{*Class} (or \texttt{cls} if
given). If the method for that class existed and was changed in the
process, a ``redefined'' message is issued. If - instead of a method \texttt{fun}
\begin{itemize}
\item a symbol specifying another class \texttt{cls2} is given, the method from
\end{itemize}
that class is used (explicit inheritance). When the value of the global
variable \hyperref[refD.html-Dbg]{*Dbg} is non-=NIL=, the current line number
and file name (if any) are stored in the \texttt{*Dbg} property of \texttt{sym}. See
also \texttt{OO Concepts}, \texttt{de}, \texttt{undef}, \hyperref[refC.html-class]{class},
\hyperref[refR.html-rel]{rel}, \hyperref[refV.html-var]{var},
\hyperref[refM.html-method]{method}, \hyperref[refS.html-send]{send} and
\hyperref[refT.html-try]{try}.


\begin{verbatim}
: (dm start> ()
   (super)
   (mapc 'start> (: fields))
   (mapc 'start> (: arrays)) )

: (dm foo> . +OtherClass)  # Explicitly inherit 'foo>' from '+OtherClass'
\end{verbatim}

 
\section{(do 'flg|num ['any | (NIL 'any . prg) | (T 'any . prg) ..]) -> any}
\label{sec-8-1-4-36}


Counted loop with multiple conditional exits: The body is executed at
most \texttt{num} times (or never (if the first argument is \texttt{NIL}), or an
infinite number of times (if the first argument is \texttt{T})). If a clause
has \texttt{NIL} or \texttt{T} as its CAR, the clause's second element is evaluated as
a condition and - if the result is \texttt{NIL} or non-=NIL=, respectively -
the \texttt{prg} is executed and the result returned. Otherwise (if count drops
to zero), the result of the last expression is returned. See also \texttt{loop}
and \texttt{for}.


\begin{verbatim}
: (do 4 (printsp 'OK))
OK OK OK OK -> OK
: (do 4 (printsp 'OK) (T (= 3 3) (printsp 'done)))
OK done -> done
\end{verbatim}

 
\section{(doc ['sym1] ['sym2])}
\label{sec-8-1-4-37}


Opens a browser, and tries to display the reference documentation for
\texttt{sym1}. \texttt{sym2} may be the name of a browser. If not given, the value of
the environment variable \texttt{BROWSER}, or the \texttt{w3m} browser is tried. If
\texttt{sym1} is \texttt{NIL}, the \hyperref[ref.html]{PicoLisp Reference} manual is opened.
See also \hyperref[ref.html-fun]{Function Reference} and \texttt{vi}.


\begin{verbatim}
: (doc '+)  # Function reference
-> T
: (doc '+relation)  # Class reference
-> T
: (doc)  # Reference manual
-> T
:  (doc 'vi "firefox")  # Use alternative browser
-> T
\end{verbatim}


\chapter{Functions starting with E}
\label{sec-8-1-5}


 
\section{*Err}
\label{sec-8-1-5-1}


A global variable holding a (possibly empty) \texttt{prg} body, which will be
executed during error processing. See also \texttt{Error Handling}, \texttt{*Msg} and
\texttt{\textasciicircum{}.


\begin{verbatim}
: (de *Err (prinl "Fatal error!"))
-> ((prinl "Fatal error!"))
: (/ 3 0)
!? (/ 3 0)
Div/0
Fatal error!
$
\end{verbatim}

 
\section{*Ext}
\label{sec-8-1-5-2}


A global variable holding a sorted list of cons pairs. The CAR of each
pair specifies an external symbol offset (suitable for \texttt{ext}), and the
CDR should be a function taking a single external symbol as an argument.
This function should return a list, with the value for that symbol in
its CAR, and the property list (in the format used by \texttt{getl} and \texttt{putl})
in its CDR. The symbol will be set to this value and property list upon
access. Typically this function will access the corresponding symbol in
a remote database process. See also \texttt{qsym} and \texttt{external symbols}.


\begin{verbatim}
### On the local machine ###
: (setq *Ext  # Define extension functions
   (mapcar
      '((@Host @Ext)
         (cons @Ext
            (curry (@Host @Ext (Sock)) (Obj)
               (when (or Sock (setq Sock (connect @Host 4040)))
                  (ext @Ext
                     (out Sock (pr (cons 'qsym Obj)))
                     (prog1 (in Sock (rd))
                        (unless @
                           (close Sock)
                           (off Sock) ) ) ) ) ) ) )
      '("10.10.12.1" "10.10.12.2" "10.10.12.3" "10.10.12.4")
      (20 40 60 80) ) )

### On the remote machines ###
(de go ()
   ...
   (task (port 4040)                      # Set up background query server
      (let? Sock (accept @)               # Accept a connection
         (unless (fork)                   # In child process
            (in Sock
               (while (rd)                # Handle requests
                  (sync)
                  (out Sock
                     (pr (eval @)) ) ) )
            (bye) )                       # Exit child process
         (close Sock) ) )
   (forked)                               # Close task in children
   ...
\end{verbatim}

 
\section{+Entity}
\label{sec-8-1-5-3}


Base class of all database objects. See also \texttt{+relation} and \texttt{Database}.

Messages to entity objects include


\begin{verbatim}
zap> ()              # Clean up relational structures, for removal from the DB
url> (Tab)           # Call the GUI on that object (in optional Tab)
upd> (X Old)         # Callback method when object is created/modified/deleted
has> (Var Val)       # Check if value is present
put> (Var Val)       # Put a new value
put!> (Var Val)      # Put a new value, single transaction
del> (Var Val)       # Delete value (also partial)
del!> (Var Val)      # Delete value (also partial), single transaction
inc> (Var Val)       # Increment numeric value
inc!> (Var Val)      # Increment numeric value, single transaction
dec> (Var Val)       # Decrement numeric value
dec!> (Var Val)      # Decrement numeric value, single transaction
mis> (Var Val)       # Return error message if value or type mismatch
lose1> (Var)         # Delete relational structures for a single attribute
lose> (Lst)          # Delete relational structures (excluding 'Lst')
lose!> ()            # Delete relational structures, single transaction
keep1> (Var)         # Restore relational structures for single attribute
keep> (Lst)          # Restore relational structures (excluding 'Lst')
keep?> (Lst)         # Test for restauration (excluding 'Lst')
keep!> ()            # Restore relational structures, single transaction
set> (Val)           # Set the value (type, i.e. class list)
set!> (Val)          # Set the value, single transaction
clone> ()            # Object copy
clone!> ()           # Object copy, single transaction
\end{verbatim}

 
\section{(e . prg) -> any}
\label{sec-8-1-5-4}


Used in a breakpoint. Evaluates \texttt{prg} in the execution environment, or
the currently executed expression if \texttt{prg} is not given. See also
\texttt{debug}, \texttt{!}, \texttt{\textasciicircum{} and \texttt{*Dbg}.


\begin{verbatim}
: (! + 3 4)
(+ 3 4)
! (e)
-> 7
\end{verbatim}

 
\section{(echo ['cnt ['cnt]] | ['sym ..]) -> sym}
\label{sec-8-1-5-5}


Reads the current input channel, and writes to the current output
channel. If \texttt{cnt} is given, only that many bytes are actually echoed. In
case of two \texttt{cnt} arguments, the first one specifies the number of bytes
to skip in the input stream. Otherwise, if one or more \texttt{sym} arguments
are given, the echo process stops as soon as one of the symbol's names
is encountered in the input stream. In this case the name will be read
and returned, but not written. Returns non-=NIL= if the operation was
successfully completed. See also \texttt{from}.


\begin{verbatim}
: (in "x.l" (echo))  # Display file on console
 ..

: (out "x2.l" (in "x.l" (echo)))  # Copy file "x.l" to "x2.l"
\end{verbatim}

 
\section{(edit 'sym ..) -> NIL}
\label{sec-8-1-5-6}


Edits the value and property list of the argument symbol(s) by calling
the \texttt{vim} editor on a temporary file with these data. When closing the
editor, the modified data are read and stored into the symbol(s). During
the edit session, individual symbols are separated by the pattern
\texttt{(********)}. These separators should not be modified. When moving the
cursor to the beginning of a symbol (no matter if internal, transient or
external), and hitting `=K=', that symbol is added to the currently
edited symbols. Hitting `=Q=' will go back one step and return to the
previously edited list of symbols.

\texttt{edit} is especially useful for browsing through the database (with
`=K=' and `=Q='), inspecting external symbols, but care must be taken
when modifying any data as then the \hyperref[ref.html-er]{entity/relation}
mechanisms are circumvented, and \texttt{commit} has to be called manually if
the changes should be persistent.

Another typical use case is inserting or removing \texttt{!} breakpoints at
arbitrary code locations, or doing other temporary changes to the code
for debugging purposes.

See also \texttt{update}, \texttt{show} and \texttt{vi}.


\begin{verbatim}
: (edit (db 'nr '+Item 1))  # Edit a database symbol
### 'vim' shows this ###
{3-1} (+Item)
   nr 1
   inv 100
   pr 29900
   sup {2-1}  # (+CuSu)
   nm "Main Part"

(********)
### Hitting 'K' on the '{' of '{2-1} ###
{2-1} (+CuSu)
   nr 1
   plz "3425"
   mob "37 176 86303"
   tel "37 4967 6846-0"
   fax "37 4967 68462"
   nm "Active Parts Inc."
   nm2 "East Division"
   ort "Freetown"
   str "Wildcat Lane"
   em "info@api.tld"

(********)

{3-1} (+Item)
   nr 1
   inv 100
   pr 29900
   sup {2-1}  # (+CuSu)
   nm "Main Part"

(********)
### Entering ':q' in vim ###
-> NIL
\end{verbatim}

 
\section{(env ['lst] | ['sym 'val] ..) -> lst}
\label{sec-8-1-5-7}


Return a list of symbol-value pairs of all dynamically bound symbols if
called without arguments, or of the symbols or symbol-value pairs in
\texttt{lst}, or the explicitly given \texttt{sym}-=val= arguments. See also \texttt{bind}
and \texttt{job}.


\begin{verbatim}
: (env)
-> NIL
: (let (A 1 B 2) (env))
-> ((A . 1) (B . 2))
: (let (A 1 B 2) (env '(A B)))
-> ((B . 2) (A . 1))
: (let (A 1 B 2) (env 'X 7 '(A B (C . 3)) 'Y 8))
-> ((Y . 8) (C . 3) (B . 2) (A . 1) (X . 7))
\end{verbatim}

 
\section{(eof ['flg]) -> flg}
\label{sec-8-1-5-8}


Returns the end-of-file status of the current input channel. If \texttt{flg} is
non-=NIL=, the channel's status is forced to end-of-file, so that the
next call to \texttt{eof} will return \texttt{T}, and calls to \texttt{char}, \texttt{peek}, \texttt{line},
\texttt{from}, \texttt{till}, \texttt{read} or \texttt{skip} will return \texttt{NIL}. Note that \texttt{eof}
cannot be used with the binary \texttt{rd} function. See also \texttt{eol}.


\begin{verbatim}
: (in "file" (until (eof) (println (line T))))
...
\end{verbatim}

 
\section{(eol) -> flg}
\label{sec-8-1-5-9}


Returns the end-of-line status of the current input channel. See also
\texttt{eof}.


\begin{verbatim}
: (make (until (prog (link (read)) (eol))))  # Read line into a list
a b c (d e f) 123
-> (a b c (d e f) 123)
\end{verbatim}

 
\section{equal/2}
\label{sec-8-1-5-10}


\hyperref[ref.html-pilog]{Pilog} predicate that succeeds if the two arguments
are equal. See also ===, \texttt{different/2} and \texttt{member/2}.


\begin{verbatim}
: (? (equal 3 4))
-> NIL
: (? (equal @N 7))
 @N=7
-> NIL
\end{verbatim}

 
\section{(err 'sym . prg) -> any}
\label{sec-8-1-5-11}


Redirects the standard error stream to \texttt{sym} during the execution of
\texttt{prg}. The current standard error stream will be saved and restored
appropriately. If the argument is \texttt{NIL}, the current output stream will
be used. Otherwise, \texttt{sym} is taken as a file name (opened in ``append''
mode if the first character is ``+''), where standard error is to be
written to. See also \texttt{in}, \texttt{out} and \texttt{ctl}.


\begin{verbatim}
: (err "/dev/null"             # Suppress error messages
   (call 'ls 'noSuchFile) )
-> NIL
\end{verbatim}

 
\section{(errno) -> cnt}
\label{sec-8-1-5-12}


(64-bit version only) Returns the value of the standard I/O `errno'
variable. See also \texttt{native}.


\begin{verbatim}
: (in "foo")                           # Produce an error
!? (in "foo")
"foo" -- Open error: No such file or directory
? (errno)
-> 2                                   # Returned 'ENOENT'
\end{verbatim}

 
\section{(eval 'any ['cnt ['lst]]) -> any}
\label{sec-8-1-5-13}


Evaluates \texttt{any}. Note that because of the standard argument evaluation,
\texttt{any} is actually evaluated twice. If a binding environment offset \texttt{cnt}
is given, the second evaluation takes place in the corresponding
environment, and an optional \texttt{lst} of excluded symbols can be supplied.
See also \texttt{run} and \texttt{up}.


\begin{verbatim}
: (eval (list '+ 1 2 3))
-> 6
: (setq X 'Y  Y 7)
-> 7
: X
-> Y
: Y
-> 7
: (eval X)
-> 7
\end{verbatim}

 
\section{(expDat 'sym) -> dat}
\label{sec-8-1-5-14}


Expands a \texttt{date} string according to the current \texttt{locale} (delimiter,
and order of year, month and day). Accepts abbreviated input, without
delimiter and with only the day, or the day and month, or the day, month
and year of current century. See also \texttt{datStr}, \texttt{day}, \texttt{expTel}.


\begin{verbatim}
: (date)
-> 733133
: (date (date))
-> (2007 5 31)
: (expDat "31")
-> 733133
: (expDat "315")
-> 733133
: (expDat "3105")
-> 733133
: (expDat "31057")
-> 733133
: (expDat "310507")
-> 733133
: (expDat "2007-05-31")
-> 733133
: (expDat "7-5-31")
-> 733133

: (locale "DE" "de")
-> NIL
: (expDat "31.5")
-> 733133
: (expDat "31.5.7")
-> 733133
\end{verbatim}

 
\section{(expTel 'sym) -> sym}
\label{sec-8-1-5-15}


Expands a telephone number string. Multiple spaces or hyphens are
coalesced. A leading \texttt{+} or \texttt{00} is removed, a leading \texttt{0} is replaced
with the current country code. Otherwise, \texttt{NIL} is returned. See also
\texttt{telStr}, \texttt{expDat} and \texttt{locale}.


\begin{verbatim}
: (expTel "+49 1234 5678-0")
-> "49 1234 5678-0"
: (expTel "0049 1234 5678-0")
-> "49 1234 5678-0"
: (expTel "01234 5678-0")
-> NIL
: (locale "DE" "de")
-> NIL
: (expTel "01234 5678-0")
-> "49 1234 5678-0"
\end{verbatim}

 
\section{(expr 'sym) -> fun}
\label{sec-8-1-5-16}


Converts a C-function (``subr'') to a Lisp-function. Useful only for
normal functions (i.e. functions that evaluate all arguments). See also
\texttt{subr}.


\begin{verbatim}
: car
-> 67313448
: (expr 'car)
-> (@ (pass $385260187))
: (car (1 2 3))
-> 1
\end{verbatim}

 
\section{(ext 'cnt . prg) -> any}
\label{sec-8-1-5-17}


During the execution of \texttt{prg}, all \texttt{external symbols} processed by \texttt{rd},
\texttt{pr} or \texttt{udp} are modified by an offset \texttt{cnt} suitable for mapping via
the \texttt{*Ext} mechanism. All external symbol's file numbers are decremented
by \texttt{cnt} during output, and incremented by \texttt{cnt} during input.


\begin{verbatim}
: (out 'a (ext 5 (pr '({6-2} ({8-9} . a) ({7-7} . b)))))
-> ({6-2} ({8-9} . a) ({7-7} . b))

: (in 'a (rd))
-> ({2} ({3-9} . a) ({2-7} . b))

: (in 'a (ext 5 (rd)))
-> ({6-2} ({8-9} . a) ({7-7} . b))
\end{verbatim}

 
\section{(ext? 'any) -> sym | NIL}
\label{sec-8-1-5-18}


Returns the argument \texttt{any} when it is an existing external symbol,
otherwise \texttt{NIL}. See also \texttt{sym?}, \texttt{box?}, \texttt{str?}, \texttt{extern} and \texttt{lieu}.


\begin{verbatim}
: (ext? *DB)
-> {1}
: (ext? 'abc)
-> NIL
: (ext? "abc")
-> NIL
: (ext? 123)
-> NIL
\end{verbatim}

 
\section{(extend cls) -> cls}
\label{sec-8-1-5-19}


Extends the class \texttt{cls}, by storing it in the global variable \texttt{*Class}.
As a consequence, all following method, relation and class variable
definitions are applied to that class. See also \texttt{OO Concepts}, \texttt{class},
\texttt{dm}, \texttt{var}, \texttt{rel}, \texttt{type} and \texttt{isa}.


\begin{verbatim}
\end{EXAMPLE}

 

**** =(extern 'sym) -> sym | NIL=

Creates or finds an external symbol. If a symbol with the name =sym= is
already extern, it is returned. Otherwise, a new external symbol is
returned. =NIL= is returned if =sym= does not exist in the database. See
also =intern= and =ext?=.

\begin{EXAMPLE}
    : (extern "A1b")
    -> {A1b}
    : (extern "{A1b}")
    -> {A1b}
\end{verbatim}

 
\section{(extra ['any ..]) -> any}
\label{sec-8-1-5-20}


Can only be used inside methods. Sends the current message to the
current object \texttt{This}, this time starting the search for a method at the
remaining branches of the inheritance tree of the class where the
current method was found. See also \texttt{OO Concepts}, \texttt{super}, \texttt{method},
\texttt{meth}, \texttt{send} and \texttt{try}.


\begin{verbatim}
(dm key> (C)            # 'key>' method of the '+Uppc' class
   (uppc (extra C)) )   # Convert 'key>' of extra classes to upper case
\end{verbatim}

 
\section{(extract 'fun 'lst ..) -> lst}
\label{sec-8-1-5-21}


Applies \texttt{fun} to each element of \texttt{lst}. When additional \texttt{lst} arguments
are given, their elements are also passed to \texttt{fun}. Returns a list of
all non-=NIL= values returned by \texttt{fun}. \texttt{(extract 'fun 'lst)} is
equivalent to \texttt{(mapcar 'fun (filter 'fun 'lst))} or, for non-NIL
results, to \texttt{(mapcan '((X) (and (fun X) (cons @))) 'lst)}. See also
\texttt{filter}, \texttt{find}, \texttt{pick} and \texttt{mapcan}.


\begin{verbatim}
: (setq A NIL  B 1  C NIL  D 2  E NIL  F 3)
-> 3
: (filter val '(A B C D E F))
-> (B D F)
: (extract val '(A B C D E F))
-> (1 2 3)
\end{verbatim}


\chapter{Functions starting with F}
\label{sec-8-1-6}


 
\section{*Fork}
\label{sec-8-1-6-1}


A global variable holding a (possibly empty) \texttt{prg} body, to be executed
after a call to \texttt{fork} in the child process.


\begin{verbatim}
: (push '*Fork '(off *Tmp))   # Clear '*Tmp' in child process
-> (off *Tmp)
\end{verbatim}

 
\section{+Fold}
\label{sec-8-1-6-2}


Prefix class for maintaining \texttt{fold=ed indexes to =+String} relations.
Typically used in combination with the \texttt{+Ref} or \texttt{+Idx} prefix classes.
See also \texttt{Database}.


\begin{verbatim}
(rel nm (+Fold +Idx +String))   # Item Description
...
(rel tel (+Fold +Ref +String))  # Phone number
\end{verbatim}

 
\section{(fail) -> lst}
\label{sec-8-1-6-3}


Constructs an empty \hyperref[ref.html-pilog]{Pilog} query, i.e. a query that
will aways fail. See also \texttt{goal}.


\begin{verbatim}
(dm clr> ()                # Clear query chart in search dialogs
   (query> This (fail)) )
\end{verbatim}

 
\section{fail/0}
\label{sec-8-1-6-4}


\hyperref[ref.html-pilog]{Pilog} predicate that always fails. See also
\texttt{true/0}.


\begin{verbatim}
:  (? (fail))
-> NIL
\end{verbatim}

 
\section{(fetch 'tree 'any) -> any}
\label{sec-8-1-6-5}


Fetches a value for the key \texttt{any} from a database tree. See also \texttt{tree}
and \texttt{store}.


\begin{verbatim}
: (fetch (tree 'nr '+Item) 2)
-> {3-2}
\end{verbatim}

 
\section{(fifo 'var ['any ..]) -> any}
\label{sec-8-1-6-6}


Implements a first-in-first-out structure using a circular list. When
called with \texttt{any} arguments, they will be concatenated to end of the
structure. Otherwise, the first element is removed from the structure
and returned. See also \texttt{queue}, \texttt{push}, \texttt{pop}, \texttt{rot} and \texttt{circ}.


\begin{verbatim}
: (fifo 'X 1)
-> 1
: (fifo 'X 2 3)
-> 3
: X
-> (3 1 2 .)
: (fifo 'X)
-> 1
: (fifo 'X)
-> 2
: X
-> (3 .)
\end{verbatim}

 
\section{(file) -> (sym1 sym2 . num) | NIL}
\label{sec-8-1-6-7}


Returns for the current input channel the path name \texttt{sym1}, the file
name \texttt{sym2}, and the current line number \texttt{num}. If the current input
channel is not a file, \texttt{NIL} is returned. See also \texttt{info}, \texttt{in} and
\texttt{load}.


\begin{verbatim}
: (load (pack (car (file)) "localFile.l"))  # Load a file in same directory
\end{verbatim}

 
\section{(fill 'any ['sym|lst]) -> any}
\label{sec-8-1-6-8}


Fills a pattern \texttt{any}, by substituting \texttt{sym}, or all symbols in \texttt{lst},
or - if no second argument is given - each pattern symbol in \texttt{any} (see
\texttt{pat?}), with its current value. \texttt{@} itself is not considered a pattern
symbol here. In any case, expressions following the symbol \texttt{\textasciicircum{} should
evaluate to lists which are then (destructively) spliced into the
result. See also \texttt{match}.


\begin{verbatim}
: (setq  @X 1234  @Y (1 2 3 4))
-> (1 2 3 4)
: (fill '@X)
-> 1234
: (fill '(a b (c @X) ((@Y . d) e)))
-> (a b (c 1234) (((1 2 3 4) . d) e))
: (let X 2 (fill (1 X 3) 'X))
-> (1 2 3)

: (fill (1 ^ (list 'a 'b 'c) 9))
-> (1 a b c 9)

: (match '(This is @X) '(This is a pen))
-> T
: (fill '(Got ^ @X))
-> (Got a pen)
\end{verbatim}

 
\section{(filter 'fun 'lst ..) -> lst}
\label{sec-8-1-6-9}


Applies \texttt{fun} to each element of \texttt{lst}. When additional \texttt{lst} arguments
are given, their elements are also passed to \texttt{fun}. Returns a list of
all elements of \texttt{lst} where \texttt{fun} returned non-=NIL=. See also \texttt{fish},
\texttt{find}, \texttt{pick} and \texttt{extract}.


\begin{verbatim}
: (filter num? (1 A 2 (B) 3 CDE))
-> (1 2 3)
\end{verbatim}

 
\section{(fin 'any) -> num|sym}
\label{sec-8-1-6-10}


Returns \texttt{any} if it is an atom, otherwise the CDR of its last cell. See
also \texttt{last} and \texttt{tail}.


\begin{verbatim}
: (fin 'a)
-> a
: (fin '(a . b))
-> b
: (fin '(a b . c))
-> c
: (fin '(a b c))
-> NIL
\end{verbatim}

 
\section{(finally exe . prg) -> any}
\label{sec-8-1-6-11}


\texttt{prg} is executed, then \texttt{exe} is evaluated, and the result of \texttt{prg} is
returned. \texttt{exe} will also be evaluated if \texttt{prg} does not terminate
normally due to a runtime error or a call to \texttt{throw}. See also \texttt{bye},
\texttt{catch}, \texttt{quit} and \texttt{Error Handling}.


\begin{verbatim}
: (finally (prinl "Done!")
   (println 123)
   (quit)
   (println 456) )
123
Done!
: (catch 'A
   (finally (prinl "Done!")
      (println 1)
      (throw 'A 123)
      (println 2) ) )
1
Done!
-> 123
\end{verbatim}

 
\section{(find 'fun 'lst ..) -> any}
\label{sec-8-1-6-12}


Applies \texttt{fun} to successive elements of \texttt{lst} until non-=NIL= is
returned. Returns that element, or \texttt{NIL} if \texttt{fun} did not return
non-=NIL= for any element of \texttt{lst}. When additional \texttt{lst} arguments are
given, their elements are also passed to \texttt{fun}. See also \texttt{seek}, \texttt{pick}
and \texttt{filter}.


\begin{verbatim}
: (find pair (1 A 2 (B) 3 CDE))
-> (B)
: (find '((A B) (> A B)) (1 2 3 4 5 6) (6 5 4 3 2 1))
-> 4
: (find > (1 2 3 4 5 6) (6 5 4 3 2 1))  # shorter
-> 4
\end{verbatim}

 
\section{(fish 'fun 'any) -> lst}
\label{sec-8-1-6-13}


Applies \texttt{fun} to each element - and recursively to all sublists - of
\texttt{any}. Returns a list of all items where \texttt{fun} returned non-=NIL=. See
also \texttt{filter}.


\begin{verbatim}
: (fish gt0 '(a -2 (1 b (-3 c 2)) 3 d -1))
-> (1 2 3)
: (fish sym? '(a -2 (1 b (-3 c 2)) 3 d -1))
-> (a b c d)
\end{verbatim}

 
\section{(flg? 'any) -> flg}
\label{sec-8-1-6-14}


Returns \texttt{T} when the argument \texttt{any} is either \texttt{NIL} or \texttt{T}. See also
\texttt{bool}. \texttt{(flg? X)} is equivalent to \texttt{(or (not X) (=T X))}.


\begin{verbatim}
: (flg? (= 3 3))
-> T
: (flg? (= 3 4))
-> T
: (flg? (+ 3 4))
-> NIL
\end{verbatim}

 
\section{(flip 'lst ['cnt]) -> lst}
\label{sec-8-1-6-15}


Returns \texttt{lst} (destructively) reversed. Without the optional \texttt{cnt}
argument, the whole list is flipped, otherwise only the first \texttt{cnt}
elements. See also \texttt{reverse} and \texttt{rot}.


\begin{verbatim}
: (flip (1 2 3 4))         # Flip all  four elements
-> (4 3 2 1)
: (flip (1 2 3 4 5 6) 3)   # Flip only the first three elements
-> (3 2 1 4 5 6)
\end{verbatim}

 
\section{(flush) -> flg}
\label{sec-8-1-6-16}


Flushes the current output stream by writing all buffered data. A call
to \texttt{flush} for standard output is done automatically before a call to
\texttt{key}. Returns \texttt{T} when successful. See also \texttt{rewind}.


\begin{verbatim}
: (flush)
-> T
\end{verbatim}

 
\section{(fmt64 'num) -> sym}
\label{sec-8-1-6-17}


\texttt{(fmt64 'sym) -> num}

Converts a number \texttt{num} to a string in base--64 notation, or a base--64
formatted string to a number. The digits are represented with the
characters \texttt{0} - \texttt{9}, \texttt{:}, \texttt{;}, \texttt{A} - \texttt{Z} and \texttt{a} - \texttt{z}. This format is
used internally for the names of \texttt{external symbols} in the 32-bit
version. See also \texttt{hax}, \texttt{hex}, \texttt{bin} and \texttt{oct}.


\begin{verbatim}
: (fmt64 9)
-> "9"
: (fmt64 10)
-> ":"
: (fmt64 11)
-> ";"
: (fmt64 12)
-> "A"
: (fmt64 "100")
-> 4096
\end{verbatim}

 
\section{(fold 'any ['cnt]) -> sym}
\label{sec-8-1-6-18}


Folding to a canonical form: If \texttt{any} is not a symbol, \texttt{NIL} is
returned. Otherwise, a new transient symbol with all digits and all
letters of \texttt{any}, converted to lower case, is returned. If the \texttt{cnt}
argument is given, the result is truncated to that length (or not
truncated if \texttt{cnt} is zero). Otherwise \texttt{cnt} defaults to 24. See also
\texttt{lowc}.


\begin{verbatim}
: (fold " 1A 2-b/3")
-> "1a2b3"
: (fold " 1A 2-B/3" 3)
-> "1a2"
\end{verbatim}

 
\section{fold/3}
\label{sec-8-1-6-19}


\hyperref[ref.html-pilog]{Pilog} predicate that succeeds if the first argument,
after \texttt{fold=ing it to a canonical form, is a /prefix/ of the folded string representation of the result of applying the =get} algorithm to
the following arguments. Typically used as filter predicate in
\texttt{select/3} database queries. See also \texttt{pre?}, \texttt{isa/2}, \texttt{same/3},
\texttt{bool/3}, \texttt{range/3}, \texttt{head/3}, \texttt{part/3} and \texttt{tolr/3}.


\begin{verbatim}
: (?
   @Nr (1 . 5)
   @Nm "main"
   (select (@Item)
      ((nr +Item @Nr) (nm +Item @Nm))
      (range @Nr @Item nr)
      (fold @Nm @Item nm) ) )
 @Nr=(1 . 5) @Nm="main" @Item={3-1}
-> NIL
\end{verbatim}

 
\section{(for sym 'num ['any | (NIL 'any . prg) | (T 'any . prg) ..]) -> any}
\label{sec-8-1-6-20}


\texttt{(for sym|(sym2 . sym) 'lst ['any | (NIL 'any . prg) | (T 'any . prg) ..]) -> any}

\texttt{(for (sym|(sym2 . sym) 'any1 'any2 [. prg]) ['any | (NIL 'any . prg) | (T 'any . prg) ..]) -> any}

Conditional loop with local variable(s) and multiple conditional exits:
In the first form, the value of \texttt{sym} is saved, \texttt{sym} is bound to \texttt{1},
and the body is executed with increasing values up to (and including)
\texttt{num}. In the second form, the value of \texttt{sym} is saved, \texttt{sym} is
subsequently bound to the elements of \texttt{lst}, and the body is executed
each time. In the third form, the value of \texttt{sym} is saved, and \texttt{sym} is
bound to \texttt{any1}. If \texttt{sym2} is given, it is treated as a counter
variable, first bound to 1 and then incremented for each execution of
the body. While the condition \texttt{any2} evaluates to non-=NIL=, the body is
repeatedly executed and, if \texttt{prg} is given, \texttt{sym} is re-bound to the
result of its evaluation. If a clause has \texttt{NIL} or \texttt{T} as its CAR, the
clause's second element is evaluated as a condition and - if the result
is \texttt{NIL} or non-=NIL=, respectively - the \texttt{prg} is executed and the
result returned. If the body is never executed, \texttt{NIL} is returned. See
also \texttt{do} and \texttt{loop}.


\begin{verbatim}
: (for (N 1 (>= 8 N) (inc N)) (printsp N))
1 2 3 4 5 6 7 8 -> 8
: (for (L (1 2 3 4 5 6 7 8) L) (printsp (pop 'L)))
1 2 3 4 5 6 7 8 -> 8
: (for X (1 a 2 b) (printsp X))
1 a 2 b -> b
: (for ((I . L) '(a b c d e f) L (cddr L)) (println I L))
1 (a b c d e f)
2 (c d e f)
3 (e f)
-> (e f)
: (for (I . X) '(a b c d e f) (println I X))
1 a
2 b
3 c
4 d
5 e
6 f
-> f
\end{verbatim}

 
\section{(fork) -> pid | NIL}
\label{sec-8-1-6-21}


Forks a child process. Returns \texttt{NIL} in the child, and the child's
process ID \texttt{pid} in the parent. In the child, the \texttt{VAL} of the global
variable \texttt{*Fork} (should be a \texttt{prg}) is executed. See also \texttt{pipe} and
\texttt{tell}.


\begin{verbatim}
: (unless (fork) (do 5 (println 'OK) (wait 1000)) (bye))
-> NIL
OK                                              # Child's output
: OK
OK
OK
OK
\end{verbatim}

 
\section{(forked)}
\label{sec-8-1-6-22}


Installs maintenance code in \texttt{*Fork} to close server sockets and clean
up \texttt{*Run} code in child processes. Should only be called immediately
after \texttt{task}.


\begin{verbatim}
: (task -60000 60000 (msg 'OK))     # Install timer task
-> (-60000 60000 (msg 'OK))
: (forked)                          # No timer in child processes
-> (task -60000)
: *Run
-> ((-60000 56432 (msg 'OK)))
: *Fork
-> ((task -60000) (del '(saveHistory) '*Bye))
\end{verbatim}

 
\section{(format 'num ['cnt ['sym1 ['sym2]]]) -> sym}
\label{sec-8-1-6-23}


\texttt{(format 'sym|lst ['cnt ['sym1 ['sym2]]]) -> num}

Converts a number \texttt{num} to a string, or a string \texttt{sym|lst} to a number.
In both cases, optionally a precision \texttt{cnt}, a decimal-separator \texttt{sym1}
and a thousands-separator \texttt{sym2} can be supplied. Returns \texttt{NIL} if the
conversion is unsuccessful. See also \hyperref[ref.html-num-io]{Numbers} and
\texttt{round}.


\begin{verbatim}
: (format 123456789)                   # Integer conversion
-> "123456789"
: (format 123456789 2)                 # Fixed point
-> "1234567.89"
: (format 123456789 2 ",")             # Comma as decimal-separator
-> "1234567,89"
: (format 123456789 2 "," ".")         # and period as thousands-separator
-> "1.234.567,89"

: (format "123456789")                 # String to number
-> 123456789
: (format (1 "23" (4 5 6)))
-> 123456
: (format "1234567.89" 4)              # scaled to four digits
-> 12345678900
: (format "1.234.567,89")              # separators not recognized
-> NIL
: (format "1234567,89" 4 ",")
-> 12345678900
: (format "1.234.567,89" 4 ",")        # thousands-separator not recognized
-> NIL
: (format "1.234.567,89" 4 "," ".")
-> 12345678900
\end{verbatim}

 
\section{(free 'cnt) -> (sym . lst)}
\label{sec-8-1-6-24}


Returns, for the \texttt{cnt}'th database file, the next available symbol \texttt{sym}
(i.e. the first symbol greater than any symbol in the database), and the
list \texttt{lst} of free symbols. See also \texttt{seq}, \texttt{zap} and \texttt{dbck}.


\begin{verbatim}
: (pool "x")      # A new database
-> T
: (new T)         # Create a new symbol
-> {2}
: (new T)         # Create another symbol
-> {3}
: (commit)        # Commit changes
-> T
: (zap '{2})      # Delete the first symbol
-> {2}
: (free 1)        # Show free list
-> ({4})          # {3} was the last symbol allocated
: (commit)        # Commit the deletion of {2}
-> T
: (free 1)        # Now {2} is in the free list
-> ({4} {2})
\end{verbatim}

 
\section{(from 'any ..) -> sym}
\label{sec-8-1-6-25}


Skips the current input channel until one of the strings \texttt{any} is found,
and starts subsequent reading from that point. The found \texttt{any} argument,
or \texttt{NIL} (if none is found) is returned. See also \texttt{till} and \texttt{echo}.


\begin{verbatim}
: (and (from "val='") (till "'" T))
test val='abc'
-> "abc"
\end{verbatim}

 
\section{(full 'any) -> bool}
\label{sec-8-1-6-26}


Returns \texttt{NIL} if \texttt{any} is a non-empty list with at least one \texttt{NIL}
element, otherwise \texttt{T}. \texttt{(full X)} is equivalent to
\texttt{(not (memq NIL X))}.


\begin{verbatim}
: (full (1 2 3))
-> T
: (full (1 NIL 3))
-> NIL
: (full 123)
-> T
\end{verbatim}

 
\section{(fun? 'any) -> any}
\label{sec-8-1-6-27}


Returns \texttt{NIL} when the argument \texttt{any} is neither a number suitable for a
code-pointer, nor a list suitable for a lambda expression (function).
Otherwise a number is returned for a code-pointer, \texttt{T} for a function
without arguments, and a single formal parameter or a list of formal
parameters for a function. See also \texttt{getd}.


\begin{verbatim}
: (fun? 1000000000)              # Might be a code pointer
-> 1000000000
: (fun? 100000000000000)         # Too big for a code pointer
-> NIL
: (fun? 1000000001)              # Cannot be a code pointer (odd)
-> NIL
: (fun? '((A B) (* A B)))        # Lambda expression
-> (A B)
: (fun? '((A B) (* A B) . C))    # Not a lambda expression
-> NIL
: (fun? '(1 2 3 4))              # Not a lambda expression
-> NIL
: (fun? '((A 2 B) (* A B)))      # Not a lambda expression
-> NIL
\end{verbatim}


\chapter{Functions starting with G}
\label{sec-8-1-7}


 
\section{(gc ['cnt]) -> cnt | NIL}
\label{sec-8-1-7-1}


Forces a garbage collection. When \texttt{cnt} is given, so many megabytes of
free cells are reserved, increasing the heap size if necessary. If \texttt{cnt}
is zero, all currently unused heap blocks are purged, decreasing the
heap size if possible. See also \texttt{heap}.


\begin{verbatim}
: (gc)
-> NIL
: (heap)
-> 2
: (gc 4)
-> 4
: (heap)
-> 5
\end{verbatim}

 
\section{(ge0 'any) -> num | NIL}
\label{sec-8-1-7-2}


Returns \texttt{num} when the argument is a number and greater or equal zero,
otherwise \texttt{NIL}. See also \texttt{lt0}, \texttt{le0}, \texttt{gt0}, ==0= and \texttt{n0}.


\begin{verbatim}
: (ge0 -2)
-> NIL
: (ge0 3)
-> 3
: (ge0 0)
-> 0
\end{verbatim}

 
\section{(genKey 'var 'cls ['hook ['num1 ['num2]]]) -> num}
\label{sec-8-1-7-3}


Generates a key for a database tree. If a minimal key \texttt{num1} and/or a
maximal key \texttt{num2} is given, the next free number in that range is
returned. Otherwise, the current maximal key plus one is returned. See
also \texttt{useKey}, \texttt{genStrKey} and \texttt{maxKey}.


\begin{verbatim}
: (maxKey (tree 'nr '+Item))
-> 8
: (genKey 'nr '+Item)
-> 9
\end{verbatim}

 
\section{(genStrKey 'sym 'var 'cls ['hook]) -> sym}
\label{sec-8-1-7-4}


Generates a unique string for a database tree, by prepending as many ``\#''
sequences as necessary. See also \texttt{genKey}.


\begin{verbatim}
: (genStrKey "ben" 'nm '+User)
-> "# ben"
\end{verbatim}

 
\section{(get 'sym1|lst ['sym2|cnt ..]) -> any}
\label{sec-8-1-7-5}


Fetches a value \texttt{any} from the properties of a symbol, or from a list.
From the first argument \texttt{sym1|lst}, values are retrieved in successive
steps by either extracting the value (if the next argument is zero) or a
property from a symbol, the =asoq=ed element (if the next argument is a
symbol), the n'th element (if the next argument is a positive number) or
the n'th CDR (if the next argument is a negative number) from a list.
See also \texttt{put}, \texttt{;} and \texttt{:}.


\begin{verbatim}
: (put 'X 'a 1)
-> 1
: (get 'X 'a)
-> 1
: (put 'Y 'link 'X)
-> X
: (get 'Y 'link)
-> X
: (get 'Y 'link 'a)
-> 1
: (get '((a (b . 1) (c . 2)) (d (e . 3) (f . 4))) 'a 'b)
-> 1
: (get '((a (b . 1) (c . 2)) (d (e . 3) (f . 4))) 'd 'f)
-> 4
: (get '(X Y Z) 2)
-> Y
: (get '(X Y Z) 2 'link 'a)
-> 1
\end{verbatim}

 
\section{(getd 'any) -> fun | NIL}
\label{sec-8-1-7-6}


Returns \texttt{fun} if \texttt{any} is a symbol that has a function definition,
otherwise \texttt{NIL}. See also \texttt{fun?}.


\begin{verbatim}
: (getd '+)
-> 67327232
: (getd 'script)
-> ((File . @) (load File))
: (getd 1)
-> NIL
\end{verbatim}

 
\section{(getl 'sym1|lst1 ['sym2|cnt ..]) -> lst}
\label{sec-8-1-7-7}


Fetches the complete property list \texttt{lst} from a symbol. That symbol is
\texttt{sym1} (if no other arguments are given), or a symbol found by applying
the \texttt{get} algorithm to \texttt{sym1|lst1} and the following arguments. See also
\texttt{putl} and \texttt{maps}.


\begin{verbatim}
: (put 'X 'a 1)
-> 1
: (put 'X 'b 2)
-> 2
: (put 'X 'flg T)
-> T
: (getl 'X)
-> (flg (2 . b) (1 . a))
\end{verbatim}

 
\section{(glue 'any 'lst) -> sym}
\label{sec-8-1-7-8}


Builds a new transient symbol (string) by \texttt{pack=ing the =any} argument
between the individual elements of \texttt{lst}. See also \texttt{text}.


\begin{verbatim}
: (glue "," '(a b c d))
-> "a,b,c,d"
\end{verbatim}

 
\section{(goal '([pat 'any ..] . lst) ['sym 'any ..]) -> lst}
\label{sec-8-1-7-9}


Constructs a \hyperref[ref.html-pilog]{Pilog} query list from the list of
clauses \texttt{lst}. The head of the argument list may consist of a sequence
of pattern symbols (Pilog variables) and expressions, which are used
together with the optional \texttt{sym} and \texttt{any} arguments to form an initial
environment. See also \texttt{prove} and \texttt{fail}.


\begin{verbatim}
: (goal '((likes John @X)))
-> (((1 (0) NIL ((likes John @X)) NIL T)))
: (goal '(@X 'John (likes @X @Y)))
-> (((1 (0) NIL ((likes @X @Y)) NIL ((0 . @X) 1 . John) T)))
\end{verbatim}

 
\section{(group 'lst) -> lst}
\label{sec-8-1-7-10}


Builds a list of lists, by grouping all elements of \texttt{lst} with the same
CAR into a common sublist. See also \hyperref[ref.html-cmp]{Comparing}, \texttt{by},
\texttt{sort} and \texttt{uniq}.


\begin{verbatim}
: (group '((1 . a) (1 . b) (1 . c) (2 . d) (2 . e) (2 . f)))
-> ((1 a b c) (2 d e f))
: (by name group '("x" "x" "y" "z" "x" "z")))
-> (("x" "x" "x") ("y") ("z" "z"))
: (by length group '(123 (1 2) "abcd" "xyz" (1 2 3 4) "XY"))
-> ((123 "xyz") ((1 2) "XY") ("abcd" (1 2 3 4))
\end{verbatim}

 
\section{(gt0 'any) -> num | NIL}
\label{sec-8-1-7-11}


Returns \texttt{num} when the argument is a number and greater than zero,
otherwise \texttt{NIL}. See also \texttt{lt0}, \texttt{le0}, \texttt{ge0}, ==0= and \texttt{n0}.


\begin{verbatim}
: (gt0 -2)
-> NIL
: (gt0 3)
-> 3
\end{verbatim}


\chapter{Functions starting with H}
\label{sec-8-1-8}


 
\section{*Hup}
\label{sec-8-1-8-1}


Global variable holding a (possibly empty) \texttt{prg} body, which will be
executed when a SIGHUP signal is sent to the current process. See also
\texttt{alarm}, \texttt{sigio} and \texttt{*Sig[12]}.


\begin{verbatim}
: (de *Hup (msg 'SIGHUP))
-> *Hup
\end{verbatim}

 
\section{+Hook}
\label{sec-8-1-8-2}


Prefix class for \texttt{+relation=s, typically =+Link} or \texttt{+Joint}. In
essence, this maintains an local database in the referred object. See
also \texttt{Database}.


\begin{verbatim}
(rel sup (+Hook +Link) (+Sup))   # Supplier
(rel nr (+Key +Number) sup)      # Item number, unique per supplier
(rel dsc (+Ref +String) sup)     # Item description, indexed per supplier
\end{verbatim}

 
\section{(hash 'any) -> cnt}
\label{sec-8-1-8-3}


Generates a 16-bit number (1--65536) from \texttt{any}, suitable as a hash
value for various purposes, like randomly balanced \texttt{idx} structures. See
also \texttt{cache} and \texttt{seed}.


\begin{verbatim}
: (hash 0)
-> 1
: (hash 1)
-> 55682
: (hash "abc")
-> 45454
\end{verbatim}

 
\section{(hax 'num) -> sym}
\label{sec-8-1-8-4}


\texttt{(hax 'sym) -> num}

Converts a number \texttt{num} to a string in hexadecimal/alpha notation, or a
hexadecimal/alpha formatted string to a number. The digits are
represented with `=@=' (zero) and the letters `=A=' - `=O=' (from
``alpha'' to ``omega''). This format is used internally for the names of
\texttt{external symbols} in the 64-bit version. See also \texttt{fmt64}, \texttt{hex}, \texttt{bin}
and \texttt{oct}.


\begin{verbatim}
: (hax 7)
-> "G"
: (hax 16)
-> "A@"
: (hax 255)
-> "OO"
: (hax "A")
-> 1
\end{verbatim}

 
\section{(hd 'sym ['cnt]) -> NIL}
\label{sec-8-1-8-5}


Displays a hexadecimal dump of the file given by \texttt{sym}, limited to \texttt{cnt}
lines. See also \texttt{proc}.


\begin{verbatim}
:  (hd "lib.l" 4)
00000000  23 20 32 33 64 65 63 30 39 61 62 75 0A 23 20 28  # 23dec09abu.# (
00000010  63 29 20 53 6F 66 74 77 61 72 65 20 4C 61 62 2E  c) Software Lab.
00000020  20 41 6C 65 78 61 6E 64 65 72 20 42 75 72 67 65   Alexander Burge
00000030  72 0A 0A 28 64 65 20 74 61 73 6B 20 28 4B 65 79  r..(de task (Key
-> NIL
\end{verbatim}

 
\section{(head 'cnt|lst 'lst) -> lst}
\label{sec-8-1-8-6}


Returns a new list made of the first \texttt{cnt} elements of \texttt{lst}. If \texttt{cnt}
is negative, it is added to the length of \texttt{lst}. If the first argument
is a \texttt{lst}, \texttt{head} is a predicate function returning that argument list
if it is \texttt{equal} to the head of the second argument, and \texttt{NIL}
otherwise. See also \texttt{tail}.


\begin{verbatim}
: (head 3 '(a b c d e f))
-> (a b c)
: (head 0 '(a b c d e f))
-> NIL
: (head 10 '(a b c d e f))
-> (a b c d e f)
: (head -2 '(a b c d e f))
-> (a b c d)
: (head '(a b c) '(a b c d e f))
-> (a b c)
\end{verbatim}

 
\section{head/3}
\label{sec-8-1-8-7}


\hyperref[ref.html-pilog]{Pilog} predicate that succeeds if the first (string)
argument is a prefix of the string representation of the result of
applying the \texttt{get} algorithm to the following arguments. Typically used
as filter predicate in \texttt{select/3} database queries. See also \texttt{pre?},
\texttt{isa/2}, \texttt{same/3}, \texttt{bool/3}, \texttt{range/3}, \texttt{fold/3}, \texttt{part/3} and \texttt{tolr/3}.


\begin{verbatim}
: (?
   @Nm "Muller"
   @Tel "37"
   (select (@CuSu)
      ((nm +CuSu @Nm) (tel +CuSu @Tel))
      (tolr @Nm @CuSu nm)
      (head @Tel @CuSu tel) )
   (val @Name @CuSu nm)
   (val @Phone @CuSu tel) )
 @Nm="Muller" @Tel="37" @CuSu={2-3} @Name="Miller" @Phone="37 4773 82534"
-> NIL
\end{verbatim}

 
\section{(heap 'flg) -> cnt}
\label{sec-8-1-8-8}


Returns the total size of the cell heap space in megabytes. If \texttt{flg} is
non-=NIL=, the size of the currently free space is returned. See also
\texttt{stack} and \texttt{gc}.


\begin{verbatim}
: (gc 4)
-> 4
: (heap)
-> 5
: (heap T)
-> 4
\end{verbatim}

 
\section{(hear 'cnt) -> cnt}
\label{sec-8-1-8-9}


Uses the file descriptor \texttt{cnt} as an asynchronous command input channel.
Any executable list received via this channel will be executed in the
background. As this mechanism is also used for inter-family
communication (see \texttt{tell}), \texttt{hear} is usually only called explicitly by
a top level parent process.


\begin{verbatim}
: (call 'mkfifo "fifo/cmd")
-> T
: (hear (open "fifo/cmd"))
-> 3
\end{verbatim}

 
\section{(here ['sym]) -> sym}
\label{sec-8-1-8-10}


Echoes the current input stream until \texttt{sym} is encountered, or until end
of file. See also \texttt{echo}.


\begin{verbatim}
$ cat hello.l
(html 0 "Hello" "lib.css" NIL
   (<h2> NIL "Hello")
   (here) )
<p>Hello!</p>
<p>This is a test.</p>

$ pil @lib/http.l @lib/xhtml.l hello.l
HTTP/1.0 200 OK
Server: PicoLisp
Date: Sun, 03 Jun 2007 11:41:27 GMT
Cache-Control: max-age=0
Cache-Control: no-cache
Content-Type: text/html; charset=utf-8

<!DOCTYPE html PUBLIC "-//W3C//DTD XHTML 1.0 Strict//EN" "http://www.w3.org/TR/xhtml1/DTD/xhtml1-strict.dtd">
<html xmlns="http://www.w3.org/1999/xhtml" xml:lang="en" lang="en">
<head>
<title>Hello</title>
<link rel="stylesheet" href="http://:/lib.css" type="text/css"/>
</head>
<body><h2>Hello</h2>
<p>Hello!</p>
<p>This is a test.</p>
</body>
</html>
\end{verbatim}

 
\section{(hex 'num ['num]) -> sym}
\label{sec-8-1-8-11}


\texttt{(hex 'sym) -> num}

Converts a number \texttt{num} to a hexadecimal string, or a hexadecimal string
\texttt{sym} to a number. In the first case, if the second argument is given,
the result is separated by spaces into groups of such many digits. See
also \texttt{bin}, \texttt{oct}, \texttt{fmt64}, \texttt{hax} and \texttt{format}.


\begin{verbatim}
: (hex 273)
-> "111"
: (hex "111")
-> 273
: (hex 1234567 4)
-> "12 D687"
\end{verbatim}

 
\section{(host 'any) -> sym}
\label{sec-8-1-8-12}


Returns the hostname corresponding to the given IP address. See also
\texttt{*Adr}.


\begin{verbatim}
: (host "80.190.158.9")
-> "www.leo.org"
\end{verbatim}


\chapter{Functions starting with I}
\label{sec-8-1-9}


 
\section{+Idx}
\label{sec-8-1-9-1}


Prefix class for maintaining non-unique full-text indexes to \texttt{+String}
relations, a subclass of \texttt{+Ref}. Accepts optional arguments for the
minimally indexed substring length (defaults to 3), and a \texttt{+Hook}
attribute. Often used in combination with the \texttt{+Sn} soundex index, or
the \texttt{+Fold} index prefix classes. See also \texttt{Database}.


\begin{verbatim}
(rel nm (+Sn +Idx +String))  # Name
\end{verbatim}

 
\section{+index}
\label{sec-8-1-9-2}


Abstract base class of all database B-Tree index relations (prefix
classes for \texttt{+relation=s). The class hierarchy includes =+Key}, \texttt{+Ref}
and \texttt{+Idx}. See also \texttt{Database}.


\begin{verbatim}
(isa '+index Rel)  # Check for an index relation
\end{verbatim}

 
\section{(id 'num ['num]) -> sym}
\label{sec-8-1-9-3}


\texttt{(id 'sym [NIL]) -> num}

\texttt{(id 'sym T) -> (num . num)}

Converts one or two numbers to an external symbol, or an external symbol
to a number or a pair of numbers.


\begin{verbatim}
: (id 7)
-> {7}
: (id 1 2)
-> {2}
: (id '{1-2})
-> 2
: (id '{1-2} T)
-> (1 . 2)
\end{verbatim}

 
\section{(idx 'var 'any 'flg) -> lst (idx 'var 'any) -> lst (idx 'var) -> lst}
\label{sec-8-1-9-4}


Maintains an index tree in \texttt{var}, and checks for the existence of \texttt{any}.
If \texttt{any} is contained in \texttt{var}, the corresponding subtree is returned,
otherwise \texttt{NIL}. In the first form, \texttt{any} is destructively inserted into
the tree if \texttt{flg} is non-=NIL= (and \texttt{any} was not already there), or
deleted from the tree if \texttt{flg} is \texttt{NIL}. The second form only checks for
existence, but does not change the index tree. In the third form (when
called with a single \texttt{var} argument) the contents of the tree are
returned as a sorted list. If all elements are inserted in sorted order,
the tree degenerates into a linear list. See also \texttt{lup}, \texttt{hash},
\texttt{depth}, \texttt{sort}, \texttt{balance} and \texttt{member}.


\begin{verbatim}
: (idx 'X 'd T)                              # Insert data
-> NIL
: (idx 'X 2 T)
-> NIL
: (idx 'X '(a b c) T)
-> NIL
: (idx 'X 17 T)
-> NIL
: (idx 'X 'A T)
-> NIL
: (idx 'X 'd T)
-> (d (2 NIL 17 NIL A) (a b c))              # 'd' already existed
: (idx 'X T T)
-> NIL
: X                                          # View the index tree
-> (d (2 NIL 17 NIL A) (a b c) NIL T)
: (idx 'X 'A)                                # Check for 'A'
-> (A)
: (idx 'X 'B)                                # Check for 'B'
-> NIL
: (idx 'X)
-> (2 17 A d (a b c) T)                      # Get list
: (idx 'X 17 NIL)                            # Delete '17'
-> (17 NIL A)
: X
-> (d (2 NIL A) (a b c) NIL T)               # View it again
: (idx 'X)
-> (2 A d (a b c) T)                         # '17' is deleted
\end{verbatim}

 
\section{(if 'any1 'any2 . prg) -> any}
\label{sec-8-1-9-5}


Conditional execution: If the condition \texttt{any1} evaluates to non-=NIL=,
\texttt{any2} is evaluated and returned. Otherwise, \texttt{prg} is executed and the
result returned. See also \texttt{cond}, \texttt{when} and \texttt{if2}.


\begin{verbatim}
: (if (> 4 3) (println 'OK) (println 'Bad))
OK
-> OK
: (if (> 3 4) (println 'OK) (println 'Bad))
Bad
-> Bad
\end{verbatim}

 
\section{(if2 'any1 'any2 'any3 'any4 'any5 . prg) -> any}
\label{sec-8-1-9-6}


Four-way conditional execution for two conditions: If both conditions
\texttt{any1} and \texttt{any2} evaluate to non-=NIL=, \texttt{any3} is evaluated and
returned. Otherwise, \texttt{any4} or \texttt{any5} is evaluated and returned if
\texttt{any1} or \texttt{any2} evaluate to non-=NIL=, respectively. If none of the
conditions evaluate to non-=NIL=, \texttt{prg} is executed and the result
returned. See also \texttt{if} and \texttt{cond}.


\begin{verbatim}
: (if2 T T 'both 'first 'second 'none)
-> both
: (if2 T NIL 'both 'first 'second 'none)
-> first
: (if2 NIL T 'both 'first 'second 'none)
-> second
: (if2 NIL NIL 'both 'first 'second 'none)
-> none
\end{verbatim}

 
\section{(ifn 'any1 'any2 . prg) -> any}
\label{sec-8-1-9-7}


Conditional execution (``If not''): If the condition \texttt{any1} evaluates to
\texttt{NIL}, \texttt{any2} is evaluated and returned. Otherwise, \texttt{prg} is executed
and the result returned.


\begin{verbatim}
: (ifn (= 3 4) (println 'OK) (println 'Bad))
OK
-> OK
\end{verbatim}

 
\section{(import lst) -> NIL}
\label{sec-8-1-9-8}


Wrapper function for \texttt{intern}. Typically used to import symbols from
other namespaces, as created by \texttt{symbols}. \texttt{lst} should be a list of
symbols. An import conflict error is issued when a symbol with the same
name already exists in the current namespace. See also \texttt{pico} and
\texttt{local}.


\begin{verbatim}
: (import libA~foo libB~bar)
-> NIL
\end{verbatim}

 
\section{(in 'any . prg) -> any}
\label{sec-8-1-9-9}


Opens \texttt{any} as input channel during the execution of \texttt{prg}. The current
input channel will be saved and restored appropriately. If the argument
is \texttt{NIL}, standard input is used. If the argument is a symbol, it is
used as a file name (opened for reading \emph{and} writing if the first
character is ``=+=''). If it is a positive number, it is used as the
descriptor of an open file. If it is a negative number, the saved input
channel such many levels above the current one is used. Otherwise (if it
is a list), it is taken as a command with arguments, and a pipe is
opened for input. See also \texttt{ipid}, \texttt{call}, \texttt{load}, \texttt{file}, \texttt{out}, \texttt{err},
\texttt{poll}, \texttt{pipe} and \texttt{ctl}.


\begin{verbatim}
: (in "a" (list (read) (read) (read)))  # Read three items from file "a"
-> (123 (a b c) def)
\end{verbatim}

 
\section{(inc 'num) -> num (inc 'var ['num]) -> num}
\label{sec-8-1-9-10}


The first form returns the value of \texttt{num} incremented by 1. The second
form increments the \texttt{VAL} of \texttt{var} by 1, or by \texttt{num}. If the first
argument is \texttt{NIL}, it is returned immediately. \texttt{(inc 'num)} is
equivalent to \texttt{(+ 'num 1)} and \texttt{(inc 'var)} is equivalent to
\texttt{(set 'var (+ var 1))}. See also \texttt{dec} and \texttt{+}.


\begin{verbatim}
: (inc 7)
-> 8
: (inc -1)
-> 0
: (zero N)
-> 0
: (inc 'N)
-> 1
: (inc 'N 7)
-> 8
: N
-> 8

: (setq L (1 2 3 4))
-> (1 2 3 4)
: (inc (cdr L))
-> 3
: L
-> (1 3 3 4)
\end{verbatim}

 
\section{(inc! 'obj 'sym ['num]) -> num}
\label{sec-8-1-9-11}


\hyperref[ref.html-trans]{Transaction} wrapper function for \texttt{inc}. \texttt{num}
defaults to 1. Note that for incrementing a property value of an entity
typically the \texttt{inc!>} message is used. See also \texttt{new!}, \texttt{set!} and
\texttt{put!}.


\begin{verbatim}
(inc! Obj 'cnt 0)  # Incrementing a property of a non-entity object
\end{verbatim}

 
\section{(index 'any 'lst) -> cnt | NIL}
\label{sec-8-1-9-12}


Returns the \texttt{cnt} position of \texttt{any} in \texttt{lst}, or \texttt{NIL} if it is not
found. See also \texttt{offset}.


\begin{verbatim}
: (index 'c '(a b c d e f))
-> 3
: (index '(5 6) '((1 2) (3 4) (5 6) (7 8)))
-> 3
\end{verbatim}

 
\section{(info 'any) -> (cnt|T dat . tim)}
\label{sec-8-1-9-13}


Returns information about a file with the name \texttt{any}: The current size
\texttt{cnt} in bytes, and the modification date and time (UTC). For
directories, \texttt{T} is returned instead of the a size. See also \texttt{dir},
\texttt{date}, \texttt{time} and \texttt{lines}.


\begin{verbatim}
$ ls -l x.l
-rw-r--r--   1 abu      users         208 Jun 17 08:58 x.l
$ pil +
: (info "x.l")
-> (208 730594 . 32315)
: (stamp 730594 32315)
-> "2000-06-17 08:58:35"
\end{verbatim}

 
\section{(init 'tree ['any1] ['any2]) -> lst}
\label{sec-8-1-9-14}


Initializes a structure for stepping iteratively through a database
tree. \texttt{any1} and \texttt{any2} may specify a range of keys. If \texttt{any2} is
greater than \texttt{any1}, the traversal will be in opposite direction. See
also \texttt{tree}, \texttt{step}, \texttt{iter} and \texttt{scan}.


\begin{verbatim}
: (init (tree 'nr '+Item) 3 5)
-> (((3 . 5) ((3 NIL . {3-3}) (4 NIL . {3-4}) (5 NIL . {3-5}) (6 NIL . {3-6}) (7 NIL . {3-8}))))
\end{verbatim}

 
\section{(insert 'cnt 'lst 'any) -> lst}
\label{sec-8-1-9-15}


Inserts \texttt{any} into \texttt{lst} at position \texttt{cnt}. This is a non-destructive
operation. See also \texttt{remove}, \texttt{place}, \texttt{append}, \texttt{delete} and \texttt{replace}.


\begin{verbatim}
: (insert 3 '(a b c d e) 777)
-> (a b 777 c d e)
: (insert 1 '(a b c d e) 777)
-> (777 a b c d e)
: (insert 9 '(a b c d e) 777)
-> (a b c d e 777)
\end{verbatim}

 
\section{(intern 'sym) -> sym}
\label{sec-8-1-9-16}


Creates or finds an internal symbol. If a symbol with the name \texttt{sym} is
already intern, it is returned. Otherwise, \texttt{sym} is interned and
returned. See also \texttt{symbols}, \texttt{zap}, \texttt{extern} and ======.


\begin{verbatim}
: (intern "abc")
-> abc
: (intern 'car)
-> car
: ((intern (pack "c" "a" "r")) (1 2 3))
-> 1
\end{verbatim}

 
\section{(ipid) -> pid | NIL}
\label{sec-8-1-9-17}


Returns the corresponding process ID when the current input channel is
reading from a pipe, otherwise \texttt{NIL}. See also \texttt{opid}, \texttt{in}, \texttt{pipe} and
\texttt{load}.


\begin{verbatim}
: (in '(ls "-l") (println (line T)) (kill (ipid)))
"total 7364"
-> T
\end{verbatim}

 
\section{(isa 'cls|typ 'obj) -> obj | NIL}
\label{sec-8-1-9-18}


Returns \texttt{obj} when it is an object that inherits from \texttt{cls} or \texttt{type}.
See also \texttt{OO Concepts}, \texttt{class}, \texttt{type}, \texttt{new} and \texttt{object}.


\begin{verbatim}
: (isa '+Address Obj)
-> {1-17}
: (isa '(+Male +Person) Obj)
-> NIL
\end{verbatim}

 
\section{isa/2}
\label{sec-8-1-9-19}


\hyperref[ref.html-pilog]{Pilog} predicate that succeeds if the second argument
is of the type or class given by the first argument, according to the
\texttt{isa} function. Typically used in \texttt{db/3} or \texttt{select/3} database queries.
See also \texttt{same/3}, \texttt{bool/3}, \texttt{range/3}, \texttt{head/3}, \texttt{fold/3}, \texttt{part/3} and
\texttt{tolr/3}.


\begin{verbatim}
: (? (db nm +Person @Prs) (isa +Woman @Prs) (val @Nm @Prs nm))
 @Prs={2-Y} @Nm="Alexandra of Denmark"
 @Prs={2-1I} @Nm="Alice Maud Mary"
 @Prs={2-F} @Nm="Anne"
 @Prs={2-j} @Nm="Augusta Victoria".   # Stop
\end{verbatim}

 
\section{(iter 'tree ['fun] ['any1] ['any2] ['flg])}
\label{sec-8-1-9-20}


Iterates through a database tree by applying \texttt{fun} to all values. \texttt{fun}
defaults to \texttt{println}. \texttt{any1} and \texttt{any2} may specify a range of keys. If
\texttt{any2} is greater than \texttt{any1}, the traversal will be in opposite
direction. Note that the keys need not to be atomic, depending on the
application's index structure. If \texttt{flg} is non-=NIL=, partial keys are
skipped. See also \texttt{tree}, \texttt{scan}, \texttt{init} and \texttt{step}.


\begin{verbatim}
: (iter (tree 'nr '+Item))
{3-1}
{3-2}
{3-3}
{3-4}
{3-5}
{3-6}
{3-8}
-> {7-1}
: (iter (tree 'nr '+Item) '((This) (println (: nm))))
"Main Part"
"Spare Part"
"Auxiliary Construction"
"Enhancement Additive"
"Metal Fittings"
"Gadget Appliance"
"Testartikel"
-> {7-1}
\end{verbatim}


\chapter{Functions starting with J}
\label{sec-8-1-10}


 
\section{+Joint}
\label{sec-8-1-10-1}


Class for bidirectional object relations, a subclass of \texttt{+Link}. Expects
a (symbolic) attribute, and list of classes as \texttt{type} of the referred
database object (of class \texttt{+Entity}). A \texttt{+Joint} corresponds to two
\texttt{+Link=s, where the attribute argument is the relation of the back-link in the referred object. See also =Database}.


\begin{verbatim}
(class +Ord +Entity)                   # Order class
(rel pos (+List +Joint) ord (+Pos))    # List of positions in that order
...
(class +Pos +Entity)    # Position class
(rel ord (+Joint)       # Back-link to the parent order
\end{verbatim}

 
\section{(job 'lst . prg) -> any}
\label{sec-8-1-10-2}


Executes a job within its own environment (as specified by symbol-value
pairs in \texttt{lst}). The current values of all symbols are saved, the
symbols are bound to the values in \texttt{lst}, \texttt{prg} is executed, then the
(possibly modified) symbol values are (destructively) stored in the
environment list, and the symbols are restored to their original values.
The return value is the result of \texttt{prg}. Typically used in \texttt{curried}
functions and \texttt{*Run} tasks. See also \texttt{env}, \texttt{bind}, \texttt{let}, \texttt{use} and
\texttt{state}.


\begin{verbatim}
: (de tst ()
   (job '((A . 0) (B . 0))
      (println (inc 'A) (inc 'B 2)) ) )
-> tst
: (tst)
1 2
-> 2
: (tst)
2 4
-> 4
: (tst)
3 6
-> 6
: (pp 'tst)
(de tst NIL
   (job '((A . 3) (B . 6))
      (println (inc 'A) (inc 'B 2)) ) )
-> tst
\end{verbatim}

 
\section{(journal 'any ..) -> T}
\label{sec-8-1-10-3}


Reads journal data from the files with the names \texttt{any}, and writes all
changes to the database. See also \texttt{pool}.


\begin{verbatim}
: (journal "db.log")
-> T
\end{verbatim}


\chapter{Functions starting with K}
\label{sec-8-1-11}


 
\section{+Key}
\label{sec-8-1-11-1}


Prefix class for maintaining unique indexes to \texttt{+relation=s, a subclass of =+index}. Accepts an optional argument for a \texttt{+Hook} attribute. See
also \texttt{Database}.


\begin{verbatim}
(rel nr (+Need +Key +Number))  # Mandatory, unique Customer/Supplier number
\end{verbatim}

 
\section{(key ['cnt]) -> sym}
\label{sec-8-1-11-2}


Returns the next character from standard input as a single-character
transient symbol. The console is set to raw mode. While waiting for a
key press, a \texttt{select} system call is executed for all file descriptors
and timers in the \texttt{VAL} of the global variable \texttt{*Run}. If \texttt{cnt} is
non-=NIL=, that amount of milliseconds is waited maximally, and \texttt{NIL} is
returned upon timeout. See also \texttt{raw} and \texttt{wait}.


\begin{verbatim}
: (key)           # Wait for a key
-> "a"            # 'a' pressed
\end{verbatim}

 
\section{(kill 'pid ['cnt]) -> flg}
\label{sec-8-1-11-3}


Sends a signal with the signal number \texttt{cnt} (or SIGTERM if \texttt{cnt} is not
given) to the process with the ID \texttt{pid}. Returns \texttt{T} if successful.


\begin{verbatim}
: (kill *Pid 20)                                # Stop current process

[2]+  Stopped               pil +               # Unix shell
$ fg                                            # Job control: Foreground
pil +
-> T                                            # 'kill' was successful
\end{verbatim}


\chapter{Functions starting with L}
\label{sec-8-1-12}


 
\section{*Led}
\label{sec-8-1-12-1}


A global variable holding a (possibly empty) \texttt{prg} body that implements
a ``Line editor''. When non-=NIL=, it should return a single symbol
(string) upon execution.


\begin{verbatim}
: (de *Led "(bye)")
# *Led redefined
-> *Led
: $                                    # Exit
\end{verbatim}

 
\section{+Link}
\label{sec-8-1-12-2}


Class for object relations, a subclass of \texttt{+relation}. Expects a list of
classes as \texttt{type} of the referred database object (of class \texttt{+Entity}).
See also \texttt{Database}.


\begin{verbatim}
(rel sup (+Ref +Link) NIL (+CuSu))  # Supplier (class Customer/Supplier)
\end{verbatim}

 
\section{+List}
\label{sec-8-1-12-3}


Prefix class for a list of identical relations. Objects of that class
maintain a list of Lisp data of uniform type. See also \texttt{Database}.


\begin{verbatim}
(rel pos (+List +Joint) ord (+Pos))  # Positions
(rel nm (+List +Fold +Ref +String))  # List of folded and indexed names
(rel val (+Ref +List +Number))       # Indexed list of numeric values
\end{verbatim}

 
\section{(last 'lst) -> any}
\label{sec-8-1-12-4}


Returns the last element of \texttt{lst}. See also \texttt{fin} and \texttt{tail}.


\begin{verbatim}
: (last (1 2 3 4))
-> 4
: (last '((a b) c (d e f)))
-> (d e f)
\end{verbatim}

 
\section{(later 'var . prg) -> var}
\label{sec-8-1-12-5}


Executes \texttt{prg} in a \texttt{pipe}'ed child process. The return value of \texttt{prg}
will later be available in \texttt{var}.


\begin{verbatim}
: (prog1  # Parallel background calculation of square numbers
   (mapcan '((N) (later (cons) (* N N))) (1 2 3 4))
   (wait NIL (full @)) )
-> (1 4 9 16)
\end{verbatim}

 
\section{(ld) -> any}
\label{sec-8-1-12-6}


\texttt{load=s the last file edited with =vi}.


\begin{verbatim}
: (vi 'main)
-> T
: (ld)
# main redefined
-> go
\end{verbatim}

 
\section{(le0 'any) -> num | NIL}
\label{sec-8-1-12-7}


Returns \texttt{num} when the argument is a number less or equal zero,
otherwise \texttt{NIL}. See also \texttt{lt0}, \texttt{ge0}, \texttt{gt0}, ==0= and \texttt{n0}.


\begin{verbatim}
: (le0 -2)
-> -2
: (le0 0)
-> 0
: (le0 3)
-> NIL
\end{verbatim}

 
\section{(leaf 'tree) -> any}
\label{sec-8-1-12-8}


Returns the first leaf (i.e. the value of the smallest key) in a
database tree. See also \texttt{tree}, \texttt{minKey}, \texttt{maxKey} and \texttt{step}.


\begin{verbatim}
: (leaf (tree 'nr '+Item))
-> {3-1}
: (db 'nr '+Item (minKey (tree 'nr '+Item)))
-> {3-1}
\end{verbatim}

 
\section{(length 'any) -> cnt | T}
\label{sec-8-1-12-9}


Returns the ``length'' of \texttt{any}. For numbers this is the number of decimal
digits in the value (plus 1 for negative values), for symbols it is the
number of characters in the name, and for lists it is the number of
elements (or \texttt{T} for circular lists). See also \texttt{size}.


\begin{verbatim}
: (length "abc")
-> 3
: (length "äbc")
-> 3
: (length 123)
-> 3
: (length (1 (2) 3))
-> 3
: (length (1 2 3 .))
-> T
\end{verbatim}

 
\section{(let sym 'any . prg) -> any}
\label{sec-8-1-12-10}


\texttt{(let (sym 'any ..) . prg) -> any}

Defines local variables. The value of the symbol \texttt{sym} - or the values
of the symbols \texttt{sym} in the list of the second form - are saved and the
symbols are bound to the evaluated \texttt{any} arguments. \texttt{prg} is executed,
then the symbols are restored to their original values. The result of
\texttt{prg} is returned. It is an error condition to pass \texttt{NIL} as a \texttt{sym}
argument. See also \texttt{let?}, \texttt{bind}, \texttt{recur}, \texttt{with}, \texttt{for}, \texttt{job} and
\texttt{use}.


\begin{verbatim}
: (setq  X 123  Y 456)
-> 456
: (let X "Hello" (println X))
"Hello"
-> "Hello"
: (let (X "Hello" Y "world") (prinl X " " Y))
Hello world
-> "world"
: X
-> 123
: Y
-> 456
\end{verbatim}

 
\section{(let? sym 'any . prg) -> any}
\label{sec-8-1-12-11}


Conditional local variable binding and execution: If \texttt{any} evalutes to
\texttt{NIL}, \texttt{NIL} is returned. Otherwise, the value of the symbol \texttt{sym} is
saved and \texttt{sym} is bound to the evaluated \texttt{any} argument. \texttt{prg} is
executed, then \texttt{sym} is restored to its original value. The result of
\texttt{prg} is returned. It is an error condition to pass \texttt{NIL} as the \texttt{sym}
argument. \texttt{(let? sym 'any ..)} is equivalent to
\texttt{(when 'any (let sym @ ..))}. See also \texttt{let}, \texttt{bind}, \texttt{job} and \texttt{use}.


\begin{verbatim}
: (setq Lst (1 NIL 2 NIL 3))
-> (1 NIL 2 NIL 3)
: (let? A (pop 'Lst) (println 'A A))
A 1
-> 1
: (let? A (pop 'Lst) (println 'A A))
-> NIL
\end{verbatim}

 
\section{(lieu 'any) -> sym | NIL}
\label{sec-8-1-12-12}


Returns the argument \texttt{any} when it is an external symbol and currently
manifest in heap space, otherwise \texttt{NIL}. See also \texttt{ext?}.


\begin{verbatim}
: (lieu *DB)
-> {1}
\end{verbatim}

 
\section{(line 'flg ['cnt ..]) -> lst|sym}
\label{sec-8-1-12-13}


Reads a line of characters from the current input channel. End of line
is recognized as linefeed (hex ``0A''), carriage return (hex ``0D''), or the
combination of both. (Note that a single carriage return may not work on
network connections, because the character look-ahead to distinguish
from return+linefeed can block the connection.) If \texttt{flg} is \texttt{NIL}, a
list of single-character transient symbols is returned. When \texttt{cnt}
arguments are given, subsequent characters of the input line are grouped
into sublists, to allow parsing of fixed field length records. If \texttt{flg}
is non-=NIL=, strings are returned instead of single-character lists.
\texttt{NIL} is returned upon end of file. See also \texttt{char}, \texttt{till} and \texttt{eof}.


\begin{verbatim}
: (line)
abcdefghijkl
-> ("a" "b" "c" "d" "e" "f" "g" "h" "i" "j" "k" "l")
: (line T)
abcdefghijkl
-> "abcdefghijkl"
: (line NIL 1 2 3)
abcdefghijkl
-> (("a") ("b" "c") ("d" "e" "f") "g" "h" "i" "j" "k" "l")
: (line T 1 2 3)
abcdefghijkl
-> ("a" "bc" "def" "g" "h" "i" "j" "k" "l")
\end{verbatim}

 
\section{(lines 'any ..) -> cnt}
\label{sec-8-1-12-14}


Returns the sum of the number of lines in the files with the names
\texttt{any}, or \texttt{NIL} if none was found. See also \texttt{info}.


\begin{verbatim}
: (lines "x.l")
-> 11
\end{verbatim}

 
\section{(link 'any ..) -> any}
\label{sec-8-1-12-15}


Links one or several new elements \texttt{any} to the end of the list in the
current \texttt{make} environment. This operation is efficient also for long
lists, because a pointer to the last element of the list is maintained.
\texttt{link} returns the last linked argument. See also \texttt{yoke}, \texttt{chain} and
\texttt{made}.


\begin{verbatim}
: (make
   (println (link 1))
   (println (link 2 3)) )
1
3
-> (1 2 3)
\end{verbatim}

 
\section{(lint 'sym) -> lst}
\label{sec-8-1-12-16}


\texttt{(lint 'sym 'cls) -> lst}

\texttt{(lint '(sym . cls)) -> lst}

Checks the function definition or file contents (in the first form), or
the method body of sym (second and third form), for possible pitfalls.
Returns an association list of diagnoses, where \texttt{var} indicates improper
variables, \texttt{dup} duplicate parameters, \texttt{def} an undefined function,
\texttt{bnd} an unbound variable, and \texttt{use} unused variables. See also
\texttt{noLint}, \texttt{lintAll}, \texttt{debug}, \texttt{trace} and \texttt{*Dbg}.


\begin{verbatim}
: (de foo (R S T R)     # 'T' is a improper parameter, 'R' is duplicated
   (let N 7             # 'N' is unused
      (bar X Y) ) )     # 'bar' is undefined, 'X' and 'Y' are not bound
-> foo
: (lint 'foo)
-> ((var T) (dup R) (def bar) (bnd Y X) (use N))
\end{verbatim}

 
\section{(lintAll ['sym ..]) -> lst}
\label{sec-8-1-12-17}


Applies \texttt{lint} to \texttt{all} internal symbols - and optionally to all files
\texttt{sym} - and returns a list of diagnoses. See also \texttt{noLint}.


\begin{verbatim}
: (more (lintAll "file1.l" "file2.l"))
...
\end{verbatim}

 
\section{(lisp 'sym ['fun]) -> num}
\label{sec-8-1-12-18}


(64-bit version only) Installs under the tag \texttt{sym} a callback function
\texttt{fun}, and returns a pointer \texttt{num} suitable to be passed to a C function
via `native'. If \texttt{fun} is \texttt{NIL}, the corresponding entry is freed.
Maximally 24 callback functions can be installed that way. `fun' should
be a function of maximally five numbers, and should return a number.
``Numbers'' in this context are 64-bit scalars, and may not only represent
integers, but also pointers or other encoded data. See also \texttt{native}.


\begin{verbatim}
(load "lib/native.l")

(gcc "ltest" NIL
   (cbTest (Fun) cbTest 'N Fun) )

long cbTest(int(*fun)(int,int,int,int,int)) {
   return fun(1,2,3,4,5);
}
/**/

: (cbTest
   (lisp 'cbTest
      '((A B C D E)
         (msg (list A B C D E))
         (* A B C D E) ) ) )
(1 2 3 4 5)
-> 120
\end{verbatim}

 
\section{(list 'any ['any ..]) -> lst}
\label{sec-8-1-12-19}


Returns a list of all \texttt{any} arguments. See also \texttt{cons}.


\begin{verbatim}
: (list 1 2 3 4)
-> (1 2 3 4)
: (list 'a (2 3) "OK")
-> (a (2 3) "OK")
\end{verbatim}

 
\section{lst/3}
\label{sec-8-1-12-20}


\hyperref[ref.html-pilog]{Pilog} predicate that returns subsequent list
elements, after applying the \texttt{get} algorithm to that object and the
following arguments. Often used in database queries. See also \texttt{map/3}.


\begin{verbatim}
: (? (db nr +Ord 1 @Ord) (lst @Pos @Ord pos))
 @Ord={3-7} @Pos={4-1}
 @Ord={3-7} @Pos={4-2}
 @Ord={3-7} @Pos={4-3}
-> NIL
\end{verbatim}

 
\section{(lst? 'any) -> flg}
\label{sec-8-1-12-21}


Returns \texttt{T} when the argument \texttt{any} is a (possibly empty) list (\texttt{NIL} or
a cons pair cell). See also \texttt{pair}.


\begin{verbatim}
: (lst? NIL)
-> T
: (lst? (1 . 2))
-> T
: (lst? (1 2 3))
-> T
\end{verbatim}

 
\section{(listen 'cnt1 ['cnt2]) -> cnt | NIL}
\label{sec-8-1-12-22}


Listens at a socket descriptor \texttt{cnt1} (as received by \texttt{port}) for an
incoming connection, and returns the new socket descriptor \texttt{cnt}. While
waiting for a connection, a \texttt{select} system call is executed for all
file descriptors and timers in the \texttt{VAL} of the global variable \texttt{*Run}.
If \texttt{cnt2} is non-=NIL=, that amount of milliseconds is waited maximally,
and \texttt{NIL} is returned upon timeout. The global variable \texttt{*Adr} is set to
the IP address of the client. See also \texttt{accept}, \texttt{connect}, \texttt{*Adr}.


\begin{verbatim}
: (setq *Socket
   (listen (port 6789) 60000) )  # Listen at port 6789 for max 60 seconds
-> 4
: *Adr
-> "127.0.0.1"
\end{verbatim}

 
\section{(lit 'any) -> any}
\label{sec-8-1-12-23}


Returns the literal (i.e. quoted) value of \texttt{any}, by \texttt{cons=ing it with the =quote} function if necessary.


\begin{verbatim}
: (lit T)
-> T
: (lit 1)
-> 1
: (lit '(1))
-> (1)
: (lit '(a))
-> '(a)
\end{verbatim}

 
\section{(load 'any ..) -> any}
\label{sec-8-1-12-24}


Loads all \texttt{any} arguments. Normally, the name of each argument is taken
as a file to be executed in a read-eval loop. The argument semantics are
identical to that of \texttt{in}, with the exception that if an argument is a
symbol and its first character is a hyphen `-', then that argument is
parsed as an executable list (without the surrounding parentheses). When
\texttt{any} is \texttt{T}, all remaining command line arguments are \texttt{load=ed recursively. When =any} is \texttt{NIL}, standard input is read, a prompt is
issued before each read operation, the results are printed to standard
output (read-eval-print loop), and \texttt{load} terminates when an empty line
is entered. In any case, \texttt{load} terminates upon end of file, or when
\texttt{NIL} is read. The index for transient symbols is cleared before and
after the load, so that all transient symbols in a file have a local
scope. If the namespace was switched (with \texttt{symbols}) while executing a
file, it is restored to the previous one. Returns the value of the last
evaluated expression. See also \texttt{script}, \texttt{ipid}, \texttt{call}, \texttt{file}, \texttt{in},
\texttt{out} and \texttt{str}.


\begin{verbatim}
: (load "lib.l" "-* 1 2 3")
-> 6
\end{verbatim}

 
\section{(loc 'sym 'lst) -> sym}
\label{sec-8-1-12-25}


Locates in \texttt{lst} a \texttt{transient} symbol with the same name as \texttt{sym}.
Allows to get hold of otherwise inaccessible symbols. See also ======.


\begin{verbatim}
: (loc "X" curry)
-> "X"
: (== @ "X")
-> NIL
\end{verbatim}

 
\section{(local lst) -> sym}
\label{sec-8-1-12-26}


Wrapper function for \texttt{zap}. Typically used to create namespace-local
symbols. \texttt{lst} should be a list of symbols. See also \texttt{pico}, \texttt{symbols},
\texttt{import} and \texttt{intern}.


\begin{verbatim}
(symbols 'myLib 'pico)

(local bar foo)
(de foo (A)  # 'foo' is local to 'myLib'
   ...
(de bar (B)  # 'bar' is local to 'myLib'
   ...
\end{verbatim}

 
\section{(locale 'sym1 'sym2 ['sym ..])}
\label{sec-8-1-12-27}


Sets the current locale to that given by the country file \texttt{sym1} and the
language file \texttt{sym2} (both located in the ``loc/'' directory), and
optional application-specific directories \texttt{sym}. The locale influences
the language, and numerical, date and other formats. See also \texttt{*Uni},
\texttt{datStr}, \texttt{strDat}, \texttt{expDat}, \texttt{day}, \texttt{telStr}, \texttt{expTel} and and \texttt{money}.


\begin{verbatim}
: (locale "DE" "de" "app/loc/")
-> "Zip"
: ,"Yes"
-> "Ja"
\end{verbatim}

 
\section{(lock ['sym]) -> cnt | NIL}
\label{sec-8-1-12-28}


Write-locks an external symbol \texttt{sym} (file record locking), or the whole
database root file if \texttt{sym} is \texttt{NIL}. Returns \texttt{NIL} if successful, or
the ID of the process currently holding the lock. When \texttt{sym} is
non-=NIL=, the lock is released at the next top level call to \texttt{commit}
or \texttt{rollback}, otherwise only when another database is opened with
\texttt{pool}, or when the process terminates. See also \texttt{*Solo}.


\begin{verbatim}
: (lock '{1})        # Lock single object
-> NIL
: (lock)             # Lock whole database
-> NIL
\end{verbatim}

 
\section{(loop ['any | (NIL 'any . prg) | (T 'any . prg) ..]) -> any}
\label{sec-8-1-12-29}


Endless loop with multiple conditional exits: The body is executed an
unlimited number of times. If a clause has \texttt{NIL} or \texttt{T} as its CAR, the
clause's second element is evaluated as a condition and - if the result
is \texttt{NIL} or non-=NIL=, respectively - the \texttt{prg} is executed and the
result returned. See also \texttt{do} and \texttt{for}.


\begin{verbatim}
: (let N 3
   (loop
      (prinl N)
      (T (=0 (dec 'N)) 'done) ) )
3
2
1
-> done
\end{verbatim}

 
\section{(low? 'any) -> sym | NIL}
\label{sec-8-1-12-30}


Returns \texttt{any} when the argument is a string (symbol) that starts with a
lowercase character. See also \texttt{lowc} and \texttt{upp?}


\begin{verbatim}
: (low? "a")
-> "a"
: (low? "A")
-> NIL
: (low? 123)
-> NIL
: (low? ".")
-> NIL
\end{verbatim}

 
\section{(lowc 'any) -> any}
\label{sec-8-1-12-31}


Lower case conversion: If \texttt{any} is not a symbol, it is returned as it
is. Otherwise, a new transient symbol with all characters of \texttt{any},
converted to lower case, is returned. See also \texttt{uppc}, \texttt{fold} and
\texttt{low?}.


\begin{verbatim}
: (lowc 123)
-> 123
: (lowc "ABC")
-> "abc"
\end{verbatim}

 
\section{(lt0 'any) -> num | NIL}
\label{sec-8-1-12-32}


Returns \texttt{num} when the argument is a number and less than zero,
otherwise \texttt{NIL}. See also \texttt{le0}, \texttt{ge0}, \texttt{gt0}, ==0= and \texttt{n0}.


\begin{verbatim}
: (lt0 -2)
-> -2
: (lt0 3)
-> NIL
\end{verbatim}

 
\section{(lup 'lst 'any) -> lst}
\label{sec-8-1-12-33}


\texttt{(lup 'lst 'any 'any2) -> lst}

Looks up \texttt{any} in the CAR-elements of cells stored in the index tree
\texttt{lst}, as built-up by \texttt{idx}. In the first form, the first found cell is
returned, in the second form a list of all cells whose CAR is in the
range \texttt{any} .. \texttt{any2}. See also \texttt{assoc}.


\begin{verbatim}
: (idx 'A 'a T)
-> NIL
: (idx 'A (1 . b) T)
-> NIL
: (idx 'A 123 T)
-> NIL
: (idx 'A (1 . a) T)
-> NIL
: (idx 'A (1 . c) T)
-> NIL
: (idx 'A (2 . d) T)
-> NIL
: (idx 'A)
-> (123 a (1 . a) (1 . b) (1 . c) (2 . d))
: (lup A 1)
-> (1 . b)
: (lup A 2)
-> (2 . d)
: (lup A 1 1)
-> ((1 . a) (1 . b) (1 . c))
: (lup A 1 2)
-> ((1 . a) (1 . b) (1 . c) (2 . d))
\end{verbatim}


\chapter{Functions starting with M}
\label{sec-8-1-13}


 
\section{*Msg}
\label{sec-8-1-13-1}


A global variable holding the last recently issued error message. See
also \texttt{Error Handling}, \texttt{*Err} and \texttt{\textasciicircum{}.


\begin{verbatim}
: (+ 'A 2)
!? (+ 'A 2)
A -- Number expected
?
:
: *Msg
-> "Number expected"
\end{verbatim}

 
\section{+Mis}
\label{sec-8-1-13-2}


Prefix class to explicitly specify validation functions for
\texttt{+relation=s. Expects a function that takes a value and an entity object, and returns =NIL} if everything is correct, or an error string.
See also \texttt{Database}.


\begin{verbatim}
(class +Ord +Entity)            # Order class
(rel pos (+Mis +List +Joint)    # List of positions in that order
   ((Val Obj)
      (when (memq NIL Val)
         "There are empty positions" ) )
   ord (+Pos) )
\end{verbatim}

 
\section{(macro prg) -> any}
\label{sec-8-1-13-3}


Substitues all \texttt{pat?} symbols in \texttt{prg} (using \texttt{fill}), and executes the
result with \texttt{run}. Used occasionally to call functions which otherwise
do not evaluate their arguments.


\begin{verbatim}
: (de timerMessage (@N . @Prg)
   (setq @N (- @N))
   (macro
      (task @N 0 . @Prg) ) )
-> timerMessage
: (timerMessage 6000 (println 'Timer 6000))
-> (-6000 0 (println 'Timer 6000))
: (timerMessage 12000 (println 'Timer 12000))
-> (-12000 0 (println 'Timer 12000))
: (more *Run)
(-12000 2616 (println 'Timer 12000))
(-6000 2100 (println 'Timer 6000))
-> NIL
: Timer 6000
Timer 12000
...
\end{verbatim}

 
\section{(made ['lst1 ['lst2]]) -> lst}
\label{sec-8-1-13-4}


Initializes a new list value for the current \texttt{make} environment. All
list elements already produced with \texttt{chain} and \texttt{link} are discarded,
and \texttt{lst1} is used instead. Optionally, \texttt{lst2} can be specified as the
new linkage cell, otherwise the last cell of \texttt{lst1} is used. When called
without arguments, \texttt{made} does not modify the environment. In any case,
the current list is returned.


\begin{verbatim}
: (make
   (link 'a 'b 'c)         # Link three items
   (println (made))        # Print current list (a b c)
   (made (1 2 3))          # Discard it, start new with (1 2 3)
   (link 4) )              # Link 4
(a b c)
-> (1 2 3 4)
\end{verbatim}

 
\paragraph{=(mail `any `cnt `sym1 `sym2|lst1 `sym3 `lst2 . prg)'=}
\label{sec-8-1-13-5}


Sends an eMail via SMTP to a mail server at host \texttt{any}, port \texttt{cnt}.
\texttt{sym1} should be the ``from'' address, \texttt{sym2|lst1} the ``to'' address(es),
and \texttt{sym3} the subject. \texttt{lst2} is a list of attachments, each one
specified by three elements for path, name and mime type. \texttt{prg}
generates the mail body with \texttt{prEval}. See also \texttt{connect}.


\begin{verbatim}
(mail "localhost" 25                               # Local mail server
   "a@bc.de"                                       # "From" address
   "abu@software-lab.de"                           # "To" address
   "Testmail"                                      # Subject
   (quote
      "img/go.png" "go.png" "image/png"            # First attachment
      "img/7fach.gif" "7fach.gif" "image/gif" )    # Second attachment
   "Hello,"                                        # First line
   NIL                                             # (empty line)
   (prinl (pack "This is mail #" (+ 3 4))) )       # Third line
\end{verbatim}

 
\section{(make .. [(made 'lst ..)] .. [(link 'any ..)] ..) -> any}
\label{sec-8-1-13-6}


Initializes and executes a list-building process with the \texttt{made},
\texttt{chain}, \texttt{link} and \texttt{yoke} functions, and returns the result list. For
efficiency, pointers to the head and the tail of the list are maintained
internally.


\begin{verbatim}
: (make (link 1) (link 2 3) (link 4))
-> (1 2 3 4)
: (make (made (1 2 3)) (link 4))
-> (1 2 3 4)
\end{verbatim}

 
\section{(map 'fun 'lst ..) -> lst}
\label{sec-8-1-13-7}


Applies \texttt{fun} to \texttt{lst} and all successive CDRs. When additional \texttt{lst}
arguments are given, they are passed to \texttt{fun} in the same way. Returns
the result of the last application. See also \texttt{mapc}, \texttt{maplist},
\texttt{mapcar}, \texttt{mapcon}, \texttt{mapcan} and \texttt{filter}.


\begin{verbatim}
: (map println (1 2 3 4) '(A B C))
(1 2 3 4) (A B C)
(2 3 4) (B C)
(3 4) (C)
(4) NIL
-> NIL
\end{verbatim}

 
\section{map/3}
\label{sec-8-1-13-8}


\hyperref[ref.html-pilog]{Pilog} predicate that returns a list and subsequent
CDRs of that list, after applying the \texttt{get} algorithm to that object and
the following arguments. Often used in database queries. See also
\texttt{lst/3}.


\begin{verbatim}
: (? (db nr +Ord 1 @Ord) (map @L @Ord pos))
 @Ord={3-7} @L=({4-1} {4-2} {4-3})
 @Ord={3-7} @L=({4-2} {4-3})
 @Ord={3-7} @L=({4-3})
-> NIL
\end{verbatim}

 
\section{(mapc 'fun 'lst ..) -> any}
\label{sec-8-1-13-9}


Applies \texttt{fun} to each element of \texttt{lst}. When additional \texttt{lst} arguments
are given, their elements are also passed to \texttt{fun}. Returns the result
of the last application. See also \texttt{map}, \texttt{maplist}, \texttt{mapcar}, \texttt{mapcon},
\texttt{mapcan} and \texttt{filter}.


\begin{verbatim}
: (mapc println (1 2 3 4) '(A B C))
1 A
2 B
3 C
4 NIL
-> NIL
\end{verbatim}

 
\section{(mapcan 'fun 'lst ..) -> lst}
\label{sec-8-1-13-10}


Applies \texttt{fun} to each element of \texttt{lst}. When additional \texttt{lst} arguments
are given, their elements are also passed to \texttt{fun}. Returns a
(destructively) concatenated list of all results. See also \texttt{map},
\texttt{mapc}, \texttt{maplist}, \texttt{mapcar}, \texttt{mapcon}, \texttt{filter}.


\begin{verbatim}
: (mapcan reverse '((a b c) (d e f) (g h i)))
-> (c b a f e d i h g)
\end{verbatim}

 
\section{(mapcar 'fun 'lst ..) -> lst}
\label{sec-8-1-13-11}


Applies \texttt{fun} to each element of \texttt{lst}. When additional \texttt{lst} arguments
are given, their elements are also passed to \texttt{fun}. Returns a list of
all results. See also \texttt{map}, \texttt{mapc}, \texttt{maplist}, \texttt{mapcon}, \texttt{mapcan} and
\texttt{filter}.


\begin{verbatim}
: (mapcar + (1 2 3) (4 5 6))
-> (5 7 9)
: (mapcar '((X Y) (+ X (* Y Y))) (1 2 3 4) (5 6 7 8))
-> (26 38 52 68)
\end{verbatim}

 
\section{(mapcon 'fun 'lst ..) -> lst}
\label{sec-8-1-13-12}


Applies \texttt{fun} to \texttt{lst} and all successive CDRs. When additional \texttt{lst}
arguments are given, they are passed to \texttt{fun} in the same way. Returns a
(destructively) concatenated list of all results. See also \texttt{map},
\texttt{mapc}, \texttt{maplist}, \texttt{mapcar}, \texttt{mapcan} and \texttt{filter}.


\begin{verbatim}
: (mapcon copy '(1 2 3 4 5))
-> (1 2 3 4 5 2 3 4 5 3 4 5 4 5 5)
\end{verbatim}

 
\section{(maplist 'fun 'lst ..) -> lst}
\label{sec-8-1-13-13}


Applies \texttt{fun} to \texttt{lst} and all successive CDRs. When additional \texttt{lst}
arguments are given, they are passed to \texttt{fun} in the same way. Returns a
list of all results. See also \texttt{map}, \texttt{mapc}, \texttt{mapcar}, \texttt{mapcon},
\texttt{mapcan} and \texttt{filter}.


\begin{verbatim}
: (maplist cons (1 2 3) '(A B C))
-> (((1 2 3) A B C) ((2 3) B C) ((3) C))
\end{verbatim}

 
\section{(maps 'fun 'sym ['lst ..]) -> any}
\label{sec-8-1-13-14}


Applies \texttt{fun} to all properties of \texttt{sym}. When additional \texttt{lst}
arguments are given, their elements are also passed to \texttt{fun}. Returns
the result of the last application. See also \texttt{putl} and \texttt{getl}.


\begin{verbatim}
: (put 'X 'a 1)
-> 1
: (put 'X 'b 2)
-> 2
: (put 'X 'flg T)
-> T
: (getl 'X)
-> (flg (2 . b) (1 . a))
: (maps println 'X '(A B))
flg A
(2 . b) B
(1 . a) NIL
-> NIL
\end{verbatim}

 
\section{(mark 'sym|0 ['NIL | 'T | '0]) -> flg}
\label{sec-8-1-13-15}


Tests, sets or resets a mark for \texttt{sym} in the database (for a second
argument of \texttt{NIL}, \texttt{T} or \texttt{0}, respectively), and returns the old value.
The marks are local to the current process (not stored in the database),
and vanish when the process terminates. If the first argument is zero,
all marks are cleared.


\begin{verbatim}
: (pool "db")
-> T
: (mark '{1} T)      # Mark
-> NIL
: (mark '{1})        # Test
-> T                 # -> marked
: (mark '{1} 0)      # Unmark
-> T
: (mark '{1})        # Test
-> NIL               # -> unmarked
\end{verbatim}

 
\section{(match 'lst1 'lst2) -> flg}
\label{sec-8-1-13-16}


Takes \texttt{lst1} as a pattern to be matched against \texttt{lst2}, and returns \texttt{T}
when successful. Atoms must be equal, and sublists must match
recursively. Symbols in the pattern list with names starting with an
at-mark ``=@='' (see \texttt{pat?}) are taken as wildcards. They can match zero,
one or more elements, and are bound to the corresponding data. See also
\texttt{chop}, \texttt{split} and \texttt{fill}.


\begin{verbatim}
: (match '(@A is @B) '(This is a test))
-> T
: @A
-> (This)
: @B
-> (a test)
: (match '(@X (d @Y) @Z) '((a b c) (d (e f) g) h i))
-> T
: @X
-> ((a b c))
: @Y
-> ((e f) g)
: @Z
-> (h i)
\end{verbatim}

 
\section{(max 'any ..) -> any}
\label{sec-8-1-13-17}


Returns the largest of all \texttt{any} arguments. See also
\hyperref[refM.html-min]{min} and \hyperref[ref.html-cmp]{Comparing}.


\begin{verbatim}
: (max 2 'a 'z 9)
-> z
: (max (5) (2 3) 'X)
-> (5)
\end{verbatim}

 
\section{(maxKey 'tree ['any1 ['any2]]) -> any}
\label{sec-8-1-13-18}


Returns the largest key in a database tree. If a minimal key \texttt{any1}
and/or a maximal key \texttt{any2} is given, the largest key from that range is
returned. See also \texttt{tree}, \texttt{leaf}, \texttt{minKey} and \texttt{genKey}.


\begin{verbatim}
: (maxKey (tree 'nr '+Item))
-> 7
: (maxKey (tree 'nr '+Item) 3 5)
-> 5
\end{verbatim}

 
\section{(maxi 'fun 'lst ..) -> any}
\label{sec-8-1-13-19}


Applies \texttt{fun} to each element of \texttt{lst}. When additional \texttt{lst} arguments
are given, their elements are also passed to \texttt{fun}. Returns that element
from \texttt{lst} for which \texttt{fun} returned a maximal value. See also \texttt{mini} and
\texttt{sort}.


\begin{verbatim}
: (setq A 1  B 2  C 3)
-> 3
: (maxi val '(A B C))
-> C
: (maxi                          # Symbol with largest list value
   '((X)
      (and (pair (val X)) (size @)) )
   (what) )
-> *History
\end{verbatim}

 
\section{(member 'any 'lst) -> any}
\label{sec-8-1-13-20}


Returns the tail of \texttt{lst} that starts with \texttt{any} when \texttt{any} is a member
of \texttt{lst}, otherwise \texttt{NIL}. See also \texttt{memq}, \texttt{assoc} and \texttt{idx}.


\begin{verbatim}
: (member 3 (1 2 3 4 5 6))
-> (3 4 5 6)
: (member 9 (1 2 3 4 5 6))
-> NIL
: (member '(d e f) '((a b c) (d e f) (g h i)))
-> ((d e f) (g h i))
\end{verbatim}

 
\section{member/2}
\label{sec-8-1-13-21}


\hyperref[ref.html-pilog]{Pilog} predicate that succeeds if the the first
argument is a member of the list in the second argument. See also
\texttt{equal/2} and \texttt{member}.


\begin{verbatim}
:  (? (member @X (a b c)))
 @X=a
 @X=b
 @X=c
-> NIL
\end{verbatim}

 
\section{(memq 'any 'lst) -> any}
\label{sec-8-1-13-22}


Returns the tail of \texttt{lst} that starts with \texttt{any} when \texttt{any} is a member
of \texttt{lst}, otherwise \texttt{NIL}. ==== is used for comparison (pointer
equality). See also \texttt{member}, \texttt{mmeq}, \texttt{asoq}, \texttt{delq} and
\hyperref[ref.html-cmp]{Comparing}.


\begin{verbatim}
: (memq 'c '(a b c d e f))
-> (c d e f)
: (memq (2) '((1) (2) (3)))
-> NIL
\end{verbatim}

 
\section{(meta 'obj|typ 'sym ['sym2|cnt ..]) -> any}
\label{sec-8-1-13-23}


Fetches a property value \texttt{any}, by searching the property lists of the
classes and superclasses of \texttt{obj}, or the classes in \texttt{typ}, for the
property key \texttt{sym}, and by applying the \texttt{get} algorithm to the following
optional arguments. See also \texttt{var:}.


\begin{verbatim}
: (setq A '(B))            # Be 'A' an object of class 'B'
-> (B)
: (put 'B 'a 123)
-> 123
: (meta 'A 'a)             # Fetch 'a' from 'B'
-> 123
\end{verbatim}

 
\section{(meth 'obj ['any ..]) -> any}
\label{sec-8-1-13-24}


This function is usually not called directly, but is used by = dm= as a
template to initialize the \texttt{VAL} of message symbols. It searches for
itself in the methods of \texttt{obj} and its classes and superclasses, and
executes that method. An error =''Bad message''= is issued if the search
is unsuccessful. See also \texttt{OO Concepts}, \texttt{method}, \texttt{send} and \texttt{try}.


\begin{verbatim}
: meth
-> 67283504    # Value of 'meth'
: stop>
-> 67283504    # Value of any message
\end{verbatim}

 
\section{(method 'msg 'obj) -> fun}
\label{sec-8-1-13-25}


Returns the function body of the method that would be executed upon
sending the message \texttt{msg} to the object \texttt{obj}. If the message cannot be
located in \texttt{obj}, its classes and superclasses, \texttt{NIL} is returned. See
also \texttt{OO Concepts}, \texttt{send}, \texttt{try}, \texttt{meth}, \texttt{super}, \texttt{extra}, \texttt{class}.


\begin{verbatim}
: (method 'mis> '+Number)
-> ((Val Obj) (and Val (not (num? Val)) "Numeric input expected"))
\end{verbatim}

 
\section{(min 'any ..) -> any}
\label{sec-8-1-13-26}


Returns the smallest of all \texttt{any} arguments. See also
\hyperref[refM.html-max]{max} and \hyperref[ref.html-cmp]{Comparing}.


\begin{verbatim}
: (min 2 'a 'z 9)
-> 2
: (min (5) (2 3) 'X)
-> X
\end{verbatim}

 
\section{(minKey 'tree ['any1 ['any2]]) -> any}
\label{sec-8-1-13-27}


Returns the smallest key in a database tree. If a minimal key \texttt{any1}
and/or a maximal key \texttt{any2} is given, the smallest key from that range
is returned. See also \texttt{tree}, \texttt{leaf}, \texttt{maxKey} and \texttt{genKey}.


\begin{verbatim}
: (minKey (tree 'nr '+Item))
-> 1
: (minKey (tree 'nr '+Item) 3 5)
-> 3
\end{verbatim}

 
\section{(mini 'fun 'lst ..) -> any}
\label{sec-8-1-13-28}


Applies \texttt{fun} to each element of \texttt{lst}. When additional \texttt{lst} arguments
are given, their elements are also passed to \texttt{fun}. Returns that element
from \texttt{lst} for which \texttt{fun} returned a minimal value. See also \texttt{maxi} and
\texttt{sort}.


\begin{verbatim}
: (setq A 1  B 2  C 3)
-> 3
: (mini val '(A B C))
-> A
\end{verbatim}

 
\section{(mix 'lst cnt|'any ..) -> lst}
\label{sec-8-1-13-29}


Builds a list from the elements of the argument \texttt{lst}, as specified by
the following \texttt{cnt|'any} arguments. If such an argument is a number, the
\texttt{cnt}'th element from \texttt{lst} is taken, otherwise that argument is
evaluated and the result is used.


\begin{verbatim}
: (mix '(a b c d) 3 4 1 2)
-> (c d a b)
: (mix '(a b c d) 1 'A 4 'D)
-> (a A d D)
\end{verbatim}

 
\section{(mmeq 'lst 'lst) -> any}
\label{sec-8-1-13-30}


Returns the tail of the second argument \texttt{lst} that starts with a member
of the first argument \texttt{lst}, otherwise \texttt{NIL}. ==== is used for
comparison (pointer equality). See also \texttt{member}, \texttt{memq}, \texttt{asoq} and
\texttt{delq}.


\begin{verbatim}
: (mmeq '(a b c) '(d e f))
-> NIL
: (mmeq '(a b c) '(d b x))
-> (b x)
\end{verbatim}

 
\section{(money 'num ['sym]) -> sym}
\label{sec-8-1-13-31}


Formats a number \texttt{num} into a digit string with two decimal places,
according to the current \texttt{locale}. If an additional currency name is
given, it is appended (separated by a space). See also \texttt{telStr},
\texttt{datStr} and \texttt{format}.


\begin{verbatim}
: (money 123456789)
-> "1,234,567.89"
: (money 12345 "EUR")
-> "123.45 EUR"
: (locale "DE" "de")
-> NIL
: (money 123456789 "EUR")
-> "1.234.567,89 EUR"
\end{verbatim}

 
\section{(more 'lst ['fun]) -> flg}
\label{sec-8-1-13-32}


\texttt{(more 'cls) -> any}

Displays the elements of \texttt{lst} (first form), or the type and methods of
\texttt{cls} (second form). \texttt{fun} defaults to \texttt{print}. In the second form, the
method definitions of \texttt{cls} are pretty-printed with \texttt{pp}. After each
step, \texttt{more} waits for console input, and terminates when a non-empty
line is entered. In that case, \texttt{T} is returned, otherwise (when end of
data is reached) \texttt{NIL}. See also \texttt{query} and \texttt{show}.


\begin{verbatim}
: (more (all))                         # Display all internal symbols
inc>
leaf
nil
inc!
accept.                                # Stop
-> T

: (more (all) show)                    # 'show' all internal symbols
inc> 67292896
   *Dbg ((859 . "lib/db.l"))

leaf ((Tree) (let (Node (cdr (root Tree)) X) (while (val Node) (setq X (cadr @) Node (car @))) (cddr X)))
   *Dbg ((173 . "lib/btree.l"))

nil 67284680
   T (((@X) (@ not (-> @X))))
.                                      # Stop
-> T

: (more '+Link)                        # Display a class
(+relation)

(dm mis> (Val Obj)
   (and
      Val
      (nor (isa (: type) Val) (canQuery Val))
      "Type error" ) )

(dm T (Var Lst)
   (unless (=: type (car Lst)) (quit "No Link" Var))
   (super Var (cdr Lst)) )

-> NIL
\end{verbatim}

 
\section{(msg 'any ['any ..]) -> any}
\label{sec-8-1-13-33}


Prints \texttt{any} with \texttt{print}, followed by all \texttt{any} arguments (printed with
\texttt{prin}) and a newline, to standard error. The first \texttt{any} argument is
returned.


\begin{verbatim}
: (msg (1 a 2 b 3 c) " is a mixed " "list")
(1 a 2 b 3 c) is a mixed list
-> (1 a 2 b 3 c)
\end{verbatim}


\chapter{Functions starting with N}
\label{sec-8-1-14}


 
\section{+Need}
\label{sec-8-1-14-1}


Prefix class for mandatory =+relation=s. Note that this does not enforce
any requirements by itself, it only returns an error message if the
\texttt{mis>} message is explicitly called, e.g. by GUI functions. See also
\texttt{Database}.


\begin{verbatim}
(rel nr (+Need +Key +Number))  # Item number is mandatory
\end{verbatim}

 
\section{+Number}
\label{sec-8-1-14-2}


Class for numeric relations, a subclass of \texttt{+relation}. Accepts an
optional argument for the fixpoint scale (currently not used). See also
\texttt{Database}.


\begin{verbatim}
(rel pr (+Number) 2)  # Price, with two decimal places
\end{verbatim}

 

\section*{(n== `any ..) -> flg}
\label{sec-8-1-14-3}


Returns \texttt{T} when not all \texttt{any} arguments are the same (pointer
equality). \texttt{(n== `any ..)} is equivalent to \texttt{(not (== `any ..))}. See
also \texttt{==} and \hyperref[ref.html-cmp]{Comparing}.


\begin{verbatim}
: (n== 'a 'a)
-> NIL
: (n== (1) (1))
-> T
\end{verbatim}


 
\section{(n0 'any) -> flg}
\label{sec-8-1-14-4}


Returns \texttt{T} when \texttt{any} is not a number with value zero. See also ==0=,
\texttt{lt0}, \texttt{le0}, \texttt{ge0} and \texttt{gt0}.


\begin{verbatim}
: (n0 (- 6 3 2 1))
-> NIL
: (n0 'a)
-> T
\end{verbatim}

 
\section{(nT 'any) -> flg}
\label{sec-8-1-14-5}


Returns \texttt{T} when \texttt{any} is not the symbol \texttt{T}. See also
\hyperref[ref_.html-T]{=T}.


\begin{verbatim}
: (nT 0)
-> T
: (nT "T")
-> T
: (nT T)
-> NIL
\end{verbatim}

 
\section{(name 'sym ['sym2]) -> sym}
\label{sec-8-1-14-6}


Returns, if \texttt{sym2} is not given, a new transient symbol with the name of
\texttt{sym}. Otherwise \texttt{sym} must be a transient symbol, and its name is
changed to that of \texttt{sym2} (note that this may give inconsistencies if
the symbol is still referred to from other namespaces). See also \texttt{str},
\texttt{sym}, \texttt{symbols}, \texttt{zap} and \texttt{intern}.


\begin{verbatim}
: (name 'abc)
-> "abc"
: (name "abc")
-> "abc"
: (name '{abc})
-> "abc"
: (name (new))
-> NIL
: (de foo (Lst) (car Lst))  # 'foo' calls 'car'
-> foo
: (intern (name (zap 'car) "xxx"))  # Globally change the name of 'car'
-> xxx
: (xxx (1 2 3))
-> 1
: (pp 'foo)
(de foo (Lst)
   (xxx Lst) )                      # Name changed
-> foo
: (foo (1 2 3))                     # 'foo' still works
-> 1
: (car (1 2 3))                     # Reader returns a new 'car' symbol
!? (car (1 2 3))
car -- Undefined
?
\end{verbatim}

 
\section{(nand 'any ..) -> flg}
\label{sec-8-1-14-7}


Logical NAND. The expressions \texttt{any} are evaluated from left to right. If
\texttt{NIL} is encountered, \texttt{T} is returned immediately. Else \texttt{NIL} is
returned. \texttt{(nand ..)} is equivalent to \texttt{(not (and ..))}.


\begin{verbatim}
: (nand (lt0 7) (read))
-> T
: (nand (lt0 -7) (read))
abc
-> NIL
: (nand (lt0 -7) (read))
NIL
-> T
\end{verbatim}

 
\section{(native 'cnt1|sym1 'cnt2|sym2 'sym|lst 'any ..) -> any}
\label{sec-8-1-14-8}


(64-bit version only) Calls a native C function. The first argument
should specify a shared object library, either =''@''= (the current main
program), \texttt{sym1} (a library path name), or \texttt{cnt1} (a library handle
obtained by a previous call). The second argument should be a symbol
name \texttt{sym2}, or a function pointer \texttt{cnt2} obtained by a previous call).
Practically, the first two arguments will be always passed as transient
symbols, which will get the library handle and function pointer assigned
as values to be cached and used in subsequent calls. The third \texttt{sym|lst}
argument is a return value specification, while all following arguments
are the arguments to the native function.

The return value specification may either be one of the atoms


\begin{verbatim}
 NIL   void
 B     byte     # Byte (unsigned 8 bit)
 C     char     # Character (UTF-8, 1-3 bytes)
 I     int      # Integer (signed 32 bit)
 N     long     # Long or pointer (signed 64 bit)
 S     string   # String (UTF-8)
-1.0   float    # Scaled fixpoint number
+1.0   double   # Scaled fixpoint number
\end{verbatim}

or nested lists of these atoms with size specifications to denote arrays
and structures, e.g.


\begin{verbatim}
(N . 4)        # long[4];           -> (1 2 3 4)
(N (C . 4))    # {long; char[4];}   -> (1234 ("a" "b" "c" NIL))
(N (B . 7))    # {long; byte[7];}   -> (1234 (1 2 3 4 5 6 7))
\end{verbatim}

Arguments can be

\begin{itemize}
\item integers (up to 64-bit) or pointers, passed as numbers
\item fixpoint numbers, passed as cons pairs consisting of a the value and
   the scale. If the scale is positive, the number is passed as a
   \texttt{double}, otherwise as a \texttt{float}.
\item strings, passed as symbols, or
\item structures, passed as lists with
\begin{itemize}
\item a variable in the CAR (to recieve the returned structure data,
      ignored when the CAR is \texttt{NIL})
\item a cons pair for the size and value specification in the CADR (see
      above), and
\item an optional sequence of initialization items in the CDDR, where
      each may be
\begin{itemize}
\item a positive number, stored as an unsigned byte value
\item a negative number, whose absolute value is stored as an
         unsigned integer
\item a pair \texttt{(num . cnt)} where `=num=' is stored in a field of
         `=cnt=' bytes
\item a pair \texttt{(sym . cnt)} where `=sym=' is stored as a
         null-terminated string in a field of `=cnt=' bytes
\end{itemize}
If the last CDR of the initialization sequence is a number, it is
      used as a fill-byte value for the remaining space in the
      structure.
\end{itemize}
\end{itemize}

\texttt{native} takes care of allocating memory for strings, arrays or
structures, and frees that memory when done.

The number of fixpoint arguments is limited to six. For NaN or negative
infinity \texttt{NIL}, and for positive infinity \texttt{T} is returned.


\begin{verbatim}
: (native "@" "getenv" 'S "TERM")  # Same as (sys "TERM")
-> "xterm"

: (native "@" "printf" 'I "abc%d%s^J" (+ 3 4) (pack "X" "Y" "Z"))
abc7XYZ
-> 8

: (native "@" "printf" 'I "This is %.3f^J" (123456 . 1000))
This is 123.456
-> 16

: (use Tim
   (native "@" "time" NIL '(Tim (8 B . 8)))  # time_t 8   # Get time_t structure
   (native "@" "localtime" '(I . 9) (cons NIL (8) Tim)) ) # Read local time
-> (32 18 13 31 11 109 4 364 0)  # 13:18:32, Dec. 31st, 2009
\end{verbatim}

The C function may in turn call a function


\begin{verbatim}
long lisp(char*, long, long, long, long, long);
\end{verbatim}

which accepts a symbol name as the first argument, and up to 5 numbers.
\texttt{lisp()} calls that symbol with the five numbers, and expects a numeric
return value. ``Numbers'' in this context are 64-bit scalars, and may not
only represent integers, but also pointers or other encoded data. See
also \texttt{errno} and \texttt{lisp}.

 
\section{(need 'cnt ['lst ['any]]) -> lst}
\label{sec-8-1-14-9}


\texttt{(need 'cnt ['num|sym]) -> lst}

Produces a list of at least \texttt{cnt} elements. When called without optional
arguments, a list of \texttt{cnt} \texttt{NIL}'s is returned. When \texttt{lst} is given, it
is extended to the left (if \texttt{cnt} is positive) or (destructively) to the
right (if \texttt{cnt} is negative) with \texttt{any} elements. In the second form, a
list of \texttt{cnt} atomic values is returned. See also \texttt{range}.


\begin{verbatim}
: (need 5)
-> (NIL NIL NIL NIL NIL)  # Allocate 5 cells
: (need 5 '(a b c))
-> (NIL NIL a b c)
: (need -5 '(a b c))
-> (a b c NIL NIL)
: (need 5 '(a b c) " ")  # String alignment
-> (" " " " a b c)
: (need 7 0)
-> (0 0 0 0 0 0 0)
\end{verbatim}

 
\section{(new ['flg|num] ['typ ['any ..]]) -> obj}
\label{sec-8-1-14-10}


Creates and returns a new object. If \texttt{flg} is given and non-=NIL=, the
new object will be an external symbol (created in database file 1 if
\texttt{T}, or in the corresponding database file if \texttt{num} is given). \texttt{typ}
(typically a list of classes) is assigned to the \texttt{VAL}, and the initial
\texttt{T} message is sent with the arguments \texttt{any} to the new object. If no
\texttt{T} message is defined for the classes in \texttt{typ} or their superclasses,
the \texttt{any} arguments should evaluate to alternating keys and values for
the initialization of the new object. See also \texttt{box}, \texttt{object}, \texttt{class},
\texttt{type}, \texttt{isa}, \texttt{send} and \texttt{Database}.


\begin{verbatim}
: (new)
-> $134426427
: (new T '(+Address))
-> {1A;3}
\end{verbatim}

 
\section{(new! 'typ ['any ..]) -> obj}
\label{sec-8-1-14-11}


\hyperref[ref.html-trans]{Transaction} wrapper function for \texttt{new}.
\texttt{(new! '(+Cls) 'key 'val ...)} is equivalent to
\texttt{(dbSync) (new (db: +Cls) '(+Cls) 'key 'val ...) (commit 'upd)}. See
also \texttt{set!}, \texttt{put!} and \texttt{inc!}.


\begin{verbatim}
: (new! '(+Item)  # Create a new item
   'nr 2                      # Item number
   'nm "Spare Part"           # Description
   'sup (db 'nr '+CuSu 2)     # Supplier
   'inv 100                   # Inventory
   pr 12.50 )                 # Price
\end{verbatim}

 
\section{(next) -> any}
\label{sec-8-1-14-12}


Can only be used inside functions with a variable number of arguments
(with \texttt{@}). Returns the next argument from the internal list. See also
\texttt{args}, \texttt{arg}, \texttt{rest}, and \texttt{pass}.


\begin{verbatim}
: (de foo @ (println (next)))          # Print next argument
-> foo
: (foo)
NIL
-> NIL
: (foo 123)
123
-> 123
\end{verbatim}

 
\section{(nil . prg) -> NIL}
\label{sec-8-1-14-13}


Executes \texttt{prg}, and returns \texttt{NIL}. See also \texttt{t}, \texttt{prog}, \texttt{prog1} and
\texttt{prog2}.


\begin{verbatim}
: (nil (println 'OK))
OK
-> NIL
\end{verbatim}

 
\section{nil/1}
\label{sec-8-1-14-14}


\hyperref[ref.html-pilog]{Pilog} predicate expects an argument variable, and
succeeds if that variable is bound to \texttt{NIL}. See also \texttt{not/1}.


\begin{verbatim}
: (? @X NIL (nil @X))
 @X=NIL
-> NIL
\end{verbatim}

 
\section{(noLint 'sym)}
\label{sec-8-1-14-15}


\texttt{(noLint 'sym|(sym . cls) 'sym2)}

Excludes the check for a function definition of \texttt{sym} (in the first
form), or for variable binding and usage of \texttt{sym2} in the function
definition, file contents or method body of \texttt{sym} (second form), during
calls to \texttt{lint}. See also \texttt{lintAll}.


\begin{verbatim}
: (de foo ()
   (bar FreeVariable) )
-> foo
: (lint 'foo)
-> ((def bar) (bnd FreeVariable))
: (noLint 'bar)
-> bar
: (noLint 'foo 'FreeVariable)
-> (foo . FreeVariable)
: (lint 'foo)
-> NIL
\end{verbatim}

 
\section{(nond ('any1 . prg1) ('any2 . prg2) ..) -> any}
\label{sec-8-1-14-16}


Negated (``non-cond'') multi-way conditional: If any of the \texttt{anyN}
conditions evaluates to \texttt{NIL}, \texttt{prgN} is executed and the result
returned. Otherwise (all conditions evaluate to non-=NIL=), \texttt{NIL} is
returned. See also \texttt{cond}, \texttt{ifn} and \texttt{unless}.


\begin{verbatim}
: (nond
   ((= 3 3) (println 1))
   ((= 3 4) (println 2))
   (NIL (println 3)) )
2
-> 2
\end{verbatim}

 
\section{(nor 'any ..) -> flg}
\label{sec-8-1-14-17}


Logical NOR. The expressions \texttt{any} are evaluated from left to right. If
a non-=NIL= value is encountered, \texttt{NIL} is returned immediately. Else
\texttt{T} is returned. \texttt{(nor ..)} is equivalent to \texttt{(not (or ..))}.


\begin{verbatim}
: (nor (lt0 7) (= 3 4))
-> T
\end{verbatim}

 
\section{(not 'any) -> flg}
\label{sec-8-1-14-18}


Logical negation. Returns \texttt{T} if \texttt{any} evaluates to \texttt{NIL}.


\begin{verbatim}
: (not (== 'a 'a))
-> NIL
: (not (get 'a 'a))
-> T
\end{verbatim}

 
\section{not/1}
\label{sec-8-1-14-19}


\hyperref[ref.html-pilog]{Pilog} predicate that succeeds if and only if the
goal cannot be proven. See also \texttt{nil/1}, \texttt{true/0} and \texttt{fail/0}.


\begin{verbatim}
: (? (equal 3 4))
-> NIL
: (? (not (equal 3 4)))
-> T
\end{verbatim}

 
\section{(nth 'lst 'cnt ..) -> lst}
\label{sec-8-1-14-20}


Returns the tail of \texttt{lst} starting from the \texttt{cnt}'th element of \texttt{lst}.
Successive \texttt{cnt} arguments operate on the results in the same way.
\texttt{(nth 'lst 2)} is equivalent to \texttt{(cdr 'lst)}. See also \texttt{get}.


\begin{verbatim}
: (nth '(a b c d) 2)
-> (b c d)
: (nth '(a (b c) d) 2 2)
-> (c)
: (cdadr '(a (b c) d))
-> (c)
\end{verbatim}

 
\section{(num? 'any) -> num | NIL}
\label{sec-8-1-14-21}


Returns \texttt{any} when the argument \texttt{any} is a number, otherwise \texttt{NIL}.


\begin{verbatim}
: (num? 123)
-> 123
: (num? (1 2 3))
-> NIL
\end{verbatim}


\chapter{Functions starting with O}
\label{sec-8-1-15}


 
\section{*Once}
\label{sec-8-1-15-1}


Holds an \texttt{idx} tree of already \texttt{load=ed source locations (as returned by =file}) See also \texttt{once}.


\begin{verbatim}
: *Once
-> (("lib/" "misc.l" . 11) (("lib/" "http.l" . 9) (("lib/" "form.l" . 11))))
\end{verbatim}

 
\section{*OS}
\label{sec-8-1-15-2}


A global constant holding the name of the operating system. Possible
values include =''Linux''=, =''FreeBSD''=, =''Darwin''= or =''Cygwin''=.


\begin{verbatim}
: *OS
-> "Linux"
\end{verbatim}

 
\section{(obj (typ var [hook] val ..) var2 val2 ..) -> obj}
\label{sec-8-1-15-3}


Finds or creates a database object (using \texttt{request}) corresponding to
\texttt{(typ var [hook] val ..)}, and initializes additional properties using
the \texttt{varN} and \texttt{valN} arguments.


\begin{verbatim}
: (obj ((+Item) nr 2) nm "Spare Part" sup `(db 'nr '+CuSu 2) inv 100 pr 1250)
-> {3-2}
\end{verbatim}

 
\section{(object 'sym 'any ['sym2 'any2 ..]) -> obj}
\label{sec-8-1-15-4}


Defines \texttt{sym} to be an object with the value (or type) \texttt{any}. The
property list is initialized with all optionally supplied key-value
pairs. See also \texttt{OO Concepts}, \texttt{new}, \texttt{type} and \texttt{isa}.


\begin{verbatim}
: (object 'Obj '(+A +B +C) 'a 1 'b 2 'c 3)
-> Obj
: (show 'Obj)
Obj (+A +B +C)
   c 3
   b 2
   a 1
-> Obj
\end{verbatim}

 
\section{(oct 'num ['num]) -> sym}
\label{sec-8-1-15-5}


\texttt{(oct 'sym) -> num}

Converts a number \texttt{num} to an octal string, or an octal string \texttt{sym} to
a number. In the first case, if the second argument is given, the result
is separated by spaces into groups of such many digits. See also \texttt{bin},
\texttt{hex}, \texttt{fmt64}, \texttt{hax} and \texttt{format}.


\begin{verbatim}
: (oct 73)
-> "111"
: (oct "111")
-> 73
: (oct 1234567 3)
-> "4 553 207"
\end{verbatim}

 
\section{(off var ..) -> NIL}
\label{sec-8-1-15-6}


Stores \texttt{NIL} in all \texttt{var} arguments. See also \texttt{on}, \texttt{onOff}, \texttt{zero} and
\texttt{one}.


\begin{verbatim}
: (off A B)
-> NIL
: A
-> NIL
: B
-> NIL
\end{verbatim}

 
\section{(offset 'lst1 'lst2) -> cnt | NIL}
\label{sec-8-1-15-7}


Returns the \texttt{cnt} position of the tail list \texttt{lst1} in \texttt{lst2}, or \texttt{NIL}
if it is not found. See also \texttt{index} and \texttt{tail}.


\begin{verbatim}
: (offset '(c d e f) '(a b c d e f))
-> 3
: (offset '(c d e) '(a b c d e f))
-> NIL
\end{verbatim}

 
\section{(on var ..) -> T}
\label{sec-8-1-15-8}


Stores \texttt{T} in all \texttt{var} arguments. See also \texttt{off}, \texttt{onOff}, \texttt{zero} and
\texttt{one}.


\begin{verbatim}
: (on A B)
-> T
: A
-> T
: B
-> T
\end{verbatim}

 
\section{(once . prg) -> any}
\label{sec-8-1-15-9}


Executes \texttt{prg} once, when the current file is \texttt{load=ed the first time. Subsequent loads at a later time will not execute =prg}, and \texttt{once}
returns \texttt{NIL}. See also \texttt{*Once}.


\begin{verbatim}
(once
   (zero *Cnt1 *Cnt2)  # Init counters
   (load "file1.l" "file2.l") )  # Load other files
\end{verbatim}

 
\section{(one var ..) -> 1}
\label{sec-8-1-15-10}


Stores \texttt{1} in all \texttt{var} arguments. See also \texttt{zero}, \texttt{on}, \texttt{off} and
\texttt{onOff}.


\begin{verbatim}
: (one A B)
-> 1
: A
-> 1
: B
-> 1
\end{verbatim}

 
\section{(onOff var ..) -> flg}
\label{sec-8-1-15-11}


Logically negates the values of all \texttt{var} arguments. Returns the new
value of the last symbol. See also \texttt{on}, \texttt{off}, \texttt{zero} and \texttt{one}.


\begin{verbatim}
: (onOff A B)
-> T
: A
-> T
: B
-> T
: (onOff A B)
-> NIL
: A
-> NIL
: B
-> NIL
\end{verbatim}

 
\section{(open 'any ['flg]) -> cnt | NIL}
\label{sec-8-1-15-12}


Opens the file with the name \texttt{any} in read/write mode (or read-only if
\texttt{flg} is non-=NIL=), and returns a file descriptor \texttt{cnt} (or \texttt{NIL} on
error). A leading ``=@='' character in \texttt{any} is substituted with the
PicoLisp Home Directory, as it was remembered during interpreter
startup. If \texttt{flg} is \texttt{NIL} and the file does not exist, it is created.
The file descriptor can be used in subsequent calls to \texttt{in} and \texttt{out}.
See also \texttt{close} and \texttt{poll}.


\begin{verbatim}
: (open "x")
-> 3
\end{verbatim}

 
\section{(opid) -> pid | NIL}
\label{sec-8-1-15-13}


Returns the corresponding process ID when the current output channel is
writing to a pipe, otherwise \texttt{NIL}. See also \texttt{ipid} and \texttt{out}.


\begin{verbatim}
: (out '(cat) (call 'ps "-p" (opid)))
  PID TTY          TIME CMD
 7127 pts/3    00:00:00 cat
-> T
\end{verbatim}

 
\section{(opt) -> sym}
\label{sec-8-1-15-14}


Return the next command line argument (``option'', as would be processed
by \texttt{load}) as a string, and remove it from the remaining command line
arguments. See also \hyperref[ref.html-invoc]{Invocation} and \texttt{argv}.


\begin{verbatim}
$ pil  -"de f () (println 'opt (opt))"  -f abc  -bye
opt "abc"
\end{verbatim}

 
\section{(or 'any ..) -> any}
\label{sec-8-1-15-15}


Logical OR. The expressions \texttt{any} are evaluated from left to right. If a
non-=NIL= value is encountered, it is returned immediately. Else the
result of the last expression is returned.


\begin{verbatim}
: (or (= 3 3) (read))
-> T
: (or (= 3 4) (read))
abc
-> abc
\end{verbatim}

 
\section{or/2}
\label{sec-8-1-15-16}


\hyperref[ref.html-pilog]{Pilog} predicate that takes an arbitrary number of
clauses, and succeeds if one of them can be proven. See also \texttt{not/1}.


\begin{verbatim}
: (?
   (or
      ((equal 3 @X) (equal @X 4))
      ((equal 7 @X) (equal @X 7)) ) )
 @X=7
-> NIL
\end{verbatim}

 
\section{(out 'any . prg) -> any}
\label{sec-8-1-15-17}


Opens \texttt{any} as output channel during the execution of \texttt{prg}. The current
output channel will be saved and restored appropriately. If the argument
is \texttt{NIL}, standard output is used. If the argument is a symbol, it is
used as a file name (opened in ``append'' mode if the first character is
``=+=''). If it is a positve number, it is used as the descriptor of an
open file. If it is a negative number, the saved output channel such
many levels above the current one is used. Otherwise (if it is a list),
it is taken as a command with arguments, and a pipe is opened for
output. See also \texttt{opid}, = call=, \texttt{in}, = err=, = ctl=, \texttt{pipe}, = poll=,
= close= and \texttt{load}.


\begin{verbatim}
: (out "a" (println 123 '(a b c) 'def))  # Write one line to file "a"
-> def
\end{verbatim}


\chapter{Functions starting with P}
\label{sec-8-1-16}


 
\section{*PPid}
\label{sec-8-1-16-1}


A global constant holding the process-id of the parent picolisp process,
or \texttt{NIL} if the current process is a top level process.


\begin{verbatim}
: (println *PPid *Pid)
NIL 5286

: (unless (fork) (println *PPid *Pid) (bye))
5286 5522
\end{verbatim}

 
\section{*Pid}
\label{sec-8-1-16-2}


A global constant holding the current process-id.


\begin{verbatim}
: *Pid
-> 6386
: (call "ps")  # Show processes
  PID TTY          TIME CMD
 .... ...      ........ .....
 6386 pts/1    00:00:00 pil   # <- current process
 6388 pts/1    00:00:00 ps
-> T
\end{verbatim}

 
\section{*Prompt}
\label{sec-8-1-16-3}


Global variable holding a (possibly empty) \texttt{prg} body, which is executed
\begin{itemize}
\item and the result =prin=ted - every time before a prompt is output to the
\end{itemize}
console in the ``read-eval-print-loop'' (REPL).


\begin{verbatim}
: (de *Prompt (pack "[" (stamp) "]"))
# *Prompt redefined
-> *Prompt
[2011-10-11 16:50:05]: (+ 1 2 3)
-> 6
[2011-10-11 16:50:11]:
\end{verbatim}

 
\section{(pack 'any ..) -> sym}
\label{sec-8-1-16-4}


Returns a transient symbol whose name is concatenated from all arguments
\texttt{any}. A \texttt{NIL} arguments contributes nothing to the result string, a
number is converted to a digit string, a symbol supplies the characters
of its name, and for a list its elements are taken. See also \texttt{text} and
\texttt{glue}.


\begin{verbatim}
: (pack 'car " is " 1 '(" symbol " name))
-> "car is 1 symbol name"
\end{verbatim}

 
\section{(pad 'cnt 'any) -> sym}
\label{sec-8-1-16-5}


Returns a transient symbol with \texttt{any} \texttt{pack=ed with leading '0' characters, up to a field width of =cnt}. See also \texttt{format} and \texttt{align}.


\begin{verbatim}
: (pad 5 1)
-> "00001"
: (pad 5 123456789)
-> "123456789"
\end{verbatim}

 
\section{(pair 'any) -> any}
\label{sec-8-1-16-6}


Returns \texttt{any} when the argument a cons pair cell. See also \texttt{atom}.


\begin{verbatim}
: (pair NIL)
-> NIL
: (pair (1 . 2))
-> (1 . 2)
: (pair (1 2 3))
-> (1 2 3)
\end{verbatim}

 
\section{part/3}
\label{sec-8-1-16-7}


\hyperref[ref.html-pilog]{Pilog} predicate that succeeds if the first argument,
after \texttt{fold=ing it to a canonical form, is a /substring/ of the folded string representation of the result of applying the =get} algorithm to
the following arguments. Typically used as filter predicate in
\texttt{select/3} database queries. See also \texttt{sub?}, \texttt{isa/2}, \texttt{same/3},
\texttt{bool/3}, \texttt{range/3}, \texttt{head/3}, \texttt{fold/3} and \texttt{tolr/3}.


\begin{verbatim}
: (?
   @Nr (1 . 5)
   @Nm "part"
   (select (@Item)
      ((nr +Item @Nr) (nm +Item @Nm))
      (range @Nr @Item nr)
      (part @Nm @Item nm) ) )
 @Nr=(1 . 5) @Nm="part" @Item={3-1}                                              @Nr=(1 . 5) @Nm="part" @Item={3-2}
-> NIL
\end{verbatim}

 
\section{(pass 'fun ['any ..]) -> any}
\label{sec-8-1-16-8}


Passes to \texttt{fun} all arguments \texttt{any}, and all remaining variable
arguments (\texttt{@}) as they would be returned by \texttt{rest}. \texttt{(pass 'fun 'any)}
is equivalent to \texttt{(apply 'fun (rest) 'any)}. See also \texttt{apply}.


\begin{verbatim}
: (de bar (A B . @)
   (println 'bar A B (rest)) )
-> bar
: (de foo (A B . @)
   (println 'foo A B)
   (pass bar 1)
   (pass bar 2) )
-> foo
: (foo 'a 'b 'c 'd 'e 'f)
foo a b
bar 1 c (d e f)
bar 2 c (d e f)
-> (d e f)
\end{verbatim}

 
\section{(pat? 'any) -> pat | NIL}
\label{sec-8-1-16-9}


Returns \texttt{any} when the argument \texttt{any} is a symbol whose name starts with
an at-mark ``=@='', otherwise \texttt{NIL}.


\begin{verbatim}
: (pat? '@)
-> @
: (pat? "@Abc")
-> "@Abc"
: (pat? "ABC")
-> NIL
: (pat? 123)
-> NIL
\end{verbatim}

 
\section{(patch 'lst 'any . prg) -> any}
\label{sec-8-1-16-10}


Destructively replaces all sub-expressions of \texttt{lst}, that \texttt{match} the
pattern \texttt{any}, by the result of the execution of \texttt{prg}. See also
\texttt{daemon} and \texttt{redef}.


\begin{verbatim}
: (pp 'hello)
(de hello NIL
   (prinl "Hello world!") )
-> hello

: (patch hello 'prinl 'println)
-> NIL
: (pp 'hello)
(de hello NIL
   (println "Hello world!") )
-> hello

: (patch hello '(prinl @S) (fill '(println "We said: " . @S)))
-> NIL
: (hello)
We said: Hello world!
-> "Hello world!"
\end{verbatim}

 
\section{(path 'any) -> sym}
\label{sec-8-1-16-11}


Substitutes any leading ``=@='' character in the \texttt{any} argument with the
PicoLisp Home Directory, as it was remembered during interpreter
startup. Optionally, the name may be preceded by a ``=+='' character (as
used by \texttt{in} and \texttt{out}). This mechanism is used internally by all I/O
functions. See also \hyperref[ref.html-invoc]{Invocation}, \texttt{basename} and
\texttt{dirname}.


\begin{verbatim}
$ /usr/bin/picolisp /usr/lib/picolisp/lib.l
: (path "a/b/c")
-> "a/b/c"
: (path "@a/b/c")
-> "/usr/lib/picolisp/a/b/c"
: (path "+@a/b/c")
-> "+/usr/lib/picolisp/a/b/c"
\end{verbatim}

 
\section{(peek) -> sym}
\label{sec-8-1-16-12}


Single character look-ahead: Returns the same character as the next call
to \texttt{char} would return. See also \texttt{skip}.


\begin{verbatim}
$ cat a
# Comment
abcd
$ pil +
: (in "a" (list (peek) (char)))
-> ("#" "#")
\end{verbatim}

 
\section{permute/2}
\label{sec-8-1-16-13}


\hyperref[ref.html-pilog]{Pilog} predicate that succeeds if the second argument
is a permutation of the list in the second argument. See also
\texttt{append/3}.


\begin{verbatim}
: (? (permute (a b c) @X))
 @X=(a b c)
 @X=(a c b)
 @X=(b a c)
 @X=(b c a)
 @X=(c a b)
 @X=(c b a)
-> NIL
\end{verbatim}

 
\section{(pick 'fun 'lst ..) -> any}
\label{sec-8-1-16-14}


Applies \texttt{fun} to successive elements of \texttt{lst} until non-=NIL= is
returned. Returns that value, or \texttt{NIL} if \texttt{fun} did not return non-=NIL=
for any element of \texttt{lst}. When additional \texttt{lst} arguments are given,
their elements are also passed to \texttt{fun}. \texttt{(pick 'fun 'lst)} is
equivalent to \texttt{(fun (find 'fun 'lst))}. See also \texttt{seek}, \texttt{find} and
\texttt{extract}.


\begin{verbatim}
: (setq A NIL  B 1  C NIL  D 2  E NIL  F 3)
-> 3
: (find val '(A B C D E))
-> B
: (pick val '(A B C D E))
-> 1
\end{verbatim}

 
\section{pico}
\label{sec-8-1-16-15}


(64-bit version only) A global constant holding the initial (default)
namespace of internal symbols. Its value is a cons pair of two `=idx='
trees, one for symbols with short names and one for symbols with long
names (more than 7 bytes in the name). See also \texttt{symbols}, \texttt{import} and
\texttt{intern}.


\begin{verbatim}
: (symbols)
-> pico
: (cdr pico)
-> (rollback (*NoTrace (ledSearch (expandTab (********)) *CtryCode ...
\end{verbatim}

 
\section{(pil ['any ..]) -> sym}
\label{sec-8-1-16-16}


Returns the path name to the \texttt{pack=ed =any} arguments in the directory
``.pil/'' in the user's home directory. See also \texttt{tmp}.


\begin{verbatim}
: (pil "history")  # Path to the line editor's history file
-> "/home/app/.pil/history"
\end{verbatim}

 
\section{(pilog 'lst . prg) -> any}
\label{sec-8-1-16-17}


Evaluates a \hyperref[ref.html-pilog]{Pilog} query, and executes \texttt{prg} for each
result set with all Pilog variables bound to their matching values. See
also \texttt{solve}, \texttt{?}, \texttt{goal} and \texttt{prove}.


\begin{verbatim}
: (pilog '((append @X @Y (a b c))) (println @X '- @Y))
NIL - (a b c)
(a) - (b c)
(a b) - (c)
(a b c) - NIL
-> NIL
\end{verbatim}

 
\section{(pipe exe) -> cnt}
\label{sec-8-1-16-18}


\texttt{(pipe exe . prg) -> any}

Executes \texttt{exe} in a \texttt{fork}'ed child process (which terminates
thereafter). In the first form, \texttt{pipe} just returns a file descriptor to
read from the standard output of that process. In the second form, it
opens the standard output of that process as input channel during the
execution of \texttt{prg}. The current input channel will be saved and restored
appropriately. See also \texttt{later}, \texttt{ipid}, \texttt{in} and \texttt{out}.


\begin{verbatim}
: (pipe                                # equivalent to 'any'
   (prinl "(a b # Comment^Jc d)")         # (child process)
   (read) )                               # (parent process)
-> (a b c d)
: (pipe                                # pipe through an external program
   (out '(tr "[a-z]" "[A-Z]")             # (child process)
      (prinl "abc def ghi") )
   (line T) )                             # (parent process)
-> "ABC DEF GHI"
\end{verbatim}

 
\section{(place 'cnt 'lst 'any) -> lst}
\label{sec-8-1-16-19}


Places \texttt{any} into \texttt{lst} at position \texttt{cnt}. This is a non-destructive
operation. See also \texttt{insert}, \texttt{remove}, \texttt{append}, \texttt{delete} and
\texttt{replace}.


\begin{verbatim}
: (place 3 '(a b c d e) 777)
-> (a b 777 d e)
: (place 1 '(a b c d e) 777)
-> (777 b c d e)
: (place 9 '(a b c d e) 777)
-> (a b c d e 777)
\end{verbatim}

 
\section{(poll 'cnt) -> cnt | NIL}
\label{sec-8-1-16-20}


Checks for the availability of data for reading on the file descriptor
\texttt{cnt}. See also \texttt{open}, \texttt{in} and \texttt{close}.


\begin{verbatim}
: (and (poll *Fd) (in @ (read)))  # Prevent blocking
\end{verbatim}

 
\section{(pool ['sym1 ['lst] ['sym2] ['sym3]]) -> T}
\label{sec-8-1-16-21}


Opens the file \texttt{sym1} as a database file in read/write mode. If the file
does not exist, it is created. A currently open database is closed.
\texttt{lst} is a list of block size scale factors (i.e. numbers), defaulting
to (2) (for a single file with a 256 byte block size). If \texttt{lst} is
given, an individual database file is opened for each item. If \texttt{sym2} is
non-=NIL=, it is opened in append-mode as an asynchronous replication
journal. If \texttt{sym3} is non-=NIL=, it is opened for reading and appending,
to be used as a synchronous transaction log during \texttt{commit=s. See also =dbs}, \texttt{*Dbs} and \texttt{journal}.


\begin{verbatim}
: (pool "/dev/hda2")
-> T

: *Dbs
-> (1 2 2 4)
: (pool "dbFile" *Dbs)
-> T
:
abu:~/pico  ls -l dbFile*
-rw-r--r-- 1 abu abu 256 2007-06-11 07:57 dbFile1
-rw-r--r-- 1 abu abu  13 2007-06-11 07:57 dbFile2
-rw-r--r-- 1 abu abu  13 2007-06-11 07:57 dbFile3
-rw-r--r-- 1 abu abu  13 2007-06-11 07:57 dbFile4
\end{verbatim}

 
\section{(pop 'var) -> any}
\label{sec-8-1-16-22}


Pops the first element (CAR) from the stack in \texttt{var}. See also \texttt{push},
\texttt{queue}, \texttt{cut}, \texttt{del} and \texttt{fifo}.


\begin{verbatim}
: (setq S '((a b c) (1 2 3)))
-> ((a b c) (1 2 3))
: (pop S)
-> a
: (pop (cdr S))
-> 1
: (pop 'S)
-> (b c)
: S
-> ((2 3))
\end{verbatim}

 
\section{(port ['T] 'cnt|(cnt . cnt) ['var]) -> cnt}
\label{sec-8-1-16-23}


Opens a TCP-Port \texttt{cnt} (or a UDP-Port if the first argument is \texttt{T}), and
returns a socket descriptor suitable as an argument for \texttt{listen} or
\texttt{accept} (or \texttt{udp}, respectively). If \texttt{cnt} is zero, some free port
number is allocated. If a pair of \texttt{cnt=s is given instead, it should be a range of numbers which are tried in turn. When =var} is given, it is
bound to the port number.


\begin{verbatim}
: (port 0 'A)                       # Allocate free port
-> 4
: A
-> 1034                             # Got 1034
: (port (4000 . 4008) 'A)           # Try one of these ports
-> 5
: A
-> 4002
\end{verbatim}

 
\section{(pp 'sym) -> sym}
\label{sec-8-1-16-24}


\texttt{(pp 'sym 'cls) -> sym}

\texttt{(pp '(sym . cls)) -> sym}

Pretty-prints the function or method definition of \texttt{sym}. The output
format would regenerate that same definition when read and executed. See
also \texttt{pretty}, \texttt{debug} and \texttt{vi}.


\begin{verbatim}
: (pp 'tab)
(de tab (Lst . @)
   (for N Lst
      (let V (next)
         (and (gt0 N) (space (- N (length V))))
         (prin V)
         (and
            (lt0 N)
            (space (- 0 N (length V))) ) ) )
   (prinl) )
-> tab

: (pp 'has> '+Entity)
(dm has> (Var Val)
   (or
      (nor Val (get This Var))
      (has> (meta This Var) Val (get This Var)) ) )
-> has>

: (more (can 'has>) pp)
(dm (has> . +relation) (Val X)
   (and (= Val X) X) )

(dm (has> . +Fold) (Val X)
   (extra
      Val
      (if (= Val (fold Val)) (fold X) X) ) )

(dm (has> . +Entity) (Var Val)
   (or
      (nor Val (get This Var))
      (has> (meta This Var) Val (get This Var)) ) )

(dm (has> . +List) (Val X)
   (and
      Val
      (or
         (extra Val X)
         (find '((X) (extra Val X)) X) ) ) )

(dm (has> . +Bag) (Val X)
   (and
      Val
      (or (super Val X) (car (member Val X))) ) )
\end{verbatim}

 
\section{(pr 'any ..) -> any}
\label{sec-8-1-16-25}


Binary print: Prints all \texttt{any} arguments to the current output channel
in encoded binary format. See also \texttt{rd}, \texttt{tell}, \texttt{hear} and \texttt{wr}.


\begin{verbatim}
: (out "x" (pr 7 "abc" (1 2 3) 'a))  # Print to "x"
-> a
: (hd "x")
00000000  04 0E 0E 61 62 63 01 04 02 04 04 04 06 03 05 61  ...abc.........a
-> NIL
\end{verbatim}

 
\section{(prEval 'prg ['cnt]) -> any}
\label{sec-8-1-16-26}


Executes \texttt{prg}, similar to \texttt{run}, by evaluating all expressions in \texttt{prg}
(within the binding environment given by \texttt{cnt-1}). As a side effect, all
atomic expressions will be printed with \texttt{prinl}. See also \texttt{eval}.


\begin{verbatim}
: (let Prg 567
   (prEval
      '("abc" (prinl (+ 1 2 3)) Prg 987) ) )
abc
6
567
987
-> 987
\end{verbatim}

 
\section{(pre? 'any1 'any2) -> any2 | NIL}
\label{sec-8-1-16-27}


Returns \texttt{any2} when the string representation of \texttt{any1} is a prefix of
the string representation of \texttt{any2}. See also \texttt{sub?}.


\begin{verbatim}
: (pre? "abc" "abcdef")
-> "abcdef"
: (pre? "def" "abcdef")
-> NIL
: (pre? (+ 3 4) "7fach")
-> "7fach"
: (pre? NIL "abcdef")
-> "abcdef"
\end{verbatim}

 
\section{(pretty 'any 'cnt)}
\label{sec-8-1-16-28}


Pretty-prints \texttt{any}. If \texttt{any} is an atom, or a list with a \texttt{size} not
greater than 12, it is =print=ed as is. Otherwise, only the opening
parenthesis and the CAR of the list is printed, all other elementes are
pretty-printed recursively indented by three spaces, followed by a space
and the corresponding closing parenthesis. The initial indentation level
\texttt{cnt} defaults to zero. See also \texttt{pp}.


\begin{verbatim}
: (pretty '(a (b c d) (e (f (g) (h) (i)) (j (k) (l) (m))) (n o p) q))
(a
   (b c d)
   (e
      (f (g) (h) (i))
      (j (k) (l) (m)) )
   (n o p)
   q )-> ")"
\end{verbatim}

 
\section{(prin 'any ..) -> any}
\label{sec-8-1-16-29}


Prints the string representation of all \texttt{any} arguments to the current
output channel. No space or newline is printed between individual items,
or after the last item. For lists, all elements are \texttt{prin}'ted
recursively. See also \texttt{prinl}.


\begin{verbatim}
: (prin 'abc 123 '(a 1 b 2))
abc123a1b2-> (a 1 b 2)
\end{verbatim}

 
\section{(prinl 'any ..) -> any}
\label{sec-8-1-16-30}


Prints the string representation of all \texttt{any} arguments to the current
output channel, followed by a newline. No space or newline is printed
between individual items. For lists, all elements are \texttt{prin}'ted
recursively. See also \texttt{prin}.


\begin{verbatim}
: (prinl 'abc 123 '(a 1 b 2))
abc123a1b2
-> (a 1 b 2)
\end{verbatim}

 
\section{(print 'any ..) -> any}
\label{sec-8-1-16-31}


Prints all \texttt{any} arguments to the current output channel. If there is
more than one argument, a space is printed between successive arguments.
No space or newline is printed after the last item. See also \texttt{println},
\texttt{printsp}, \texttt{sym} and \texttt{str}


\begin{verbatim}
: (print 123)
123-> 123
: (print 1 2 3)
1 2 3-> 3
: (print '(a b c) 'def)
(a b c) def-> def
\end{verbatim}

 
\section{(println 'any ..) -> any}
\label{sec-8-1-16-32}


Prints all \texttt{any} arguments to the current output channel, followed by a
newline. If there is more than one argument, a space is printed between
successive arguments. See also \texttt{print}, \texttt{printsp}.


\begin{verbatim}
: (println '(a b c) 'def)
(a b c) def
-> def
\end{verbatim}

 
\section{(printsp 'any ..) -> any}
\label{sec-8-1-16-33}


Prints all \texttt{any} arguments to the current output channel, followed by a
space. If there is more than one argument, a space is printed between
successive arguments. See also \texttt{print}, \texttt{println}.


\begin{verbatim}
: (printsp '(a b c) 'def)
(a b c) def -> def
\end{verbatim}

 
\section{(prior 'lst1 'lst2) -> lst | NIL}
\label{sec-8-1-16-34}


Returns the cell in \texttt{lst2} which immediately precedes the cell \texttt{lst1},
or \texttt{NIL} if \texttt{lst1} is not found in \texttt{lst2} or is the very first cell.
==== is used for comparison (pointer equality). See also \texttt{offset} and
\texttt{memq}.


\begin{verbatim}
: (setq L (1 2 3 4 5 6))
-> (1 2 3 4 5 6)
: (setq X (cdddr L))
-> (4 5 6)
: (prior X L)
-> (3 4 5 6)
\end{verbatim}

 
\section{(proc 'sym ..) -> T}
\label{sec-8-1-16-35}


Shows a list of processes with command names given by the \texttt{sym}
arguments, using the system \texttt{ps} utility. See also \texttt{hd}.


\begin{verbatim}
: (proc 'pil)
  PID  PPID  STARTED  SIZE %CPU WCHAN  CMD
16993  3267 12:38:21  1516  0.5 -      /usr/bin/picolisp /usr/lib/picolisp/lib.l /usr/bin/pil +
15731  1834 12:36:35  2544  0.1 -      /usr/bin/picolisp /usr/lib/picolisp/lib.l /usr/bin/pil app/main.l -main -go +
15823 15731 12:36:44  2548  0.0 -        /usr/bin/picolisp /usr/lib/picolisp/lib.l /usr/bin/pil app/main.l -main -go +
-> T
\end{verbatim}

 
\section{(prog . prg) -> any}
\label{sec-8-1-16-36}


Executes \texttt{prg}, and returns the result of the last expression. See also
\texttt{nil}, \texttt{t}, \texttt{prog1} and \texttt{prog2}.


\begin{verbatim}
: (prog (print 1) (print 2) (print 3))
123-> 3
\end{verbatim}

 
\section{(prog1 'any1 . prg) -> any1}
\label{sec-8-1-16-37}


Executes all arguments, and returns the result of the first expression
\texttt{any1}. See also \texttt{nil}, \texttt{t}, \texttt{prog} and \texttt{prog2}.


\begin{verbatim}
: (prog1 (print 1) (print 2) (print 3))
123-> 1
\end{verbatim}

 
\section{(prog2 'any1 'any2 . prg) -> any2}
\label{sec-8-1-16-38}


Executes all arguments, and returns the result of the second expression
\texttt{any2}. See also \texttt{nil}, \texttt{t}, \texttt{prog} and \texttt{prog1}.


\begin{verbatim}
: (prog2 (print 1) (print 2) (print 3))
123-> 2
\end{verbatim}

 
\section{(prop 'sym1|lst ['sym2|cnt ..] 'sym) -> var}
\label{sec-8-1-16-39}


Fetches a property for a property key \texttt{sym} from a symbol. That symbol
is \texttt{sym1} (if no other arguments are given), or a symbol found by
applying the \texttt{get} algorithm to \texttt{sym1|lst} and the following arguments.
The property (the cell, not just its value) is returned, suitable for
direct (destructive) manipulations with functions expecting a \texttt{var}
argument. See also \texttt{::}.


\begin{verbatim}
: (put 'X 'cnt 0)
-> 0
: (prop 'X 'cnt)
-> (0 . cnt)
: (inc (prop 'X 'cnt))        # Directly manipulate the property value
-> 1
: (get 'X 'cnt)
-> 1
\end{verbatim}

 
\section{(protect . prg) -> any}
\label{sec-8-1-16-40}


Executes \texttt{prg}, and returns the result of the last expression. If a
signal is received during that time, its handling will be delayed until
the execution of \texttt{prg} is completed. See also \texttt{alarm},
\hyperref[refH.html-Hup]{*Hup}, \hyperref[refS.html\-Sig1]{*Sig\footnote{DEFINITION NOT FOUND: 12 } and \texttt{kill}.


\begin{verbatim}
: (protect (journal "db1.log" "db2.log"))
-> T
\end{verbatim}

 
\section{(prove 'lst ['lst]) -> lst}
\label{sec-8-1-16-41}


The \hyperref[ref.html-pilog]{Pilog} interpreter. Tries to prove the query list
in the first argument, and returns an association list of symbol-value
pairs, or \texttt{NIL} if not successful. The query list is modified as a side
effect, allowing subsequent calls to \texttt{prove} for further results. The
optional second argument may contain a list of symbols; in that case the
successful matches of rules defined for these symbols will be traced.
See also \texttt{goal}, \texttt{->} and \texttt{unify}.


\begin{verbatim}
: (prove (goal '((equal 3 3))))
-> T
: (prove (goal '((equal 3 @X))))
-> ((@X . 3))
: (prove (goal '((equal 3 4))))
-> NIL
\end{verbatim}

 
\section{(prune ['flg])}
\label{sec-8-1-16-42}


Optimizes memory usage by pruning in-memory leaf nodes of database
trees. Typically called repeatedly during heavy data imports. If \texttt{flg}
is non-=NIL=, further pruning will be disabled. See also \texttt{lieu}.


\begin{verbatim}
(in File1
   (while (someData)
      (new T '(+Cls1) ..)
      (at (0 . 10000) (commit) (prune)) ) )
(in File2
   (while (moreData)
      (new T '(+Cls2) ..)
      (at (0 . 10000) (commit) (prune)) ) )
(commit)
(prune T)
\end{verbatim}

 
\section{(push 'var 'any ..) -> any}
\label{sec-8-1-16-43}


Implements a stack using a list in \texttt{var}. The \texttt{any} arguments are
cons'ed in front of the value list. See also \texttt{push1}, \texttt{pop}, \texttt{queue} and
\texttt{fifo}.


\begin{verbatim}
: (push 'S 3)              # Use the VAL of 'S' as a stack
-> 3
: S
-> (3)
: (push 'S 2)
-> 2
: (push 'S 1)
-> 1
: S
-> (1 2 3)
: (push S 999)             # Now use the CAR of the list in 'S'
-> 999
: (push S 888 777)
-> 777
: S
-> ((777 888 999 . 1) 2 3)
\end{verbatim}

 
\section{(push1 'var 'any ..) -> any}
\label{sec-8-1-16-44}


Maintains a unique list in \texttt{var}. Each \texttt{any} argument is cons'ed in
front of the value list only if it is not already a \texttt{member} of that
list. See also \texttt{push}, \texttt{pop} and \texttt{queue}.


\begin{verbatim}
: (push1 'S 1 2 3)
-> 3
: S
-> (3 2 1)
: (push1 'S 2 4)
-> 4
: S
-> (4 3 2 1)
\end{verbatim}

 
\section{(put 'sym1|lst ['sym2|cnt ..] 'sym|0 'any) -> any}
\label{sec-8-1-16-45}


Stores a new value \texttt{any} for a property key \texttt{sym} (or in the value cell
for zero) in a symbol. That symbol is \texttt{sym1} (if no other arguments are
given), or a symbol found by applying the \texttt{get} algorithm to \texttt{sym1|lst}
and the following arguments. See also ==:=.


\begin{verbatim}
: (put 'X 'a 1)
-> 1
: (get 'X 'a)
-> 1
: (prop 'X 'a)
-> (1 . a)

: (setq L '(A B C))
-> (A B C)
: (setq B 'D)
-> D
: (put L 2 0 'p 5)  # Store '5' under the 'p' propery of the value of 'B'
-> 5
: (getl 'D)
-> ((5 . p))
\end{verbatim}

 
\section{(put! 'obj 'sym 'any) -> any}
\label{sec-8-1-16-46}


\hyperref[ref.html-trans]{Transaction} wrapper function for \texttt{put}. Note that
for setting property values of entities typically the \texttt{put!>} message is
used. See also \texttt{new!}, \texttt{set!} and \texttt{inc!}.


\begin{verbatim}
(put! Obj 'cnt 0)  # Setting a property of a non-entity object
\end{verbatim}

 
\section{(putl 'sym1|lst1 ['sym2|cnt ..] 'lst) -> lst}
\label{sec-8-1-16-47}


Stores a complete new property list \texttt{lst} in a symbol. That symbol is
\texttt{sym1} (if no other arguments are given), or a symbol found by applying
the \texttt{get} algorithm to \texttt{sym1|lst1} and the following arguments. All
previously defined properties for that symbol are lost. See also \texttt{getl}
and \texttt{maps}.


\begin{verbatim}
: (putl 'X '((123 . a) flg ("Hello" . b)))
-> ((123 . a) flg ("Hello" . b))
: (get 'X 'a)
-> 123
: (get 'X 'b)
-> "Hello"
: (get 'X 'flg)
-> T
\end{verbatim}

 
\section{(pwd) -> sym}
\label{sec-8-1-16-48}


Returns the path to the current working directory. See also \texttt{dir} and
\texttt{cd}.


\begin{verbatim}
: (pwd)
-> "/home/app/"
\end{verbatim}


\chapter{Functions starting with Q}
\label{sec-8-1-17}


 
\section{(qsym . sym) -> lst}
\label{sec-8-1-17-1}


Returns a cons pair of the value and property list of \texttt{sym}. See also
\texttt{quote}, \texttt{val} and \texttt{getl}.


\begin{verbatim}
: (setq A 1234)
-> 1234
: (put 'A 'a 1)
-> 1
: (put 'A 'b 2)
-> 2
: (put 'A 'f T)
-> T
: (qsym . A)
-> (1234 f (2 . b) (1 . a))
\end{verbatim}

 
\section{(quote . any) -> any}
\label{sec-8-1-17-2}


Returns \texttt{any} unevaluated. The reader recognizes the single quote char
='= as a macro for this function. See also \texttt{lit}.


\begin{verbatim}
: 'a
-> a
: '(foo a b c)
-> (foo a b c)
: (quote (quote (quote a)))
-> ('('(a)))
\end{verbatim}

 
\section{(query 'lst ['lst]) -> flg}
\label{sec-8-1-17-3}


Handles an interactive \hyperref[ref.html-pilog]{Pilog} query. The two \texttt{lst}
arguments are passed to \texttt{prove}. \texttt{query} displays each result, waits for
console input, and terminates when a non-empty line is entered. See also
\texttt{?}, \texttt{pilog} and \texttt{solve}.


\begin{verbatim}
: (query (goal '((append @X @Y (a b c)))))
 @X=NIL @Y=(a b c)
 @X=(a) @Y=(b c).   # Stop
-> NIL
\end{verbatim}

 
\section{(queue 'var 'any) -> any}
\label{sec-8-1-17-4}


Implements a queue using a list in \texttt{var}. The \texttt{any} argument is
(destructively) concatenated to the end of the value list. See also
\texttt{push}, \texttt{pop} and \texttt{fifo}.


\begin{verbatim}
: (queue 'A 1)
-> 1
: (queue 'A 2)
-> 2
: (queue 'A 3)
-> 3
: A
-> (1 2 3)
: (pop 'A)
-> 1
: A
-> (2 3)
\end{verbatim}

 
\section{(quit ['any ['any]])}
\label{sec-8-1-17-5}


Stops current execution. If no arguments are given, all pending
\texttt{finally} expressions are executed and control is returned to the top
level read-eval-print loop. Otherwise, an error handler is entered. The
first argument can be some error message, and the second might be the
reason for the error. See also \texttt{Error Handling}.


\begin{verbatim}
: (de foo (X) (quit "Sorry, my error" X))
-> foo
: (foo 123)                                  # 'X' is bound to '123'
123 -- Sorry, my error                       # Error entered
? X                                          # Inspect 'X'
-> 123
?                                            # Empty line: Exit
:
\end{verbatim}


\chapter{Functions starting with R}
\label{sec-8-1-18}


 
\section{*Run}
\label{sec-8-1-18-1}


This global variable can hold a list of \texttt{prg} expressions which are used
during \texttt{key}, \texttt{sync}, \texttt{wait} and \texttt{listen}. The first element of each
expression must either be a positive number (thus denoting a file
descriptor to wait for) or a negative number (denoting a timeout value
in milliseconds (in that case another number must follow to hold the
remaining time)). A \texttt{select} system call is performed with these values,
and the corresponding \texttt{prg} body is executed when input data are
available or when a timeout occurred. See also \texttt{task}.


\begin{verbatim}
: (de *Run (-2000 0 (println '2sec)))     # Install 2-sec-timer
-> *Run
: 2sec                                    # Prints "2sec" every 2 seconds
2sec
2sec
                                          # (Ctrl-D) Exit
$
\end{verbatim}

 
\section{+Ref}
\label{sec-8-1-18-2}


Prefix class for maintaining non-unique indexes to \texttt{+relation=s, a subclass of =+index}. Accepts an optional argument for a \texttt{+Hook}
attribute. See also \texttt{Database}.


\begin{verbatim}
(rel tel (+Fold +Ref +String))  # Phone number with folded, non-unique index
\end{verbatim}

 
\section{+Ref2}
\label{sec-8-1-18-3}


Prefix class for maintaining a secondary (``backing'') index to
\texttt{+relation=s. Can only be used as a prefix class to =+Key} or \texttt{+Ref}. It
maintains an index in the current (sub)class, in addition to that in one
of the superclasses, to allow (sub)class-specific queries. See also
\texttt{Database}.


\begin{verbatim}
(class +Ord +Entity)             # Order class
(rel nr (+Need +Key +Number))    # Order number
...
(class +EuOrd +Ord)              # EU-specific order subclass
(rel nr (+Ref2 +Key +Number))    # Order number with backing index
\end{verbatim}

 
\section{+relation}
\label{sec-8-1-18-4}


Abstract base class of all database releations. Relation objects are
usually defined with \texttt{rel}. The class hierarchy includes the classes
\texttt{+Any}, \texttt{+Bag}, \texttt{+Bool}, \texttt{+Number}, \texttt{+Date}, \texttt{+Time}, \texttt{+Symbol},
\texttt{+String}, \texttt{+Link}, \texttt{+Joint} and \texttt{+Blob}, and the prefix classes
\texttt{+Hook}, \texttt{+index}, \texttt{+Key}, \texttt{+Ref}, \texttt{+Ref2}, \texttt{+Idx}, \texttt{+Sn}, \texttt{+Fold},
\texttt{+Aux}, \texttt{+UB}, \texttt{+Dep}, \texttt{+List}, \texttt{+Need}, \texttt{+Mis} and \texttt{+Alt}. See also
\texttt{Database} and \texttt{+Entity}.

Messages to relation objects include


\begin{verbatim}
mis> (Val Obj)       # Return error if mismatching type or value
has> (Val X)         # Check if the value is present
put> (Obj Old New)   # Put new value
rel> (Obj Old New)   # Maintain relational strutures
lose> (Obj Val)      # Delete relational structures
keep> (Obj Val)      # Restore deleted relational structures
zap> (Obj Val)       # Clean up relational structures
\end{verbatim}

 
\section{(rand ['cnt1 'cnt2] | ['T]) -> cnt | flg}
\label{sec-8-1-18-5}


Returns a pseudo random number in the range cnt1 .. cnt2 (or
--2147483648 .. +2147483647 if no arguments are given). If the argument
is \texttt{T}, a boolean value \texttt{flg} is returned. See also \texttt{seed}.


\begin{verbatim}
: (rand 3 9)
-> 3
: (rand 3 9)
-> 7
\end{verbatim}

 
\section{(range 'num1 'num2 ['num3]) -> lst}
\label{sec-8-1-18-6}


Produces a list of numbers in the range \texttt{num1} through \texttt{num2}. When
\texttt{num3} is non-=NIL=), it is used to increment \texttt{num1} (if it is smaller
than \texttt{num2}) or to decrement \texttt{num1} (if it is greater than \texttt{num2}). See
also \texttt{need}.


\begin{verbatim}
: (range 1 6)
-> (1 2 3 4 5 6)
: (range 6 1)
-> (6 5 4 3 2 1)
: (range -3 3)
-> (-3 -2 -1 0 1 2 3)
: (range 3 -3 2)
-> (3 1 -1 -3)
\end{verbatim}

 
\section{range/3}
\label{sec-8-1-18-7}


\hyperref[ref.html-pilog]{Pilog} predicate that succeeds if the first argument
is in the range of the result of applying the \texttt{get} algorithm to the
following arguments. Typically used as filter predicate in \texttt{select/3}
database queries. See also \texttt{Comparing}, \texttt{isa/2}, \texttt{same/3}, \texttt{bool/3},
\texttt{head/3}, \texttt{fold/3}, \texttt{part/3} and \texttt{tolr/3}.


\begin{verbatim}
: (?
   @Nr (1 . 5)  # Numbers between 1 and 5
   @Nm "part"
   (select (@Item)
      ((nr +Item @Nr) (nm +Item @Nm))
      (range @Nr @Item nr)
      (part @Nm @Item nm) ) )
 @Nr=(1 . 5) @Nm="part" @Item={3-1}                                              @Nr=(1 . 5) @Nm="part" @Item={3-2}
-> NIL
\end{verbatim}

 
\section{(rank 'any 'lst ['flg]) -> lst}
\label{sec-8-1-18-8}


Searches a ranking list. \texttt{lst} should be sorted. Returns the element
from \texttt{lst} with a maximal CAR less or equal to \texttt{any} (if \texttt{flg} is
\texttt{NIL}), or with a minimal CAR greater or equal to \texttt{any} (if \texttt{flg} is
non-=NIL=), or \texttt{NIL} if no match is found. See also \texttt{assoc} and
\hyperref[ref.html-cmp]{Comparing}.


\begin{verbatim}
: (rank 0 '((1 . a) (100 . b) (1000 . c)))
-> NIL
: (rank 50 '((1 . a) (100 . b) (1000 . c)))
-> (1 . a)
: (rank 100 '((1 . a) (100 . b) (1000 . c)))
-> (100 . b)
: (rank 300 '((1 . a) (100 . b) (1000 . c)))
-> (100 . b)
: (rank 9999 '((1 . a) (100 . b) (1000 . c)))
-> (1000 . c)
: (rank 50 '((1000 . a) (100 . b) (1 . c)) T)
-> (100 . b)
\end{verbatim}

 
\section{(raw ['flg]) -> flg}
\label{sec-8-1-18-9}


Console mode control function. When called without arguments, it returns
the current console mode (\texttt{NIL} for ``cooked mode''). Otherwise, the
console is set to the new state. See also \texttt{key}.


\begin{verbatim}
$ pil
: (raw)
-> NIL
$ pil +
: (raw)
-> T
\end{verbatim}

 
\section{(rc 'sym 'any1 ['any2]) -> any}
\label{sec-8-1-18-10}


Fetches a value from a resource file \texttt{sym}, or stores a value \texttt{any2} in
that file, using a key \texttt{any1}. All values are stored in a list in the
file, using \texttt{assoc}. During the whole operation, the file is exclusively
locked with \texttt{ctl}.


\begin{verbatim}
: (info "a.rc")               # File exists?
-> NIL                        # No
: (rc "a.rc" 'a 1)            # Store 1 for 'a'
-> 1
: (rc "a.rc" 'b (2 3 4))      # Store (2 3 4) for 'b'
-> (2 3 4)
: (rc "a.rc" 'c 'b)           # Store 'b' for 'c'
-> b
: (info "a.rc")               # Check file
-> (28 733124 . 61673)
: (in "a.rc" (echo))          # Display it
((c . b) (b 2 3 4) (a . 1))
-> T
: (rc "a.rc" 'c)              # Fetch value for 'c'
-> b
: (rc "a.rc" @)               # Fetch value for 'b'
-> (2 3 4)
\end{verbatim}

 
\section{(rd ['sym]) -> any}
\label{sec-8-1-18-11}


\texttt{(rd 'cnt) -> num | NIL}

Binary read: Reads one item from the current input channel in encoded
binary format. When called with a \texttt{cnt} argument (second form), that
number of raw bytes (in big endian format if \texttt{cnt} is positive,
otherwise little endian) is read as a single number. Upon end of file,
if the \texttt{sym} argument is given, it is returned, otherwise \texttt{NIL}. See
also \texttt{pr}, \texttt{tell}, \texttt{hear} and \texttt{wr}.


\begin{verbatim}
: (out "x" (pr 'abc "EOF" 123 "def"))
-> "def"
: (in "x" (rd))
-> abc
: (in "x"
   (make
      (use X
         (until (== "EOF" (setq X (rd "EOF")))  # '==' detects end of file
            (link X) ) ) ) )
-> (abc "EOF" 123 "def")  # as opposed to reading a symbol "EOF"

: (in "/dev/urandom" (rd 20))
-> 396737673456823753584720194864200246115286686486
\end{verbatim}

 
\section{(read ['sym1 ['sym2]]) -> any}
\label{sec-8-1-18-12}


Reads one item from the current input channel. \texttt{NIL} is returned upon
end of file. When called without arguments, an arbitrary Lisp expression
is read. Otherwise, a token (a number, or an internal or transient
symbol) is read. In that case, \texttt{sym1} specifies which set of characters
to accept for continuous symbol names (in addition to the standard
alphanumerical characters), and \texttt{sym2} an optional comment character.
See also \texttt{any}, \texttt{str}, \texttt{skip} and \texttt{eof}.


\begin{verbatim}
: (list (read) (read) (read))    # Read three things from console
123                              # a number
abcd                             # a symbol
(def                             # and a list
ghi
jkl
)
-> (123 abcd (def ghi jkl))
: (make (while (read "_" "#") (link @)))
abc = def_ghi("xyz"+-123) # Comment
NIL
-> (abc "=" def_ghi "(" "xyz" "+" "-" 123 ")")
\end{verbatim}

 
\section{(recur fun) -> any}
\label{sec-8-1-18-13}


\texttt{(recurse ..) -> any}

Implements anonymous recursion, by defining the function \texttt{recurse} on
the fly. During the execution of \texttt{fun}, the symbol \texttt{recurse} is bound to
the function definition \texttt{fun}. See also \texttt{let} and \texttt{lambda}.


\begin{verbatim}
: (de fibonacci (N)
   (when (lt0 N)
      (quit "Bad fibonacci" N) )
   (recur (N)
      (if (> 2 N)
         1
         (+
            (recurse (dec N))
            (recurse (- N 2)) ) ) ) )
-> fibonacci
: (fibonacci 22)
-> 28657
: (fibonacci -7)
-7 -- Bad fibonacci
\end{verbatim}

 
\section{(redef sym . fun) -> sym}
\label{sec-8-1-18-14}


Redefines \texttt{sym} in terms of itself. The current definition is saved in a
new symbol, which is substituted for each occurrence of \texttt{sym} in \texttt{fun},
and which is also returned. See also \texttt{de}, \texttt{undef}, \texttt{daemon} and
\texttt{patch}.


\begin{verbatim}
: (de hello () (prinl "Hello world!"))
-> hello
: (pp 'hello)
(de hello NIL
   (prinl "Hello world!") )
-> hello

: (redef hello (A B)
   (println 'Before A)
   (prog1 (hello) (println 'After B)) )
-> "hello"
: (pp 'hello)
(de hello (A B)
   (println 'Before A)
   (prog1 ("hello") (println 'After B)) )
-> hello
: (hello 1 2)
Before 1
Hello world!
After 2
-> "Hello world!"

: (redef * @
   (msg (rest))
   (pass *) )
-> "*"
: (* 1 2 3)
(1 2 3)
-> 6

: (redef + @
   (pass (ifn (num? (next)) pack +) (arg)) )
-> "+"
: (+ 1 2 3)
-> 6
: (+ "a" 'b '(c d e))
-> "abcde"
\end{verbatim}

 
\section{(rel var lst [any ..]) -> any}
\label{sec-8-1-18-15}


Defines a relation for \texttt{var} in the current class \texttt{*Class}, using \texttt{lst}
as the list of classes for that relation, and possibly additional
arguments \texttt{any} for its initialization. See also
\hyperref[ref.html-dbase]{Database}, \hyperref[refC.html-class]{class},
\hyperref[refE.html-extend]{extend}, \hyperref[refD.html-dm]{dm} and
\hyperref[refV.html-var]{var}.


\begin{verbatim}
(class +Person +Entity)
(rel nm  (+List +Ref +String))            # Names
(rel tel (+Ref +String))                  # Telephone
(rel adr (+Joint) prs (+Address))         # Address

(class +Address +Entity)
(rel cit (+Need +Hook +Link) (+City))     # City
(rel str (+List +Ref +String) cit)        # Street
(rel prs (+List +Joint) adr (+Person))    # Inhabitants

(class +City +Entity)
(rel nm  (+List +Ref +String))            # Zip / Names
\end{verbatim}

 
\section{(release 'sym) -> NIL}
\label{sec-8-1-18-16}


Releases the mutex represented by the file `sym'. This is the reverse
operation of \texttt{acquire}.


\begin{verbatim}
: (release "sema1")
-> NIL
\end{verbatim}

 
\section{remote/2}
\label{sec-8-1-18-17}


\hyperref[ref.html-pilog]{Pilog} predicate for remote database queries. It
takes a list and an arbitrary number of clauses. The list should contain
a Pilog variable for the result in the CAR, and a list of resources in
the CDR. The clauses will be evaluated on remote machines according to
these resources. Each resource must be a cons pair of two functions, an
``out'' function in the CAR, and an ``in'' function in the CDR. See also
\texttt{*Ext}, \texttt{select/3} and \texttt{db/3}.


\begin{verbatim}
(setq *Ext           # Set up external offsets
   (mapcar
      '((@Host @Ext)
         (cons @Ext
            (curry (@Host @Ext (Sock)) (Obj)
               (when (or Sock (setq Sock (connect @Host 4040)))
                  (ext @Ext
                     (out Sock (pr (cons 'qsym Obj)))
                     (prog1 (in Sock (rd))
                        (unless @
                           (close Sock)
                           (off Sock) ) ) ) ) ) ) )
      '("localhost")
      '(20) ) )

(de rsrc ()  # Simple resource handler, ignoring errors or EOFs
   (extract
      '((@Ext Host)
         (let? @Sock (connect Host 4040)
            (cons
               (curry (@Ext @Sock) (X)  # out
                  (ext @Ext (out @Sock (pr X))) )
               (curry (@Ext @Sock) ()  # in
                  (ext @Ext (in @Sock (rd))) ) ) ) )
      '(20)
      '("localhost") ) )

: (?
   @Nr (1 . 3)
   @Sup 2
   @Rsrc (rsrc)
   (remote (@Item . @Rsrc)
      (db nr +Item @Nr @Item)
      (val @Sup @Item sup nr) )
   (show @Item) )
{L-2} (+Item)
   pr 1250
   inv 100
   sup {K-2}
   nm Spare Part
   nr 2
 @Nr=(1 . 3) @Sup=2 @Rsrc=((((X) (ext 20 (out 16 (pr X)))) NIL (ext 20 (in 16 (rd))))) @Item={L-2}
-> NIL
\end{verbatim}

 
\section{(remove 'cnt 'lst) -> lst}
\label{sec-8-1-18-18}


Removes the element at position \texttt{cnt} from \texttt{lst}. This is a
non-destructive operation. See also \texttt{insert}, \texttt{place}, \texttt{append},
\texttt{delete} and \texttt{replace}.


\begin{verbatim}
: (remove 3 '(a b c d e))
-> (a b d e)
: (remove 1 '(a b c d e))
-> (b c d e)
: (remove 9 '(a b c d e))
-> (a b c d e)
\end{verbatim}

 
\section{(repeat) -> lst}
\label{sec-8-1-18-19}


Makes the current \hyperref[ref.html-pilog]{Pilog} definition ``tail recursive'',
by closing the previously defined clauses in the T property to a
circular list. See also \texttt{be}.


\begin{verbatim}
(be a (1))     # Define three facts
(be a (2))
(be a (3))
(repeat)       # Unlimited supply

: (? (a @N))
 @N=1
 @N=2
 @N=3
 @N=1
 @N=2
 @N=3.         # Stop
-> NIL
\end{verbatim}

 
\section{repeat/0}
\label{sec-8-1-18-20}


\hyperref[ref.html-pilog]{Pilog} predicate that always succeeds, also on
backtracking. See also \texttt{repeat} and \texttt{true/0}.


\begin{verbatim}
: (be int (@N)       # Generate unlimited supply of integers
   (@ zero *N)
   (repeat)          # Repeat from here
   (@N inc '*N) )
-> int

:  (? (int @X))
 @X=1
 @X=2
 @X=3
 @X=4.               # Stop
-> NIL
\end{verbatim}

 
\section{(replace 'lst 'any1 'any2 ..) -> lst}
\label{sec-8-1-18-21}


Replaces in \texttt{lst} all occurrences of \texttt{any1} with \texttt{any2}. For optional
additional argument pairs, this process is repeated. This is a
non-destructive operation. See also \texttt{append}, \texttt{delete}, \texttt{insert},
\texttt{remove} and \texttt{place}.


\begin{verbatim}
: (replace '(a b b a) 'a 'A)
-> (A b b A)
: (replace '(a b b a) 'b 'B)
-> (a B B a)
: (replace '(a b b a) 'a 'B 'b 'A)
-> (B A A B)
\end{verbatim}

 
\section{(request 'typ 'var ['hook] 'val ..) -> obj}
\label{sec-8-1-18-22}


Returns a database object. If a matching object cannot be found (using
\texttt{db}), a new object of the given type is created (using \texttt{new}). See also
\texttt{obj}.


\begin{verbatim}
: (request '(+Item) 'nr 2)
-> {3-2}
\end{verbatim}

 
\section{(rest) -> lst}
\label{sec-8-1-18-23}


Can only be used inside functions with a variable number of arguments
(with \texttt{@}). Returns the list of all remaining arguments from the
internal list. See also \texttt{args}, \texttt{next}, \texttt{arg} and \texttt{pass}.


\begin{verbatim}
: (de foo @ (println (rest)))
-> foo
: (foo 1 2 3)
(1 2 3)
-> (1 2 3)
\end{verbatim}

(retract) -> lst==

Removes a \hyperref[ref.html-pilog]{Pilog} fact or rule. See also \texttt{be},
\texttt{clause}, \texttt{asserta} and \texttt{assertz}.


\begin{verbatim}
: (be a (1))
-> a
: (be a (2))
-> a
: (be a (3))
-> a

: (retract '(a (2)))
-> (((1)) ((3)))

:  (? (a @N))
 @N=1
 @N=3
-> NIL
\end{verbatim}

 
\section{retract/1}
\label{sec-8-1-18-24}


\hyperref[ref.html-pilog]{Pilog} predicate that removes a fact or rule. See
also \texttt{retract}, \texttt{asserta/1} and \texttt{assertz/1}.


\begin{verbatim}
: (be a (1))
-> a
: (be a (2))
-> a
: (be a (3))
-> a

: (? (retract (a 2)))
-> T
: (rules 'a)
1 (be a (1))
2 (be a (3))
-> a
\end{verbatim}

 
\section{(reverse 'lst) -> lst}
\label{sec-8-1-18-25}


Returns a reversed copy of \texttt{lst}. See also \texttt{flip}.


\begin{verbatim}
: (reverse (1 2 3 4))
-> (4 3 2 1)
\end{verbatim}

 
\section{(rewind) -> flg}
\label{sec-8-1-18-26}


Sets the file position indicator for the current output stream to the
beginning of the file, and truncates the file length to zero. Returns
\texttt{T} when successful. See also \texttt{flush}.


\begin{verbatim}
: (out "a" (prinl "Hello world"))
-> "Hello world"
: (in "a" (echo))
Hello world
-> T
: (info "a")
-> (12 733216 . 53888)
: (out "a" (rewind))
-> T
: (info "a")
-> (0 733216 . 53922)
\end{verbatim}

 
\section{(rollback) -> T}
\label{sec-8-1-18-27}


Cancels a transaction, by discarding all modifications of external
symbols. See also \texttt{commit}.


\begin{verbatim}
: (pool "db")
-> T
# .. Modify external objects ..
: (rollback)            # Rollback
-> T
\end{verbatim}

 
\section{(root 'tree) -> (num . sym)}
\label{sec-8-1-18-28}


Returns the root of a database index tree, with the number of entries in
\texttt{num}, and the base node in \texttt{sym}. See also \texttt{tree}.


\begin{verbatim}
: (root (tree 'nr '+Item))
-> (7 . {7-1})
\end{verbatim}

 
\section{(rot 'lst ['cnt]) -> lst}
\label{sec-8-1-18-29}


Rotate: The contents of the cells of \texttt{lst} are (destructively) shifted
right, and the value from the last cell is stored in the first cell.
Without the optional \texttt{cnt} argument, the whole list is rotated,
otherwise only the first \texttt{cnt} elements. See also \texttt{flip} .


\begin{verbatim}
: (rot (1 2 3 4))             # Rotate all four elements
-> (4 1 2 3)
: (rot (1 2 3 4 5 6) 3)       # Rotate only the first three elements
-> (3 1 2 4 5 6)
\end{verbatim}

 
\section{(round 'num1 'num2) -> sym}
\label{sec-8-1-18-30}


Formats a number \texttt{num1} with \texttt{num2} decimal places, according to the
current scale \texttt{*Scl}. \texttt{num2} defaults to 3. See also
\hyperref[ref.html-num-io]{Numbers} and \texttt{format}.


\begin{verbatim}
: (scl 4)               # Set scale to 4
-> 4
: (round 123456)        # Format with three decimal places
-> "12.346"
: (round 123456 2)      # Format with two decimal places
-> "12.35"
: (format 123456 *Scl)  # Format with full precision
-> "12.3456"
\end{verbatim}

 
\section{(rules 'sym ..) -> sym}
\label{sec-8-1-18-31}


Prints all rules defined for the \texttt{sym} arguments. See also
\hyperref[ref.html-pilog]{Pilog} and \texttt{be}.


\begin{verbatim}
: (rules 'member 'append)
1 (be member (@X (@X . @)))
2 (be member (@X (@ . @Y)) (member @X @Y))
1 (be append (NIL @X @X))
2 (be append ((@A . @X) @Y (@A . @Z)) (append @X @Y @Z))
-> append
\end{verbatim}

 
\section{(run 'any ['cnt ['lst]]) -> any}
\label{sec-8-1-18-32}


If \texttt{any} is an atom, \texttt{run} behaves like \texttt{eval}. Otherwise \texttt{any} is a
list, which is evaluated in sequence. The last result is returned. If a
binding environment offset \texttt{cnt} is given, that evaluation takes place
in the corresponding environment, and an optional \texttt{lst} of excluded
symbols can be supplied. See also \texttt{up}.


\begin{verbatim}
: (run '((println (+ 1 2 3)) (println 'OK)))
6
OK
-> OK
\end{verbatim}


\chapter{Functions starting with S}
\label{sec-8-1-19}


 
\section{*Scl}
\label{sec-8-1-19-1}


A global variable holding the current fixpoint input scale. See also
\hyperref[ref.html-num-io]{Numbers} and \texttt{scl}.


\begin{verbatim}
: (str "123.45")  # Default value of '*Scl' is 0
-> (123)
: (setq *Scl 3)
-> 3
: (str "123.45")
-> (123450)
\end{verbatim}

 
\section{*Sig1}
\label{sec-8-1-19-2}


\texttt{*Sig2}

Global variables holding (possibly empty) \texttt{prg} bodies, which will be
executed when a SIGUSR1 signal (or a SIGUSR2 signal, respectively) is
sent to the current process. See also \texttt{alarm}, \texttt{sigio} and \texttt{*Hup}.


\begin{verbatim}
: (de *Sig1 (msg 'SIGUSR1))
-> *Sig1
\end{verbatim}

 
\section{*Solo}
\label{sec-8-1-19-3}


A global variable indicating exclusive database access. Its value is \texttt{0}
initially, set to \texttt{T} (or \texttt{NIL}) during cooperative database locks when
\texttt{lock} is successfully called with a \texttt{NIL} (or non-=NIL=) argument. See
also \texttt{*Zap}.


\begin{verbatim}
: *Solo
-> 0
: (lock *DB)
-> NIL
: *Solo
-> NIL
: (rollback)
-> T
: *Solo
-> 0
: (lock)
-> NIL
: *Solo
-> T
: (rollback)
-> T
: *Solo
-> T
\end{verbatim}

 
\section{+Sn}
\label{sec-8-1-19-4}


Prefix class for maintaining indexes according to a modified soundex
algorithm, for tolerant name searches, to \texttt{+String} relations. Typically
used in combination with the \texttt{+Idx} prefix class. See also \texttt{Database}.


\begin{verbatim}
(rel nm (+Sn +Idx +String))  # Name
\end{verbatim}

 
\section{+String}
\label{sec-8-1-19-5}


Class for string (transient symbol) relations, a subclass of \texttt{+Symbol}.
Accepts an optional argument for the string length (currently not used).
See also \texttt{Database}.


\begin{verbatim}
(rel nm (+Sn +Idx +String))  # Name, indexed by soundex and substrings
\end{verbatim}

 
\section{+Symbol}
\label{sec-8-1-19-6}


Class for symbolic relations, a subclass of \texttt{+relation}. Objects of that
class typically maintain internal symbols, as opposed to the more
often-used \texttt{+String} for transient symbols. See also \texttt{Database}.


\begin{verbatim}
(rel perm (+List +Symbol))  # Permission list
\end{verbatim}

 
\section{same/3}
\label{sec-8-1-19-7}


\hyperref[ref.html-pilog]{Pilog} predicate that succeeds if the first argument
matches the result of applying the \texttt{get} algorithm to the following
arguments. Typically used as filter predicate in \texttt{select/3} database
queries. See also \texttt{isa/2}, \texttt{bool/3}, \texttt{range/3}, \texttt{head/3}, \texttt{fold/3},
\texttt{part/3} and \texttt{tolr/3}.


\begin{verbatim}
: (?
   @Nr 2
   @Nm "Spare"
   (select (@Item)
      ((nr +Item @Nr) (nm +Item @Nm))
      (same @Nr @Item nr)
      (head @Nm @Item nm) ) )
 @Nr=2 @Nm="Spare" @Item={3-2}
\end{verbatim}

 
\section{(scan 'tree ['fun] ['any1] ['any2] ['flg])}
\label{sec-8-1-19-8}


Scans through a database tree by applying \texttt{fun} to all key-value pairs.
\texttt{fun} should be a function accepting two arguments for key and value. It
defaults to \texttt{println}. \texttt{any1} and \texttt{any2} may specify a range of keys. If
\texttt{any2} is greater than \texttt{any1}, the traversal will be in opposite
direction. Note that the keys need not to be atomic, depending on the
application's index structure. If \texttt{flg} is non-=NIL=, partial keys are
skipped. See also \texttt{tree}, \texttt{iter}, \texttt{init} and \texttt{step}.


\begin{verbatim}
: (scan (tree 'nm '+Item))
("ASLRSNSTRSTN" {3-3} . T) {3-3}
("Additive" {3-4}) {3-4}
("Appliance" {3-6}) {3-6}
("Auxiliary Construction" . {3-3}) {3-3}
("Construction" {3-3}) {3-3}
("ENNSNNTTTF" {3-4} . T) {3-4}
("Enhancement Additive" . {3-4}) {3-4}
("Fittings" {3-5}) {3-5}
("GTSTFLNS" {3-6} . T) {3-6}
("Gadget Appliance" . {3-6}) {3-6}
...

: (scan (tree 'nm '+Item) println NIL T T)  # 'flg' is non-NIL
("Auxiliary Construction" . {3-3}) {3-3}
("Enhancement Additive" . {3-4}) {3-4}
("Gadget Appliance" . {3-6}) {3-6}
("Main Part" . {3-1}) {3-1}
("Metal Fittings" . {3-5}) {3-5}
("Spare Part" . {3-2}) {3-2}
("Testartikel" . {3-8}) {3-8}
-> {7-6}
\end{verbatim}

 
\section{(scl 'num) -> num}
\label{sec-8-1-19-9}


Sets \texttt{*Scl} globally to \texttt{num}. See also \hyperref[ref.html-num-io]{Numbers}.


\begin{verbatim}
: (scl 0)
-> 0
: (str "123.45")
-> (123)
: (scl 1)
-> 1
: (read)
123.45
-> 1235
: (scl 3)
-> 3
: (str "123.45")
-> (123450)
\end{verbatim}

 
\section{(script 'any ..) -> any}
\label{sec-8-1-19-10}


The first \texttt{any} argument is \texttt{load=ed, with the remaining arguments =pass=ed as variable arguments. They can be accessed with =next}, \texttt{arg},
\texttt{args} and \texttt{rest}.


\begin{verbatim}
$ cat x
(* (next) (next))

$ pil +
: (script "x" 3 4)
-> 12
\end{verbatim}

 
\section{(sect 'lst 'lst) -> lst}
\label{sec-8-1-19-11}


Returns the intersection of the \texttt{lst} arguments. See also \texttt{diff}.


\begin{verbatim}
: (sect (1 2 3 4) (3 4 5 6))
-> (3 4)
: (sect (1 2 3) (4 5 6))
-> NIL
\end{verbatim}

 
\section{(seed 'any) -> cnt}
\label{sec-8-1-19-12}


Initializes the random generator's seed, and returns a pseudo random
number in the range --2147483648 .. +2147483647. See also \texttt{rand} and
\texttt{hash}.


\begin{verbatim}
: (seed "init string")
-> 2015582081
: (rand)
-> -706917003
: (rand)
-> 1224196082

: (seed (time))
-> 128285383
\end{verbatim}

 
\section{(seek 'fun 'lst ..) -> lst}
\label{sec-8-1-19-13}


Applies \texttt{fun} to \texttt{lst} and all successive CDRs, until non-=NIL= is
returned. Returns the tail of \texttt{lst} starting with that element, or \texttt{NIL}
if \texttt{fun} did not return non-=NIL= for any element of \texttt{lst}. When
additional \texttt{lst} arguments are given, they are passed to \texttt{fun} in the
same way. See also \texttt{find}, \texttt{pick}.


\begin{verbatim}
: (seek '((X) (> (car X) 9)) (1 5 8 12 19 22))
-> (12 19 22)
\end{verbatim}

 
\section{(select [var ..] cls [hook|T] [var val ..]) -> obj | NIL}
\label{sec-8-1-19-14}


Interactive database function, loosely modelled after the SQL `=SELECT='
command. A (limited) front-end to the Pilog \texttt{select/3} predicate. When
called with only a \texttt{cls} argument, \texttt{select} steps through all objects of
that class, and \texttt{show=s their complete contents (this is analog to 'SELECT * from CLS'). If =cls} is followed by attribute/value
specifications, the search is limited to these values (this is analog to
`SELECT * from CLS where VAR = VAL'). If between the \texttt{select} function
and \texttt{cls} one or several attribute names are supplied, only these
attribute (instead of the full \texttt{show}) are printed. These attribute
specifications may also be lists, then the \texttt{get} algorithm will be used
to retrieve related data. See also \texttt{update}, \texttt{Database} and
\hyperref[ref.html-pilog]{Pilog}.


\begin{verbatim}
: (select +Item)                       # Show all items
{3-1} (+Item)
   nr 1
   pr 29900
   inv 100
   sup {2-1}
   nm "Main Part"

{3-2} (+Item)
   nr 2
   pr 1250
   inv 100
   sup {2-2}
   nm "Spare Part"
.                                      # Stop
-> {3-2}

: (select +Item nr 3)                  # Show only item 3
{3-3} (+Item)
   nr 3
   sup {2-1}
   pr 15700
   nm "Auxiliary Construction"
   inv 100
.                                      # Stop
-> {3-3}

# Show selected attributes for items 3 through 3
: (select nr nm pr (sup nm) +Item nr (3 . 5))
3 "Auxiliary Construction" 157.00 "Active Parts Inc." {3-3}
4 "Enhancement Additive" 9.99 "Seven Oaks Ltd." {3-4}
5 "Metal Fittings" 79.80 "Active Parts Inc." {3-5}
-> NIL
\end{verbatim}

 
\section{select/3}
\label{sec-8-1-19-15}


\hyperref[ref.html-pilog]{Pilog} database predicate that allows combined
searches over \texttt{+index} and other relations. It takes a list of Pilog
variables, a list of generator clauses, and an arbitrary number of
filter clauses. The functionality is described in detail in
\hyperref[select.html]{The `select' Predicate}. See also \texttt{db/3}, \texttt{isa/2},
\texttt{same/3}, \texttt{bool/3}, \texttt{range/3}, \texttt{head/3}, \texttt{fold/3}, \texttt{part/3}, \texttt{tolr/3}
and \texttt{remote/2}.


\begin{verbatim}
: (?
   @Nr (2 . 5)          # Select all items with numbers between 2 and 5
   @Sup "Active"        # and suppliers matching "Active"
   (select (@Item)                                  # Bind results to '@Item"
      ((nr +Item @Nr) (nm +CuSu @Sup (sup +Item)))  # Generator clauses
      (range @Nr @Item nr)                          # Filter clauses
      (part @Sup @Item sup nm) ) )
 @Nr=(2 . 5) @Sup="Active" @Item={3-3}
 @Nr=(2 . 5) @Sup="Active" @Item={3-5}
-> NIL
\end{verbatim}

 
\section{(send 'msg 'obj ['any ..]) -> any}
\label{sec-8-1-19-16}


Sends the message \texttt{msg} to the object \texttt{obj}, optionally with arguments
\texttt{any}. If the message cannot be located in \texttt{obj}, its classes and
superclasses, an error =''Bad message''= is issued. See also
\texttt{OO Concepts}, \texttt{try}, \texttt{method}, \texttt{meth}, \texttt{super} and \texttt{extra}.


\begin{verbatim}
: (send 'stop> Dlg)  # Equivalent to (stop> Dlg)
-> NIL
\end{verbatim}

 
\section{(seq 'cnt|sym1) -> sym | NIL}
\label{sec-8-1-19-17}


Sequential single step: Returns the \emph{first} external symbol in the
\texttt{cnt}'th database file, or the \emph{next} external symbol following \texttt{sym1}
in the database, or \texttt{NIL} when the end of the database is reached. See
also \texttt{free}.


\begin{verbatim}
: (pool "db")
-> T
: (seq *DB)
-> {2}
: (seq @)
-> {3}
\end{verbatim}

 
\section{(set 'var 'any ..) -> any}
\label{sec-8-1-19-18}


Stores new values \texttt{any} in the \texttt{var} arguments. See also \texttt{setq}, \texttt{val},
\texttt{con} and \texttt{def}.


\begin{verbatim}
: (set 'L '(a b c)  (cdr L) '999)
-> 999
: L
-> (a 999 c)
\end{verbatim}

 
\section{(set! 'obj 'any) -> any}
\label{sec-8-1-19-19}


\hyperref[ref.html-trans]{Transaction} wrapper function for \texttt{set}. Note that
for setting the value of entities typically the \texttt{set!>} message is used.
See also \texttt{new!}, \texttt{put!} and \texttt{inc!}.


\begin{verbatim}
(set! Obj (* Count Size))  # Setting a non-entity object to a numeric value
\end{verbatim}

 
\section{(setq var 'any ..) -> any}
\label{sec-8-1-19-20}


Stores new values \texttt{any} in the \texttt{var} arguments. See also \texttt{set}, \texttt{val}
and \texttt{def}.


\begin{verbatim}
: (setq  A 123  B (list A A))  # Set 'A' to 123, then 'B' to (123 123)
-> (123 123)
\end{verbatim}

 
\section{(show 'any ['sym|cnt ..]) -> any}
\label{sec-8-1-19-21}


Shows the name, value and property list of a symbol found by applying
the \texttt{get} algorithm to \texttt{any} and the following arguments. See also
\texttt{edit} and \texttt{view}.


\begin{verbatim}
: (setq A 123456)
-> 123456
: (put 'A 'x 1)
-> 1
: (put 'A 'lst (9 8 7))
-> (9 8 7)
: (put 'A 'flg T)
-> T

: (show 'A)
A 123456
   flg
   lst (9 8 7)
   x 1
-> A

: (show 'A 'lst 2)
-> 8
\end{verbatim}

 
\section{show/1}
\label{sec-8-1-19-22}


\hyperref[ref.html-pilog]{Pilog} predicate that always succeeds, and shows the
name, value and property list of the argument symbol. See also \texttt{show}.


\begin{verbatim}
: (? (db nr +Item 2 @Item) (show @Item))
{3-2} (+Item)
   nm "Spare Part"
   nr 2
   pr 1250
   inv 100
   sup {2-2}
 @Item={3-2}
-> NIL
\end{verbatim}

 
\section{(sigio ['cnt [. prg]]) -> cnt | prg}
\label{sec-8-1-19-23}


Sets a signal handler \texttt{prg} for SIGIO on the file descriptor \texttt{cnt}. If
called without arguments, the currently installed handler is returned.
See also \texttt{alarm}, \texttt{*Hup} and \texttt{*Sig[12]}.


\begin{verbatim}
# First session
: (sigio (setq *SigSock (port T 4444))  # Register signal handler at UDP port
   (while (udp *SigSock)                # Queue all received data
      (fifo '*SigQueue @) ) )
-> 3

# Second session
: (for I 7 (udp "localhost" 4444 I))  # Send numbers to first session

# First session
: (fifo '*SigQueue)
-> 1
: (fifo '*SigQueue)
-> 2
\end{verbatim}

 
\section{(size 'any) -> cnt}
\label{sec-8-1-19-24}


Returns the ``size'' of \texttt{any}. For numbers this is the number of bytes
needed for the value, for external symbols it is the number of bytes it
would occupy in the database, for other symbols it is the number of
bytes occupied by the UTF--8 representation of the name, and for lists
it is the total number of cells in this list and all its sublists. See
also \texttt{length}.


\begin{verbatim}
: (size "abc")
-> 3
: (size "äbc")
-> 4
: (size 127)  # One byte
-> 1
: (size 128)  # Two bytes (eight bits plus sign bit!)
-> 2
: (size (1 (2) 3))
-> 4
: (size (1 2 3 .))
-> 3
\end{verbatim}

 
\section{(skip ['any]) -> sym}
\label{sec-8-1-19-25}


Skips all whitespace (and comments if \texttt{any} is given) in the input
stream. Returns the next available character, or \texttt{NIL} upon end of file.
See also \texttt{peek} and \texttt{eof}.


\begin{verbatim}
$ cat a
# Comment
abcd
$ pil +
: (in "a" (skip "#"))
-> "a"
\end{verbatim}

 
\section{(solve 'lst [. prg]) -> lst}
\label{sec-8-1-19-26}


Evaluates a \hyperref[ref.html-pilog]{Pilog} query and, returns the list of
result sets. If \texttt{prg} is given, it is executed for each result set, with
all Pilog variables bound to their matching values, and returns a list
of the results. See also \texttt{pilog}, \texttt{?}, \texttt{goal} and \texttt{prove}.


\begin{verbatim}
: (solve '((append @X @Y (a b c))))
-> (((@X) (@Y a b c)) ((@X a) (@Y b c)) ((@X a b) (@Y c)) ((@X a b c) (@Y)))

: (solve '((append @X @Y (a b c))) @X)
-> (NIL (a) (a b) (a b c))
\end{verbatim}

 
\section{(sort 'lst ['fun]) -> lst}
\label{sec-8-1-19-27}


Sorts \texttt{lst} by destructively exchanging its elements. If \texttt{fun} is given,
it is used as a ``less than'' predicate for comparisons. Typically, \texttt{sort}
is used in combination with \hyperref[refB.html-by]{by}, giving shorter and
often more efficient solutions than with the predicate function. See
also \hyperref[ref.html-cmp]{Comparing}, \texttt{group}, \texttt{maxi}, \texttt{mini} and \texttt{uniq}.


\begin{verbatim}
: (sort '(a 3 1 (1 2 3) d b 4 T NIL (a b c) (x y z) c 2))
-> (NIL 1 2 3 4 a b c d (1 2 3) (a b c) (x y z) T)
: (sort '(a 3 1 (1 2 3) d b 4 T NIL (a b c) (x y z) c 2) >)
-> (T (x y z) (a b c) (1 2 3) d c b a 4 3 2 1 NIL)
: (by cadr sort '((1 4 3) (5 1 3) (1 2 4) (3 8 5) (6 4 5)))
-> ((5 1 3) (1 2 4) (1 4 3) (6 4 5) (3 8 5))
\end{verbatim}

 
\section{(space ['cnt]) -> cnt}
\label{sec-8-1-19-28}


Prints \texttt{cnt} spaces, or a single space when \texttt{cnt} is not given.


\begin{verbatim}
: (space)
 -> 1
: (space 1)
 -> 1
: (space 2)
  -> 2
\end{verbatim}

 
\section{(sp? 'any) -> flg}
\label{sec-8-1-19-29}


Returns \texttt{T} when the argument \texttt{any} is \texttt{NIL}, or if it is a string
(symbol) that consists only of whitespace characters.


\begin{verbatim}
: (sp? "  ")
-> T
: (sp? "ABC")
-> NIL
: (sp? 123)
-> NIL
\end{verbatim}

 
\section{(split 'lst 'any ..) -> lst}
\label{sec-8-1-19-30}


Splits \texttt{lst} at all places containing an element \texttt{any} and returns the
resulting list of sublists. See also \texttt{stem}.


\begin{verbatim}
: (split (1 a 2 b 3 c 4 d 5 e 6) 'e 3 'a)
-> ((1) (2 b) (c 4 d 5) (6))
: (mapcar pack (split (chop "The quick brown fox") " "))
-> ("The" "quick" "brown" "fox")
\end{verbatim}

 
\section{(sqrt 'num ['flg]) -> num}
\label{sec-8-1-19-31}


Returns the square root of the \texttt{num} argument. If \texttt{flg} is given and
non-=NIL=, the result will be rounded.


\begin{verbatim}
: (sqrt 64)
-> 8
: (sqrt 1000)
-> 31
: (sqrt 1000 T)
-> 32
: (sqrt 10000000000000000000000000000000000000000)
-> 100000000000000000000
\end{verbatim}

 
\section{(stack ['cnt]) -> cnt | (.. sym . cnt)}
\label{sec-8-1-19-32}


(64-bit version only) Maintains the stack segment size for coroutines.
If called without a \texttt{cnt} argument, or if already one or more
\hyperref[ref.html-coroutines]{coroutines} are running, the current size in
megabytes is returned. Otherwise, the stack segment size is set to the
new value (default 4 MB). If there are running coroutines, their tags
will be \texttt{cons=ed in front of the size. See also =heap}.


\begin{verbatim}
: (stack)         # Get current stack segment size
-> 4
: (stack 10)      # Set to 10 MB
-> 10
: (let N 0 (recur (N) (recurse (inc N))))
!? (recurse (inc N))
Stack overflow
? N
-> 109181
?

: (co "routine" (yield 7))  # Create two coroutines
-> 7
: (co "routine2" (yield 8))
-> 8
: (stack)
-> ("routine2" "routine" . 4)
\end{verbatim}

 
\section{(stamp ['dat 'tim]|['T]) -> sym}
\label{sec-8-1-19-33}


Returns a date-time string in the form ``YYYY-MM-DD HH:MM:SS''. If \texttt{dat}
and \texttt{tim} is missing, the current date and time is used. If \texttt{T} is
passed, the current Coordinated Universal Time (UTC) is used instead.
See also \texttt{date} and \texttt{time}.


\begin{verbatim}
: (stamp)
-> "2000-09-12 07:48:04"
: (stamp (date) 0)
-> "2000-09-12 00:00:00"
: (stamp (date 2000 1 1) (time 12 0 0))
-> "2000-01-01 12:00:00"
\end{verbatim}

 
\section{(state 'var (sym|lst exe [. prg]) ..) -> any}
\label{sec-8-1-19-34}


Implements a finite state machine. The variable \texttt{var} holds the current
state as a symbolic value. When a clause is found that contains the
current state in its CAR \texttt{sym|lst} value, and where the \texttt{exe} in its
CADR evaluates to non-=NIL=, the current state will be set to that
value, the body \texttt{prg} in the CDDR will be executed, and the result
returned. \texttt{T} is a catch-all for any state. If no state-condition
matches, \texttt{NIL} is returned. See also \texttt{case}, \texttt{cond} and \texttt{job}.


\begin{verbatim}
: (de tst ()
   (job '((Cnt . 4))
      (state '(start)
         (start 'run
            (printsp 'start) )
         (run (and (gt0 (dec 'Cnt)) 'run)
            (printsp 'run) )
         (run 'stop
            (printsp 'run) )
         (stop 'start
            (setq Cnt 4)
            (println 'stop) ) ) ) )
-> tst
: (do 12 (tst))
start run run run run stop
start run run run run stop
-> stop
: (pp 'tst)
(de tst NIL
   (job '((Cnt . 4))
      (state '(start)
      ...
-> tst
: (do 3 (tst))
start run run -> run
: (pp 'tst)
(de tst NIL
   (job '((Cnt . 2))
      (state '(run)
      ...
-> tst
\end{verbatim}

 
\section{(stem 'lst 'any ..) -> lst}
\label{sec-8-1-19-35}


Returns the tail of \texttt{lst} that does not contain any of the \texttt{any}
arguments. \texttt{(stem 'lst 'any ..)} is equivalent to
\texttt{(last (split 'lst 'any ..))}. See also \texttt{tail} and \texttt{split}.


\begin{verbatim}
: (stem (chop "abc/def\\ghi") "/" "\\")
-> ("g" "h" "i")
\end{verbatim}

 
\section{(step 'lst ['flg]) -> any}
\label{sec-8-1-19-36}


Single-steps iteratively through a database tree. \texttt{lst} is a structure
as received from \texttt{init}. If \texttt{flg} is non-=NIL=, partial keys are
skipped. See also \texttt{tree}, \texttt{scan}, \texttt{iter}, \texttt{leaf} and \texttt{fetch}.


\begin{verbatim}
: (setq Q (init (tree 'nr '+Item) 3 5))
-> (((3 . 5) ((3 NIL . {3-3}) (4 NIL . {3-4}) (5 NIL . {3-5}) (6 NIL . {3-6}) (7 NIL . {3-8}))))
: (get (step Q) 'nr)
-> 3
: (get (step Q) 'nr)
-> 4
: (get (step Q) 'nr)
-> 5
: (get (step Q) 'nr)
-> NIL
\end{verbatim}

 
\section{(store 'tree 'any1 'any2 ['(num1 . num2)])}
\label{sec-8-1-19-37}


Stores a value \texttt{any2} for the key \texttt{any1} in a database tree. \texttt{num1} is a
database file number, as used in \texttt{new} (defaulting to 1), and \texttt{num2} a
database block size (defaulting to 256). When \texttt{any2} is \texttt{NIL}, the
corresponding entry is deleted from the tree. See also \texttt{tree} and
\texttt{fetch}.


\begin{verbatim}
: (store (tree 'nr '+Item) 2 '{3-2})
\end{verbatim}

 
\section{(str 'sym ['sym1]) -> lst}
\label{sec-8-1-19-38}


\texttt{(str 'lst) -> sym}

In the first form, the string \texttt{sym} is parsed into a list. This
mechanism is also used by \texttt{load}. If \texttt{sym1} is given, it should specify
a set of characters, and \texttt{str} will then return a list of tokens analog
to \texttt{read}. The second form does the reverse operation by building a
string from a list. See also \texttt{any}, \texttt{name} and \texttt{sym}.


\begin{verbatim}
: (str "a (1 2) b")
-> (a (1 2) b)
: (str '(a "Hello" DEF))
-> "a \"Hello\" DEF"
: (str "a*3+b*4" "_")
-> (a "*" 3 "+" b "*" 4)
\end{verbatim}

 
\section{(strDat 'sym) -> dat}
\label{sec-8-1-19-39}


Converts a string \texttt{sym} in the date format of the current \texttt{locale} to a
\texttt{date}. See also \texttt{expDat}, \texttt{\$dat} and \texttt{datStr}.


\begin{verbatim}
: (strDat "2007-06-01")
-> 733134
: (strDat "01.06.2007")
-> NIL
: (locale "DE" "de")
-> NIL
: (strDat "01.06.2007")
-> 733134
: (strDat "1.6.2007")
-> 733134
\end{verbatim}

 
\section{(strip 'any) -> any}
\label{sec-8-1-19-40}


Strips all leading \texttt{quote} symbols from \texttt{any}.


\begin{verbatim}
: (strip 123)
-> 123
: (strip '''(a))
-> (a)
: (strip (quote quote a b c))
-> (a b c)
\end{verbatim}

 
\section{(str? 'any) -> sym | NIL}
\label{sec-8-1-19-41}


Returns the argument \texttt{any} when it is a transient symbol (string),
otherwise \texttt{NIL}. See also \texttt{sym?}, \texttt{box?} and \texttt{ext?}.


\begin{verbatim}
: (str? 123)
-> NIL
: (str? '{ABC})
-> NIL
: (str? 'abc)
-> NIL
: (str? "abc")
-> "abc"
\end{verbatim}

 
\section{(sub? 'any1 'any2) -> any2 | NIL}
\label{sec-8-1-19-42}


Returns \texttt{any2} when the string representation of \texttt{any1} is a substring
of the string representation of \texttt{any2}. See also \texttt{pre?}.


\begin{verbatim}
: (sub? "def" "abcdef")
-> T
: (sub? "abb" "abcdef")
-> NIL
: (sub? NIL "abcdef")
-> T
\end{verbatim}

 
\section{(subr 'sym) -> num}
\label{sec-8-1-19-43}


Converts a Lisp-function that was previously converted with \texttt{expr} back
to a C-function.


\begin{verbatim}
: car
-> 67313448
: (expr 'car)
-> (@ (pass $385260187))
: (subr 'car)
-> 67313448
: car
-> 67313448
\end{verbatim}

 
\section{(sum 'fun 'lst ..) -> num}
\label{sec-8-1-19-44}


Applies \texttt{fun} to each element of \texttt{lst}. When additional \texttt{lst} arguments
are given, their elements are also passed to \texttt{fun}. Returns the sum of
all numeric values returned from \texttt{fun}.


\begin{verbatim}
: (setq A 1  B 2  C 3)
-> 3
: (sum val '(A B C))
-> 6
: (sum                           # Total size of symbol list values
   '((X)
      (and (pair (val X)) (size @)) )
   (what) )
-> 32021
\end{verbatim}

 
\section{(super ['any ..]) -> any}
\label{sec-8-1-19-45}


Can only be used inside methods. Sends the current message to the
current object \texttt{This}, this time starting the search for a method at the
superclass(es) of the class where the current method was found. See also
\texttt{OO Concepts}, \texttt{extra}, \texttt{method}, \texttt{meth}, \texttt{send} and \texttt{try}.


\begin{verbatim}
(dm stop> ()         # 'stop>' method of current class
   (super)           # Call the 'stop>' method of the superclass
   ... )             # other things
\end{verbatim}

 
\section{(sym 'any) -> sym}
\label{sec-8-1-19-46}


Generate the printed representation of \texttt{any} into the name of a new
symbol \texttt{sym}. This is the reverse operation of \texttt{any}. See also \texttt{name}
and \texttt{str}.


\begin{verbatim}
: (sym '(abc "Hello" 123))
-> "(abc \"Hello\" 123)"
\end{verbatim}

 
\section{(sym? 'any) -> flg}
\label{sec-8-1-19-47}


Returns \texttt{T} when the argument \texttt{any} is a symbol. See also \texttt{str?}, \texttt{box?}
and \texttt{ext?}.


\begin{verbatim}
: (sym? 'a)
-> T
: (sym? NIL)
-> T
: (sym? 123)
-> NIL
: (sym? '(a b))
-> NIL
\end{verbatim}

 
\section{(symbols) -> sym}
\label{sec-8-1-19-48}


\texttt{(symbols 'sym1) -> sym2}

\texttt{(symbols 'sym1 'sym2) -> sym3}

(64-bit version only) Creates and manages namespaces of internal
symbols: In the first form, the current namespace is returned. In the
second form, the current namespace is set to \texttt{sym1}, and the previous
namespace \texttt{sym2} is returned. In the third form, \texttt{sym1} is assigned a
\texttt{balance=d copy of an existing namespace =sym2} and becomes the new
current namespace, returning the previous namespace \texttt{sym3}. See also
\texttt{pico}, \texttt{local}, \texttt{import} and \texttt{intern}.


\begin{verbatim}
: (symbols 'myLib 'pico)
-> pico
: (de foo (X)
   (bar (inx X)) )
-> foo

: (symbols 'pico)
-> myLib
: (pp 'foo)
(de foo . NIL)
-> foo
: (pp 'myLib~foo)
(de "foo" (X)
   ("bar" ("inx" X)) )
-> "foo"

: (symbols 'myLib)
-> pico
: (pp 'foo)
(de foo (X)
   (bar (inx X)) )
-> foo
\end{verbatim}

 
\section{(sync) -> flg}
\label{sec-8-1-19-49}


Waits for pending data from all family processes. While other processes
are still sending data (via the \texttt{tell} mechanism), a \texttt{select} system
call is executed for all file descriptors and timers in the \texttt{VAL} of the
global variable \texttt{*Run}. When used in a non-database context, \texttt{(tell)}
should be called in the end to inform the parent process that it may
grant synchronization to other processes waiting for \texttt{sync}. In a
database context, where \texttt{sync} is usually called by \texttt{dbSync}, this is
not necessary because it is done internally by \texttt{commit} or \texttt{rollback}.
See also \texttt{key} and \texttt{wait}.


\begin{verbatim}
: (or (lock) (sync))       # Ensure database consistency
-> T                       # (numeric process-id if lock failed)
\end{verbatim}

 
\section{(sys 'any ['any]) -> sym}
\label{sec-8-1-19-50}


Returns or sets a system environment variable.


\begin{verbatim}
: (sys "TERM")  # Get current value
-> "xterm"
: (sys "TERM" "vt100")  # Set new value
-> "vt100"
: (sys "TERM")
-> "vt100"
\end{verbatim}


\chapter{Functions starting with T}
\label{sec-8-1-20}


 
\section{*Tmp}
\label{sec-8-1-20-1}


A global variable holding the temporary directory name created with
\texttt{tmp}. See also \texttt{*Bye}.


\begin{verbatim}
: *Bye
-> ((saveHistory) (and *Tmp (call 'rm "-r" *Tmp)))
: (tmp "foo" 123)
-> "/home/app/.pil/tmp/27140/foo123"
: *Tmp
-> "/home/app/.pil/tmp/27140/"
\end{verbatim}

 
\section{*Tsm}
\label{sec-8-1-20-2}


A global variable which may hold a cons pair of two strings with escape
sequences, to switch on and off an alternative transient symbol markup.
If set, \texttt{print} will output these sequences to the console instead of
the standard double quote markup characters.


\begin{verbatim}
: (de *Tsm "^[[4m" . "^[[24m")   # vt100 escape sequences for underline
-> *Tsm
: Hello world
-> Hello world
: (off *Tsm)
-> NIL
: "Hello world"                  # No underlining
-> "Hello world"
\end{verbatim}

 
\section{+Time}
\label{sec-8-1-20-3}


Class for clock time values (as calculated by \texttt{time}), a subclass of
\texttt{+Number}. See also \texttt{Database}.


\begin{verbatim}
(rel tim (+Time))  # Time of the day
\end{verbatim}

 
\section{T}
\label{sec-8-1-20-4}


A global constant, evaluating to itself. \texttt{T} is commonly returned as the
boolean value ``true'' (though any non-=NIL= values could be used). It
represents the absolute maximum, as it is larger than any other object.
As a property key, it is used to store \hyperref[ref.html-pilog]{Pilog}
clauses, and inside Pilog clauses it is the \emph{cut} operator. See also
\texttt{NIL} and and \hyperref[ref.html-cmp]{Comparing}.


\begin{verbatim}
: T
-> T
: (= 123 123)
-> T
: (get 'not T)
-> ((@P (1 -> @P) T (fail)) (@P))
\end{verbatim}

 
\section{This}
\label{sec-8-1-20-5}


Holds the current object during method execution (see \hyperref[ref.html-oop]{OO Concepts}), or inside the body of a \texttt{with} statement. As it is a normal
symbol, however, it can be used in normal bindings anywhere. See also
\texttt{isa}, \texttt{:}, ==:=, \texttt{::} and \texttt{var:}.


\begin{verbatim}
: (with 'X (println 'This 'is This))
This is X
-> X
: (put 'X 'a 1)
-> 1
: (put 'X 'b 2)
-> 2
: (put 'Y 'a 111)
-> 111
: (put 'Y 'b 222)
-> 222
: (mapcar '((This) (cons (: a) (: b))) '(X Y))
-> ((1 . 2) (111 . 222))
\end{verbatim}

 
\section{(t . prg) -> T}
\label{sec-8-1-20-6}


Executes \texttt{prg}, and returns \texttt{T}. See also \texttt{nil}, \texttt{prog}, \texttt{prog1} and
\texttt{prog2}.


\begin{verbatim}
: (t (println 'OK))
OK
-> T
\end{verbatim}

 
\section{(tab 'lst 'any ..) -> NIL}
\label{sec-8-1-20-7}


Print all \texttt{any} arguments in a tabular format. \texttt{lst} should be a list of
numbers, specifying the field width for each argument. All items in a
column will be left-aligned for negative numbers, otherwise
right-aligned. See also \texttt{align}, \texttt{center} and \texttt{wrap}.


\begin{verbatim}
: (let Fmt (-3 14 14)
   (tab Fmt "Key" "Rand 1" "Rand 2")
   (tab Fmt "---" "------" "------")
   (for C '(A B C D E F)
      (tab Fmt C (rand) (rand)) ) )
Key        Rand 1        Rand 2
---        ------        ------
A               0    1481765933
B     -1062105905    -877267386
C      -956092119     812669700
D       553475508   -1702133896
E      1344887256   -1417066392
F      1812158119   -1999783937
-> NIL
\end{verbatim}

 
\section{(tail 'cnt|lst 'lst) -> lst}
\label{sec-8-1-20-8}


Returns the last \texttt{cnt} elements of \texttt{lst}. If \texttt{cnt} is negative, it is
added to the length of \texttt{lst}. If the first argument is a \texttt{lst}, \texttt{tail}
is a predicate function returning that argument list if it is \texttt{equal} to
the tail of the second argument, and \texttt{NIL} otherwise. \texttt{(tail -2 Lst)} is
equivalent to \texttt{(nth Lst 3)}. See also \texttt{offset}, \texttt{head}, \texttt{last} and
\texttt{stem}.


\begin{verbatim}
: (tail 3 '(a b c d e f))
-> (d e f)
: (tail -2 '(a b c d e f))
-> (c d e f)
: (tail 0 '(a b c d e f))
-> NIL
: (tail 10 '(a b c d e f))
-> (a b c d e f)
: (tail '(d e f) '(a b c d e f))
-> (d e f)
\end{verbatim}

 
\section{(task 'num ['num] [sym 'any ..] [. prg]) -> lst}
\label{sec-8-1-20-9}


A front-end to the \texttt{*Run} global. If called with only a single \texttt{num}
argument, the corresponding entry is removed from the value of \texttt{*Run}.
Otherwise, a new entry is created. If an entry with that key already
exists, an error is issued. For negative numbers, a second number must
be supplied. If \texttt{sym=/=any} arguments are given, a \texttt{job} environment is
built for thie \texttt{*Run} entry. See also \texttt{forked} and \texttt{timeout}.


\begin{verbatim}
: (task -10000 5000 N 0 (msg (inc 'N)))            # Install task
-> (-10000 5000 (job '((N . 0)) (msg (inc 'N))))   # for every 10 seconds
: 1                                                # ... after 5 seconds
2                                                  # ... after 10 seconds
3                                                  # ... after 10 seconds
(task -10000)                                      # remove again
-> NIL

: (task (port T 4444) (eval (udp @)))              # Receive RPC via UDP
-> (3 (eval (udp @)))

# Another session (on the same machine)
: (udp "localhost" 4444 '(println *Pid))  # Send RPC message
-> (println *Pid)
\end{verbatim}

 
\section{(telStr 'sym) -> sym}
\label{sec-8-1-20-10}


Formats a telephone number according to the current \texttt{locale}. If the
string head matches the local country code, it is replaced with \texttt{0},
otherwise \texttt{+} is prepended. See also \texttt{expTel}, \texttt{datStr}, \texttt{money} and
\texttt{format}.


\begin{verbatim}
: (telStr "49 1234 5678-0")
-> "+49 1234 5678-0"
: (locale "DE" "de")
-> NIL
: (telStr "49 1234 5678-0")
-> "01234 5678-0"
\end{verbatim}

 
\section{(tell ['cnt] 'sym ['any ..]) -> any}
\label{sec-8-1-20-11}


Family IPC: Send an executable list \texttt{(sym any ..)} to all family members
(i.e. all children of the current process, and all other children of the
parent process, see \texttt{fork}) for automatic execution. When the \texttt{cnt}
argument is given and non-zero, it should be the PID of such a process,
and the list will be sent only to that process. \texttt{tell} is also used
internally by \texttt{commit} to notify about database changes. When called
without arguments, no message is actually sent, and the parent process
may grant \texttt{sync} to the next waiting process. See also \texttt{hear}.


\begin{verbatim}
: (call 'ps "x")                          # Show processes
  PID TTY      STAT   TIME COMMAND
  ..
 1321 pts/0    S      0:00 /usr/bin/picolisp ..  # Parent process
 1324 pts/0    S      0:01 /usr/bin/picolisp ..  # First child
 1325 pts/0    S      0:01 /usr/bin/picolisp ..  # Second child
 1326 pts/0    R      0:00 ps x
-> T
: *Pid                                    # We are the second child
-> 1325
: (tell 'println '*Pid)                   # Ask all others to print their Pid's
1324
-> *Pid
\end{verbatim}

 
\section{(test 'any . prg)}
\label{sec-8-1-20-12}


Executes \texttt{prg}, and issues an \texttt{error} if the result does not \texttt{match} the
\texttt{any} argument. See also \texttt{assert}.


\begin{verbatim}
: (test 12 (* 3 4))
-> NIL
: (test 12 (+ 3 4))
((+ 3 4))
12 -- 'test' failed
?
\end{verbatim}

 
\section{(text 'any1 'any ..) -> sym}
\label{sec-8-1-20-13}


Builds a new transient symbol (string) from the string representation of
\texttt{any1}, by replacing all occurrences of an at-mark ``=@='', followed by
one of the letters''\texttt{1}'' through ``=9='', and''\texttt{A}'' through ``=Z='', with the
corresponding \texttt{any} argument. In this context ``=@A='' refers to the 10th
argument. A literal at-mark in the text can be represented by two
successive at-marks. See also \texttt{pack} and \texttt{glue}.


\begin{verbatim}
: (text "abc @1 def @2" 'XYZ 123)
-> "abc XYZ def 123"
: (text "a@@bc.@1" "de")
-> "a@bc.de"
\end{verbatim}

 
\section{(tim\$ 'tim ['flg]) -> sym}
\label{sec-8-1-20-14}


Formats a \texttt{time} \texttt{tim}. If \texttt{flg} is \texttt{NIL}, the format is HH:MM,
otherwise it is HH:MM:SS. See also \texttt{\$tim} and \texttt{dat\$}.


\begin{verbatim}
: (tim$ (time))
-> "10:57"
: (tim$ (time) T)
-> "10:57:56"
\end{verbatim}

 
\section{(timeout ['num])}
\label{sec-8-1-20-15}


Sets or refreshes a timeout value in the \texttt{*Run} global, so that the
current process executes \texttt{bye} after the given period. If called without
arguments, the timeout is removed. See also \texttt{task}.


\begin{verbatim}
: (timeout 3600000)           # Timeout after one hour
-> (-1 3600000 (bye))
: *Run                        # Look after a few seconds
-> ((-1 3574516 (bye)))
\end{verbatim}

 
\section{(throw 'sym 'any)}
\label{sec-8-1-20-16}


Non-local jump into a previous \texttt{catch} environment with the jump label
\texttt{sym} (or \texttt{T} as a catch-all). Any pending \texttt{finally} expressions are
executed, local symbol bindings are restored, open files are closed and
internal data structures are reset appropriately, as the environment was
at the time when the corresponding \texttt{catch} was called. Then \texttt{any} is
returned from that \texttt{catch}. See also \texttt{quit}.


\begin{verbatim}
: (de foo (N)
   (println N)
   (throw 'OK) )
-> foo
: (let N 1  (catch 'OK (foo 7))  (println N))
7
1
-> 1
\end{verbatim}

 
\section{(tick (cnt1 . cnt2) . prg) -> any}
\label{sec-8-1-20-17}


Executes \texttt{prg}, then (destructively) adds the number of elapsed user
ticks to the \texttt{cnt1} parameter, and the number of elapsed system ticks to
the \texttt{cnt2} parameter. Thus, \texttt{cnt1} and \texttt{cnt2} will finally contain the
total number of user and system time ticks spent in \texttt{prg} and all
functions called (this works also for recursive functions). For
execution profiling, \texttt{tick} is usually inserted into words with \texttt{prof},
and removed with \texttt{unprof}. See also \texttt{usec}.


\begin{verbatim}
: (de foo ()                        # Define function with empty loop
   (tick (0 . 0) (do 100000000)) )
-> foo
: (foo)                             # Execute it
-> NIL
: (pp 'foo)
(de foo NIL
   (tick (97 . 0) (do 100000000)) ) # 'tick' incremented 'cnt1' by 97
-> foo
\end{verbatim}

 
\section{(till 'any ['flg]) -> lst|sym}
\label{sec-8-1-20-18}


Reads from the current input channel till a character contained in \texttt{any}
is found (or until end of file if \texttt{any} is \texttt{NIL}). If \texttt{flg} is \texttt{NIL}, a
list of single-character transient symbols is returned. Otherwise, a
single string is returned. See also \texttt{from} and \texttt{line}.


\begin{verbatim}
: (till ":")
abc:def
-> ("a" "b" "c")
: (till ":" T)
abc:def
-> "abc"
\end{verbatim}

 
\section{(time ['T]) -> tim}
\label{sec-8-1-20-19}


\texttt{(time 'tim) -> (h m s)}

\texttt{(time 'h 'm ['s]) -> tim | NIL}

\texttt{(time '(h m [s])) -> tim | NIL}

Calculates the time of day, represented as the number of seconds since
midnight. When called without arguments, the current local time is
returned. When called with a \texttt{T} argument, the time of the last call to
\texttt{date} is returned. When called with a single number \texttt{tim}, it is taken
as a time value and a list with the corresponding hour, minute and
second is returned. When called with two or three numbers (or a list of
two or three numbers) for the hour, minute (and optionally the second),
the corresponding time value is returned (or \texttt{NIL} if they do not
represent a legal time). See also \texttt{date}, \texttt{stamp}, \texttt{usec}, \texttt{tim\$} and
\texttt{\$tim}.


\begin{verbatim}
: (time)                         # Now
-> 32334
: (time 32334)                   # Now
-> (8 58 54)
: (time 25 30)                   # Illegal time
-> NIL
\end{verbatim}

 
\section{(tmp ['any ..]) -> sym}
\label{sec-8-1-20-20}


Returns the path name to the \texttt{pack=ed =any} arguments in a process-local
temporary directory. The directory name consists of the path to
``.pil/tmp/'' in the user's home directory, followed by the current
process ID \texttt{*Pid}. This directory is automatically created if necessary,
and removed upon termination of the process (\texttt{bye}). See also \texttt{pil},
\texttt{*Tmp} and \texttt{*Bye} .


\begin{verbatim}
: *Pid
-> 27140
: (tmp "foo" 123)
-> "/home/app/.pil/tmp/27140/foo123"
: (out (tmp "foo" 123) (println 'OK))
-> OK
: (dir (tmp))
-> ("foo123")
: (in (tmp "foo" 123) (read))
-> OK
\end{verbatim}

 
\section{tolr/3}
\label{sec-8-1-20-21}


\hyperref[ref.html-pilog]{Pilog} predicate that succeeds if the first argument
is either a \emph{substring} or a \texttt{+Sn} \emph{soundex} match of the result of
applying the \texttt{get} algorithm to the following arguments. Typically used
as filter predicate in \texttt{select/3} database queries. See also \texttt{isa/2},
\texttt{same/3}, \texttt{bool/3}, \texttt{range/3}, \texttt{head/3}, \texttt{fold/3} and \texttt{part/3}.


\begin{verbatim}
: (?
   @Nr (1 . 5)
   @Nm "Sven"
   (select (@CuSu)
      ((nr +CuSu @Nr) (nm +CuSu @Nm))
      (range @Nr @CuSu nr)
      (tolr @Nm @CuSu nm) )
   (val @Name @CuSu nm) )
 @Nr=(1 . 5) @Nm="Sven" @CuSu={2-2} @Name="Seven Oaks Ltd."
\end{verbatim}

 
\section{(touch 'sym) -> sym}
\label{sec-8-1-20-22}


When \texttt{sym} is an external symbol, it is marked as ``modified'' so that
upon a later \texttt{commit} it will be written to the database file. An
explicit call of \texttt{touch} is only necessary when the value or properties
of \texttt{sym} are indirectly modified.


\begin{verbatim}
: (get '{2} 'lst)
-> (1 2 3 4 5)
: (set (cdr (get (touch '{2}) 'lst)) 999)    # Only read-access, need 'touch'
-> 999
: (get '{2} 'lst)                            # Modified second list element
-> (1 999 3 4 5)
\end{verbatim}

 
\section{(trace 'sym) -> sym}
\label{sec-8-1-20-23}


\texttt{(trace 'sym 'cls) -> sym}

\texttt{(trace '(sym . cls)) -> sym}

Inserts a \texttt{\$} trace function call at the beginning of the function or
method body of \texttt{sym}, so that trace information will be printed before
and after execution. Built-in functions (C-function pointer) are
automatically converted to Lisp expressions (see \texttt{expr}). See also
\texttt{*Dbg}, \texttt{traceAll} and \texttt{untrace}, \texttt{debug} and \texttt{lint}.


\begin{verbatim}
: (trace '+)
-> +
: (+ 3 4)
 + : 3 4
 + = 7
-> 7
\end{verbatim}

 
\section{(traceAll ['lst]) -> sym}
\label{sec-8-1-20-24}


Traces all Lisp level functions by inserting a \texttt{\$} function call at the
beginning. \texttt{lst} may contain symbols which are to be excluded from that
process. In addition, all symbols in the global variable \texttt{*NoTrace} are
excluded. See also \texttt{trace}, \texttt{untrace} and \texttt{*Dbg}.


\begin{verbatim}
: (traceAll)      # Trace all Lisp level functions
-> balance
\end{verbatim}

 
\section{(tree 'var 'cls ['hook]) -> tree}
\label{sec-8-1-20-25}


Returns a data structure specifying a database index tree. \texttt{var} and
\texttt{cls} determine the relation, with an optional \texttt{hook} object. See also
\texttt{root}, \texttt{fetch}, \texttt{store}, \texttt{count}, \texttt{leaf}, \texttt{minKey}, \texttt{maxKey}, \texttt{init},
\texttt{step}, \texttt{scan}, \texttt{iter}, \texttt{prune}, \texttt{zapTree} and \texttt{chkTree}.


\begin{verbatim}
: (tree 'nm '+Item)
-> (nm . +Item)
\end{verbatim}

 
\section{(trim 'lst) -> lst}
\label{sec-8-1-20-26}


Returns a copy of \texttt{lst} with all trailing whitespace characters or \texttt{NIL}
elements removed. See also \texttt{clip}.


\begin{verbatim}
: (trim (1 NIL 2 NIL NIL))
-> (1 NIL 2)
: (trim '(a b " " " "))
-> (a b)
\end{verbatim}

 
\section{true/0}
\label{sec-8-1-20-27}


\hyperref[ref.html-pilog]{Pilog} predicate that always succeeds. See also
\texttt{fail/0} and \texttt{repeat/0}.


\begin{verbatim}
:  (? (true))
-> T
\end{verbatim}

 
\section{(try 'msg 'obj ['any ..]) -> any}
\label{sec-8-1-20-28}


Tries to send the message \texttt{msg} to the object \texttt{obj}, optionally with
arguments \texttt{any}. If \texttt{obj} is not an object, or if the message cannot be
located in \texttt{obj}, in its classes or superclasses, \texttt{NIL} is returned. See
also \texttt{OO Concepts}, \texttt{send}, \texttt{method}, \texttt{meth}, \texttt{super} and \texttt{extra}.


\begin{verbatim}
: (try 'msg> 123)
-> NIL
: (try 'html> 'a)
-> NIL
\end{verbatim}

 
\section{(type 'any) -> lst}
\label{sec-8-1-20-29}


Return the type (list of classes) of the object \texttt{sym}. See also
\texttt{OO Concepts}, \texttt{isa}, \texttt{class}, \texttt{new} and \texttt{object}.


\begin{verbatim}
: (type '{1A;3})
(+Address)
: (type '+DnButton)
-> (+Tiny +Rid +JS +Able +Button)
\end{verbatim}


\chapter{Functions starting with U}
\label{sec-8-1-21}


 
\section{*Uni}
\label{sec-8-1-21-1}


A global variable holding an \texttt{idx} tree, with all unique data that were
collected with the comma (=,=) read-macro. Typically used for
localization. See also \texttt{Read-Macros} and \texttt{locale}.


\begin{verbatim}
: (off *Uni)            # Clear
-> NIL
: ,"abc"                # Collect a transient symbol
-> "abc"
: ,(1 2 3)              # Collect a list
-> (1 2 3)
: *Uni
-> ("abc" NIL (1 2 3))
\end{verbatim}

 
\section{+UB}
\label{sec-8-1-21-2}


Prefix class for \texttt{+Aux} to maintain an UB-Tree index instead of the
direct values. This allows efficient range access to multidimensional
data. Only numeric keys are supported. See also \texttt{Database}.


\begin{verbatim}
(class +Pos +Entity)
(rel x (+UB +Aux +Ref +Number) (y z))
(rel y (+Number))
(rel z (+Number))

: (scan (tree 'x '+Pos))
...
(664594005183881683 . {B}) {B}
(899018453307525604 . {C}) {C}  # UBKEY of (516516 690628 706223)
(943014863198293414 . {2}) {2}
(988682500781514058 . {A}) {A}
(994667870851824704 . {8}) {8}
(1016631364991047263 . {:}) {:}
...

: (show '{C})
{C} (+Pos)
   z 706223
   y 690628
   x 516516
-> {C}

# Discrete queries work the same way as without the +UB prefix
: (db 'x '+Pos 516516 'y 690628 'z 706223)
-> {C}
: (aux 'x '+Pos 516516 690628 706223)
-> {C}
: (? (db x +Pos (516516 690628 706223) @Pos))
 @Pos={C}
-> NIL

# Efficient range queries are are possible now
: (?
   @X (416511 . 616519)
   @Y (590621 . 890629)
   @Z (606221 . 906229)
   (select (@@)
      ((x +Pos (@X @Y @Z)))   # Range query
      (range @X @@ x)         # Filter
      (range @Y @@ y)
      (range @Z @@ z) ) )
 @X=(416511 . 616519) @Y=(590621 . 890629) @Z=(606221 . 906229) @@={C}
 @X=(416511 . 616519) @Y=(590621 . 890629) @Z=(606221 . 906229) @@={8}
\end{verbatim}

 
\section{(u) -> T}
\label{sec-8-1-21-3}


Removes \texttt{!} all breakpoints in all subexpressions of the current
breakpoint. Typically used when single-stepping a function or method
with \texttt{debug}. See also \texttt{d} and \texttt{unbug}.


\begin{verbatim}
! (u)                         # Unbug subexpression(s) at breakpoint
-> T
\end{verbatim}

 
\section{(udp 'any1 'any2 'any3) -> any}
\label{sec-8-1-21-4}


\texttt{(udp 'cnt) -> any}

Simple unidirectional sending/receiving of UDP packets. In the first
form, \texttt{any3} is sent to a UDP server listening at host \texttt{any1}, port
\texttt{any2}. In the second form, one item is received from a UDP socket
\texttt{cnt}, established with \texttt{port}. See also \texttt{listen} and \texttt{connect}.


\begin{verbatim}
# First session
: (port T 6666)
-> 3
: (udp 3)  # Receive a datagram

# Second session (on the same machine)
: (udp "localhost" 6666 '(a b c))
-> (a b c)

# First session
-> (a b c)
\end{verbatim}

 
\section{(ultimo 'y 'm) -> cnt}
\label{sec-8-1-21-5}


Returns the \texttt{date} of the last day of the month \texttt{m} in the year \texttt{y}. See
also \texttt{day} and \texttt{week}.


\begin{verbatim}
: (date (ultimo 2007 1))
-> (2007 1 31)
: (date (ultimo 2007 2))
-> (2007 2 28)
: (date (ultimo 2004 2))
-> (2004 2 29)
: (date (ultimo 2000 2))
-> (2000 2 29)
: (date (ultimo 1900 2))
-> (1900 2 28)
\end{verbatim}

 
\section{(unbug 'sym) -> T}
\label{sec-8-1-21-6}


\texttt{(unbug 'sym 'cls) -> T}

\texttt{(unbug '(sym . cls)) -> T}

Removes all \texttt{!} breakpoints in the function or method body of sym, as
inserted with \texttt{debug} or \texttt{d}, or directly with \texttt{edit}. See also \texttt{u}.


\begin{verbatim}
: (pp 'tst)
(de tst (N)
   (! println (+ 3 N)) )         # 'tst' has a breakpoint '!'
-> tst
: (unbug 'tst)                   # Unbug it
-> T
: (pp 'tst)                      # Restore
(de tst (N)
   (println (+ 3 N)) )
\end{verbatim}

 
\section{(undef 'sym) -> fun}
\label{sec-8-1-21-7}


\texttt{(undef 'sym 'cls) -> fun}

\texttt{(undef '(sym . cls)) -> fun}

Undefines the function or method \texttt{sym}. Returns the previous definition.
See also \texttt{de}, \texttt{dm}, \texttt{def} and \texttt{redef}.


\begin{verbatim}
: (de hello () "Hello world!")
-> hello
: hello
-> (NIL "Hello world!")
: (undef 'hello)
-> (NIL "Hello world!")
: hello
-> NIL
\end{verbatim}

 
\section{(unify 'any) -> lst}
\label{sec-8-1-21-8}


Unifies \texttt{any} with the current \hyperref[ref.html-pilog]{Pilog} environment at
the current level and with a value of \texttt{NIL}, and returns the new
environment or \texttt{NIL} if not successful. See also \texttt{prove} and \texttt{->}.


\begin{verbatim}
: (? (@A unify '(@B @C)))
 @A=(((NIL . @C) 0 . @C) ((NIL . @B) 0 . @B) T)
\end{verbatim}

 
\section{(uniq 'lst) -> lst}
\label{sec-8-1-21-9}


Returns a unique list, by eleminating all duplicate elements from \texttt{lst}.
See also \hyperref[ref.html-cmp]{Comparing}, \texttt{sort} and \texttt{group}.


\begin{verbatim}
: (uniq (2 4 6 1 2 3 4 5 6 1 3 5))
-> (2 4 6 1 3 5)
\end{verbatim}

 
\section{uniq/2}
\label{sec-8-1-21-10}


\hyperref[ref.html-pilog]{Pilog} predicate that succeeds if the first argument
is not yet stored in the second argument's index structure. \texttt{idx} is
used internally storing for the values and checking for uniqueness. See
also \texttt{member/2}.


\begin{verbatim}
: (? (uniq a @Z))       # Remember 'a'
 @Z=NIL                 # Succeeded

: (? (uniq b @Z))       # Remember 'b'
 @Z=NIL                 # Succeeded

: (? (uniq a @Z))       # Remembered 'a'?
-> NIL                  # Yes: Not unique
\end{verbatim}

 
\section{(unless 'any . prg) -> any}
\label{sec-8-1-21-11}


Conditional execution: When the condition \texttt{any} evaluates to non-=NIL=,
\texttt{NIL} is returned. Otherwise \texttt{prg} is executed and the result returned.
See also \texttt{when}.


\begin{verbatim}
: (unless (= 3 3) (println 'Strange 'result))
-> NIL
: (unless (= 3 4) (println 'Strange 'result))
Strange result
-> result
\end{verbatim}

 
\section{(until 'any . prg) -> any}
\label{sec-8-1-21-12}


Conditional loop: While the condition \texttt{any} evaluates to \texttt{NIL}, \texttt{prg} is
repeatedly executed. If \texttt{prg} is never executed, \texttt{NIL} is returned.
Otherwise the result of \texttt{prg} is returned. See also \texttt{while}.


\begin{verbatim}
: (until (=T (setq N (read)))
   (println 'square (* N N)) )
4
square 16
9
square 81
T
-> 81
\end{verbatim}

 
\section{(untrace 'sym) -> sym}
\label{sec-8-1-21-13}


\texttt{(untrace 'sym 'cls) -> sym}

\texttt{(untrace '(sym . cls)) -> sym}

Removes the \texttt{\$} trace function call at the beginning of the function or
method body of \texttt{sym}, so that no more trace information will be printed
before and after execution. Built-in functions (C-function pointer) are
automatically converted to their original form (see \texttt{subr}). See also
\texttt{trace} and \texttt{traceAll}.


\begin{verbatim}
: (trace '+)                           # Trace the '+' function
-> +
: +
-> (@ ($ + @ (pass $385455126)))       # Modified for tracing
: (untrace '+)                         # Untrace '+'
-> +
: +
-> 67319120                            # Back to original form
\end{verbatim}

 
\section{(up [cnt] sym ['val]) -> any}
\label{sec-8-1-21-14}


Looks up (or modifies) the \texttt{cnt}'th previously saved value of \texttt{sym} in
the corresponding enclosing environment. If \texttt{cnt} is not given, 1 is
used. See also \texttt{eval}, \texttt{run} and \texttt{env}.


\begin{verbatim}
: (let N 1 ((quote (N) (println N (up N))) 2))
2 1
-> 1
: (let N 1 ((quote (N) (println N (up N) (up N 7))) 2) N)
2 1 7
-> 7
\end{verbatim}

 
\section{(upd sym ..) -> lst}
\label{sec-8-1-21-15}


Synchronizes the internal state of all passed (external) symbols by
passing them to \texttt{wipe}. \texttt{upd} is the standard function passed to
\texttt{commit} during database \texttt{transactions}.


\begin{verbatim}
(commit 'upd)  # Commit changes, informing all sister processes
\end{verbatim}

 
\section{(update 'obj ['var]) -> obj}
\label{sec-8-1-21-16}


Interactive database function for modifying external symbols. When
called only with an \texttt{obj} argument, \texttt{update} steps through the value and
all properties of that object (and recursively also through
substructures) and allows to edit them with the console line editor.
When the \texttt{var} argument is given, only that single property is handed to
the editor. To delete a propery, \texttt{NIL} must be explicitly entered.
\texttt{update} will correctly handle all \hyperref[ref.html-er]{entity/relation}
mechanisms. See also \texttt{select}, \texttt{edit} and \texttt{Database}.


\begin{verbatim}
: (show '{3-1})            # Show item 1
{3-1} (+Item)
   nr 1
   pr 29900
   inv 100
   sup {2-1}
   nm "Main Part"
-> {3-1}

: (update '{3-1} 'pr)      # Update the prices of that item
{3-1} pr 299.00            # The cursor is right behind "299.00"
-> {3-1}
\end{verbatim}

 
\section{(upp? 'any) -> sym | NIL}
\label{sec-8-1-21-17}


Returns \texttt{any} when the argument is a string (symbol) that starts with an
uppercase character. See also \texttt{uppc} and \texttt{low?}


\begin{verbatim}
: (upp? "A")
-> T
: (upp? "a")
-> NIL
: (upp? 123)
-> NIL
: (upp? ".")
-> NIL
\end{verbatim}

 
\section{(uppc 'any) -> any}
\label{sec-8-1-21-18}


Upper case conversion: If \texttt{any} is not a symbol, it is returned as it
is. Otherwise, a new transient symbol with all characters of \texttt{any},
converted to upper case, is returned. See also \texttt{lowc}, \texttt{fold} and
\texttt{upp?}.


\begin{verbatim}
: (uppc 123)
-> 123
: (uppc "abc")
-> "ABC"
: (uppc 'car)
-> "CAR"
\end{verbatim}

 
\section{(use sym . prg) -> any}
\label{sec-8-1-21-19}


\texttt{(use (sym ..) . prg) -> any}

Defines local variables. The value of the symbol \texttt{sym} - or the values
of the symbols \texttt{sym} in the list of the second form - are saved, \texttt{prg}
is executed, then the symbols are restored to their original values.
During execution of \texttt{prg}, the values of the symbols can be temporarily
modified. The return value is the result of \texttt{prg}. See also \texttt{bind},
\texttt{job} and \texttt{let}.


\begin{verbatim}
: (setq  X 123  Y 456)
-> 456
: (use (X Y) (setq  X 3  Y 4) (* X Y))
-> 12
: X
-> 123
: Y
-> 456
\end{verbatim}

 
\section{(useKey 'var 'cls ['hook]) -> num}
\label{sec-8-1-21-20}


Generates or reuses a key for a database tree, by randomly trying to
locate a free number. See also \texttt{genKey}.


\begin{verbatim}
: (maxKey (tree 'nr '+Item))
-> 8
: (useKey 'nr '+Item)
-> 12
\end{verbatim}

 
\section{(usec) -> num}
\label{sec-8-1-21-21}


Returns the number the microseconds since interpreter startup. See also
\texttt{time} and \texttt{tick}.


\begin{verbatim}
: (usec)
-> 1154702479219050
\end{verbatim}


\chapter{Functions starting with V}
\label{sec-8-1-22}


 
\section{(val 'var) -> any}
\label{sec-8-1-22-1}


Returns the current value of \texttt{var}. See also \texttt{setq}, \texttt{set} and \texttt{def}.


\begin{verbatim}
: (setq L '(a b c))
-> (a b c)
: (val 'L)
-> (a b c)
: (val (cdr L))
-> b
\end{verbatim}

 
\section{val/3}
\label{sec-8-1-22-2}


\hyperref[ref.html-pilog]{Pilog} predicate that returns the value of an
object's attribute. Typically used in database queries. The first
argument is a Pilog variable to bind the value, the second is the
object, and the third and following arguments are used to apply the
\texttt{get} algorithm to that object. See also \texttt{db/3} and \texttt{select/3}.


\begin{verbatim}
: (?
   (db nr +Item (2 . 5) @Item)   # Fetch articles 2 through 5
   (val @Nm @Item nm)            # Get item description
   (val @Sup @Item sup nm) )     # and supplier's name
 @Item={3-2} @Nm="Spare Part" @Sup="Seven Oaks Ltd."                             @Item={3-3} @Nm="Auxiliary Construction" @Sup="Active Parts Inc."
 @Item={3-4} @Nm="Enhancement Additive" @Sup="Seven Oaks Ltd."
 @Item={3-5} @Nm="Metal Fittings" @Sup="Active Parts Inc."
-> NIL
\end{verbatim}

 
\section{(var sym . any) -> any}
\label{sec-8-1-22-3}


\texttt{(var (sym . cls) . any) -> any}

Defines a class variable \texttt{sym} with the initial value \texttt{any} for the
current class, implicitly given by the value of the global variable
\texttt{*Class}, or - in the second form - for the explicitly given class cls.
See also \hyperref[ref.html-oop]{OO Concepts}, \texttt{rel} and \texttt{var:}.


\begin{verbatim}
: (class +A)
-> +A
: (var a . 1)
-> 1
: (var b . 2)
-> 2
: (show '+A)
+A NIL
   b 2
   a 1
-> +A
\end{verbatim}

 
\section{(var: sym) -> any}
\label{sec-8-1-22-4}


Fetches the value of a class variable \texttt{sym} for the current object
\texttt{This}, by searching the property lists of its class(es) and
supperclasses. See also \texttt{OO Concepts}, \texttt{var}, \texttt{with}, \texttt{meta}, \texttt{:}, ==:=
and \texttt{::}.


\begin{verbatim}
: (object 'O '(+A) 'a 9 'b 8)
-> O
: (with 'O (list (: a) (: b) (var: a) (var: b)))
-> (9 8 1 2)
\end{verbatim}

 
\section{(version ['flg]) -> lst}
\label{sec-8-1-22-5}


Prints the current version as a string of dot-separated numbers, and
returns the current version as a list of numbers. The JVM- and
C-versions print an additional ``JVM'' or ``C'', respectively, separated by
a space. When \texttt{flg} is non-NIL, printing is suppressed.


\begin{verbatim}
$ pil -version
3.0.1.22
: (version T)
-> (3 0 1 22)
\end{verbatim}

 
\section{(vi 'sym) -> sym}
\label{sec-8-1-22-6}


\texttt{(vi 'sym 'cls) -> sym}

\texttt{(vi '(sym . cls)) -> sym}

\texttt{(vi) -> NIL}

Opens the ``vi'' editor on the function or method definition of \texttt{sym}. A
call to \texttt{ld} thereafter will \texttt{load} the modified file. See also \texttt{doc},
\texttt{edit}, \texttt{*Dbg}, \texttt{debug} and \texttt{pp}.


\begin{verbatim}
: (vi 'url> '+CuSu)  # Edit the method's source code, then exit from 'vi'
-> T
\end{verbatim}

 
\section{(view 'lst ['T]) -> any}
\label{sec-8-1-22-7}


Views \texttt{lst} as tree-structured ASCII graphics. When the \texttt{T} argument is
given, \texttt{lst} should be a binary tree structure (as generated by \texttt{idx}),
which is then shown as a left-rotated tree. See also \texttt{pretty} and
\texttt{show}.


\begin{verbatim}
: (balance 'I '(a b c d e f g h i j k l m n o))
-> NIL
: I
-> (h (d (b (a) c) f (e) g) l (j (i) k) n (m) o)

: (view I)
+-- h
|
+---+-- d
|   |
|   +---+-- b
|   |   |
|   |   +---+-- a
|   |   |
|   |   +-- c
|   |
|   +-- f
|   |
|   +---+-- e
|   |
|   +-- g
|
+-- l
|
+---+-- j
|   |
|   +---+-- i
|   |
|   +-- k
|
+-- n
|
+---+-- m
|
+-- o
-> NIL

: (view I T)
         o
      n
         m
   l
         k
      j
         i
h
         g
      f
         e
   d
         c
      b
         a
-> NIL
\end{verbatim}


\chapter{Functions starting with W}
\label{sec-8-1-23}

 
\section{(wait ['cnt] . prg) -> any}
\label{sec-8-1-23-1}


Waits for a condition. While the result of the execution of \texttt{prg} is
\texttt{NIL}, a \texttt{select} system call is executed for all file descriptors and
timers in the \texttt{VAL} of the global variable \texttt{*Run}. When \texttt{cnt} is
non-=NIL=, the waiting time is limited to \texttt{cnt} milliseconds. Returns
the result of \texttt{prg}. See also \texttt{key} and \texttt{sync}.


\begin{verbatim}
: (wait 2000)                                # Wait 2 seconds
-> NIL
: (prog
   (zero *Cnt)
   (setq *Run                                # Install background loop
      '((-2000 0 (println (inc '*Cnt)))) )   # Increment '*Cnt' every 2 sec
   (wait NIL (> *Cnt 6))                     # Wait until > 6
   (off *Run) )
1                                            # Waiting ..
2
3
4
5
6
7
-> NIL
\end{verbatim}

 
\section{(week 'dat) -> num}
\label{sec-8-1-23-2}


Returns the number of the week for a given \texttt{date} \texttt{dat}. See also \texttt{day},
\texttt{ultimo}, \texttt{datStr} and \texttt{strDat}.


\begin{verbatim}
: (datStr (date))
-> "2007-06-01"
: (week (date))
-> 22
\end{verbatim}

 
\section{(when 'any . prg) -> any}
\label{sec-8-1-23-3}


Conditional execution: When the condition \texttt{any} evaluates to non-=NIL=,
\texttt{prg} is executed and the result is returned. Otherwise \texttt{NIL} is
returned. See also \texttt{unless}.


\begin{verbatim}
: (when (> 4 3) (println 'OK) (println 'Good))
OK
Good
-> Good
\end{verbatim}

 
\section{(while 'any . prg) -> any}
\label{sec-8-1-23-4}


Conditional loop: While the condition \texttt{any} evaluates to non-=NIL=,
\texttt{prg} is repeatedly executed. If \texttt{prg} is never executed, \texttt{NIL} is
returned. Otherwise the result of \texttt{prg} is returned. See also \texttt{until}.


\begin{verbatim}
: (while (read)
   (println 'got: @) )
abc
got: abc
1234
got: 1234
NIL
-> 1234
\end{verbatim}

 
\section{(what 'sym) -> lst}
\label{sec-8-1-23-5}


Returns a list of all internal symbols that match the pattern string
\texttt{sym}. See also \texttt{match}, \texttt{who} and \texttt{can}.


\begin{verbatim}
: (what "cd@dr")
-> (cdaddr cdaadr cddr cddddr cdddr cddadr cdadr)
\end{verbatim}

 
\section{(who 'any) -> lst}
\label{sec-8-1-23-6}


Returns a list of all functions or method definitions that contain the
atom or pattern \texttt{any}. See also \texttt{match}, \texttt{what} and \texttt{can}.


\begin{verbatim}
: (who 'caddr)                         # Who is using 'caddr'?
-> ($dat lint1 expDat datStr $tim tim$ mail _gen dat$ datSym)

: (who "Type error")
-> ((mis> . +Link) *Uni (mis> . +Joint))

: (more (who "Type error") pp)         # Pretty print all results
(dm (mis> . +Link) (Val Obj)
   (and
      Val
      (nor (isa (: type) Val) (canQuery Val))
      "Type error" ) )
.                                      # Stop
-> T
\end{verbatim}

 
\section{(wipe 'sym|lst) -> sym|lst}
\label{sec-8-1-23-7}


Clears the \texttt{VAL} and the property list of \texttt{sym}, or of all symbols in
the list \texttt{lst}. When a symbol is an external symbol, its state is also
set to ``not loaded''. Does nothing when \texttt{sym} is an external symbol that
has been modified or deleted (``dirty'').


\begin{verbatim}
: (setq A (1 2 3 4))
-> (1 2 3 4)
: (put 'A 'a 1)
-> 1
: (put 'A 'b 2)
-> 2
: (show 'A)
A (1 2 3 4)
   b 2
   a 1
-> A
: (wipe 'A)
-> A
: (show 'A)
A NIL
-> A
\end{verbatim}

 
\section{(with 'sym . prg) -> any}
\label{sec-8-1-23-8}


Saves the current object \texttt{This} and sets it to the new value \texttt{sym}. Then
\texttt{prg} is executed, and \texttt{This} is restored to its previous value. The
return value is the result of \texttt{prg}. Used typically to access the local
data of \texttt{sym} in the same manner as inside a method body. \texttt{prg} is not
executed (and \texttt{NIL} is returned) when \texttt{sym} is \texttt{NIL}. \texttt{(with 'X . prg)}
is equivalent to \texttt{(let? This 'X . prg)}.


\begin{verbatim}
: (put 'X 'a 1)
-> 1
: (put 'X 'b 2)
-> 2
: (with 'X (list (: a) (: b)))
-> (1 2)
\end{verbatim}

 
\section{(wr 'cnt ..) -> cnt}
\label{sec-8-1-23-9}


Writes all \texttt{cnt} arguments as raw bytes to the current output channel.
See also \texttt{rd} and \texttt{pr}.


\begin{verbatim}
: (out "x" (wr 1 255 257))  # Write to "x"
-> 257
: (hd "x")
00000000  01 FF 01                                         ...
-> NIL
\end{verbatim}

 
\section{(wrap 'cnt 'lst) -> sym}
\label{sec-8-1-23-10}


Returns a transient symbol with all characters in \texttt{lst} \texttt{pack=ed in lines with a maximal length of =cnt}. See also \texttt{tab}, \texttt{align} and
\texttt{center}.


\begin{verbatim}
: (wrap 20 (chop "The quick brown fox jumps over the lazy dog"))
-> "The quick brown fox^Jjumps over the lazy^Jdog"
: (wrap 8 (chop "The quick brown fox jumps over the lazy dog"))
-> "The^Jquick^Jbrown^Jfox^Jjumps^Jover the^Jlazy dog"
\end{verbatim}



\chapter{Functions starting with X}
\label{sec-8-1-24}


 
\section{(xchg 'var 'var ..) -> any}
\label{sec-8-1-24-1}


Exchange the values of successive \texttt{var} argument pairs.


\begin{verbatim}
: (setq  A 1  B 2  C '(a b c))
-> (a b c)
: (xchg  'A C  'B (cdr C))
-> 2
: A
-> a
: B
-> b
: C
-> (1 2 c)
\end{verbatim}

 
\section{(xor 'any 'any) -> flg}
\label{sec-8-1-24-2}


Returns T if exactly one of the arguments evaluates to non-=NIL=.


\begin{verbatim}
: (xor T NIL)
-> T
: (xor T T)
-> NIL
\end{verbatim}

 
\section{(x| 'num ..) -> num}
\label{sec-8-1-24-3}


Returns the bitwise \texttt{XOR} of all \texttt{num} arguments. When one of the
arguments evaluates to \texttt{NIL}, it is returned immediately. See also \texttt{\&},
\texttt{|} and \texttt{bit?}.


\begin{verbatim}
: (x| 2 7)
-> 5
: (x| 2 7 1)
-> 4
\end{verbatim}



\chapter{Functions starting with Y}
\label{sec-8-1-25}


 
\section{(yield 'any ['sym]) -> any}
\label{sec-8-1-25-1}


(64-bit version only) Transfers control from the current
\hyperref[ref.html-coroutines]{coroutine} back to the caller (when the \texttt{sym}
tag is not given), or to some other coroutine (specified by \texttt{sym}) to
continue execution at the point where that coroutine had called \texttt{yield}
before. In the first case, the value \texttt{any} will be returned from the
corresponding \texttt{co} call, in the second case it will be the return value
of that \texttt{yield} call. See also \texttt{stack}, \texttt{catch} and \texttt{throw}.


\begin{verbatim}
: (co "rt1"                            # Start first routine
   (msg (yield 1) " in rt1 from rt2")  # Return '1', wait for value from "rt2"
   7 )                                 # Then return '7'
-> 1

: (co "rt2"                            # Start second routine
   (yield 2 "rt1") )                   # Send '2' to "rt1"
2 in rt1 from rt2
-> 7
\end{verbatim}

 
\section{(yoke 'any ..) -> any}
\label{sec-8-1-25-2}


Inserts one or several new elements \texttt{any} in front of the list in the
current \texttt{make} environment. \texttt{yoke} returns the last inserted argument.
See also \texttt{link}, \texttt{chain} and \texttt{made}.


\begin{verbatim}
: (make (link 2 3) (yoke 1) (link 4))
-> (1 2 3 4)
\end{verbatim}



\chapter{Functions starting with Z}
\label{sec-8-1-26}


 
\section{*Zap}
\label{sec-8-1-26-1}


A global variable holding a list and a pathname. If given, and the value
of \texttt{*Solo} is \texttt{NIL}, external symbols which are no longer accessible can
be collected in the CAR, e.g. during DB tree processing, and written to
the file in the CDR at the next \texttt{commit}. A (typically periodic) call to
\texttt{zap\_} will clean them up later.


\begin{verbatim}
: (setq *Zap '(NIL . "db/app/_zap"))
-> "db/app/_zap"
\end{verbatim}

 
\section{(zap 'sym) -> sym}
\label{sec-8-1-26-2}


``Delete'' the symbol \texttt{sym}. For internal symbols, that means to remove it
from the internal index, effectively transforming it to a transient
symbol. For external symbols, it means to mark it as ``deleted'', so that
upon a later \texttt{commit} it will be removed from the database file. See
also \texttt{intern}.


\begin{verbatim}
: (de foo (Lst) (car Lst))          # 'foo' calls 'car'
-> foo
: (zap 'car)                        # Delete the symbol 'car'
-> "car"
: (pp 'foo)
(de foo (Lst)
   ("car" Lst) )                    # 'car' is now a transient symbol
-> foo
: (foo (1 2 3))                     # 'foo' still works
-> 1
: (car (1 2 3))                     # Reader returns a new 'car' symbol
!? (car (1 2 3))
car -- Undefined
?
\end{verbatim}

 
\section{(zapTree 'sym)}
\label{sec-8-1-26-3}


Recursively deletes a tree structure from the database. See also \texttt{tree},
\texttt{chkTree} and \texttt{prune}.


\begin{verbatim}
: (zapTree (cdr (root (tree 'nm '+Item))))
\end{verbatim}

 
\section{(zap\_)}
\label{sec-8-1-26-4}


Delayed deletion (with \texttt{zap}) of external symbols which were collected
e.g. during DB tree processing. An auxiliary file (with the name taken
from the CDR of the value of \texttt{*Zap}, concatenated with a ``=_=''
character) is used as an intermediary file.


\begin{verbatim}
: *Zap
-> (NIL . "db/app/Z")
: (call 'ls "-l" "db/app")
...
-rw-r--r-- 1 abu abu     1536 2007-06-23 12:34 Z
-rw-r--r-- 1 abu abu     1280 2007-05-23 12:15 Z_
...
: (zap_)
...
: (call 'ls "-l" "db/app")
...
-rw-r--r-- 1 abu abu     1536 2007-06-23 12:34 Z_
...
\end{verbatim}

              
\section{(zero var ..) -> 0} 
\label{sec-8-1-26-5}


Stores \texttt{0} in all \texttt{var} arguments. See also \texttt{one}, \texttt{on}, \texttt{off} and
\texttt{onOff}.


\begin{verbatim}
: (zero A B)
-> 0
: A
-> 0
: B
-> 0
\end{verbatim}


% Local variables:
% mode: latex
% TeX-master: "../../editor.tex"
% End:
