%%%%%%%%%%%%%%%%%%%%% chapter.tex %%%%%%%%%%%%%%%%%%%%%%%%%%%%%%%%%
%
% sample chapter
%
% Use this file as a template for your own input.
%
%%%%%%%%%%%%%%%%%%%%%%%% Springer-Verlag %%%%%%%%%%%%%%%%%%%%%%%%%%
%\motto{Use the template \emph{chapter.tex} to style the various elements of your chapter content.}

\chapter{Symbols starting with B}
\label{cha:funct-ref-B-symbols-starting-B}
 
\section*{\texttt{*Blob}}
\label{sec:funct-ref-B-*blob}


A global variable holding the pathname of the database blob directory.
See also \texttt{blob}.


\begin{wideverbatim}
: *Blob
-> "blob/app/"
\end{wideverbatim}

 
\section*{\texttt{*Bye}}
\label{sec:funct-ref-B-*bye}


A global variable holding a (possibly empty) \texttt{prg} body, to be executed
just before the termination of the PicoLisp interpreter. See also \texttt{bye}
and \texttt{tmp}.


\begin{wideverbatim}
: (push1 '*Bye '(call 'rm "myfile.tmp"))  # Remove a temporary file
-> (call 'rm "myfile.tmp")
\end{wideverbatim}

 
\section*{\texttt{+Bag}}
\label{sec:funct-ref-B-+bag}


Class for a list of arbitrary relations, a subclass of \texttt{+relation}.
Objects of that class maintain a list of heterogeneous relations.
Typically used in combination with the \texttt{+List} prefix class, to maintain
small two-dimensional tables within oubjects. See also \texttt{Database}.


\begin{wideverbatim}
(rel pos (+List +Bag)         # Positions
   ((+Ref +Link) NIL (+Item))    # Item
   ((+Number) 2)                 # Price
   ((+Number))                   # Quantity
   ((+String))                   # Memo text
   ((+Number) 2) )               # Total amount
\end{wideverbatim}

 
\section*{\texttt{+Blob}}
\label{sec:funct-ref-B-+blob}


Class for blob relations, a subclass of \texttt{+relation}. Objects of that
class maintain blobs, as stubs in database objects pointing to actual
files for arbitrary (often binary) data. The files themselves reside
below the path specified by the \texttt{*Blob} variable. See also \texttt{Database}.


\begin{wideverbatim}
(rel jpg (+Blob))  # Picture
\end{wideverbatim}


\section*{\texttt{+Bool}}
\label{sec:funct-ref-B-+bool}

Class for boolean relations, a subclass of \texttt{+relation}. Objects of that
class expect either \texttt{T} or \texttt{NIL} as value (though, as always, only
non-\texttt{NIL} will be physically stored in objects). See also \texttt{Database}.


\begin{wideverbatim}
(rel ok (+Ref +Bool))  # Indexed flag
\end{wideverbatim}

 
\section*{\texttt{(balance 'var 'lst ['flg])}}
\label{sec:funct-ref-B-(balance-'var-'lst-['flg])}


Builds a balanced binary \texttt{idx} tree in \texttt{var}, from the sorted list in
\texttt{lst}. Normally (if random or, in the worst case, ordered data) are
inserted with \texttt{idx}, the tree will not be balanced. But if \texttt{lst} is
properly sorted, its contents will be inserted in an optimally balanced
way. If \texttt{flg} is non-\texttt{NIL}, the index tree will be augmented instead of
being overwritten. See also \texttt{Comparing} and \texttt{sort}.


\begin{wideverbatim}
# Normal idx insert
: (off I)
-> NIL
: (for X (1 4 2 5 3 6 7 9 8) (idx 'I X T))
-> NIL
: (depth I)
-> 7

# Balanced insert
: (balance 'I (sort (1 4 2 5 3 6 7 9 8)))
-> NIL
: (depth I)
-> 4

# Augment
: (balance 'I (sort (10 40 20 50 30 60 70 90 80)) T)
-> NIL
: (idx 'I)
-> (1 2 3 4 5 6 7 8 9 10 20 30 40 50 60 70 80 90)
\end{wideverbatim}

 
\section*{\texttt{(basename 'any) -> sym}}
\label{sec:funct-ref-B-(basename-'any)-->-sym}


Returns the filename part of a path name \texttt{any}. See also \texttt{dirname} and
\texttt{path}.


\begin{wideverbatim}
: (basename "a/b/c/d")
-> "d"
\end{wideverbatim}

 
\section*{\texttt{(be sym . any) -> sym}}
\label{sec:funct-ref-B-(be-sym-.-any)-->-sym}


Declares a \hyperref[ref.html-pilog]{Pilog} fact or rule for the \texttt{sym}
argument, by concatenating the \texttt{any} argument to the \texttt{T} property of
\texttt{sym}. See also \texttt{clause}, \texttt{asserta}, \texttt{assertz}, \texttt{retract}, \texttt{goal} and
\texttt{prove}.


\begin{wideverbatim}
: (be likes (John Mary))
-> likes
: (be likes (John @X) (likes @X wine) (likes @X food))
-> likes
: (get 'likes T)
-> (((John Mary)) ((John @X) (likes @X wine) (likes @X food)))
: (? (likes John @X))
 @X=Mary
-> NIL
\end{wideverbatim}

 
\section*{\texttt{(beep) -> any}}
\label{sec:funct-ref-B-(beep)-->-any}


Send the bell character to the console. See also \texttt{prin} and \texttt{char}.


\begin{wideverbatim}
: (beep)
-> "^G"
\end{wideverbatim}

 
\section*{\texttt{(bench . prg) -> any}}
\label{sec:funct-ref-B-(bench-.-prg)-->-any}


Benchmarks \texttt{prg}, by printing the time it took to execute, and returns
the result. See also \texttt{usec}.


\begin{wideverbatim}
: (bench (wait 2000))
1.996 sec
-> NIL
\end{wideverbatim}

 
\section*{\texttt{(bin 'num ['num]) -> sym}}
\label{sec:funct-ref-B-(bin-'num-['num])-->-sym}


\texttt{(bin 'sym) -> num}

Converts a number \texttt{num} to a binary string, or a binary string \texttt{sym} to
a number. In the first case, if the second argument is given, the result
is separated by spaces into groups of such many digits. See also \texttt{oct},
\texttt{hex}, \texttt{fmt64}, \texttt{hax} and \texttt{format}.


\begin{wideverbatim}
: (bin 73)
-> "1001001"
: (bin "1001001")
-> 73
: (bin 1234567 4)
-> "100 1011 0101 1010 0001 11"
\end{wideverbatim}

 
\section*{\texttt{(bind 'sym|lst . prg) -> any}}
\label{sec:funct-ref-B-(bind-'sym|lst-.-prg)-->-any}


Binds value(s) to symbol(s). The first argument must evaluate to a
symbol, or a list of symbols or symbol-value pairs. The values of these
symbols are saved (and the symbols bound to the values in the case of
pairs), \texttt{prg} is executed, then the symbols are restored to their
original values. During execution of \texttt{prg}, the values of the symbols
can be temporarily modified. The return value is the result of \texttt{prg}.
See also \texttt{let}, \texttt{job} and \texttt{use}.


\begin{wideverbatim}
: (setq X 123)                               # X is 123
-> 123
: (bind 'X (setq X "Hello") (println X))  # Set X to "Hello", print it
"Hello"
-> "Hello"
: (bind '((X . 3) (Y . 4)) (println X Y) (* X Y))
3 4
-> 12
: X
-> 123                                       # X is restored to 123
\end{wideverbatim}

 
\section*{\texttt{(bit? 'num ..) -> num | NIL}}
\label{sec:funct-ref-B-(bit?-'num-..)-->-num-|-nil}


Returns the first \texttt{num} argument when all bits which are 1 in the first
argument are also 1 in all following arguments, otherwise \texttt{NIL}. When
one of those arguments evaluates to \texttt{NIL}, it is returned immediately.
See also \texttt{\&}, \texttt{|} and \texttt{x|}.


\begin{wideverbatim}
: (bit? 7 15 255)
-> 7
: (bit? 1 3)
-> 1
: (bit? 1 2)
-> NIL
\end{wideverbatim}

 
\section*{\texttt{(blob 'obj 'sym) -> sym}}
\label{sec:funct-ref-B-(blob-'obj-'sym)-->-sym}


Returns the blob file name for \texttt{var} in \texttt{obj}. See also \texttt{*Blob}, \texttt{blob!}
and \texttt{pack}.


\begin{wideverbatim}
: (show (db 'nr '+Item 1))
{3-1} (+Item)
   jpg
   pr 29900
   inv 100
   sup {2-1}
   nm "Main Part"
   nr 1
-> {3-1}
: (blob '{3-1} 'jpg)
-> "blob/app/3/-/1.jpg"
\end{wideverbatim}

 
\section*{\texttt{(blob! 'obj 'sym 'file)}}
\label{sec:funct-ref-B-(blob!-'obj-'sym-'file)}


Stores the contents of \texttt{file} in a \texttt{blob}. See also \texttt{put!>}.


\begin{wideverbatim}
(blob! *ID 'jpg "picture.jpg")
\end{wideverbatim}

 
\section*{\texttt{(bool 'any) -> flg}}
\label{sec:funct-ref-B-(bool-'any)-->-flg}


Returns \texttt{T} when the argument \texttt{any} is non-\texttt{NIL}. This function is only
needed when \texttt{T} is strictly required for a ``true'' condition (Usually,
any non-\texttt{NIL} value is considered to be ``true''). See also \texttt{flg?}.


\begin{wideverbatim}
: (and 3 4)
-> 4
: (bool (and 3 4))
-> T
\end{wideverbatim}

 
\section*{\texttt{bool/3}}
\label{sec:funct-ref-B-bool/3}


\hyperref[ref.html-pilog]{Pilog} predicate that succeeds if the first argument
has the same truth value as the result of applying the \texttt{get} algorithm
to the following arguments. Typically used as filter predicate in
\texttt{select/3} database queries. See also \texttt{bool}, \texttt{isa/2}, \texttt{same/3},
\texttt{range/3}, \texttt{head/3}, \texttt{fold/3}, \texttt{part/3} and \texttt{tolr/3}.


\begin{wideverbatim}
: (? @OK NIL         # Find orders where the 'ok' flag is not set
   (db nr +Ord @Ord)
   (bool @OK @Ord ok) )
 @OK=NIL @Ord={3-7}
-> NIL
\end{wideverbatim}

 
\section*{\texttt{(box 'any) -> sym}}
\label{sec:funct-ref-B-(box-'any)-->-sym}


Creates and returns a new anonymous symbol. The initial value is set to
the \texttt{any} argument. See also \texttt{new} and \texttt{box?}.


\begin{wideverbatim}
: (show (box '(A B C)))
$134425627 (A B C)
-> $134425627
\end{wideverbatim}

 
\section*{\texttt{(box? 'any) -> sym | NIL}}
\label{sec:funct-ref-B-(box?-'any)-->-sym-|-nil}


Returns the argument \texttt{any} when it is an anonymous symbol, otherwise
\texttt{NIL}. See also \texttt{box}, \texttt{str?} and \texttt{ext?}.


\begin{wideverbatim}
: (box? (new))
-> $134563468
: (box? 123)
-> NIL
: (box? 'a)
-> NIL
: (box? NIL)
-> NIL
\end{wideverbatim}

 
\section*{\texttt{(by 'fun1 'fun2 'lst ..) -> lst}}
\label{sec:funct-ref-B-(by-'fun1-'fun2-'lst-..)-->-lst}


Applies \texttt{fun1} to each element of \texttt{lst}. When additional \texttt{lst} arguments
are given, their elements are also passed to \texttt{fun1}. Each result of
\texttt{fun1} is CONSed with its corresponding argument form the original
\texttt{lst}, and collected into a list which is passed to \texttt{fun2}. For the list
returned from \texttt{fun2}, the CAR elements returned by \texttt{fun1} are
(destructively) removed from each element.


\begin{wideverbatim}
: (let (A 1 B 2 C 3) (by val sort '(C A B)))
-> (A B C)
: (by '((N) (bit? 1 N)) group (3 11 6 2 9 5 4 10 12 7 8 1))
-> ((3 11 9 5 7 1) (6 2 4 10 12 8))
\end{wideverbatim}

 
\section*{(bye 'cnt|NIL)}}
\label{sec:funct-ref-B-(bye-'cnt|nil)}


Executes all pending \texttt{finally} expressions, closes all open files,
executes the \texttt{VAL} of the global variable \texttt{*Bye} (should be a \texttt{prg}),
flushes standard output, and then exits the PicoLisp interpreter. The
process return value is \texttt{cnt}, or 0 if the argument is missing or \texttt{NIL}.


\begin{wideverbatim}
: (setq *Bye '((println 'OK) (println 'bye)))
-> ((println 'OK) (println 'bye))
: (bye)
OK
bye
$
\end{wideverbatim}

