%%%%%%%%%%%%%%%%%%%%% chapter.tex %%%%%%%%%%%%%%%%%%%%%%%%%%%%%%%%%
%
% sample chapter
%
% Use this file as a template for your own input.
%
%%%%%%%%%%%%%%%%%%%%%%%% Springer-Verlag %%%%%%%%%%%%%%%%%%%%%%%%%%
%\motto{Use the template \emph{chapter.tex} to style the various elements of your chapter content.}



\chapter{Symbols starting with G}
\label{sec:func-ref-G-}


 
\section*{\texttt{(gc ['cnt]) -> cnt | NIL}}
\label{sec:func-ref-G-(gc ['cnt]) -> cnt | NIL}


Forces a garbage collection. When \texttt{cnt} is given, so many megabytes of
free cells are reserved, increasing the heap size if necessary. If \texttt{cnt}
is zero, all currently unused heap blocks are purged, decreasing the
heap size if possible. See also \texttt{heap}.


\begin{wideverbatim}
: (gc)
-> NIL
: (heap)
-> 2
: (gc 4)
-> 4
: (heap)
-> 5
\end{wideverbatim}

 
\section*{\texttt{(ge0 'any) -> num | NIL}}
\label{sec:func-ref-G-(ge0 'any) -> num | NIL}


Returns \texttt{num} when the argument is a number and greater or equal zero,
otherwise \texttt{NIL}. See also \texttt{lt0}, \texttt{le0}, \texttt{gt0}, \texttt{=0} and \texttt{n0}.


\begin{wideverbatim}
: (ge0 -2)
-> NIL
: (ge0 3)
-> 3
: (ge0 0)
-> 0
\end{wideverbatim}

 
\section*{\texttt{(genKey 'var 'cls ['hook ['num1 ['num2]]]) -> num}}
\label{sec:func-ref-G-(genKey 'var 'cls ['hook ['num1 ['num2]]]) -> num}


Generates a key for a database tree. If a minimal key \texttt{num1} and/or a
maximal key \texttt{num2} is given, the next free number in that range is
returned. Otherwise, the current maximal key plus one is returned. See
also \texttt{useKey}, \texttt{genStrKey} and \texttt{maxKey}.


\begin{wideverbatim}
: (maxKey (tree 'nr '+Item))
-> 8
: (genKey 'nr '+Item)
-> 9
\end{wideverbatim}

 
\section*{\texttt{(genStrKey 'sym 'var 'cls ['hook]) -> sym}}
\label{sec:func-ref-G-(genStrKey 'sym 'var 'cls ['hook]) -> sym}


Generates a unique string for a database tree, by prepending as many ``\#''
sequences as necessary. See also \texttt{genKey}.


\begin{wideverbatim}
: (genStrKey "ben" 'nm '+User)
-> "# ben"
\end{wideverbatim}

 
\section*{\texttt{(get 'sym1|lst ['sym2|cnt ..]) -> any}}
\label{sec:func-ref-G-(get 'sym1|lst ['sym2|cnt ..]) -> any}


Fetches a value \texttt{any} from the properties of a symbol, or from a list.
From the first argument \texttt{sym1|lst}, values are retrieved in successive
steps by either extracting the value (if the next argument is zero) or a
property from a symbol, the \texttt{asoq}ed element (if the next argument is a
symbol), the n'th element (if the next argument is a positive number) or
the n'th CDR (if the next argument is a negative number) from a list.
See also \texttt{put}, \texttt{;} and \texttt{:}.


\begin{wideverbatim}
: (put 'X 'a 1)
-> 1
: (get 'X 'a)
-> 1
: (put 'Y 'link 'X)
-> X
: (get 'Y 'link)
-> X
: (get 'Y 'link 'a)
-> 1
: (get '((a (b . 1) (c . 2)) (d (e . 3) (f . 4))) 'a 'b)
-> 1
: (get '((a (b . 1) (c . 2)) (d (e . 3) (f . 4))) 'd 'f)
-> 4
: (get '(X Y Z) 2)
-> Y
: (get '(X Y Z) 2 'link 'a)
-> 1
\end{wideverbatim}

 
\section*{\texttt{(getd 'any) -> fun | NIL}}
\label{sec:func-ref-G-(getd 'any) -> fun | NIL}


Returns \texttt{fun} if \texttt{any} is a symbol that has a function definition,
otherwise \texttt{NIL}. See also \texttt{fun?}.


\begin{wideverbatim}
: (getd '+)
-> 67327232
: (getd 'script)
-> ((File . @) (load File))
: (getd 1)
-> NIL
\end{wideverbatim}

 
\section*{\texttt{(getl 'sym1|lst1 ['sym2|cnt ..]) -> lst}}
\label{sec:func-ref-G-(getl 'sym1|lst1 ['sym2|cnt ..]) -> lst}


Fetches the complete property list \texttt{lst} from a symbol. That symbol is
\texttt{sym1} (if no other arguments are given), or a symbol found by applying
the \texttt{get} algorithm to \texttt{sym1|lst1} and the following arguments. See also
\texttt{putl} and \texttt{maps}.


\begin{wideverbatim}
: (put 'X 'a 1)
-> 1
: (put 'X 'b 2)
-> 2
: (put 'X 'flg T)
-> T
: (getl 'X)
-> (flg (2 . b) (1 . a))
\end{wideverbatim}

 
\section*{\texttt{(glue 'any 'lst) -> sym}}
\label{sec:func-ref-G-(glue 'any 'lst) -> sym}


Builds a new transient symbol (string) by \texttt{pack}ing the
\texttt{any} argument between the individual elements of \texttt{lst}.
See also \texttt{text}.


\begin{wideverbatim}
: (glue "," '(a b c d))
-> "a,b,c,d"
\end{wideverbatim}

 
\section*{\texttt{(goal '([pat 'any ..] . lst) ['sym 'any ..]) -> lst}}
\label{sec:func-ref-G-(goal '([pat 'any ..] . lst) ['sym 'any ..]) -> lst}


Constructs a \emph{Pilog} query list from the list of
clauses \texttt{lst}. The head of the argument list may consist of a sequence
of pattern symbols (Pilog variables) and expressions, which are used
together with the optional \texttt{sym} and \texttt{any} arguments to form an initial
environment. See also \texttt{prove} and \texttt{fail}.


\begin{wideverbatim}
: (goal '((likes John @X)))
-> (((1 (0) NIL ((likes John @X)) NIL T)))
: (goal '(@X 'John (likes @X @Y)))
-> (((1 (0) NIL ((likes @X @Y)) NIL ((0 . @X) 1 . John) T)))
\end{wideverbatim}

 
\section*{\texttt{(group 'lst) -> lst}}
\label{sec:func-ref-G-(group 'lst) -> lst}


Builds a list of lists, by grouping all elements of \texttt{lst} with the same
CAR into a common sublist. See also \emph{Comparing}, \texttt{by},
\texttt{sort} and \texttt{uniq}.


\begin{wideverbatim}
: (group '((1 . a) (1 . b) (1 . c) (2 . d) (2 . e) (2 . f)))
-> ((1 a b c) (2 d e f))
: (by name group '("x" "x" "y" "z" "x" "z")))
-> (("x" "x" "x") ("y") ("z" "z"))
: (by length group '(123 (1 2) "abcd" "xyz" (1 2 3 4) "XY"))
-> ((123 "xyz") ((1 2) "XY") ("abcd" (1 2 3 4))
\end{wideverbatim}

 
\section*{\texttt{(gt0 'any) -> num | NIL}}
\label{sec:func-ref-G-(gt0 'any) -> num | NIL}


Returns \texttt{num} when the argument is a number and greater than
zero, otherwise \texttt{NIL}. See also \texttt{lt0}, \texttt{le0},
\texttt{ge0}, \texttt{=0} and \texttt{n0}.


\begin{wideverbatim}
: (gt0 -2)
-> NIL
: (gt0 3)
-> 3
\end{wideverbatim}




% \input{referenc}
