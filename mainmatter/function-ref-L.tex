%%%%%%%%%%%%%%%%%%%%% chapter.tex %%%%%%%%%%%%%%%%%%%%%%%%%%%%%%%%%
%
% sample chapter
%
% Use this file as a template for your own input.
%
%%%%%%%%%%%%%%%%%%%%%%%% Springer-Verlag %%%%%%%%%%%%%%%%%%%%%%%%%%
%\motto{Use the template \emph{chapter.tex} to style the various elements of your chapter content.}

\chapter{Symbols starting with L}
\label{cha:sec:func-ref-L-functions-starting-with-L}
 
\section*{\texttt{*Led}}
\label{sec:func-ref-L-*Led}

A global variable holding a (possibly empty) \texttt{prg} body that implements
a ``Line editor''. When non-\texttt{NIL}, it should return a single symbol
(string) upon execution.


\begin{wideverbatim}
: (de *Led "(bye)")
# *Led redefined
-> *Led
: $                                    # Exit
\end{wideverbatim}

 
\section*{\texttt{+Link}}
\label{sec:func-ref-L-+Link}


Class for object relations, a subclass of \texttt{+relation}. Expects a list of
classes as \texttt{type} of the referred database object (of class \texttt{+Entity}).
See also \texttt{Database}.


\begin{wideverbatim}
(rel sup (+Ref +Link) NIL (+CuSu))  # Supplier (class Customer/Supplier)
\end{wideverbatim}

 
\section*{\texttt{+List}}
\label{sec:func-ref-L-+List}


Prefix class for a list of identical relations. Objects of that class
maintain a list of Lisp data of uniform type. See also \texttt{Database}.


\begin{wideverbatim}
(rel pos (+List +Joint) ord (+Pos))  # Positions
(rel nm (+List +Fold +Ref +String))  # List of folded and indexed names
(rel val (+Ref +List +Number))       # Indexed list of numeric values
\end{wideverbatim}

 
\section*{\texttt{(last 'lst) -> any}}
\label{sec:func-ref-L-(last 'lst) -> any}


Returns the last element of \texttt{lst}. See also \texttt{fin} and \texttt{tail}.


\begin{wideverbatim}
: (last (1 2 3 4))
-> 4
: (last '((a b) c (d e f)))
-> (d e f)
\end{wideverbatim}

 
\section*{\texttt{(later 'var . prg) -> var}}
\label{sec:func-ref-L-(later 'var . prg) -> var}


Executes \texttt{prg} in a \texttt{pipe}'ed child process. The return value of \texttt{prg}
will later be available in \texttt{var}.


\begin{wideverbatim}
: (prog1  # Parallel background calculation of square numbers
   (mapcan '((N) (later (cons) (* N N))) (1 2 3 4))
   (wait NIL (full @)) )
-> (1 4 9 16)
\end{wideverbatim}

 
\section*{\texttt{(ld) -> any}}
\label{sec:func-ref-L-(ld) -> any}


\texttt{load}s the last file edited with \texttt{vi}.


\begin{wideverbatim}
: (vi 'main)
-> T
: (ld)
# main redefined
-> go
\end{wideverbatim}

 
\section*{\texttt{(le0 'any) -> num | NIL}}
\label{sec:func-ref-L-(le0 'any) -> num | NIL}


Returns \texttt{num} when the argument is a number less or equal zero,
otherwise \texttt{NIL}. See also \texttt{lt0}, \texttt{ge0},
\texttt{gt0}, \texttt{=0} and \texttt{n0}.


\begin{wideverbatim}
: (le0 -2)
-> -2
: (le0 0)
-> 0
: (le0 3)
-> NIL
\end{wideverbatim}

 
\section*{\texttt{(leaf 'tree) -> any}}
\label{sec:func-ref-L-(leaf 'tree) -> any}


Returns the first leaf (i.e. the value of the smallest key) in a
database tree. See also \texttt{tree}, \texttt{minKey}, \texttt{maxKey} and \texttt{step}.


\begin{wideverbatim}
: (leaf (tree 'nr '+Item))
-> {3-1}
: (db 'nr '+Item (minKey (tree 'nr '+Item)))
-> {3-1}
\end{wideverbatim}

 
\section*{\texttt{(length 'any) -> cnt | T}}
\label{sec:func-ref-L-(length 'any) -> cnt | T}


Returns the ``length'' of \texttt{any}. For numbers this is the number of decimal
digits in the value (plus 1 for negative values), for symbols it is the
number of characters in the name, and for lists it is the number of
elements (or \texttt{T} for circular lists). See also \texttt{size}.


\begin{wideverbatim}
: (length "abc")
-> 3
: (length "äbc")
-> 3
: (length 123)
-> 3
: (length (1 (2) 3))
-> 3
: (length (1 2 3 .))
-> T
\end{wideverbatim}

 
\section*{\texttt{(let sym 'any . prg) -> any}}
\label{sec:func-ref-L-(let sym 'any . prg) -> any}


\texttt{(let (sym 'any ..) . prg) -> any}

Defines local variables. The value of the symbol \texttt{sym} - or the values
of the symbols \texttt{sym} in the list of the second form - are saved and the
symbols are bound to the evaluated \texttt{any} arguments. \texttt{prg} is executed,
then the symbols are restored to their original values. The result of
\texttt{prg} is returned. It is an error condition to pass \texttt{NIL} as a \texttt{sym}
argument. See also \texttt{let?}, \texttt{bind}, \texttt{recur}, \texttt{with}, \texttt{for}, \texttt{job} and
\texttt{use}.


\begin{wideverbatim}
: (setq  X 123  Y 456)
-> 456
: (let X "Hello" (println X))
"Hello"
-> "Hello"
: (let (X "Hello" Y "world") (prinl X " " Y))
Hello world
-> "world"
: X
-> 123
: Y
-> 456
\end{wideverbatim}

 
\section*{\texttt{(let? sym 'any . prg) -> any}}
\label{sec:func-ref-L-(let? sym 'any . prg) -> any}


Conditional local variable binding and execution: If \texttt{any} evalutes to
\texttt{NIL}, \texttt{NIL} is returned. Otherwise, the value of the symbol \texttt{sym} is
saved and \texttt{sym} is bound to the evaluated \texttt{any} argument. \texttt{prg} is
executed, then \texttt{sym} is restored to its original value. The result of
\texttt{prg} is returned. It is an error condition to pass \texttt{NIL} as the \texttt{sym}
argument. \texttt{(let? sym 'any ..)} is equivalent to
\texttt{(when 'any (let sym @ ..))}. See also \texttt{let}, \texttt{bind}, \texttt{job} and \texttt{use}.


\begin{wideverbatim}
: (setq Lst (1 NIL 2 NIL 3))
-> (1 NIL 2 NIL 3)
: (let? A (pop 'Lst) (println 'A A))
A 1
-> 1
: (let? A (pop 'Lst) (println 'A A))
-> NIL
\end{wideverbatim}

 
\section*{\texttt{(lieu 'any) -> sym | NIL}}
\label{sec:func-ref-L-(lieu 'any) -> sym | NIL}


Returns the argument \texttt{any} when it is an external symbol and currently
manifest in heap space, otherwise \texttt{NIL}. See also \texttt{ext?}.


\begin{wideverbatim}
: (lieu *DB)
-> {1}
\end{wideverbatim}

 
\section*{\texttt{(line 'flg ['cnt ..]) -> lst|sym}}
\label{sec:func-ref-L-(line 'flg ['cnt ..]) -> lst|sym}


Reads a line of characters from the current input channel. End of line
is recognized as linefeed (hex ``0A''), carriage return (hex ``0D''), or the
combination of both. (Note that a single carriage return may not work on
network connections, because the character look-ahead to distinguish
from return+linefeed can block the connection.) If \texttt{flg} is \texttt{NIL}, a
list of single-character transient symbols is returned. When \texttt{cnt}
arguments are given, subsequent characters of the input line are grouped
into sublists, to allow parsing of fixed field length records. If \texttt{flg}
is non-\texttt{NIL}, strings are returned instead of single-character lists.
\texttt{NIL} is returned upon end of file. See also \texttt{char}, \texttt{till} and \texttt{eof}.


\begin{wideverbatim}
: (line)
abcdefghijkl
-> ("a" "b" "c" "d" "e" "f" "g" "h" "i" "j" "k" "l")
: (line T)
abcdefghijkl
-> "abcdefghijkl"
: (line NIL 1 2 3)
abcdefghijkl
-> (("a") ("b" "c") ("d" "e" "f") "g" "h" "i" "j" "k" "l")
: (line T 1 2 3)
abcdefghijkl
-> ("a" "bc" "def" "g" "h" "i" "j" "k" "l")
\end{wideverbatim}

 
\section*{\texttt{(lines 'any ..) -> cnt}}
\label{sec:func-ref-L-(lines 'any ..) -> cnt}


Returns the sum of the number of lines in the files with the names
\texttt{any}, or \texttt{NIL} if none was found. See also \texttt{info}.


\begin{wideverbatim}
: (lines "x.l")
-> 11
\end{wideverbatim}

 
\section*{\texttt{(link 'any ..) -> any}}
\label{sec:func-ref-L-(link 'any ..) -> any}


Links one or several new elements \texttt{any} to the end of the list in the
current \texttt{make} environment. This operation is efficient also for long
lists, because a pointer to the last element of the list is maintained.
\texttt{link} returns the last linked argument. See also \texttt{yoke}, \texttt{chain} and
\texttt{made}.


\begin{wideverbatim}
: (make
   (println (link 1))
   (println (link 2 3)) )
1
3
-> (1 2 3)
\end{wideverbatim}

 
\section*{\texttt{(lint 'sym) -> lst}}
\label{sec:func-ref-L-(lint 'sym) -> lst}


\texttt{(lint 'sym 'cls) -> lst}

\texttt{(lint '(sym . cls)) -> lst}

Checks the function definition or file contents (in the first form), or
the method body of sym (second and third form), for possible pitfalls.
Returns an association list of diagnoses, where \texttt{var} indicates improper
variables, \texttt{dup} duplicate parameters, \texttt{def} an undefined function,
\texttt{bnd} an unbound variable, and \texttt{use} unused variables. See also
\texttt{noLint}, \texttt{lintAll}, \texttt{debug}, \texttt{trace} and \texttt{*Dbg}.


\begin{wideverbatim}
: (de foo (R S T R)     # 'T' is a improper parameter, 'R' is duplicated
   (let N 7             # 'N' is unused
      (bar X Y) ) )     # 'bar' is undefined, 'X' and 'Y' are not bound
-> foo
: (lint 'foo)
-> ((var T) (dup R) (def bar) (bnd Y X) (use N))
\end{wideverbatim}

 
\section*{\texttt{(lintAll ['sym ..]) -> lst}}
\label{sec:func-ref-L-(lintAll ['sym ..]) -> lst}


Applies \texttt{lint} to \texttt{all} internal symbols - and optionally to all files
\texttt{sym} - and returns a list of diagnoses. See also \texttt{noLint}.


\begin{wideverbatim}
: (more (lintAll "file1.l" "file2.l"))
...
\end{wideverbatim}

 
\section*{\texttt{(lisp 'sym ['fun]) -> num}}
\label{sec:func-ref-L-(lisp 'sym ['fun]) -> num}


(64-bit version only) Installs under the tag \texttt{sym} a callback function
\texttt{fun}, and returns a pointer \texttt{num} suitable to be passed to a C function
via `native'. If \texttt{fun} is \texttt{NIL}, the corresponding entry is freed.
Maximally 24 callback functions can be installed that way. `fun' should
be a function of maximally five numbers, and should return a number.
``Numbers'' in this context are 64-bit scalars, and may not only represent
integers, but also pointers or other encoded data. See also \texttt{native}.


\begin{wideverbatim}
(load "lib/native.l")

(gcc "ltest" NIL
   (cbTest (Fun) cbTest 'N Fun) )

long cbTest(int(*fun)(int,int,int,int,int)) {
   return fun(1,2,3,4,5);
}
/**/

: (cbTest
   (lisp 'cbTest
      '((A B C D E)
         (msg (list A B C D E))
         (* A B C D E) ) ) )
(1 2 3 4 5)
-> 120
\end{wideverbatim}

 
\section*{\texttt{(list 'any ['any ..]) -> lst}}
\label{sec:func-ref-L-(list 'any ['any ..]) -> lst}


Returns a list of all \texttt{any} arguments. See also \texttt{cons}.


\begin{wideverbatim}
: (list 1 2 3 4)
-> (1 2 3 4)
: (list 'a (2 3) "OK")
-> (a (2 3) "OK")
\end{wideverbatim}

 
\section*{\texttt{lst/3}}
\label{sec:func-ref-L-lst/3}


\emph{Pilog} predicate that returns subsequent list
elements, after applying the \texttt{get} algorithm to that object and the
following arguments. Often used in database queries. See also \texttt{map/3}.


\begin{wideverbatim}
: (? (db nr +Ord 1 @Ord) (lst @Pos @Ord pos))
 @Ord={3-7} @Pos={4-1}
 @Ord={3-7} @Pos={4-2}
 @Ord={3-7} @Pos={4-3}
-> NIL
\end{wideverbatim}

 
\section*{\texttt{(lst? 'any) -> flg}}
\label{sec:func-ref-L-(lst? 'any) -> flg}


Returns \texttt{T} when the argument \texttt{any} is a (possibly empty) list (\texttt{NIL} or
a cons pair cell). See also \texttt{pair}.


\begin{wideverbatim}
: (lst? NIL)
-> T
: (lst? (1 . 2))
-> T
: (lst? (1 2 3))
-> T
\end{wideverbatim}

 
\section*{\texttt{(listen 'cnt1 ['cnt2]) -> cnt | NIL}}
\label{sec:func-ref-L-(listen 'cnt1 ['cnt2]) -> cnt | NIL}


Listens at a socket descriptor \texttt{cnt1} (as received by \texttt{port}) for an
incoming connection, and returns the new socket descriptor \texttt{cnt}. While
waiting for a connection, a \texttt{select} system call is executed for all
file descriptors and timers in the \texttt{VAL} of the global variable \texttt{*Run}.
If \texttt{cnt2} is non-\texttt{NIL}, that amount of milliseconds is waited maximally,
and \texttt{NIL} is returned upon timeout. The global variable \texttt{*Adr} is set to
the IP address of the client. See also \texttt{accept}, \texttt{connect}, \texttt{*Adr}.


\begin{wideverbatim}
: (setq *Socket
   (listen (port 6789) 60000) )  # Listen at port 6789 for max 60 seconds
-> 4
: *Adr
-> "127.0.0.1"
\end{wideverbatim}

 
\section*{\texttt{(lit 'any) -> any}}
\label{sec:func-ref-L-(lit 'any) -> any}


Returns the literal (i.e. quoted) value of \texttt{any}, by \texttt{cons}ing it with the \texttt{quote} function if necessary.


\begin{wideverbatim}
: (lit T)
-> T
: (lit 1)
-> 1
: (lit '(1))
-> (1)
: (lit '(a))
-> '(a)
\end{wideverbatim}

 
\section*{\texttt{(load 'any ..) -> any}}
\label{sec:func-ref-L-(load 'any ..) -> any}


Loads all \texttt{any} arguments. Normally, the name of each argument is taken
as a file to be executed in a read-eval loop. The argument semantics are
identical to that of \texttt{in}, with the exception that if an argument is a
symbol and its first character is a hyphen `-', then that argument is
parsed as an executable list (without the surrounding parentheses). When
\texttt{any} is \texttt{T}, all remaining command line arguments are \texttt{load}ed recursively. When \texttt{any} is \texttt{NIL}, standard input is read, a prompt is
issued before each read operation, the results are printed to standard
output (read-eval-print loop), and \texttt{load} terminates when an empty line
is entered. In any case, \texttt{load} terminates upon end of file, or when
\texttt{NIL} is read. The index for transient symbols is cleared before and
after the load, so that all transient symbols in a file have a local
scope. If the namespace was switched (with \texttt{symbols}) while executing a
file, it is restored to the previous one. Returns the value of the last
evaluated expression. See also \texttt{script}, \texttt{ipid}, \texttt{call}, \texttt{file}, \texttt{in},
\texttt{out} and \texttt{str}.


\begin{wideverbatim}
: (load "lib.l" "-* 1 2 3")
-> 6
\end{wideverbatim}

 
\section*{\texttt{(loc 'sym 'lst) -> sym}}
\label{sec:func-ref-L-(loc 'sym 'lst) -> sym}


Locates in \texttt{lst} a \texttt{transient} symbol with the same name as \texttt{sym}.
Allows to get hold of otherwise inaccessible symbols. See also \texttt{====}.


\begin{wideverbatim}
: (loc "X" curry)
-> "X"
: (== @ "X")
-> NIL
\end{wideverbatim}

 
\section*{\texttt{(local lst) -> sym}}
\label{sec:func-ref-L-(local lst) -> sym}


Wrapper function for \texttt{zap}. Typically used to create namespace-local
symbols. \texttt{lst} should be a list of symbols. See also \texttt{pico}, \texttt{symbols},
\texttt{import} and \texttt{intern}.


\begin{wideverbatim}
(symbols 'myLib 'pico)

(local bar foo)
(de foo (A)  # 'foo' is local to 'myLib'
   ...
(de bar (B)  # 'bar' is local to 'myLib'
   ...
\end{wideverbatim}

 
\section*{\texttt{(locale 'sym1 'sym2 ['sym ..])}}
\label{sec:func-ref-L-(locale 'sym1 'sym2 ['sym ..])}


Sets the current locale to that given by the country file \texttt{sym1} and the
language file \texttt{sym2} (both located in the ``loc/'' directory), and
optional application-specific directories \texttt{sym}. The locale influences
the language, and numerical, date and other formats. See also \texttt{*Uni},
\texttt{datStr}, \texttt{strDat}, \texttt{expDat}, \texttt{day}, \texttt{telStr}, \texttt{expTel} and and \texttt{money}.


\begin{wideverbatim}
: (locale "DE" "de" "app/loc/")
-> "Zip"
: ,"Yes"
-> "Ja"
\end{wideverbatim}

 
\section*{\texttt{(lock ['sym]) -> cnt | NIL}}
\label{sec:func-ref-L-(lock ['sym]) -> cnt | NIL}


Write-locks an external symbol \texttt{sym} (file record locking), or the whole
database root file if \texttt{sym} is \texttt{NIL}. Returns \texttt{NIL} if successful, or
the ID of the process currently holding the lock. When \texttt{sym} is
non-\texttt{NIL}, the lock is released at the next top level call to \texttt{commit}
or \texttt{rollback}, otherwise only when another database is opened with
\texttt{pool}, or when the process terminates. See also \texttt{*Solo}.


\begin{wideverbatim}
: (lock '{1})        # Lock single object
-> NIL
: (lock)             # Lock whole database
-> NIL
\end{wideverbatim}

 
\section*{\texttt{(loop ['any | (NIL 'any . prg) | (T 'any . prg) ..]) -> any}}
\label{sec:func-ref-L-(loop ['any | (NIL 'any . prg) | (T 'any . prg) ..]) -> any}


Endless loop with multiple conditional exits: The body is executed an
unlimited number of times. If a clause has \texttt{NIL} or \texttt{T} as its CAR, the
clause's second element is evaluated as a condition and - if the result
is \texttt{NIL} or non-\texttt{NIL}, respectively - the \texttt{prg} is executed and the
result returned. See also \texttt{do} and \texttt{for}.


\begin{wideverbatim}
: (let N 3
   (loop
      (prinl N)
      (T (=0 (dec 'N)) 'done) ) )
3
2
1
-> done
\end{wideverbatim}

 
\section*{\texttt{(low? 'any) -> sym | NIL}}
\label{sec:func-ref-L-(low? 'any) -> sym | NIL}


Returns \texttt{any} when the argument is a string (symbol) that starts with a
lowercase character. See also \texttt{lowc} and \texttt{upp?}


\begin{wideverbatim}
: (low? "a")
-> "a"
: (low? "A")
-> NIL
: (low? 123)
-> NIL
: (low? ".")
-> NIL
\end{wideverbatim}

 
\section*{\texttt{(lowc 'any) -> any}}
\label{sec:func-ref-L-(lowc 'any) -> any}


Lower case conversion: If \texttt{any} is not a symbol, it is returned as it
is. Otherwise, a new transient symbol with all characters of \texttt{any},
converted to lower case, is returned. See also \texttt{uppc}, \texttt{fold} and
\texttt{low?}.


\begin{wideverbatim}
: (lowc 123)
-> 123
: (lowc "ABC")
-> "abc"
\end{wideverbatim}

 
\section*{\texttt{(lt0 'any) -> num | NIL}}
\label{sec:func-ref-L-(lt0 'any) -> num | NIL}


Returns \texttt{num} when the argument is a number and less than zero,
otherwise \texttt{NIL}. See also \texttt{le0}, \texttt{ge0}, \texttt{gt0}, \texttt{=0} and \texttt{n0}.


\begin{wideverbatim}
: (lt0 -2)
-> -2
: (lt0 3)
-> NIL
\end{wideverbatim}

 
\section*{\texttt{(lup 'lst 'any) -> lst}}
\label{sec:func-ref-L-(lup 'lst 'any) -> lst}


\texttt{(lup 'lst 'any 'any2) -> lst}

Looks up \texttt{any} in the CAR-elements of cells stored in the index tree
\texttt{lst}, as built-up by \texttt{idx}. In the first form, the first found cell is
returned, in the second form a list of all cells whose CAR is in the
range \texttt{any} .. \texttt{any2}. See also \texttt{assoc}.


\begin{wideverbatim}
: (idx 'A 'a T)
-> NIL
: (idx 'A (1 . b) T)
-> NIL
: (idx 'A 123 T)
-> NIL
: (idx 'A (1 . a) T)
-> NIL
: (idx 'A (1 . c) T)
-> NIL
: (idx 'A (2 . d) T)
-> NIL
: (idx 'A)
-> (123 a (1 . a) (1 . b) (1 . c) (2 . d))
: (lup A 1)
-> (1 . b)
: (lup A 2)
-> (2 . d)
: (lup A 1 1)
-> ((1 . a) (1 . b) (1 . c))
: (lup A 1 2)
-> ((1 . a) (1 . b) (1 . c) (2 . d))
\end{wideverbatim}

