%%%%%%%%%%%%%%%%%%%%% chapter.tex %%%%%%%%%%%%%%%%%%%%%%%%%%%%%%%%%
%
% sample chapter
%
% Use this file as a template for your own input.
%
%%%%%%%%%%%%%%%%%%%%%%%% Springer-Verlag %%%%%%%%%%%%%%%%%%%%%%%%%%
%\motto{Use the template \emph{chapter.tex} to style the various elements of your chapter content.}



\chapter{Symbols starting with Q}
\label{cha:func-ref-Q-functions-starting-with-Q}
 
\section*{\texttt{(qsym . sym) -> lst}}
\label{sec:func-ref-Q-(qsym . sym) -> lst}


Returns a cons pair of the value and property list of \texttt{sym}. See also
\texttt{quote}, \texttt{val} and \texttt{getl}.


\begin{wideverbatim}
: (setq A 1234)
-> 1234
: (put 'A 'a 1)
-> 1
: (put 'A 'b 2)
-> 2
: (put 'A 'f T)
-> T
: (qsym . A)
-> (1234 f (2 . b) (1 . a))
\end{wideverbatim}

 
\section*{\texttt{(quote . any) -> any}}
\label{sec:func-ref-Q-(quote . any) -> any}


Returns \texttt{any} unevaluated. The reader recognizes the single quote char
\texttt{'} as a macro for this function. See also \texttt{lit}.


\begin{wideverbatim}
: 'a
-> a
: '(foo a b c)
-> (foo a b c)
: (quote (quote (quote a)))
-> ('('(a)))
\end{wideverbatim}

 
\section*{\texttt{(query 'lst ['lst]) -> flg}}
\label{sec:func-ref-Q-(query 'lst ['lst]) -> flg}


Handles an interactive \emph{Pilog} query. The two \texttt{lst}
arguments are passed to \texttt{prove}. \texttt{query} displays each result, waits for
console input, and terminates when a non-empty line is entered. See also
\texttt{?}, \texttt{pilog} and \texttt{solve}.


\begin{wideverbatim}
: (query (goal '((append @X @Y (a b c)))))
 @X=NIL @Y=(a b c)
 @X=(a) @Y=(b c).   # Stop
-> NIL
\end{wideverbatim}

 
\section*{\texttt{(queue 'var 'any) -> any}}
\label{sec:func-ref-Q-(queue 'var 'any) -> any}


Implements a queue using a list in \texttt{var}. The \texttt{any} argument is
(destructively) concatenated to the end of the value list. See also
\texttt{push}, \texttt{pop} and \texttt{fifo}.


\begin{wideverbatim}
: (queue 'A 1)
-> 1
: (queue 'A 2)
-> 2
: (queue 'A 3)
-> 3
: A
-> (1 2 3)
: (pop 'A)
-> 1
: A
-> (2 3)
\end{wideverbatim}

 
\section*{\texttt{(quit ['any ['any]])}}
\label{sec:func-ref-Q-(quit ['any ['any]])}


Stops current execution. If no arguments are given, all pending
\texttt{finally} expressions are executed and control is returned to the top
level read-eval-print loop. Otherwise, an error handler is entered. The
first argument can be some error message, and the second might be the
reason for the error. See also \texttt{Error Handling}.


\begin{wideverbatim}
: (de foo (X) (quit "Sorry, my error" X))
-> foo
: (foo 123)                                  # 'X' is bound to '123'
123 -- Sorry, my error                       # Error entered
? X                                          # Inspect 'X'
-> 123
?                                            # Empty line: Exit
:
\end{wideverbatim}


