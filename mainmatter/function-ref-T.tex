%%%%%%%%%%%%%%%%%%%%% chapter.tex %%%%%%%%%%%%%%%%%%%%%%%%%%%%%%%%%
%
% sample chapter
%
% Use this file as a template for your own input.
%
%%%%%%%%%%%%%%%%%%%%%%%% Springer-Verlag %%%%%%%%%%%%%%%%%%%%%%%%%%
%\motto{Use the template \emph{chapter.tex} to style the various elements of your chapter content.}



\chapter{Symbols starting with T}
\label{cha:func-ref-T-functions-starting-with-T}
 
\section*{\texttt{*Tmp}}
\label{sec:func-ref-T-*Tmp}


A global variable holding the temporary directory name created with
\texttt{tmp}. See also \texttt{*Bye}.


\begin{wideverbatim}
: *Bye
-> ((saveHistory) (and *Tmp (call 'rm "-r" *Tmp)))
: (tmp "foo" 123)
-> "/home/app/.pil/tmp/27140/foo123"
: *Tmp
-> "/home/app/.pil/tmp/27140/"
\end{wideverbatim}

 
\section*{\texttt{*Tsm}}
\label{sec:func-ref-T-*Tsm}


A global variable which may hold a cons pair of two strings with escape
sequences, to switch on and off an alternative transient symbol markup.
If set, \texttt{print} will output these sequences to the console instead of
the standard double quote markup characters.


\begin{wideverbatim}
: (de *Tsm "^[[4m" . "^[[24m")   # vt100 escape sequences for underline
-> *Tsm
: Hello world
-> Hello world
: (off *Tsm)
-> NIL
: "Hello world"                  # No underlining
-> "Hello world"
\end{wideverbatim}

 
\section*{\texttt{+Time}}
\label{sec:func-ref-T-+Time}


Class for clock time values (as calculated by \texttt{time}), a subclass of
\texttt{+Number}. See also \texttt{Database}.


\begin{wideverbatim}
(rel tim (+Time))  # Time of the day
\end{wideverbatim}

 
\section*{\texttt{T}}
\label{sec:func-ref-T-T}


A global constant, evaluating to itself. \texttt{T} is commonly returned as the
boolean value ``true'' (though any non-\texttt{NIL} values could be used). It
represents the absolute maximum, as it is larger than any other object.
As a property key, it is used to store \emph{Pilog}
clauses, and inside Pilog clauses it is the \emph{cut} operator. See also
\texttt{NIL} and and \emph{Comparing}.


\begin{wideverbatim}
: T
-> T
: (= 123 123)
-> T
: (get 'not T)
-> ((@P (1 -> @P) T (fail)) (@P))
\end{wideverbatim}

 
\section*{\texttt{This}}
\label{sec:func-ref-T-This}


Holds the current object during method execution (see \emph{OO Concepts}), or inside the body of a \texttt{with} statement. As it is a normal
symbol, however, it can be used in normal bindings anywhere. See also
\texttt{isa}, \texttt{:}, \texttt{=:}, \texttt{::} and \texttt{var:}.


\begin{wideverbatim}
: (with 'X (println 'This 'is This))
This is X
-> X
: (put 'X 'a 1)
-> 1
: (put 'X 'b 2)
-> 2
: (put 'Y 'a 111)
-> 111
: (put 'Y 'b 222)
-> 222
: (mapcar '((This) (cons (: a) (: b))) '(X Y))
-> ((1 . 2) (111 . 222))
\end{wideverbatim}

 
\section*{\texttt{(t . prg) -> T}}
\label{sec:func-ref-T-(t . prg) -> T}


Executes \texttt{prg}, and returns \texttt{T}. See also \texttt{nil}, \texttt{prog}, \texttt{prog1} and
\texttt{prog2}.


\begin{wideverbatim}
: (t (println 'OK))
OK
-> T
\end{wideverbatim}

 
\section*{\texttt{(tab 'lst 'any ..) -> NIL}}
\label{sec:func-ref-T-(tab 'lst 'any ..) -> NIL}


Print all \texttt{any} arguments in a tabular format. \texttt{lst} should be a list of
numbers, specifying the field width for each argument. All items in a
column will be left-aligned for negative numbers, otherwise
right-aligned. See also \texttt{align}, \texttt{center} and \texttt{wrap}.


\begin{wideverbatim}
: (let Fmt (-3 14 14)
   (tab Fmt "Key" "Rand 1" "Rand 2")
   (tab Fmt "---" "------" "------")
   (for C '(A B C D E F)
      (tab Fmt C (rand) (rand)) ) )
Key        Rand 1        Rand 2
---        ------        ------
A               0    1481765933
B     -1062105905    -877267386
C      -956092119     812669700
D       553475508   -1702133896
E      1344887256   -1417066392
F      1812158119   -1999783937
-> NIL
\end{wideverbatim}

 
\section*{\texttt{(tail 'cnt|lst 'lst) -> lst}}
\label{sec:func-ref-T-(tail 'cnt|lst 'lst) -> lst}


Returns the last \texttt{cnt} elements of \texttt{lst}. If \texttt{cnt} is negative, it is
added to the length of \texttt{lst}. If the first argument is a \texttt{lst}, \texttt{tail}
is a predicate function returning that argument list if it is \texttt{equal} to
the tail of the second argument, and \texttt{NIL} otherwise. \texttt{(tail -2 Lst)} is
equivalent to \texttt{(nth Lst 3)}. See also \texttt{offset}, \texttt{head}, \texttt{last} and
\texttt{stem}.


\begin{wideverbatim}
: (tail 3 '(a b c d e f))
-> (d e f)
: (tail -2 '(a b c d e f))
-> (c d e f)
: (tail 0 '(a b c d e f))
-> NIL
: (tail 10 '(a b c d e f))
-> (a b c d e f)
: (tail '(d e f) '(a b c d e f))
-> (d e f)
\end{wideverbatim}

 
\section*{\texttt{(task 'num ['num] [sym 'any ..] [. prg]) -> lst}}
\label{sec:func-ref-T-(task 'num ['num] [sym 'any ..] [. prg]) -> lst}


A front-end to the \texttt{*Run} global. If called with only a single
\texttt{num} argument, the corresponding entry is removed from the
value of \texttt{*Run}. Otherwise, a new entry is created. If an entry
with that key already exists, an error is issued. For negative
numbers, a second number must be supplied. If
\texttt{sym}/\texttt{any} arguments are given, a \texttt{job}
environment is built for thie \texttt{*Run} entry. See also
\texttt{forked} and \texttt{timeout}.


\begin{wideverbatim}
: (task -10000 5000 N 0 (msg (inc 'N)))            # Install task
-> (-10000 5000 (job '((N . 0)) (msg (inc 'N))))   # for every 10 seconds
: 1                                                # ... after 5 seconds
2                                                  # ... after 10 seconds
3                                                  # ... after 10 seconds
(task -10000)                                      # remove again
-> NIL

: (task (port T 4444) (eval (udp @)))              # Receive RPC via UDP
-> (3 (eval (udp @)))

# Another session (on the same machine)
: (udp "localhost" 4444 '(println *Pid))  # Send RPC message
-> (println *Pid)
\end{wideverbatim}

 
\section*{\texttt{(telStr 'sym) -> sym}}
\label{sec:func-ref-T-(telStr 'sym) -> sym}


Formats a telephone number according to the current \texttt{locale}. If the
string head matches the local country code, it is replaced with \texttt{0},
otherwise \texttt{+} is prepended. See also \texttt{expTel}, \texttt{datStr}, \texttt{money} and
\texttt{format}.


\begin{wideverbatim}
: (telStr "49 1234 5678-0")
-> "+49 1234 5678-0"
: (locale "DE" "de")
-> NIL
: (telStr "49 1234 5678-0")
-> "01234 5678-0"
\end{wideverbatim}

 
\section*{\texttt{(tell ['cnt] 'sym ['any ..]) -> any}}
\label{sec:func-ref-T-(tell ['cnt] 'sym ['any ..]) -> any}


Family IPC: Send an executable list \texttt{(sym any ..)} to all family members
(i.e. all children of the current process, and all other children of the
parent process, see \texttt{fork}) for automatic execution. When the \texttt{cnt}
argument is given and non-zero, it should be the PID of such a process,
and the list will be sent only to that process. \texttt{tell} is also used
internally by \texttt{commit} to notify about database changes. When called
without arguments, no message is actually sent, and the parent process
may grant \texttt{sync} to the next waiting process. See also \texttt{hear}.


\begin{wideverbatim}
: (call 'ps "x")                          # Show processes
  PID TTY      STAT   TIME COMMAND
  ..
 1321 pts/0    S      0:00 /usr/bin/picolisp ..  # Parent process
 1324 pts/0    S      0:01 /usr/bin/picolisp ..  # First child
 1325 pts/0    S      0:01 /usr/bin/picolisp ..  # Second child
 1326 pts/0    R      0:00 ps x
-> T
: *Pid                                    # We are the second child
-> 1325
: (tell 'println '*Pid)                   # Ask all others to print their Pid's
1324
-> *Pid
\end{wideverbatim}

 
\section*{\texttt{(test 'any . prg)}}
\label{sec:func-ref-T-(test 'any . prg)}


Executes \texttt{prg}, and issues an \texttt{error} if the result does not \texttt{match} the
\texttt{any} argument. See also \texttt{assert}.


\begin{wideverbatim}
: (test 12 (* 3 4))
-> NIL
: (test 12 (+ 3 4))
((+ 3 4))
12 -- 'test' failed
?
\end{wideverbatim}

 
\section*{\texttt{(text 'any1 'any ..) -> sym}}
\label{sec:func-ref-T-(text 'any1 'any ..) -> sym}


Builds a new transient symbol (string) from the string representation of
\texttt{any1}, by replacing all occurrences of an at-mark ``\texttt{@}'', followed by
one of the letters''\texttt{1}'' through ``\texttt{9}'', and''\texttt{A}'' through ``\texttt{Z}'', with the
corresponding \texttt{any} argument. In this context ``\texttt{@A}'' refers to the 10th
argument. A literal at-mark in the text can be represented by two
successive at-marks. See also \texttt{pack} and \texttt{glue}.


\begin{wideverbatim}
: (text "abc @1 def @2" 'XYZ 123)
-> "abc XYZ def 123"
: (text "a@@bc.@1" "de")
-> "a@bc.de"
\end{wideverbatim}

 
\section*{\texttt{(tim\$ 'tim ['flg]) -> sym}}
\label{sec:func-ref-T-\texttt{(timtime} \texttt{tim}. If \texttt{flg} is \texttt{NIL}, the format is HH:MM
otherwise it is HH:MM:SS. See also \texttt{\$tim} and \texttt{dat\$}.


\begin{wideverbatim}
: (tim$ (time))
-> "10:57"
: (tim$ (time) T)
-> "10:57:56"
\end{wideverbatim}

 
\section*{\texttt{(timeout ['num])}}
\label{sec:func-ref-T-(timeout ['num])}


Sets or refreshes a timeout value in the \texttt{*Run} global, so that the
current process executes \texttt{bye} after the given period. If called without
arguments, the timeout is removed. See also \texttt{task}.


\begin{wideverbatim}
: (timeout 3600000)           # Timeout after one hour
-> (-1 3600000 (bye))
: *Run                        # Look after a few seconds
-> ((-1 3574516 (bye)))
\end{wideverbatim}

 
\section*{\texttt{(throw 'sym 'any)}}
\label{sec:func-ref-T-(throw 'sym 'any)}


Non-local jump into a previous \texttt{catch} environment with the jump labe
\texttt{sym} (or \texttt{T} as a catch-all). Any pending \texttt{finally} expressions are
executed, local symbol bindings are restored, open files are closed and
internal data structures are reset appropriately, as the environment was
at the time when the corresponding \texttt{catch} was called. Then \texttt{any} is
returned from that \texttt{catch}. See also \texttt{quit}.


\begin{wideverbatim}
: (de foo (N)
   (println N)
   (throw 'OK) )
-> foo
: (let N 1  (catch 'OK (foo 7))  (println N))
7
1
-> 1
\end{wideverbatim}

 
\section*{\texttt{(tick (cnt1 . cnt2) . prg) -> any}}
\label{sec:func-ref-T-(tick (cnt1 . cnt2) . prg) -> any}


Executes \texttt{prg}, then (destructively) adds the number of elapsed user
ticks to the \texttt{cnt1} parameter, and the number of elapsed system ticks to
the \texttt{cnt2} parameter. Thus, \texttt{cnt1} and \texttt{cnt2} will finally contain the
total number of user and system time ticks spent in \texttt{prg} and all
functions called (this works also for recursive functions). For
execution profiling, \texttt{tick} is usually inserted into words with \texttt{prof},
and removed with \texttt{unprof}. See also \texttt{usec}.


\begin{wideverbatim}
: (de foo ()                        # Define function with empty loop
   (tick (0 . 0) (do 100000000)) )
-> foo
: (foo)                             # Execute it
-> NIL
: (pp 'foo)
(de foo NIL
   (tick (97 . 0) (do 100000000)) ) # 'tick' incremented 'cnt1' by 97
-> foo
\end{wideverbatim}

 
\section*{\texttt{(till 'any ['flg]) -> lst|sym}}
\label{sec:func-ref-T-(till 'any ['flg]) -> lst|sym}


Reads from the current input channel till a character contained in \texttt{any}
is found (or until end of file if \texttt{any} is \texttt{NIL}). If \texttt{flg} is \texttt{NIL}, a
list of single-character transient symbols is returned. Otherwise, a
single string is returned. See also \texttt{from} and \texttt{line}.


\begin{wideverbatim}
: (till ":")
abc:def
-> ("a" "b" "c")
: (till ":" T)
abc:def
-> "abc"
\end{wideverbatim}

 
\section*{\texttt{(time ['T]) -> tim}}
\label{sec:func-ref-T-(time ['T]) -> tim}


\texttt{(time 'tim) -> (h m s)}

\texttt{(time 'h 'm ['s]) -> tim | NIL}

\texttt{(time '(h m [s])) -> tim | NIL}

Calculates the time of day, represented as the number of seconds since
midnight. When called without arguments, the current local time is
returned. When called with a \texttt{T} argument, the time of the last call to
\texttt{date} is returned. When called with a single number \texttt{tim}, it is taken
as a time value and a list with the corresponding hour, minute and
second is returned. When called with two or three numbers (or a list of
two or three numbers) for the hour, minute (and optionally the second),
the corresponding time value is returned (or \texttt{NIL} if they do not
represent a legal time). See also \texttt{date}, \texttt{stamp}, \texttt{usec}, \texttt{tim\$} and
\texttt{\$tim}.


\begin{wideverbatim}
: (time)                         # Now
-> 32334
: (time 32334)                   # Now
-> (8 58 54)
: (time 25 30)                   # Illegal time
-> NIL
\end{wideverbatim}

 
\section*{\texttt{(tmp ['any ..]) -> sym}}
\label{sec:func-ref-T-(tmp ['any ..]) -> sym}


Returns the path name to the \texttt{pack}ed \texttt{any} arguments in a process-local
temporary directory. The directory name consists of the path to
``.pil/tmp/'' in the user's home directory, followed by the current
process ID \texttt{*Pid}. This directory is automatically created if necessary,
and removed upon termination of the process (\texttt{bye}). See also \texttt{pil},
\texttt{*Tmp} and \texttt{*Bye} .


\begin{wideverbatim}
: *Pid
-> 27140
: (tmp "foo" 123)
-> "/home/app/.pil/tmp/27140/foo123"
: (out (tmp "foo" 123) (println 'OK))
-> OK
: (dir (tmp))
-> ("foo123")
: (in (tmp "foo" 123) (read))
-> OK
\end{wideverbatim}

 
\section*{\texttt{tolr/3}}
\label{sec:func-ref-T-tolr/3}


\emph{Pilog} predicate that succeeds if the first argument
is either a \emph{substring} or a \texttt{+Sn} \emph{soundex} match of the result of
applying the \texttt{get} algorithm to the following arguments. Typically used
as filter predicate in \texttt{select/3} database queries. See also \texttt{isa/2},
\texttt{same/3}, \texttt{bool/3}, \texttt{range/3}, \texttt{head/3}, \texttt{fold/3} and \texttt{part/3}.


\begin{wideverbatim}
: (?
   @Nr (1 . 5)
   @Nm "Sven"
   (select (@CuSu)
      ((nr +CuSu @Nr) (nm +CuSu @Nm))
      (range @Nr @CuSu nr)
      (tolr @Nm @CuSu nm) )
   (val @Name @CuSu nm) )
 @Nr=(1 . 5) @Nm="Sven" @CuSu={2-2} @Name="Seven Oaks Ltd."
\end{wideverbatim}

 
\section*{\texttt{(touch 'sym) -> sym}}
\label{sec:func-ref-T-(touch 'sym) -> sym}


When \texttt{sym} is an external symbol, it is marked as ``modified'' so that
upon a later \texttt{commit} it will be written to the database file. An
explicit call of \texttt{touch} is only necessary when the value or properties
of \texttt{sym} are indirectly modified.


\begin{wideverbatim}
: (get '{2} 'lst)
-> (1 2 3 4 5)
: (set (cdr (get (touch '{2}) 'lst)) 999)    # Only read-access, need 'touch'
-> 999
: (get '{2} 'lst)                            # Modified second list element
-> (1 999 3 4 5)
\end{wideverbatim}

 
\section*{\texttt{(trace 'sym) -> sym}}
\label{sec:func-ref-T-(trace 'sym) -> sym}


\texttt{(trace 'sym 'cls) -> sym}

\texttt{(trace '(sym . cls)) -> sym}

Inserts a \texttt{\$} trace function call at the beginning of the function or
method body of \texttt{sym}, so that trace information will be printed before
and after execution. Built-in functions (C-function pointer) are
automatically converted to Lisp expressions (see \texttt{expr}). See also
\texttt{*Dbg}, \texttt{traceAll} and \texttt{untrace}, \texttt{debug} and \texttt{lint}.


\begin{wideverbatim}
: (trace '+)
-> +
: (+ 3 4)
 + : 3 4
 + = 7
-> 7
\end{wideverbatim}

 
\section*{\texttt{(traceAll ['lst]) -> sym}}
\label{sec:func-ref-T-(traceAll ['lst]) -> sym}


Traces all Lisp level functions by inserting a \texttt{\$} function call at the
beginning. \texttt{lst} may contain symbols which are to be excluded from that
process. In addition, all symbols in the global variable \texttt{*NoTrace} are
excluded. See also \texttt{trace}, \texttt{untrace} and \texttt{*Dbg}.


\begin{wideverbatim}
: (traceAll)      # Trace all Lisp level functions
-> balance
\end{wideverbatim}

 
\section*{\texttt{(tree 'var 'cls ['hook]) -> tree}}
\label{sec:func-ref-T-(tree 'var 'cls ['hook]) -> tree}


Returns a data structure specifying a database index tree. \texttt{var} and
\texttt{cls} determine the relation, with an optional \texttt{hook} object. See also
\texttt{root}, \texttt{fetch}, \texttt{store}, \texttt{count}, \texttt{leaf}, \texttt{minKey}, \texttt{maxKey}, \texttt{init},
\texttt{step}, \texttt{scan}, \texttt{iter}, \texttt{prune}, \texttt{zapTree} and \texttt{chkTree}.


\begin{wideverbatim}
: (tree 'nm '+Item)
-> (nm . +Item)
\end{wideverbatim}

 
\section*{\texttt{(trim 'lst) -> lst}}
\label{sec:func-ref-T-(trim 'lst) -> lst}


Returns a copy of \texttt{lst} with all trailing whitespace characters or \texttt{NIL}
elements removed. See also \texttt{clip}.


\begin{wideverbatim}
: (trim (1 NIL 2 NIL NIL))
-> (1 NIL 2)
: (trim '(a b " " " "))
-> (a b)
\end{wideverbatim}

 
\section*{\texttt{true/0}}
\label{sec:func-ref-T-true/0}


\emph{Pilog} predicate that always succeeds. See also
\texttt{fail/0} and \texttt{repeat/0}.


\begin{wideverbatim}
:  (? (true))
-> T
\end{wideverbatim}

 
\section*{\texttt{(try 'msg 'obj ['any ..]) -> any}}
\label{sec:func-ref-T-(try 'msg 'obj ['any ..]) -> any}


Tries to send the message \texttt{msg} to the object \texttt{obj}, optionally with
arguments \texttt{any}. If \texttt{obj} is not an object, or if the message cannot be
located in \texttt{obj}, in its classes or superclasses, \texttt{NIL} is returned. See
also \texttt{OO Concepts}, \texttt{send}, \texttt{method}, \texttt{meth}, \texttt{super} and \texttt{extra}.


\begin{wideverbatim}
: (try 'msg> 123)
-> NIL
: (try 'html> 'a)
-> NIL
\end{wideverbatim}

 
\section*{\texttt{(type 'any) -> lst}}
\label{sec:func-ref-T-(type 'any) -> lst}


Return the type (list of classes) of the object \texttt{sym}. See also
\texttt{OO Concepts}, \texttt{isa}, \texttt{class}, \texttt{new} and \texttt{object}.


\begin{wideverbatim}
: (type '{1A;3})
(+Address)
: (type '+DnButton)
-> (+Tiny +Rid +JS +Able +Button)
\end{wideverbatim}

