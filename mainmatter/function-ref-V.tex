%%%%%%%%%%%%%%%%%%%%% chapter.tex %%%%%%%%%%%%%%%%%%%%%%%%%%%%%%%%%
%
% sample chapter
%
% Use this file as a template for your own input.
%
%%%%%%%%%%%%%%%%%%%%%%%% Springer-Verlag %%%%%%%%%%%%%%%%%%%%%%%%%%
%\motto{Use the template \emph{chapter.tex} to style the various elements of your chapter content.}

\chapter{Symbols starting with V}
\label{cha:func-ref-V-functions-starting-with-V}
          
\section*{\texttt{(val 'var) -> any}}
\label{sec:func-ref-V-(val 'var) -> any}


Returns the current value of \texttt{var}. See also \texttt{setq}, \texttt{set} and \texttt{def}.


\begin{wideverbatim}
: (setq L '(a b c))
-> (a b c)
: (val 'L)
-> (a b c)
: (val (cdr L))
-> b
\end{wideverbatim}

 
\section*{\texttt{val/3}}
\label{sec:func-ref-V-val/3}


\emph{Pilog} predicate that returns the value of an
object's attribute. Typically used in database queries. The first
argument is a Pilog variable to bind the value, the second is the
object, and the third and following arguments are used to apply the
\texttt{get} algorithm to that object. See also \texttt{db/3} and \texttt{select/3}.


\begin{wideverbatim}
: (?
   (db nr +Item (2 . 5) @Item)   # Fetch articles 2 through 5
   (val @Nm @Item nm)            # Get item description
   (val @Sup @Item sup nm) )     # and supplier's name
   @Item={3-2} @Nm="Spare Part" @Sup="Seven Oaks Ltd."
   @Item={3-3} @Nm="Auxiliary Construction" @Sup="Active Parts Inc."
   @Item={3-4} @Nm="Enhancement Additive" @Sup="Seven Oaks Ltd."
   @Item={3-5} @Nm="Metal Fittings" @Sup="Active Parts Inc."
-> NIL
\end{wideverbatim}

 
\section*{\texttt{(var sym . any) -> any}}
\label{sec:func-ref-V-(var sym . any) -> any}


\texttt{(var (sym . cls) . any) -> any}

Defines a class variable \texttt{sym} with the initial value \texttt{any} for the
current class, implicitly given by the value of the global variable
\texttt{*Class}, or - in the second form - for the explicitly given class cls.
See also \emph{OO Concepts}, \texttt{rel} and \texttt{var:}.


\begin{wideverbatim}
: (class +A)
-> +A
: (var a . 1)
-> 1
: (var b . 2)
-> 2
: (show '+A)
+A NIL
   b 2
   a 1
-> +A
\end{wideverbatim}

 
\section*{\texttt{(var: sym) -> any}}
\label{sec:func-ref-V-(var: sym) -> any}


Fetches the value of a class variable \texttt{sym} for the current
object \texttt{This}, by searching the property lists of its class(es)
and supperclasses. See also \texttt{OO Concepts}, \texttt{var},
\texttt{with}, \texttt{meta}, \texttt{:}, \texttt{=:} and \texttt{::}.


\begin{wideverbatim}
: (object 'O '(+A) 'a 9 'b 8)
-> O
: (with 'O (list (: a) (: b) (var: a) (var: b)))
-> (9 8 1 2)
\end{wideverbatim}

 
\section*{\texttt{(version ['flg]) -> lst}}
\label{sec:func-ref-V-(version ['flg]) -> lst}


Prints the current version as a string of dot-separated numbers, and
returns the current version as a list of numbers. The JVM- and
C-versions print an additional ``JVM'' or ``C'', respectively, separated by
a space. When \texttt{flg} is non-NIL, printing is suppressed.


\begin{wideverbatim}
$ pil -version
3.0.1.22
: (version T)
-> (3 0 1 22)
\end{wideverbatim}

 
\section*{\texttt{(vi 'sym) -> sym}}
\label{sec:func-ref-V-(vi 'sym) -> sym}


\texttt{(vi 'sym 'cls) -> sym}

\texttt{(vi '(sym . cls)) -> sym}

\texttt{(vi) -> NIL}

Opens the ``vi'' editor on the function or method definition of \texttt{sym}. A
call to \texttt{ld} thereafter will \texttt{load} the modified file. See also \texttt{doc},
\texttt{edit}, \texttt{*Dbg}, \texttt{debug} and \texttt{pp}.


\begin{wideverbatim}
: (vi 'url> '+CuSu)  # Edit the method's source code, then exit from 'vi'
-> T
\end{wideverbatim}

 
\section*{\texttt{(view 'lst ['T]) -> any}}
\label{sec:func-ref-V-(view 'lst ['T]) -> any}


Views \texttt{lst} as tree-structured ASCII graphics. When the \texttt{T} argument is
given, \texttt{lst} should be a binary tree structure (as generated by \texttt{idx}),
which is then shown as a left-rotated tree. See also \texttt{pretty} and
\texttt{show}.


\begin{wideverbatim}
: (balance 'I '(a b c d e f g h i j k l m n o))
-> NIL
: I
-> (h (d (b (a) c) f (e) g) l (j (i) k) n (m) o)

: (view I)
+-- h
|
+---+-- d
|   |
|   +---+-- b
|   |   |
|   |   +---+-- a
|   |   |
|   |   +-- c
|   |
|   +-- f
|   |
|   +---+-- e
|   |
|   +-- g
|
+-- l
|
+---+-- j
|   |
|   +---+-- i
|   |
|   +-- k
|
+-- n
|
+---+-- m
|
+-- o
-> NIL

: (view I T)
         o
      n
         m
   l
         k
      j
         i
h
         g
      f
         e
   d
         c
      b
         a
-> NIL
\end{wideverbatim}

