%%%%%%%%%%%%%%%%%%%%% chapter.tex %%%%%%%%%%%%%%%%%%%%%%%%%%%%%%%%%
%
% sample chapter
%
% Use this file as a template for your own input.
%
%%%%%%%%%%%%%%%%%%%%%%%% Springer-Verlag %%%%%%%%%%%%%%%%%%%%%%%%%%
%\motto{Use the template \emph{chapter.tex} to style the various elements of your chapter content.}



\chapter{Symbols starting with W}
\label{cha:func-ref-W-functions-starting-with-W}

 
\section*{\texttt{(wait ['cnt] . prg) -> any}}
\label{sec:func-ref-W-(wait ['cnt] . prg) -> any}


Waits for a condition. While the result of the execution of \texttt{prg} is
\texttt{NIL}, a \texttt{select} system call is executed for all file descriptors and
timers in the \texttt{VAL} of the global variable \texttt{*Run}. When \texttt{cnt} is
non-\texttt{NIL}, the waiting time is limited to \texttt{cnt} milliseconds. Returns
the result of \texttt{prg}. See also \texttt{key} and \texttt{sync}.


\begin{wideverbatim}
: (wait 2000)                                # Wait 2 seconds
-> NIL
: (prog
   (zero *Cnt)
   (setq *Run                                # Install background loop
      '((-2000 0 (println (inc '*Cnt)))) )   # Increment '*Cnt' every 2 sec
   (wait NIL (> *Cnt 6))                     # Wait until > 6
   (off *Run) )
1                                            # Waiting ..
2
3
4
5
6
7
-> NIL
\end{wideverbatim}

 
\section*{\texttt{(week 'dat) -> num}}
\label{sec:func-ref-W-(week 'dat) -> num}


Returns the number of the week for a given \texttt{date} \texttt{dat}. See also \texttt{day},
\texttt{ultimo}, \texttt{datStr} and \texttt{strDat}.


\begin{wideverbatim}
: (datStr (date))
-> "2007-06-01"
: (week (date))
-> 22
\end{wideverbatim}

 
\section*{\texttt{(when 'any . prg) -> any}}
\label{sec:func-ref-W-(when 'any . prg) -> any}


Conditional execution: When the condition \texttt{any} evaluates to non-\texttt{NIL},
\texttt{prg} is executed and the result is returned. Otherwise \texttt{NIL} is
returned. See also \texttt{unless}.


\begin{wideverbatim}
: (when (> 4 3) (println 'OK) (println 'Good))
OK
Good
-> Good
\end{wideverbatim}

 
\section*{\texttt{(while 'any . prg) -> any}}
\label{sec:func-ref-W-(while 'any . prg) -> any}


Conditional loop: While the condition \texttt{any} evaluates to non-\texttt{NIL},
\texttt{prg} is repeatedly executed. If \texttt{prg} is never executed, \texttt{NIL} is
returned. Otherwise the result of \texttt{prg} is returned. See also \texttt{until}.


\begin{wideverbatim}
: (while (read)
   (println 'got: @) )
abc
got: abc
1234
got: 1234
NIL
-> 1234
\end{wideverbatim}

 
\section*{\texttt{(what 'sym) -> lst}}
\label{sec:func-ref-W-(what 'sym) -> lst}


Returns a list of all internal symbols that match the pattern string
\texttt{sym}. See also \texttt{match}, \texttt{who} and \texttt{can}.


\begin{wideverbatim}
: (what "cd@dr")
-> (cdaddr cdaadr cddr cddddr cdddr cddadr cdadr)
\end{wideverbatim}

 
\section*{\texttt{(who 'any) -> lst}}
\label{sec:func-ref-W-(who 'any) -> lst}


Returns a list of all functions or method definitions that contain the
atom or pattern \texttt{any}. See also \texttt{match}, \texttt{what} and \texttt{can}.


\begin{wideverbatim}
: (who 'caddr)                         # Who is using 'caddr'?
-> ($dat lint1 expDat datStr $tim tim$ mail _gen dat$ datSym)

: (who "Type error")
-> ((mis> . +Link) *Uni (mis> . +Joint))

: (more (who "Type error") pp)         # Pretty print all results
(dm (mis> . +Link) (Val Obj)
   (and
      Val
      (nor (isa (: type) Val) (canQuery Val))
      "Type error" ) )
.                                      # Stop
-> T
\end{wideverbatim}

 
\section*{\texttt{(wipe 'sym|lst) -> sym|lst}}
\label{sec:func-ref-W-(wipe 'sym|lst) -> sym|lst}


Clears the \texttt{VAL} and the property list of \texttt{sym}, or of all symbols in
the list \texttt{lst}. When a symbol is an external symbol, its state is also
set to ``not loaded''. Does nothing when \texttt{sym} is an external symbol that
has been modified or deleted (``dirty'').


\begin{wideverbatim}
: (setq A (1 2 3 4))
-> (1 2 3 4)
: (put 'A 'a 1)
-> 1
: (put 'A 'b 2)
-> 2
: (show 'A)
A (1 2 3 4)
   b 2
   a 1
-> A
: (wipe 'A)
-> A
: (show 'A)
A NIL
-> A
\end{wideverbatim}

 
\section*{\texttt{(with 'sym . prg) -> any}}
\label{sec:func-ref-W-(with 'sym . prg) -> any}


Saves the current object \texttt{This} and sets it to the new value \texttt{sym}. Then
\texttt{prg} is executed, and \texttt{This} is restored to its previous value. The
return value is the result of \texttt{prg}. Used typically to access the local
data of \texttt{sym} in the same manner as inside a method body. \texttt{prg} is not
executed (and \texttt{NIL} is returned) when \texttt{sym} is \texttt{NIL}. \texttt{(with 'X . prg)}
is equivalent to \texttt{(let? This 'X . prg)}.


\begin{wideverbatim}
: (put 'X 'a 1)
-> 1
: (put 'X 'b 2)
-> 2
: (with 'X (list (: a) (: b)))
-> (1 2)
\end{wideverbatim}

 
\section*{\texttt{(wr 'cnt ..) -> cnt}}
\label{sec:func-ref-W-(wr 'cnt ..) -> cnt}


Writes all \texttt{cnt} arguments as raw bytes to the current output channel.
See also \texttt{rd} and \texttt{pr}.


\begin{wideverbatim}
: (out "x" (wr 1 255 257))  # Write to "x"
-> 257
: (hd "x")
00000000  01 FF 01                                         ...
-> NIL
\end{wideverbatim}

 
\section*{\texttt{(wrap 'cnt 'lst) -> sym}}
\label{sec:func-ref-W-(wrap 'cnt 'lst) -> sym}


Returns a transient symbol with all characters in \texttt{lst} \texttt{pack}ed in lines with a maximal length of \texttt{cnt}. See also \texttt{tab}, \texttt{align} and
\texttt{center}.


\begin{wideverbatim}
: (wrap 20 (chop "The quick brown fox jumps over the lazy dog"))
-> "The quick brown fox^Jjumps over the lazy^Jdog"
: (wrap 8 (chop "The quick brown fox jumps over the lazy dog"))
-> "The^Jquick^Jbrown^Jfox^Jjumps^Jover the^Jlazy dog"
\end{wideverbatim}


