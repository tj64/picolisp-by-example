%%%%%%%%%%%%%%%%%%%%% chapter.tex %%%%%%%%%%%%%%%%%%%%%%%%%%%%%%%%%
%
% sample chapter
%
% Use this file as a template for your own input.
%
%%%%%%%%%%%%%%%%%%%%%%%% Springer-Verlag %%%%%%%%%%%%%%%%%%%%%%%%%%
%\motto{Use the template \emph{chapter.tex} to style the various elements of your chapter content.}



\chapter{Symbols starting with Y}
\label{cha:func-ref-Y-functions-starting-with-Y}
 
\section*{\texttt{(yield 'any ['sym]) -> any}}
\label{sec:func-ref-Y-(yield 'any ['sym]) -> any}


(64-bit version only) Transfers control from the current
\emph{coroutine} back to the caller (when the \texttt{sym}
tag is not given), or to some other coroutine (specified by \texttt{sym}) to
continue execution at the point where that coroutine had called \texttt{yield}
before. In the first case, the value \texttt{any} will be returned from the
corresponding \texttt{co} call, in the second case it will be the return value
of that \texttt{yield} call. See also \texttt{stack}, \texttt{catch} and \texttt{throw}.


\begin{wideverbatim}
: (co "rt1"                            # Start first routine
   (msg (yield 1) " in rt1 from rt2")  # Return '1', wait for value from "rt2"
   7 )                                 # Then return '7'
-> 1

: (co "rt2"                            # Start second routine
   (yield 2 "rt1") )                   # Send '2' to "rt1"
2 in rt1 from rt2
-> 7
\end{wideverbatim}

 
\section*{\texttt{(yoke 'any ..) -> any}}
\label{sec:func-ref-Y-(yoke 'any ..) -> any}


Inserts one or several new elements \texttt{any} in front of the list in the
current \texttt{make} environment. \texttt{yoke} returns the last inserted argument.
See also \texttt{link}, \texttt{chain} and \texttt{made}.


\begin{wideverbatim}
: (make (link 2 3) (yoke 1) (link 4))
-> (1 2 3 4)
\end{wideverbatim}


