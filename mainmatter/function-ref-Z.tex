%%%%%%%%%%%%%%%%%%%%% chapter.tex %%%%%%%%%%%%%%%%%%%%%%%%%%%%%%%%%
%
% sample chapter
%
% Use this file as a template for your own input.
%
%%%%%%%%%%%%%%%%%%%%%%%% Springer-Verlag %%%%%%%%%%%%%%%%%%%%%%%%%%
%\motto{Use the template \emph{chapter.tex} to style the various elements of your chapter content.}



\chapter{Symbols starting with Z}
\label{cha:func-ref-Z-functions-starting-with-Z}

 
\section*{\texttt{*Zap}}
\label{sec:func-ref-Z-*Zap}


A global variable holding a list and a pathname. If given, and the value
of \texttt{*Solo} is \texttt{NIL}, external symbols which are no longer accessible can
be collected in the CAR, e.g. during DB tree processing, and written to
the file in the CDR at the next \texttt{commit}. A (typically periodic) call to
\texttt{zap\_} will clean them up later.


\begin{wideverbatim}
: (setq *Zap '(NIL . "db/app/_zap"))
-> "db/app/_zap"
\end{wideverbatim}

 
\section*{\texttt{(zap 'sym) -> sym}}
\label{sec:func-ref-Z-(zap 'sym) -> sym}


``Delete'' the symbol \texttt{sym}. For internal symbols, that means to remove it
from the internal index, effectively transforming it to a transient
symbol. For external symbols, it means to mark it as ``deleted'', so that
upon a later \texttt{commit} it will be removed from the database file. See
also \texttt{intern}.


\begin{wideverbatim}
: (de foo (Lst) (car Lst))          # 'foo' calls 'car'
-> foo
: (zap 'car)                        # Delete the symbol 'car'
-> "car"
: (pp 'foo)
(de foo (Lst)
   ("car" Lst) )                    # 'car' is now a transient symbol
-> foo
: (foo (1 2 3))                     # 'foo' still works
-> 1
: (car (1 2 3))                     # Reader returns a new 'car' symbol
!? (car (1 2 3))
car -- Undefined
?
\end{wideverbatim}

 
\section*{\texttt{(zapTree 'sym)}}
\label{sec:func-ref-Z-(zapTree 'sym)}


Recursively deletes a tree structure from the database. See also \texttt{tree},
\texttt{chkTree} and \texttt{prune}.


\begin{wideverbatim}
: (zapTree (cdr (root (tree 'nm '+Item))))
\end{wideverbatim}

 
\section*{\texttt{(zap\_)}}
\label{sec:func-ref-Z-(zap)} 

Delayed deletion (with zap) of external symbols which were collected
e.g. during DB tree processing. An auxiliary file (with the name taken
from the CDR of the value of \texttt{*Zap}, concatenated with a
``\texttt{\_}'' character) is used as an intermediary file.


\begin{wideverbatim}
: *Zap
-> (NIL . "db/app/Z")
: (call 'ls "-l" "db/app")
...
-rw-r--r-- 1 abu abu     1536 2007-06-23 12:34 Z
-rw-r--r-- 1 abu abu     1280 2007-05-23 12:15 Z_
...
: (zap_)
...
: (call 'ls "-l" "db/app")
...
-rw-r--r-- 1 abu abu     1536 2007-06-23 12:34 Z_
...
\end{wideverbatim}

              
\section*{\texttt{(zero var ..) -> 0} }
\label{sec:(zero var ..) -> 0}

Stores \texttt{0} in all \texttt{var} arguments. See also \texttt{one}, \texttt{on}, \texttt{off} and
\texttt{onOff}.


\begin{wideverbatim}
: (zero A B)
-> 0
: A
-> 0
: B
-> 0
\end{wideverbatim}


