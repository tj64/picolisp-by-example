%%%%%%%%%%%%%%%%%%%%% chapter.tex %%%%%%%%%%%%%%%%%%%%%%%%%%%%%%%%%
%
% sample chapter
%
% Use this file as a template for your own input.
%
%%%%%%%%%%%%%%%%%%%%%%%% Springer-Verlag %%%%%%%%%%%%%%%%%%%%%%%%%%
%\motto{Use the template \emph{chapter.tex} to style the various elements of your chapter content.}

\chapter{Rosetta Code Tasks starting with W}

\section*{Walk a directory/Non-recursively}

Walk a given directory and print the \emph{names} of files matching a
given pattern.

\textbf{Note:} This task is for non-recursive methods. These tasks
should read a \emph{single directory}, not an entire directory tree. For
code examples that read entire directory trees, see
\emph{Walk Directory Tree}


\begin{wideverbatim}

(for F (dir "@src/")                         # Iterate directory
   (when (match '`(chop "s@.c") (chop F))    # Matches 's*.c'?
      (println F) ) )                        # Yes: Print it

Output:

"start.c"
"ssl.c"
"subr.c"
"sym.c"
...

\end{wideverbatim}

\pagebreak{}
\section*{Walk a directory/Recursively}

Walk a given directory \emph{tree} and print files matching a given
pattern.

\textbf{Note:} This task is for recursive methods. These tasks should
read an entire directory tree, not a \emph{single directory}. For code
examples that read a \emph{single directory}, see \emph{Walk a
  directory/Non-recursively}.


\begin{wideverbatim}

(let Dir "."
   (recur (Dir)
      (for F (dir Dir)
         (let Path (pack Dir "/" F)
            (cond
               ((=T (car (info Path)))             # Is a subdirectory?
                  (recurse Path) )                 # Yes: Recurse
               ((match '`(chop "s@.l") (chop F))   # Matches 's*.l'?
                  (println Path) ) ) ) ) ) )       # Yes: Print it

Output:

"./src64/sym.l"
"./src64/subr.l"
...

\end{wideverbatim}

\pagebreak{}
\section*{Web scraping}

Create a program that downloads the time from this URL:
\href{http://tycho.usno.navy.mil/cgi-bin/timer.pl}{http://tycho.usno.navy.mil/cgi-bin/timer.pl}
and then prints the current UTC time by extracting just the UTC time
from the web page's \emph{HTML}.

If possible, only use libraries that come at no \emph{extra} monetary
cost with the programming language and that are widely available and
popular such as \href{http://www.cpan.org/}{CPAN} for Perl or
\emph{Boost} for C++.

\begin{wideverbatim}

(load "@lib/http.l")

(client "tycho.usno.navy.mil" 80 "cgi-bin/timer.pl"
   (when (from "<BR>")
      (pack (trim (till "U"))) ) )

Output:

-> "Feb. 19, 18:11:37"

\end{wideverbatim}

\pagebreak{}
\section*{Window creation}

Display a \emph{GUI} window. The window need not have any
contents, but should respond to requests to be closed.

\begin{wideverbatim}

(load "@lib/openGl.l")

(glutInit)
(glutCreateWindow "Goodbye, World!")
(keyboardFunc '(() (bye)))
(glutMainLoop)

\end{wideverbatim}

\pagebreak{}
\section*{Window creation/X11}

Create a simple X11 application, using an X11 protocol library such as
Xlib or XCB, that draws a box and ``Hello World'' in a window.
Implementations of this task should \emph{avoid using a toolkit} as much
as possible.

\begin{wideverbatim}

The following script works in the 32-bit version, using inlined C code

#!/usr/bin/picolisp /usr/lib/picolisp/lib.l

(load "@lib/misc.l" "@lib/gcc.l")

(gcc "x11" '("-lX11") 'simpleWin)

#include <X11/Xlib.h>

any simpleWin(any ex) {
   any x = cdr(ex);
   int dx, dy;
   Display *disp;
   int scrn;
   Window win;
   XEvent ev;

   x = cdr(ex),  dx = (int)evCnt(ex,x);
   x = cdr(x),  dy = (int)evCnt(ex,x);
   x = evSym(cdr(x));
   if (disp = XOpenDisplay(NULL)) {
      char msg[bufSize(x)];

      bufString(x, msg);
      scrn = DefaultScreen(disp);
      win = XCreateSimpleWindow(disp, RootWindow(disp,scrn), 0, 0, dx, dy,
                           1, BlackPixel(disp,scrn), WhitePixel(disp,scrn) );
      XSelectInput(disp, win, ExposureMask | KeyPressMask | ButtonPressMask);
      XMapWindow(disp, win);
      for (;;) {
         XNextEvent(disp, \&ev);
         switch (ev.type) {
         case Expose:
            XDrawRectangle(disp, win, DefaultGC(disp, scrn), 10, 10, dx-20, dy-20);
            XDrawString(disp, win, DefaultGC(disp, scrn), 30, 40, msg, strlen(msg));
            break;
         case KeyPress:
         case ButtonPress:
            XCloseDisplay(disp);
            return Nil;
         }
      }
   }
   return mkStr("Can't open Display");
}
/**/

(simpleWin 300 200 "Hello World")
(bye)

\end{wideverbatim}

\pagebreak{}
\section*{Window management}

Treat windows or at least window identities as
\href{http://en.wikipedia.org/wiki/First-class\_object}{first class
objects}.

\begin{itemize}
\item
  Store window identities in variables, compare them for equality.
\item
  Provide examples of performing some of the following:

  \begin{itemize}
  \item
    hide, show, close, minimize, maximize, move, and resize a window.
  \end{itemize}
\end{itemize}

The window of interest may or may not have been created by your program.



\begin{wideverbatim}

The following works on ErsatzLisp, the Java version of PicoLisp.

\$ ersatz/pil +
: (setq
   JFrame         "javax.swing.JFrame"
   MAXIMIZED_BOTH (java (public JFrame 'MAXIMIZED_BOTH))
   ICONIFIED      (java (public JFrame 'ICONIFIED))
   Win            (java JFrame T "Window") )
-> \$ JFrame

# Compare for equality
: (== Win Win)
-> T

# Set window visible
(java Win 'setLocation 100 100)
(java Win 'setSize 400 300)
(java Win 'setVisible T)

# Hide window
(java Win 'hide)

# Show again
(java Win 'setVisible T)

# Move window
(java Win 'setLocation 200 200)

# Iconify window
(java Win 'setExtendedState
   (| (java (java Win 'getExtendedState)) ICONIFIED) )

# De-conify window
(java Win 'setExtendedState
   (\& (java (java Win 'getExtendedState)) (x| (hex "FFFFFFFF") ICONIFIED)) )

# Maximize window
(java Win 'setExtendedState
   (| (java (java Win 'getExtendedState)) MAXIMIZED_BOTH) )

# Close window
(java Win 'dispose)

\end{wideverbatim}

\pagebreak{}
\section*{Wireworld}


\href{http://en.wikipedia.org/wiki/Wireworld}{Wireworld} is a cellular
automaton with some similarities to
\emph{Conway's Game of Life}. It is
capable of doing sophisticated computations (e.g., calculating
primeness!) with appropriate programs, and is much simpler to program
for.

A wireworld arena consists of a cartesian grid of cells, each of which
can be in one of four states. All cell transitions happen
simultaneously. The cell transition rules are this:

\ctable[pos = H, center, botcap]{lll}
{% notes
}
{% rows
\FL
Input State & Output State & Condition
\ML
\texttt{empty} & \texttt{empty} & 
\\\noalign{\medskip}
\texttt{electron~head~} & \texttt{electron~tail~} & 
\\\noalign{\medskip}
\texttt{electron~tail~} & \texttt{conductor} & 
\\\noalign{\medskip}
\texttt{conductor} & \texttt{electron~head~} & if 1 or 2 cells in the
\href{http://en.wikipedia.org/wiki/Moore\_neighborhood}{neighborhood} of
the cell 
\\ & &  are in the state ``\texttt{electron head}''
\\\noalign{\medskip}
\texttt{conductor} & \texttt{conductor} & otherwise
\LL
}

To implement this task, create a program that reads a wireworld program
from a file and displays an animation of the processing. Here is a
sample description file (using "\texttt{H}" for an electron head,
"\texttt{t}" for a tail, "\texttt{.}" for a conductor and a space for
empty) you may wish to test with, which demonstrates two cycle-3
generators and an inhibit gate:

\begin{verbatim}
tH.........
.   .
   ...
.   .
Ht.. ......
\end{verbatim}

While text-only implementations of this task are possible, mapping cells
to pixels is advisable if you wish to be able to display large designs.
The logic is not significantly more complex.



\begin{wideverbatim}

This example uses 'grid' from "lib/simul.l", which maintains a two-dimensional
structure.

(load "@lib/simul.l")

(let
   (Data (in "wire.data" (make (while (line) (link @))))
      Grid (grid (length (car Data)) (length Data)) )
   (mapc
      '((G D) (mapc put G '(val .) D))
      Grid
      (apply mapcar (flip Data) list) )
   (loop
      (disp Grid T
         '((This) (pack " " (: val) " ")) )
      (wait 1000)
      (for Col Grid
         (for This Col
            (case (=: next (: val))
               ("H" (=: next "t"))
               ("t" (=: next "."))
               ("."
                  (when
                     (>=
                        2
                        (cnt # Count neighbors
                           '((Dir) (= "H" (get (Dir This) 'val)))
                           (quote
                              west east south north
                              ((X) (south (west X)))
                              ((X) (north (west X)))
                              ((X) (south (east X)))
                              ((X) (north (east X))) ) )
                        1 )
                     (=: next "H") ) ) ) ) )
      (for Col Grid  # Update
         (for This Col
            (=: val (: next)) ) )
      (prinl) ) )

\end{wideverbatim}

\begin{wideverbatim}

Output:

   +---+---+---+---+---+---+---+---+---+---+---+
 5 | t | H | . | . | . | . | . | . | . | . | . |
   +---+---+---+---+---+---+---+---+---+---+---+
 4 | . |   |   |   | . |   |   |   |   |   |   |
   +---+---+---+---+---+---+---+---+---+---+---+
 3 |   |   |   | . | . | . |   |   |   |   |   |
   +---+---+---+---+---+---+---+---+---+---+---+
 2 | . |   |   |   | . |   |   |   |   |   |   |
   +---+---+---+---+---+---+---+---+---+---+---+
 1 | H | t | . | . |   | . | . | . | . | . | . |
   +---+---+---+---+---+---+---+---+---+---+---+
     a   b   c   d   e   f   g   h   i   j   k

   +---+---+---+---+---+---+---+---+---+---+---+
 5 | . | t | H | . | . | . | . | . | . | . | . |
   +---+---+---+---+---+---+---+---+---+---+---+
 4 | H |   |   |   | . |   |   |   |   |   |   |
   +---+---+---+---+---+---+---+---+---+---+---+
 3 |   |   |   | . | . | . |   |   |   |   |   |
   +---+---+---+---+---+---+---+---+---+---+---+
 2 | H |   |   |   | . |   |   |   |   |   |   |
   +---+---+---+---+---+---+---+---+---+---+---+
 1 | t | . | . | . |   | . | . | . | . | . | . |
   +---+---+---+---+---+---+---+---+---+---+---+
     a   b   c   d   e   f   g   h   i   j   k

   +---+---+---+---+---+---+---+---+---+---+---+
 5 | H | . | t | H | . | . | . | . | . | . | . |
   +---+---+---+---+---+---+---+---+---+---+---+
 4 | t |   |   |   | . |   |   |   |   |   |   |
   +---+---+---+---+---+---+---+---+---+---+---+
 3 |   |   |   | . | . | . |   |   |   |   |   |
   +---+---+---+---+---+---+---+---+---+---+---+
 2 | t |   |   |   | . |   |   |   |   |   |   |
   +---+---+---+---+---+---+---+---+---+---+---+
 1 | . | H | . | . |   | . | . | . | . | . | . |
   +---+---+---+---+---+---+---+---+---+---+---+
     a   b   c   d   e   f   g   h   i   j   k

\end{wideverbatim}

\pagebreak{}
\section*{Word wrap}

Even today, with proportional fonts and complex layouts, there are
still \emph{cases} where you need to wrap text at a specified column.
The basic task is to wrap a paragraph of text in a simple way in your
language. If there is a way to do this that is built-in, trivial, or
provided in a standard library, show that. Otherwise implement the
\href{http://en.wikipedia.org/wiki/Word\_wrap\#Minimum\_length}{minimum
  length greedy algorithm from Wikipedia.}

Show your routine working on a sample of text at two different wrap
columns.

\textbf{Extra credit!} Wrap text using a more sophisticated algorithm
such as the Knuth and Plass TeX algorithm. If your language provides
this, you get easy extra credit, but you \emph{must reference
documentation} indicating that the algorithm is something better than a
simple minimimum length algorithm.

If you have both basic and extra credit solutions, show an example where
the two algorithms give different results.

\begin{wideverbatim}

'[http://software-lab.de/doc/refW.html#wrap wrap]' is a built-in.

: (prinl (wrap 12 (chop "The quick brown fox jumps over the lazy dog")))
The quick
brown fox
jumps over
the lazy dog
-> "The quick^Jbrown fox^Jjumps over^Jthe lazy dog"

\end{wideverbatim}

\pagebreak{}
\section*{Write float arrays to a text file}

Write two equal-sized numerical arrays `x' and `y' to a two-column text
file named `filename'.

The first column of the file contains values from an `x'-array with a
given `xprecision', the second -- values from `y'-array with
`yprecision'.

For example, considering:

\begin{wideverbatim}
   x = {1, 2, 3, 1e11};
   y = {1, 1.4142135623730951, 1.7320508075688772, 316227.76601683791}; 
                                                          /* sqrt(x) */
   xprecision = 3;
   yprecision = 5;
\end{wideverbatim}

The file is:

\begin{wideverbatim}
   1    1
   2    1.4142
   3    1.7321
   1e+011   3.1623e+005
\end{wideverbatim}

This task is intended as a subtask for \emph{Measure relative
  performance of sorting algorithms implementations}.



\begin{wideverbatim}

An exponential format like "1e11" is not supported

(setq *Xprecision 3  *Yprecision 5)

(scl 7)
(mapc
   '((X Y)
      (prinl
         (round X *Xprecision)
         "  "
         (round Y *Yprecision) ) )
   (1.0 2.0 3.0)
   (1.0 1.414213562 1.732050807) )

Output:

1.000  1.00000
2.000  1.41421
3.000  1.73205

\end{wideverbatim}

\pagebreak{}
\section*{Write to Windows event log}

Write script status to the Windows Event Log

\begin{wideverbatim}

PicoLisp doesn't run on Windows. In case of Linux, the equivalent of the event
log is the syslog. It can be written with
'[http://software-lab.de/doc/refN.html#native native]' C functions, or simply
with the 'logger' utility:

: (call 'logger "This is a test")
-> T

: (call 'logger "This" 'is "another" 'test)
-> T

\end{wideverbatim}



% \input{referenc}
