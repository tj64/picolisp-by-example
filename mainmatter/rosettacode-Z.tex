%%%%%%%%%%%%%%%%%%%%% chapter.tex %%%%%%%%%%%%%%%%%%%%%%%%%%%%%%%%%
%
% sample chapter
%
% Use this file as a template for your own input.
%
%%%%%%%%%%%%%%%%%%%%%%%% Springer-Verlag %%%%%%%%%%%%%%%%%%%%%%%%%%
%\motto{Use the template \emph{chapter.tex} to style the various elements of your chapter content.}

\chapter{Rosetta Code Tasks starting with Z}

\section*{Zebra puzzle}

The \href{http://en.wikipedia.org/wiki/Zebra\_puzzle}{Zebra puzzle},
a.k.a. Einstein's Riddle, is a logic puzzle which is to be solved
programmatically. It has several variants, one of them this:

\begin{enumerate}
\item
  There are five houses.
\item
  The English man lives in the red house.
\item
  The Swede has a dog.
\item
  The Dane drinks tea.
\item
  The green house is immediately to the left of the white house.
\item
  They drink coffee in the green house.
\item
  The man who smokes Pall Mall has birds.
\item
  In the yellow house they smoke Dunhill.
\item
  In the middle house they drink milk.
\item
  The Norwegian lives in the first house.
\item
  The man who smokes Blend lives in the house next to the house with
  cats.
\item
  In a house next to the house where they have a horse, they smoke
  Dunhill.
\item
  The man who smokes Blue Master drinks beer.
\item
  The German smokes Prince.
\item
  The Norwegian lives next to the blue house.
\item
  They drink water in a house next to the house where they smoke Blend.
\end{enumerate}

The question is, who owns the zebra?

Additionally, list the solution for all the houses. Optionally, show the
solution is unique.

cf. \emph{Dinesman's multiple-dwelling problem}



\begin{wideverbatim}

(be match (@House @Person @Drink @Pet @Cigarettes)
   (permute (red blue green yellow white) @House)
   (left-of @House white  @House green)

   (permute (Norwegian English Swede German Dane) @Person)
   (has @Person English  @House red)
   (equal @Person (Norwegian . @))
   (next-to @Person Norwegian  @House blue)

   (permute (tea coffee milk beer water) @Drink)
   (has @Drink tea  @Person Dane)
   (has @Drink coffee  @House green)
   (equal @Drink (@ @ milk . @))

   (permute (dog birds cats horse zebra) @Pet)
   (has @Pet dog  @Person Swede)

   (permute (Pall-Mall Dunhill Blend Blue-Master Prince) @Cigarettes)
   (has @Cigarettes Pall-Mall  @Pet birds)
   (has @Cigarettes Dunhill  @House yellow)
   (next-to @Cigarettes Blend  @Pet cats)
   (next-to @Cigarettes Dunhill  @Pet horse)
   (has @Cigarettes Blue-Master  @Drink beer)
   (has @Cigarettes Prince  @Person German)

   (next-to @Drink water  @Cigarettes Blend) )

\end{wideverbatim}

\begin{wideverbatim}


(be has ((@A . @X) @A (@B . @Y) @B))
(be has ((@ . @X) @A (@ . @Y) @B)
   (has @X @A @Y @B) )

(be right-of ((@A . @X) @A (@ @B . @Y) @B))
(be right-of ((@ . @X) @A (@ . @Y) @B)
   (right-of @X @A @Y @B) )

(be left-of ((@ @A . @X) @A (@B . @Y) @B))
(be left-of ((@ . @X) @A (@ . @Y) @B)
   (left-of @X @A @Y @B) )

(be next-to (@X @A @Y @B) (right-of @X @A @Y @B))
(be next-to (@X @A @Y @B) (left-of @X @A @Y @B))

Test:

(pilog '((match @House @Person @Drink @Pet @Cigarettes))
   (let Fmt (-8 -11 -8 -7 -11)
      (tab Fmt "HOUSE" "PERSON" "DRINKS" "HAS" "SMOKES")
      (mapc '(@ (pass tab Fmt))
         @House @Person @Drink @Pet @Cigarettes ) ) )

Output:

HOUSE   PERSON     DRINKS  HAS    SMOKES
yellow  Norwegian  water   cats   Dunhill
blue    Dane       tea     horse  Blend
red     English    milk    birds  Pall-Mall
green   German     coffee  zebra  Prince
white   Swede      beer    dog    Blue-Master

\end{wideverbatim}

\pagebreak{}
\section*{Zig-zag matrix}


Produce a zig-zag array. A zig-zag array is a square arrangement of the
first \texttt{N2} integers, where the numbers increase sequentially as
you zig-zag along the anti-diagonals of the array. For a graphical
representation, see
\href{http://en.wikipedia.org/wiki/Image:JPEG\_ZigZag.svg}{JPG zigzag}
(JPG uses such arrays to encode images).

For example, given \texttt{5}, produce this array:

\begin{verbatim}
 0  1  5  6 14
 2  4  7 13 15
 3  8 12 16 21
 9 11 17 20 22
10 18 19 23 24
\end{verbatim}


\begin{wideverbatim}

This example uses 'grid' from "lib/simul.l", which maintains a two-dimensional
structure and is normally used for simulations and board games.

(load "@lib/simul.l")

(de zigzag (N)
   (prog1 (grid N N)
      (let (D '(north west  south east  .)  E '(north east .)  This 'a1)
         (for Val (* N N)
            (=: val Val)
            (setq This
               (or
                  ((cadr D) ((car D) This))
                  (prog
                     (setq D (cddr D))
                     ((pop 'E) This) )
                  ((pop 'E) This) ) ) ) ) ) )

(mapc
   '((L)
      (for This L (prin (align 3 (: val))))
      (prinl) )
   (zigzag 5) )

Output:

  1  2  6  7 15
  3  5  8 14 16
  4  9 13 17 22
 10 12 18 21 23
 11 19 20 24 25


\end{wideverbatim}                 


% \input{referenc}
